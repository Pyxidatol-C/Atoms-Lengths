\section{Finite length in characteristic zero}
\label{sec:characteristic-zero}

In this section, we present the first of our two main results, which is a  method for proving the finite length property, assuming that the field has characteristic zero. Under this assumption, we will establish the finite length property for the Rado graph (Example~\ref{ex:rado-graph}) and for the vector space atoms (Example~\ref{ex:vector-space-atoms}). These are new results. We also think that the proof itself, even when applied to get already known results, is of independent interest and arguably simpler than previously known proofs.

The method that we use will work for structures that satisfy the following condition. 
 
\begin{definition}
    [Oligomorphic approximation]
    \label{def:oligomorphic-approximation} We say that a structure  $\A$ has \emph{oligomorphic approximation} if it is homogeneous and,  for every $d \in \set{1,2,\ldots}$, there exists a family $\Bclass$ of finite substructures of $\A$ such that: 
    \begin{enumerate}
        \item \label{item:oligomorphic-approximation-embedding} every finite substructure of $\A$ embeds into some $\B \in \Bclass$; and
        \item \label{item:oligomorphic-approximation-orbits} there is a common finite upper bound on the number of orbits in $\B^d$ for $\B \in \Bclass$.
    \end{enumerate}
\end{definition}

This is a relaxation of a stronger notion of  \emph{smooth approximation} known in model theory. There, the family $\Bclass$ is independent of $d$, and there are other requirements as well~\cite[p.~440]{KLM89}. 

\begin{theorem}\label{thm:have-oligomorphic-approximation}
    The following structures have oligomorphic approximation: 
    \begin{enumerate}
        \item the structure with equality only from Example~\ref{ex:equality-atoms};
        \item the vector space from Example~\ref{ex:vector-space-atoms}, for any finite field $\innerfield$;
        \item the Rado graph from Example~\ref{ex:rado-graph}.
    \end{enumerate}
\end{theorem}
Before proving the above theorem, let us observe that the dense linear order (Example~\ref{ex:order-atoms}) does not have oligomorphic approximation.


\begin{non-example}[rational numbers with order]\label{non-ex:weak-smooth-approximation}
A non-example is the rational numbers with the usual order. The finite substructures in this case are finite linear orders, and already for dimension $d=1$, a finite linear order of size $n$ will have $n$ orbits. Hence, we cannot have a common finite upper bound on the number of orbits in $\B^d$ for $\B \in \Bclass$.
\end{non-example}

\begin{proof}[Proof of Theorem~\ref{thm:have-oligomorphic-approximation}]

\ 
    \begin{enumerate}
        \item For the structure with equality only, we can choose $\Bclass$ so that it is independent of $d$, and this is simply all finite structures with equality only. For sufficiently large $\B \in \Bclass$, the number of orbits in $\B^d$ is the same as the number of orbits in $\A^d$. 
        \item For the vector space structure, as in the previous example, we can choose $\Bclass$ independently of $d$, namely the family of vector spaces of finite dimension. This argument applies to any finite field $\innerfield$.
        \item The most interesting case is the Rado graph. The witness for oligomorphic approximation will be a family of \emph{symplectic graphs}, see~\cite[Sec.~8.11]{GR01}.\footnote{We are grateful to Ehud Hrushovski for drawing our attention to this construction.} For every $n \in \set{1,2,\ldots}$ define a finite graph as follows. The set of vertices is the vector space over the two-element field with basis 
\begin{align*}
\set{e_1,\ldots,e_n, f_1,\ldots,f_n}.
\end{align*}
Since the field has two elements, we can view vertices as subsets of this basis. In this graph, there is an edge between vertices $v$ and $w$ if and only if the sets 
\begin{align*}
\setbuild{ i \in \set{1,\ldots,n}}{$e_i \in v$ and $f_i \in w$}\\
\setbuild{ i \in \set{1,\ldots,n}}{$f_i \in v$ and $e_i \in w$}
\end{align*}
have different sizes modulo two. These graphs satisfy condition~\eqref{item:oligomorphic-approximation-embedding} from Definition~\ref{def:oligomorphic-approximation}, i.e.~every finite graph embeds in some symplectic graph, see~\cite[Thm.~8.11.2]{GR01}. In Appendix~\ref{sec:appendix-symplectic}, we prove condition~\eqref{item:oligomorphic-approximation-orbits}, i.e.~that the number of orbits of $d$-tuples in symplectic graphs is uniformly bounded by a function of $d$ only.
    \end{enumerate}
\end{proof}

The main result of this section is the following theorem.

\begin{theorem}\label{thm:weak-smooth-approximation-finite-length}
    If an oligomorphic structure $\A$ has oligomorphic approximation, then it has the finite length property over any field of characteristic $0$.
\end{theorem}

Combining Theorems~\ref{thm:have-oligomorphic-approximation} and~\ref{thm:weak-smooth-approximation-finite-length}, we can get the following results, both old and new, on the finite length property.

\begin{corollary}\label{cor:weak-finite-length}
    Over any field of characteristic $0$, the following structures have  the finite length property: (a) the equality atoms; (b) the vector atoms; and (c) the Rado graph.
\end{corollary}

As mentioned in the introduction, the finite length property was already known for the equality atoms, for arbitrary fields. The results for the vector atoms and the Rado graph are new. The assumption on characteristic zero is important, at least in case of the  vector atoms, where  the finite length property is known to fail over finite fields~\cite[Sec.~4.4]{BFKM24}. Later on in this paper, we will prove the result for the Rado graph again using a different method that works for any field.

The rest of this section is devoted to proving Theorem~\ref{thm:weak-smooth-approximation-finite-length}.

\begin{proof}
    [Proof of Theorem~\ref{thm:weak-smooth-approximation-finite-length}]
Fix a structure $\A$, which has oligomorphic approximation, and  a field of characteristic zero. Since the field is fixed, we omit the field subscript and write $\Lin X$ for linear combinations of elements in $X$ that use coefficients from that field. Fix some power $d \in \set{1,2,\ldots}$.  Our goal is to show that 
$
\Lin \A^d
$
 has  the finite length property. For technical reasons, we apply the assumption on oligomorphic approximation not to $d$, but to $2d$, yielding  some class $\Bclass$ of finite structures that satisfies Definition~\ref{def:oligomorphic-approximation}.


    \begin{lemma}\label{lem:reduce-to-finite-substructures}
        For every $d \in \set{1,2,\ldots}$ we have 
        \begin{align*}
        \text{length of }\Lin \A^d
        \quad \leq \quad  \sup_{\B \in \Bclass} \text{\ length of }\Lin \B^d.
        \end{align*}
   %     where the supremum ranges over finite sets of atoms.
    \end{lemma}
    \begin{proof}
        Consider some chain of equivariant subspaces 
        \begin{align*}
        V_0 \subset V_1 \subset \ldots \subset V_n = \Lin \A^d,
        \end{align*}
        where equivariance is with respect to automorphisms of $\A$. 
        For each $i \in \set{1,\ldots,n}$, choose some vector  that is in $V_i$ but not in $V_{i-1}$. Let $S$ be the (finite) set of atoms that appear in these chosen vectors, and choose some $\B \in \Bclass$ which contains all atoms from $S$. (We can assume without loss of generality that $\Bclass$ is closed under isomorphism, and therefore every finite subset $S$ of $\A$ is contained in some $\B \in \Bclass$.) Define  
        \begin{align*}
        W_i = V_i \cap \Lin \B^d.
        \end{align*}
        By homogeneity, every automorphism of $\B$ extends to an automorphism of $\A$, and therefore the space $W_i$ is  equivariant with respect to automorphisms of $\B$.  The chain of $W_i$'s continues to be strictly growing, since it contains vectors that witness the growth of the original chain. Hence, the new chain  witnesses that the length of $\Lin \B^d$ is at least $n$. 
    \end{proof}


    Thanks to the above lemma, it remains to show that the length of $\Lin \B^d$ is bounded by some number that depends only on $d$. In our proof, this bound will be the number of orbits in $\A^{2d}$. What we have gained by moving from $\A$ to $\Bclass$ is that our vector spaces now have finite (albeit unbounded) linear dimension, and the group actions use finite (albeit unbounded) groups. This will let us leverage a central result from representation theory, called Maschke's Theorem (see e.g.~\cite[Chap.~8]{JamesLiebeck01}), which decomposes  spaces  into irreducible parts.  (In the theorem below, $\oplus$ stands for the direct sum of vector spaces, i.e.~$V \oplus W$ is the set of pairs $(v,w)$ where $v \in V$ and $w \in W$, with the operations defined coordinate-wise.)

        \begin{theorem}[Maschke's Theorem]
          Let $V$ be a finite dimensional vector space over a field of characteristic zero, equipped with an action of a finite group $G$. Then $V$ can be decomposed as 
            \begin{align*}
            V = V_1 \oplus \cdots \oplus V_n,
            \end{align*}
            where each $V_i$ is an equivariant subspace (with respect to the action of $G$)  and  is  irreducible, i.e.~the only equivariant subspaces of $V_i$ are the zero space  and the full space $V_i$.
        \end{theorem}


    We will use Maschke's Theorem to bound the length of the vector  spaces $\Lin \B^d$. 
    For two vector spaces $V$ and $W$ equipped with an action of the same group $G$, let us write 
    \begin{align*}
    \lineqfun V W
    \end{align*}
    for the set of all those linear maps from $V$ to $W$ which are equivariant with respect to the action of $G$. (The group and its action are  implicit in this notation.) Elements of this set are closed under taking linear combinations, and therefore the set can be seen as a vector space. (We do not care to equip this space with an action of $G$.) In particular, it is meaningful to talk about the dimension of this vector space. 
    \begin{lemma}\label{lem:dim-bounds-length}
        Let $V$ be a finite dimensional vector space over a field of characteristic zero, equipped with an action of a finite group $G$. Then 
        \begin{align*}
        \text{length of\, $V$} \quad \leq  \quad \text{dimension of\, }  \lineqfun V V.
        \end{align*}
    \end{lemma}
    \begin{proof}
        Apply Maschke's Theorem, yielding a decomposition 
        \begin{align*}%\label{eq:maschke-decomposition}
        V = V_1 \oplus \cdots \oplus V_n,
        \end{align*}
        where the subspaces $V_1,\ldots,V_n$ are equivariant and irreducible, with respect to the action of the group $G$. Irreducible spaces have length one by definition, and the length is additive with respect to direct sums, i.e.
        \begin{align*}
            \text{length of $V$} = \text{length of $V_1$} + \cdots + \text{length of $V_n$},
        \end{align*}
        so the length of $V$ is equal to $n$.  We will now show that the dimension is at least $n$ on the right-hand side of the inequality in the statement of the lemma.
        For every $i \in \set{1,\ldots,n}$ we can define an equivariant linear map from $V$ to itself which is the identity on $V_i$ and maps vectors from the other components to zero.  This gives us  at least $n$ equivariant linear maps from $V$ to itself. None of these maps is spanned by the other, and hence the dimension is at least $n$. 
    \end{proof}

    Thanks to Lemmas~\ref{lem:reduce-to-finite-substructures} and~\ref{lem:dim-bounds-length}, the length of the vector space with atoms $\Lin \A^d$ is bounded by the dimensions of the vector spaces
    \begin{align}
        \label{eq:endo-maps-b-d}
 \lineqfun { \Lin \B^d} {\Lin \B^d},
    \end{align}
    where $\B$ ranges over the family $\Bclass$.
    To complete the proof of the theorem, it remains to show that these dimensions  are bounded by some constant that depends only on $d$. This is done in the following lemma, which   completes the    proof of Theorem~\ref{thm:weak-smooth-approximation-finite-length}. 

    \begin{lemma}\label{lem:bound-on-dim-for-finite-set} 
        For every  $\B \in \Bclass$ the dimension of the vector space  in~\eqref{eq:endo-maps-b-d} is at most the number of orbits in $\A^{2d}$.
    \end{lemma}
    \begin{proof}
        A linear map  in the space~\eqref{eq:endo-maps-b-d} is the same thing as a square matrix indexed by $\B^d$, i.e.~a function of type  
        \begin{align*}
        {\B^d \times \B^d} \to \field.
        \end{align*}
        This function must be equivariant with respect to automorphisms of $\B$. This means that inputs in the same orbit must be mapped to the same field element. Therefore, to define  such a function, we need to choose one element of $\field$ for each orbit of the input.   Therefore, the dimension of the space in~\eqref{eq:endo-maps-b-d} is equal to the number of orbits of $\B^{2d}$, under the action of the group of automorphisms of $\B$.  This dimension is bounded by some constant that depends only on $d$, by the assumption of oligomorphic approximation. 
    \end{proof}
This completes the proof of Theorem~\ref{thm:weak-smooth-approximation-finite-length}. For convenience, we summarise the steps  in Figure~\ref{fig:proof-summary-cogless}.
\end{proof}


\begin{figure}
    \[
\begin{tikzcd}[row sep=large]
\text{length of $\Lin \A^d$} 
\ar[d, phantom, "\rotatebox{90}{$\geq$} \text{\quad (Lemma~\ref{lem:reduce-to-finite-substructures})}"']
\\
\displaystyle\sup_{\B \in \Bclass} \text{length of $\Lin \B^d$} 
\ar[d, phantom, "\rotatebox{90}{$\geq$} \text{\quad  (Lemma~\ref{lem:dim-bounds-length})}"']
\\
\displaystyle\sup_{\B \in \Bclass} \text{dimension of } \lineqfun {\Lin \B^d} {\Lin \B^d}
\ar[d, phantom, "\rotatebox{90}{$\geq$} \text{\quad( Lemma~\ref{lem:bound-on-dim-for-finite-set})}"']
\\
\displaystyle\sup_{\B \in \Bclass} \text{number of orbits in $\B^{2d}$}
\ar[d, phantom, "\rotatebox{90}{$>$} \text{\quad(by definition)}"']
\\
\infty.
\end{tikzcd}
\]
\caption{
    \label{fig:proof-summary-cogless}
    Summary of the proof of Theorem~\ref{thm:weak-smooth-approximation-finite-length}.}
\end{figure}

The inequalities shown in Figure~\ref{fig:proof-summary-cogless} give us upper bounds on the length. In the case of the equality atoms, this bound is exponential in $d$. In the case of the vector space atoms from Example~\ref{ex:vector-space-atoms}, the bound is the number of  linear dependency patterns for $2d$-tuples over a finite field. Such a pattern is described by: (a) indicating a subset of the coordinates which is a basis for the tuple; and (b) indicating the basis decompositions for the remaining coordinates. This can be done in at most exponentially many ways in $d$, and therefore the overall bound is  exponential in $d$. \textcolor{red}{In the case of the Rado graph, more exact bounds will be presented later in the paper -- where? }
