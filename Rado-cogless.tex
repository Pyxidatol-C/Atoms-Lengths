\section{Rado graph, sans cogs}

$\A$ is: 
\begin{itemize}
    \item oligomorphic if, for $d = 0, 1, 2, \ldots$, $\A^d$ only has finitely many orbits;
    \item $\FF$-oligomorphic if, for $d = 0, 1, 2, \ldots$, $\Lin_\FF \A^d$ only has finitely long chains.
\end{itemize}
\textcolor{red}{
Of note: with Stirling numbers of the second kind and Gaussian $2$-binomial coefficients,
the orbit counts are given by
\begin{align*}
    \# \N^d &= \sum_{k=0}^d \begin{Bmatrix} d \\ k \end{Bmatrix} \\
    \# \Q^d &= \sum_{k=0}^d \begin{Bmatrix} d \\ k \end{Bmatrix} k! \\
    \# \G^d &= \sum_{k=0}^d \begin{Bmatrix} d \\ k \end{Bmatrix} 2^{\binom{k}{2}} \\
    \# \V_\infty^d &= \sum_{k=0}^d \begin{bmatrix} d \\ k \end{bmatrix}_2 \\
    \# \W_\infty^d &= \sum_{k=0}^d \begin{bmatrix} d \\ k \end{bmatrix}_2 2^{\binom{kW}{2}}
\end{align*}
To introduce: 
\begin{itemize}
    \item 
    \emph{smooth approximation} by \emph{homogeneous substructures} \cite{KLM89} (N.B. `smooth approximation' from \cite[Definition~4]{MP24} seems to be entirely different)
    \item 
    \emph{rough approximation} of a homogeneous structure by finite substructures with few orbits (i.e., types)
    that cover the age of $\A$
\end{itemize}}

\subsection{Symplectic vector spaces}
Throughout this subsection let $\ff$ denote a finite field.
\begin{definition}
    A \emph{symplectic vector space} is an $\ff$-vector space $\W$ 
    equipped with a bilinear form $\omega: \W \times \W \to \ff$ that is
    \begin{itemize}
        \item alternating: $\omega(v, v) = 0$ for all $v$; and
        \item non-degenerate: if $\omega(v, w) = 0$ for all $w$ then $v = 0$.
    \end{itemize}
\end{definition}

\begin{example}
    Let $\W_n$ be the $\ff$-vector space with basis $e_1  , \ldots, e_n, f_1, \ldots, f_n$.
    Define $\omega$ by bilinearly extending
    \begin{equation}\label{eq:symplectic-basis}
        \omega(e_i, f_i) = 1 = -\omega(f_i, e_i),\quad
        \omega(-, *) = 0 \text{ elsewhere;}
        \tag{\S}
    \end{equation}
    one may straightforwardly check that $\omega$ is alternating and non-degenerate.
    Moreover, noticing that $\W_0 \subseteq \W_1 \subseteq \W_2 \subseteq \cdots$,
    we obtain a countable-dimensional symplectic vector space $\W_\infty = \bigcup_n \W_n$.
\end{example}

We will take a straight-line path to prove the following in a self-contained manner.
Detailed expositions can be found in \cite[{\S}III.3]{Ar57}.
\begin{theorem}\label{thm:symplectic-smooth-approximation}
    The symplectic vector space $\W_\infty$ is smoothly approximated by 
    $\W_0 \subseteq \W_1 \subseteq \W_2 \subseteq \cdots$.
\end{theorem}

To begin with, we will refer to vectors satisfying \eqref{eq:symplectic-basis} as a \emph{symplectic basis} 
--- indeed, they must be linearly independent.
Such bases behave very much like the usual bases.

\begin{lemma}\label{lem:symplectic-basis}
    Assume that $\W$ is a symplectic vector space that is at most countable.
    Then any finite symplectic basis $e_1, \ldots, e_n, f_1, \ldots, f_n$ 
    can be extended to a symplectic basis that spans the whole $\W$.
\end{lemma}
\begin{proof}
    Suppose that $e_1, \ldots, e_n, f_1, \ldots, f_n$ does not already span $\W$;
    take $v$ to be a witness (that is least according to some fixed enumeration of $\W$ 
    in the case it is infinite).
    Put
    \[
        e_{n+1} = v - \sum_{i=1}^n \omega(e_i, v) f_i + \sum_{i=1}^n \omega(f_i, v) e_i
    \]
    so that $\omega(e_i, e_{n+1}) = 0 = \omega(f_i, e_{n+1})$.
    By the non-degeneracy of $\omega$, there is --- rescaling if necessary 
    --- some $w$ such that $\omega(e_{n+1}, w) = 1$. 
    Now define
    \[
        f_{n+1} = w - \sum_{i=1^n} \omega(e_i, w) f_i + \sum_{i=1}^n \omega(f_i, w) e_i
    \]
    in a similar manner, 
    making $e_1, \ldots, e_n, e_{n+1}, f_1, \ldots, f_n, f_{n+1}$ a symplectic basis that spans $v$.
    We go through every element of $\W$ by continuing this way.
\end{proof}

Given two symplectic vector spaces $\W$ and $\W'$,
we call a function $f$ between $X \subseteq \W$ and $X' \subseteq \W'$ \emph{isometric} if 
$\omega(f(x_1), f(x_2)) = \omega(x_1, x_2)$ for all $x_1, x_2 \in X$.

\begin{proposition}\label{prop:symplectic-unique-by-dimension}
    $\W_0, \W_1, \W_2, \ldots, \W_\infty$ are all the countable symplectic vector spaces 
    up to linear isometric isomorphisms.
\end{proposition}
\begin{proof}
    Let $\W$ be a countable symplectic vector space.
    By Lemma~\ref{lem:symplectic-basis}, we can extend the empty symplectic basis to one that spans $\W$ 
    --- call the basis vectors $e'_i, f'_i$.
    Then $e'_i \mapsto e_i, f'_i \mapsto f_i$ exhibits a linear isomorphism from $\W$ 
    to $\W_n$ or $\W_\infty$, which one easily checks to be isometric.
\end{proof}

In particular $\W_\infty$ is the essentially unique countable symplectic vector space;
by Ryll-Nardzewski it is oligomorphic.
We need one last fact about its finite counterparts.
\begin{proposition}[Witt Extension] \label{prop:symplectic-witt-extension}
    Any isometric injective linear map $f : X \subseteq \W_n \to \W_n$ 
    can be extended to an isometric linear isomorphism $\W_n \to \W_n$.
\end{proposition}
\begin{proof}
    We proceed by induction on the dimension of $X^{\perp X}$,
    where we write
    \[
        Y^{\perp Z} = \{y \in Y \mid \forall z \in Z : \omega(y, z) = 0\}
    \]
    for subspaces $Y, Z \subseteq \W_n$.
    
    Suppose first that $X^{\perp X} = X \cap \W_n^{\perp X}$ is the zero space.
    Notice that $\dim \W_n^{\perp X} = 2n - \dim X$:
    \begin{itemize}
        \item the map 
            $\W_n \to (\W_n \xrightarrow{\text{lin.}} \ff), v \mapsto \omega(v, -)$ 
        is linear and injective since $\omega$ is bilinear and non-degenrate, 
        so for dimension reasons it is also surjective;

        \item the restriction map 
            $(\W_n \xrightarrow{\text{lin.}} \ff) \to (X \xrightarrow{\text{lin.}} \ff)$ 
        is linear and surjective, 
        since any basis for $X \subseteq \W_n$ can be extended to one for $\W_n$;
        
        \item their composition 
            $v \mapsto \omega(v, -)|_X$ 
        is therefore linear, surjective, and has kernel $\W_n^{\perp X}$.
    \end{itemize}
    It follows that $\W_n^{\perp X}$ is the orthogonal complement of $X$ in $\W_n$: 
    by assumption and the above we have 
        $X \cap \W_n^{\perp X} = \{0\}$, $X + \W_n^{\perp X} = \W_n$, 
    and $\omega$ restricted to $X \times \W_n^{\perp X}$ is the zero function;
    we will use the notation \[
        \W_n = X \ominus \W_n^{\perp X}.
    \]
    On the other hand, as $f$ is isometric, 
    $f(X)^{\perp f(X)} = f(X^{\perp X})$ must also be the zero space,
    meaning $\W_n = f(X) \ominus \W_n^{\perp f(X)}$ by the same analysis.
    But $\dim \W_n^{\perp f(X)} = 2n - \dim f(X) = 2n - \dim X = \dim \W_n^{\perp X}$,
    so by Proposition~\ref{prop:symplectic-unique-by-dimension} there is a isometric 
    linear isomorphism $g : \W_n^{\perp X} \to \W_n^{\perp f(X)}$.
    Combining $f$ and $g$ yields a linear isomorphism 
    \begin{align*}
        \W_n = X \ominus \W_n^{\perp X} &\to f(X) \ominus \W_n^{\perp f(X)} = \W_n \\
        x + y &\mapsto f(x) + g(y)
    \end{align*} 
    which is isometric.
    
    Now suppose $X^{\perp X}$ contains some non-zero vector $x$.
    By extending $x$ to a basis of $X$ we can find a complement $Y$ for $\langle x \rangle$ in $X$;
    then $X = \langle x \rangle \ominus Y$ by the assumption on $x$.
    Writing $Z = \W_n^{\perp Y}$, 
    notice $Y \subseteq \W_n^{\perp Z}$ and $\dim Y = 2n - (2n - \dim Y) = \dim \W_n^{\perp Z}$.
    It follows that $\W_n^{\perp Z} = Y$ does not contain $x$,
    i.e., that some $z \in Z$ satisfies $\omega(x, z) = 1$.
    Consider $X' = \langle x, z \rangle + Y$;
    we must have 
    \[
        X' = \langle x, z \rangle \ominus Y,
    \]
    because if $\lambda x + \mu z \in \langle x, z \rangle$ lies also in $Y \subseteq X$,  
    then $0 = \omega(x, \lambda x + \mu z) = \mu$ and so $\lambda = 0$ too.    
    Similarly, as $f(x)$ is a non-zero vector in 
        $f(X)^{\perp f(X)}$ and $f(X) = \langle f(x) \rangle \ominus f(Y)$,
    we can find a vector $z'$ orthogonal to $\W_n^{\perp f(Y)}$ and satisfying $\omega(f(x), z') = 1$.
    Hence \[
        x \mapsto f(x), z \mapsto z', y \mapsto f(y)
    \]
    defines an isometric linear embedding
    \[
        f' : X' = \langle x, z \rangle \ominus Y 
        \to \langle f(x), z' \rangle \ominus f(Y) \subseteq \W_n
    \]
    extending $f$.
    Finally, we can apply the inductive hypothesis to extend $f'$.
    Indeed, if $v = \lambda x + \mu z + y$ is in $X'^{\perp X'}$ 
    then $\lambda = \omega(v, z) = 0 = \omega(v, x) = \mu$,
    so $v = y$ belongs to $X^{\perp X}$;
    as $x \in X^{\perp X} \setminus X'^{\perp X'}$, 
    we have $\dim X'^{\perp X'} \leq \dim X^{\perp X} - 1$.
\end{proof}

Smooth approximation is now immediate.

\begin{proof}[Proof of Theorem~\ref{thm:symplectic-smooth-approximation}]
    Firstly, observe that the restriction 
        $\Aut(\W_\infty)_{\{\W_n\}} \to \Aut(\W_n)$ 
    is surjective:
    any isometric linear automorphism of $\W_n$ maps the standard symplectic basis to another symplectic basis, 
    both of which can be extended to a symplectic basis of $\W_\infty$ by Lemma~\ref{lem:symplectic-basis}.

    Now suppose $\pi \in \Aut(\W_\infty)$ maps $(x_1, \dots, x_d) \in \W_n^d$ 
    to $(y_1, \dots, y_d) \in \W_n^d$.
    By Proposition~\ref{prop:symplectic-witt-extension}, we may extend 
        $\pi|_{\langle x_1, \dots, x_d \rangle} : \langle x_1, \dots, x_d \rangle \to \W_n$ 
    to some $f \in \Aut(\W_n)$ which still maps $(x_1, \dots, x_d)$ to $(y_1, \dots, y_d)$.
\end{proof}

\begin{corollary}
    Provided $\FF$ is of characteristic $0$, the symplectic $\ff$-vector space $\W_\infty$ is $\FF$-oligomorphic.
\end{corollary}




\subsection{Symplectic graphs}
For this subsection let $\ff$ be the two-element field.
\begin{definition}
    The \emph{symplectic graph} $\widetilde \W_n$ has vertices $\W_n$ and edges
    \[
        v_1 \sim v_2 \iff \omega(v_1, v_2) = 1.
    \]
    This is indeed an undirected graph: as $\omega$ is alternating, we have $\omega(v_1, v_2) = -\omega(v_2, v_1) = \omega(v_2, v_1)$.    
\end{definition}

\begin{proposition}\label{prop:symplectic-vs-graph}
    $\Aut(\widetilde \W_n) = \Aut(\W_n)$.
\end{proposition}
\begin{proof}
    Clearly any isometric linear automorphism of $\W_n$ is a graph automorphism of $\widetilde \W_n$.
    Conversely, any $f \in \widetilde \W_n$ is evidently isometric.
    To show that $f$ is linear, take $\lambda_1, \lambda_2 \in \ff$ and $v_1, v_2 \in \W$.
    We calculate:
    \begin{align*}
        &\omega\Bigl( f(\sum_i \lambda_i v_i) - \sum_i \lambda_i f(v_i) , f(w) \Bigr) \\
        ={}& \omega\Bigl( f(\sum_i \lambda_i v_i), f(w)\Bigr) - \sum_i \lambda_i \omega\bigl(f(v_i), f(w) \bigr) \\
        ={}& \omega\Bigl( \sum_i \lambda_i v_i, w \Bigr) - \sum_i \lambda_i \omega( v_i, w ) \\
        ={}& \omega(0, w) = 0
    \end{align*}
    for all $f(w) \in f(\W_n) = \W_n$;
    since $\omega$ is non-degenerate, 
    we conclude that $f(\sum_i \lambda_i v_i) = \sum_i \lambda_i f(v_i)$.
\end{proof}

\begin{proposition}
    The number of orbits in $\widetilde \W_n^d$ is at most 
        $\prod_{i=1}^{d} (2^{i-1} + 1) = O(2^{d (d-1) / 2})$ 
    for all $n$.
\end{proposition}
\begin{proof}
    By Proposition~\ref{prop:symplectic-vs-graph} and Theorem~\ref{thm:symplectic-smooth-approximation},
    the number of orbits in $\W_\infty^d$ is an upper bound;
    \textcolor{red}{this number is the OEIS sequence A028361.}
\end{proof}

\begin{proposition}[{\cite[Theorem~8.11.2]{GR01}}]
    Every graph on at most $2n$ vertices embeds into $\widetilde \W_n$.
\end{proposition}
\begin{proof}
    Let $G$ be a graph on at most $2n$ vertices. 
    The conclusion is trivial when $n = 0$.
    Also, if $G$ contains no edges, we can choose any $2n$ of the $2^n$ vectors in 
    $\langle e_1, \ldots, e_n \rangle \subseteq \widetilde \W_n$.
    
    So suppose $n \geq 1$ and $G$ has an edge $s \sim t$.
    Let $G_{s,t}$ be the graph on vertices $G \setminus \{s, t\}$ with edges which we will specify later.
    By induction, some embedding $f : G_{s, t} \to \widetilde \W_{n-1}$ exists.
    Define $f' : G \to \widetilde \W_n$ by
    \begin{align*}
        x \in G_{s, t} &\mapsto f(x) - \llbracket x \sim s \rrbracket f_n + \llbracket x \sim t \rrbracket e_n \\
        s &\mapsto e_n \\
        t &\mapsto f_n
    \end{align*}
    where $\llbracket \phi \rrbracket$ is $1$ if $\phi$ holds and $0$ otherwise.
    Then we have $\omega(f'(x), f'(s)) = \llbracket x \sim s \rrbracket$ 
    and $\omega(f'(x), f'(t)) = \llbracket x \sim t \rrbracket$ as desired, on one hand.
    On the other,
    \begin{align*}
        \omega( f'(x_1), f'(x_2) ) 
        = \llbracket x_1 \sim x_2 \rrbracket 
        &+ \llbracket x_1 \sim s \rrbracket \llbracket x_2 \sim t \rrbracket \\
        &+ \llbracket x_1 \sim t \rrbracket \llbracket x_2 \sim s \rrbracket
    \end{align*}
    tells us how we should define the edge relation in $G_{s,t}$ for $f'$ to be an embedding of graphs.
\end{proof}

\begin{theorem}
    The Rado graph is roughly approximated by $\widetilde \W_0 \subseteq \widetilde \W_1 \subseteq \widetilde \W_2 \subseteq \cdots$.
\end{theorem}

\begin{corollary}
    Provided $\FF$ is of characteristic $0$, the Rado graph is $\FF$-oligomorphic.
\end{corollary}
