\section{Rado graph, sans cogs}

$\A$ is: 
\begin{itemize}
    \item oligomorphic if, for $d = 0, 1, 2, \ldots$, $\A^d$ only has finitely many orbits;
    \item $\FF$-oligomorphic if, for $d = 0, 1, 2, \ldots$, $\Lin_\FF \A^d$ only has finitely long chains.
\end{itemize}
\textcolor{red}{
Of note: with Stirling numbers of the second kind and Gaussian $2$-binomial coefficients,
the orbit counts are given by
\begin{align*}
    \# \N^d &= \sum_{k=0}^d \begin{Bmatrix} d \\ k \end{Bmatrix} \\
    \# \Q^d &= \sum_{k=0}^d \begin{Bmatrix} d \\ k \end{Bmatrix} k! \\
    \# \G^d &= \sum_{k=0}^d \begin{Bmatrix} d \\ k \end{Bmatrix} 2^{\binom{k}{2}} \\
    \# \V_\infty^d &= \sum_{k=0}^d \begin{bmatrix} d \\ k \end{bmatrix}_2 \\
    \# \W_\infty^d &= \sum_{k=0}^d \begin{bmatrix} d \\ k \end{bmatrix}_2 2^{\binom{k}{2}}
\end{align*}
To introduce: 
\begin{itemize}
    \item 
    \emph{smooth approximation} by \emph{homogeneous substructures} \cite{KLM89} (N.B. `smooth approximation' from \cite[Definition~4]{MP24} seems to be entirely different)
    \item 
    \emph{rough approximation} of a homogeneous structure by finite substructures with few orbits (i.e., types)
    that cover the age of $\A$
\end{itemize}}

\subsection{Symplectic vector spaces}
Throughout this subsection let $\ff$ denote a finite field.
\begin{definition}
    A \emph{symplectic vector space} is an $\ff$-vector space $\W$ 
    equipped with a bilinear form $\omega: \W \times \W \to \ff$ that is
    \begin{itemize}
        \item alternating: $\omega(v, v) = 0$ for all $v$; and
        \item non-degenerate: if $\omega(v, w) = 0$ for all $w$ then $v = 0$.
    \end{itemize}
\end{definition}

\begin{example}
    Let $\W_n$ be the $\ff$-vector space with basis $e_1  , \ldots, e_n, f_1, \ldots, f_n$.
    Define $\omega$ by bilinearly extending
    \begin{equation}\label{eq:symplectic-basis}
        \omega(e_i, f_i) = 1 = -\omega(f_i, e_i),\quad
        \omega(-, *) = 0 \text{ elsewhere;}
        \tag{\S}
    \end{equation}
    one may straightforwardly check that $\omega$ is alternating and non-degenerate.
    Moreover, noticing that $\W_0 \subseteq \W_1 \subseteq \W_2 \subseteq \cdots$,
    we obtain a countable-dimensional symplectic vector space $\W_\infty = \bigcup_n \W_n$.
\end{example}

We will refer to vectors satisfying \eqref{eq:symplectic-basis} as a \emph{symplectic basis} 
--- indeed, they must be linearly independent.
Such bases behave very much like the usual bases.

\begin{proposition}\label{prop:symplectic-basis}
    Assume that $\W$ is a symplectic vector space that is at most countable.
    Then any finite symplectic basis $e_1, \ldots, e_n, f_1, \ldots, f_n$ 
    can be extended to a symplectic basis that spans the whole $\W$.
\end{proposition}
\begin{proof}
    Suppose that $e_1, \ldots, e_n, f_1, \ldots, f_n$ does not already span $\W$;
    take $v$ to be a witness (that is least according to some fixed enumeration of $\W$ 
    in the case it is infinite).
    Put
    \[
        e_{n+1} = v - \sum_{i=1}^n \omega(e_i, v) f_i + \sum_{i=1}^n \omega(f_i, v) e_i
    \]
    so that $\omega(e_i, e_{n+1}) = 0 = \omega(f_i, e_{n+1})$.
    This cannot be the zero vector lest we contradict the choice of $v$.
    By the non-degeneracy of $\omega$, there is --- rescaling if necessary 
    --- some $w$ such that $\omega(e_{n+1}, w) = 1$. 
    Now define
    \[
        f_{n+1} = w - \sum_{i=1^n} \omega(e_i, w) f_i + \sum_{i=1}^n \omega(f_i, w) e_i
    \]
    in a similar manner, 
    making $e_1, \ldots, e_n, e_{n+1}, f_1, \ldots, f_n, f_{n+1}$ a symplectic basis that spans $v$.
    We go through every element of $\W$ by continuing this way.
\end{proof}

\color{gray}THROW IN THE APPENDIX: 

In fact, we will also make use of the ``symplectic basis and a half'' variant below. 
\begin{proposition}\label{prop:symplectic-basis-and-a-half}\textcolor{gray}
    Now assume $\W$ is a finite-dimensional symplectic vector space.
    Let 
    \begin{align*} 
        &e_1, \cdots, e_n, e_{n+1}, \cdots, e_{n+k}, \\
        &f_1, \cdots, f_n
    \end{align*}
    be linearly independent vectors satisfying \eqref{eq:symplectic-basis}.
    Then we can find the missing $f_{n+1}, \cdots, f_{n+k}$ to complete the symplectic basis.
\end{proposition}
\begin{proof}
    We first need the following notion.
    Given a subspace $V \subseteq \W$, consider its orthogonal complement
    \[
        V^\perp = \{ w \in \W \mid \forall v \in V : \omega(v, w) = 0 \}.
    \]
    It is the kernel of the composite linear map 
    \begin{align*}
        \W \to (\W \xrightarrow{\text{lin.}} \ff) \to (V \xrightarrow{\text{lin.}} \ff) \\
        w \mapsto \omega(-, w) \mapsto \omega(-, w)|_V.
    \end{align*}
    Note this map is surjective:
    the first part is injective by non-degeneracy and hence surjective for dimension reasons,
    and the second part is surjective since we can extend a basis of $V$ to one of $\W$.
    Therefore 
    \[
        \dim V^\perp = \dim \W - \dim V,
    \]
    and in particular $V^{\perp \perp}$ is precisely equal to $V$.

    Now suppose we have found $f_{n+1}, \cdots, f_{n+i}$ already.
    If $e_{n+i+1}$ were to be spanned by 
    \begin{align*}
        &e_1, \cdots, e_{n+i}, e_{n+i+2}, \cdots, e_{n+k},\\ 
        &f_1, \cdots, f_{n+i},
    \end{align*}
    it would be spanned by $e_1, \cdots, e_{n+i}, e_{n+i+2}, \cdots, e_{n+k}$ alone because of \eqref{eq:symplectic-basis};
    but this is impossible as we assumed linear independency.
    So 
    \begin{align*}
        e_{n+i+1} 
        \not\in{} &\langle e_1, \cdots, e_{n+i}, e_{n+i+2}, f_1, \cdots, f_{n+i} \rangle \\
        ={} &\langle e_1, \cdots, e_{n+i}, e_{n+i+2}, f_1, \cdots, f_{n+i} \rangle^{\perp \perp},
    \end{align*}
    i.e., some $f_{n+i+1} \in \langle e_1, \cdots, e_{n+i}, e_{n+i+2}, f_1, \cdots, f_{n+i} \rangle^\perp$ satisfies $\omega(e_{n+i+1}, f_{n+i+1}) = 1$.
\end{proof}

\color{black}
Given two symplectic vector spaces $\W$ and $\W'$,
we call a function $\alpha$ between $X \subseteq \W$ and $X' \subseteq \W'$ \emph{isometric} if 
$\omega(\alpha(x_1), \alpha(x_2)) = \omega(x_1, x_2)$ for all $x_1, x_2 \in X$.
We can make an easy observation:
\begin{lemma}\label{lem:symplectic-basis-gives-isometry}
Let $\{e_i, f_i \mid i \in I\} \subseteq \W$, $\{e'_j, f'_j \mid j \in J\} \subseteq \W'$ be two symplectic bases
and let $\alpha : I \to J$ be a bijection.
Then 
\[
    e_i \mapsto e'_{\alpha(i)}, f_i \mapsto f'_{\alpha(i)}
\]
defines an isometric linear isomorphism $\langle e_i, f_i \rangle \to \langle e'_j, f'_j \rangle$.
\end{lemma}

It then follows from Proposition~\ref{prop:symplectic-basis} that, 
up to isometric linear isomorphisms, 
$\W_0, \W_1, \W_2, \cdots, \W_\infty$ are all the countable symplectic vector spaces.
Whilst we may deduce that $\W_\infty$ is oligomorphic by appealing to Ryll-Nardzewski,
we will opt for a more direct proof that also establishes smooth approximation.

\begin{proposition}[Witt Extension] \label{prop:symplectic-witt-extension}
    Any isometric linear injection $\alpha : \langle X \rangle \subseteq \W_n \to \W_n$ 
    can be extended to an isometric linear automorphism of $\W_n$
    and in turn to one of $\W_\infty$.
\end{proposition}

\color{gray}
\begin{proof}
    To begin with, find a basis $x_1, \cdots, x_k$ for $\langle X \rangle^\perp = \{ w \in W \mid \forall x \in X : \omega(w, x) = 0 \}$ 
    and extend it to a basis \(
        x_1, \cdots, x_k, x_{k+1}, \cdots, x_d
    \) for $\langle X \rangle$.
    Notice that
    \[
        U = \langle x_{k+1}, \cdots, x_d \rangle
    \] must be a symplectic subspace:
    as it intersects with $\langle X \rangle^\perp$ trivially, 
    given any non-zero vector $u \in U$ we must have $0 \neq \omega(u, x + u') = \omega(u, u')$ for some $x \in \langle X \rangle^\perp$ and $u' \in U$.
    Hence use Proposition~\ref{prop:symplectic-basis} to find a symplectic basis
    \(
        e_1, \cdots, e_n,
        f_1, \cdots, f_n
    \)
    for $U$.
    Observe that 
    \begin{align*}
        &e_1, \cdots, e_n, x_1, \cdots, x_k, \\
        &f_1, \cdots, f_n
    \end{align*}
    form a basis for $\langle X \rangle$ and satisfy \eqref{eq:symplectic-basis}.
    On the other hand,
    \begin{align*}
        &\alpha(e_1), \cdots, \alpha(e_n), \alpha(x_1), \cdots, \alpha(x_k), \\
        &\alpha(f_1), \cdots, \alpha(f_n)
    \end{align*}
    form a basis for $\alpha(\langle X \rangle)$ and also satisfy \eqref{eq:symplectic-basis}.
    Therefore apply Proposition~\ref{prop:symplectic-basis-and-a-half} twice to find the missing 
    $y_1, \cdots, y_k$ and $y'_1, \cdots, y'_k$ to complete the two symplectic bases --- call them $\mathcal{B}$ and $\mathcal{B}'$.
    They are of the same size.

    Now, by using Proposition~\ref{prop:symplectic-basis}, extend $\mathcal{B}$ and $\mathcal{B}'$ to symplectic bases $\mathcal{C}$ and $\mathcal{C}'$ that span $\W_n$.
    These must have the same size ($2n$ namely), 
    so by Lemma~\ref{lem:symplectic-basis-gives-isometry} we obtain an isometric linear automorphism $\beta : \W_n \to \W_n$ extending $\alpha$.

    To finish, notice that $\mathcal{C}, e_{n+1}, \cdots, f_{n+1}, \cdots$ as well as $\mathcal{C}', e_{n+1}, \cdots, f_{n+1}, \cdots$ 
    form a symplectic basis spanning $\W_\infty$.
    We obtain from Lemma~\ref{lem:symplectic-basis-gives-isometry} another time an isometric linear automorphism $\gamma : \W_\infty \to \W_\infty$ extending $\beta$ 
    that is the identity almost everywhere.
\end{proof}
\color{black}


\begin{proposition}\label{prop:symplectic-oligomorphic}
    $\W_\infty^d$ has precisely \(
        \sum_{k = 0}^d \begin{bsmallmatrix}d \\ k \end{bsmallmatrix}_q \cdot q^{\binom{k}{2}}
    \) orbits under isometric linear automorphisms,
    where $q = \vert \ff \vert$ and \[
        \begin{bmatrix}d \\ k \end{bmatrix}_q 
        = \frac{
            (q^d - 1)(q^{d-1} - 1) \cdots (q^{d-k+1} - 1)
        }{
            (q^k - 1) (q^{k-1} - 1) \cdots (q^1 - 1)
        }
    \] is the q-binomial coefficient.
\end{proposition}
\begin{remark}
    To anticipate the next subsection, we note a similarity with the Rado graph: 
    in $\G^d$ there are $\sum_{k=0}^d \binom{d}{k} \cdot 2^{\binom{k}{2}}$ orbits 
    --- we may impose any edge relation on $d$ vertices.
\end{remark}

\begin{proof}
    To $(v_1, \cdots, v_d) \in \W_\infty^d$ we associate a \emph{type}, 
    which comprises the following data:
    \begin{itemize}
        \item 
        pivot indices $I \subseteq \{1, \cdots, d\}$ containing every $i$ 
        such that $v_i$ is not spanned by $v_1, \cdots, v_{i-1}$
        --- so we inductively ensure that $$\{v_{i'} \mid i' \in I, i' \leq i\}$$ is linearly independent;

        \item 
        for each $j \not\in I$, pairs $\Lambda_j \subseteq \ff \times \{i \in I \mid i < j\}$
        where $v_j = \sum_{(\lambda, i) \in \Lambda_j} \lambda v_i$;

        \item 
        a map $\Omega : \binom{I}{2} \to \ff$ defined by $\Omega(\{i < j\}) = \omega(v_i, v_j)$.
    \end{itemize}
    If $\pi : \W_\infty \to \W_\infty$ is an isometric linear automorphism,
    then $(v_1, \cdots, v_d)$ and $(\pi(v_1), \cdots, \pi(v_d))$ evidently share the same type.
    Conversely, if $(w_1, \cdots, w_d)$ has the type of $(v_i, \cdots, v_d)$,
    then \[
        v_i \mapsto w_i, i \in I
    \] extends to an isometric linear injection 
    \[
        \alpha : \langle v_1, \cdots, v_d \rangle 
        \to \langle w_1, \cdots, w_d \rangle 
        \subseteq \W_n
    \]
    for some large enough $n$.
    Observe that $\alpha$ must send $v_j \mapsto w_j$ for $j \not\in I$ too,
    and that it may be extended to an isometric linear automorphism $\pi$ of $\W_\infty$ by Propsoition~\ref{prop:symplectic-witt-extension}.
    But we can find some $(v_1, \cdots, v_d)$ that realises any given type $(I, \{\Lambda_j\}_j, \Omega)$.
    Indeed, it suffices to put
    \[
        v_i = e_i + \sum_{i < i'} \Omega(i, i') f_{i'}
    \]
    for $i \in I$ and $v_j = \sum_{\lambda, i} \in \Lambda_i \lambda v_i$ for $j \not\in I$.
    Therefore the number of types is precisely the number of orbits in $\W_\infty^d$.

    Finally, we do some combinatorics.
    Fix $0 \leq k \leq d$ and count the number of types with $|I| = k$.
    There are $2^{\binom{k}{2}}$ choices for $\Omega$ and say $\#_{d, k}$ choices for the $\Lambda_j$'s;
    the two can be chosen separately.
    In total, this gives
    \[ 
        \sum_{k=0}^d \#_{d, k} \cdot 2^{\binom{k}{2}}
    \]
    types for vectors in $\W_\infty^d$.
    So focus on $\#_{d, k}$, the number of \emph{linear types} --- i.e., $(I, \{\Lambda_j\}_j)$, ignoring $\Omega$ --- in $\W_\infty^d$.
    (Incidentally $\#_{d, k}$ is the number of orbits in $\W_\infty^d$ or, more generally, any countable-dimensional $\ff$-vector space under linear automorphisms.)
    On the small values we easily check that \begin{align*}
        &\#_{0, 0} = 1 = \begin{bsmallmatrix}0 \\ 0\end{bsmallmatrix}_q, \\
        &\#_{1, 0} = 1 = \begin{bsmallmatrix}1 \\ 0\end{bsmallmatrix}_q,\qquad
        \#_{1, 1} = 1 = \begin{bsmallmatrix}1 \\ 1\end{bsmallmatrix}_q.
    \end{align*}
    Given a linear type in $\W_\infty^d$ with $\vert I \vert = k$, 
    we either have $1 \in I$ or $I \subseteq \{2, \cdots, d\}$.
    In the first case, the linear type restricts to one of the $\#_{d-1, k-1}$ linear types in $\W_\infty^{d-1}$
    and indicates how $v_1$ is involved in the span of the $(d - 1) - (k - 1)$ non-pivot vectors.
    In the second case, the linear type is simply one of the $\#_{d-1, k}$ linear types in $\W_\infty^{d-1}$.
    Thus 
    \begin{align*}
        \#_{d, k} 
        &= q^{d-k} \cdot \#_{d-1, k-1} + \#_{d-1, k} \\
        &= q^{d-k} \cdot \begin{bmatrix}d-1 \\ k-1\end{bmatrix}_q + \begin{bmatrix}d-1 \\ k\end{bmatrix}_q.
        \qedhere
    \end{align*}
\end{proof}

\begin{theorem}\label{thm:symplectic-smooth-approximation}
    The symplectic vector space $\W_\infty$ is smoothly approximated by 
    $\W_0 \subseteq \W_1 \subseteq \W_2 \subseteq \cdots$.
\end{theorem}

\begin{corollary}
    Provided $\FF$ is of characteristic $0$, the symplectic $\ff$-vector space $\W_\infty$ is $\FF$-oligomorphic.
\end{corollary}




\subsection{Symplectic graphs}
For this subsection let $\ff$ be the two-element field.
\begin{definition}
    For $n = 0, 1, 2, \cdots$, the \emph{symplectic graph} $\widetilde \W_n$ has vertices $\W_n$ and edges
    \[
        v_1 \sim v_2 \iff \omega(v_1, v_2) = 1.
    \]
    This is indeed an undirected graph: as $\omega$ is alternating, we have $\omega(v_1, v_2) = -\omega(v_2, v_1) = \omega(v_2, v_1)$.    
\end{definition}

\begin{proposition}\label{prop:symplectic-vs-graph}
    $\Aut(\widetilde \W_n) = \Aut(\W_n)$.
\end{proposition}
\begin{proof}
    Clearly any isometric linear automorphism of $\W_n$ is a graph automorphism of $\widetilde \W_n$.
    Conversely, any $f \in \widetilde \W_n$ is evidently isometric.
    To show that $f$ is linear, take $\lambda_1, \lambda_2 \in \ff$ and $v_1, v_2 \in \W$.
    We calculate:
    \begin{align*}
        &\omega\Bigl( f(\sum_i \lambda_i v_i) - \sum_i \lambda_i f(v_i) , f(w) \Bigr) \\
        ={}& \omega\Bigl( f(\sum_i \lambda_i v_i), f(w)\Bigr) - \sum_i \lambda_i \omega\bigl(f(v_i), f(w) \bigr) \\
        ={}& \omega\Bigl( \sum_i \lambda_i v_i, w \Bigr) - \sum_i \lambda_i \omega( v_i, w ) \\
        ={}& \omega(0, w) = 0
    \end{align*}
    for all $f(w) \in f(\W_n) = \W_n$;
    since $\omega$ is non-degenerate, 
    we conclude that $f(\sum_i \lambda_i v_i) = \sum_i \lambda_i f(v_i)$.
\end{proof}

So the number of orbits in $\widetilde \W_n^d$ is precisely equal to the number of orbits in $\W_n^d$
--- in particular, it is bounded above by $\sum_{k=0}^d \begin{bsmallmatrix}d \\ k\end{bsmallmatrix}_2 \cdot 2^{\binom{k}{2}}$ independently of $n$ by Proposition~\ref{prop:symplectic-oligomorphic}.
It remains to show $\widetilde \W_0 \subseteq \widetilde \W_1 \subseteq \widetilde \W_2 \subseteq \cdots$ embeds all finite graphs:

\begin{proposition}[{\cite[Theorem~8.11.2]{GR01}}]
    Every graph on at most $2n$ vertices embeds into $\widetilde \W_n$.
\end{proposition}
\begin{proof}
    Let $G$ be a graph on at most $2n$ vertices. 
    The conclusion is trivial when $n = 0$.
    Also, if $G$ contains no edges, we can choose any $2n$ of the $2^n$ vectors in 
    $\langle e_1, \ldots, e_n \rangle \subseteq \widetilde \W_n$.
    
    So suppose $n \geq 1$ and $G$ has an edge $s \sim t$.
    Let $G_{s,t}$ be the graph on vertices $G \setminus \{s, t\}$ with edges which we will specify later.
    By induction, some embedding $f : G_{s, t} \to \widetilde \W_{n-1}$ exists.
    Define $f' : G \to \widetilde \W_n$ by
    \begin{align*}
        x \in G_{s, t} &\mapsto f(x) - \llbracket x \sim s \rrbracket f_n + \llbracket x \sim t \rrbracket e_n \\
        s &\mapsto e_n \\
        t &\mapsto f_n
    \end{align*}
    where $\llbracket \phi \rrbracket$ is $1$ if $\phi$ holds and $0$ otherwise.
    Then we have $\omega(f'(x), f'(s)) = \llbracket x \sim s \rrbracket$ 
    and $\omega(f'(x), f'(t)) = \llbracket x \sim t \rrbracket$ as desired, on one hand.
    On the other,
    \begin{align*}
        \omega( f'(x_1), f'(x_2) ) 
        = \llbracket x_1 \sim x_2 \rrbracket 
        &+ \llbracket x_1 \sim s \rrbracket \llbracket x_2 \sim t \rrbracket \\
        &+ \llbracket x_1 \sim t \rrbracket \llbracket x_2 \sim s \rrbracket
    \end{align*}
    tells us how we should define the edge relation in $G_{s,t}$ for $f'$ to be an embedding of graphs.
\end{proof}

\begin{theorem}
    The Rado graph is roughly approximated by $\widetilde \W_0 \subseteq \widetilde \W_1 \subseteq \widetilde \W_2 \subseteq \cdots$.
\end{theorem}

\begin{corollary}
    Provided $\FF$ is of characteristic $0$, the Rado graph is $\FF$-oligomorphic.
\end{corollary}
