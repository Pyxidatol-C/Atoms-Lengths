\section{Introduction}

This paper is part of a research programme focused on orbit-finite sets and structures. In this programme, one starts with an infinite relational structure $\A$ whose elements are called \emph{atoms}. Based on these, one constructs sets which are called \emph{orbit-finite}. Precise definitions will follow, but the understanding is that elements of an orbit-finite set are constructed using atoms, and there are only finitely many of them up to automorphisms of $\A$. For the theory to make sense, we must assume that $\A$ is \emph{oligomorphic}, which means that $\A^d$ has finitely many orbits for every $d$. The simplest example of an oligomorphic atom structure is what we refer to as the \emph{equality atoms}; this is the structure which has a countably infinite underlying set, and no relations except for equality. This structure, like all oligomorphic structures, arises by applying a model theoretic construction (the Fraïssé limit) to a  well-behaved class of finite structures. Figure~\ref{fig:fraisse-limits} shows other examples of such structures. 
% \begin{figure}
%     \begin{center}
%     \begin{tabular}{ll|l}
%    &  class of finite structures & its Fraïssé limit\\
%     \hline
%     1.& finite sets with equality only & infinite set with equality only \\
%     2.& finite orders & rational numbers with order \\
%     3.& finite graphs & the Rado graph \\ 
%     4.& vector spaces of finite & vector space of countably\\  
%     & \quad dimension over $\mathbb F_2$ & \quad infinite  dimension over $\mathbb F_2$
% \end{tabular}
% \end{center}
% \caption{\label{fig:fraisse-limits} Examples of Fraïssé limits}
% \end{figure}
\begin{figure}
    \begin{center}
    \begin{tabular}{lll}
   &  class of finite structures & its Fraïssé limit\\
    \toprule
    1.& finite sets with equality only & $(\N, =)$ \\
    2.& finite orders & $(\Q, \leq)$ \\
    3.& finite graphs & $(\text{the Rado graph}, \sim)$ \\ 
    4.& finite $\mathbb F_2$-vector spaces & $\mathbb{F}_2 \oplus \mathbb{F}_2 \oplus \mathbb{F}_2 \oplus \cdots$ 
    \\
    \bottomrule  
\end{tabular}
\end{center}
\caption{\label{fig:fraisse-limits} Examples of Fraïssé limits}
\end{figure}

%In the last row, we assume that the underlying field is finite, since we want the finite dimensional vector spaces to be finite structure. In the case of the two-element field, we use the name \emph{bit-vector atoms} for the structure in the last row. 

When the underlying atom structure is oligomorphic, the orbit-finite sets have a robust theory, which resembles in some ways the theory of finite sets. This theory was originally developed to describe regular languages over infinite alphabets, by considering orbit-finite versions of various automata models~\cite{bojanczykNominalMonoids2013,bojanczykAutomataTheoryNominal2014}, but it has developed to cover other models, such as orbit-finite Turing machines~\cite{bojanczykTuringMachinesAtoms2013}, or orbit-finite constraint satisfaction problems~\cite{klin2015locally}.
Also, there are programming languages with data structures that can store  orbit-finite sets~\cite{bojanczyk2012towards,bojanczyk2012imperative}, and even working implementations~\cite{kopczynski2016lois, szynwelski-phd}. For a survey of the orbit-finite programme, we refer to the lecture notes~\cite{bojanczyk_slightly}. 

Some results do not depend on the choice of the atom structure, while others do. Here is an example of the latter case that arises for orbit-finite Turing machines~\cite{bojanczykTuringMachinesAtoms2013}. If we choose the atoms  to be the Fraïssé limit of linear orders, as in row 2 of Figure~\ref{fig:fraisse-limits},  then the orbit-finite version of \textsc{p}$\neq$\textsc{np} has the same answer as the  classical version without atoms.  On the other hand, if we choose any of the other three rows of the table, then one can prove unconditionally that \textsc{p}$\neq$\textsc{np} holds in the orbit-finite setting, by leveraging problems with choice. The dependency on the underlying atom structure will play a prominent role in this paper. 


% An \emph{orbit-finite} set over this atom structure  is one that can be generated from finitely many elements by applying atom automorphisms. For example, if the atom structure has on relations except for equality (such a structure is called the \emph{equality atoms}), then its automorphisms are all permutations, and the set $\A^2$ is orbit-finite, since it can be generated by starting with two elements: a non-equal pair $(a,b)$ and an equal pair $(a,a)$; all other elements of $\A^2$ can be obtained from these two by applying permutations of the atoms.  Orbit-finite sets were introduced in order to define models of computation, such as orbit-finite monoids~\cite{bojanczykNominalMonoids2013}, orbit-finite automata~\cite{bojanczykAutomataTheoryNominal2014}, or orbit-finite Turing machines~\cite{bojanczykTuringMachinesAtoms2013}. There are also strong connections with the theory of nominal sets~\cite{PittsAM:nomsns}.

\paragraph*{Vector spaces.}
One direction of the orbit-finite programme, motivated by the study of orbit-finite weighted automata,  is focused on vector spaces~\cite{BFKM24}. In these spaces (taken over some fixed field), one can take linear combinations and apply  automorphisms of the atom structure.  We are interested in spaces which have orbit-finite dimension, which means that there the entire space can be obtained from some finite subset by using atom automorphisms and linear combinations\footnote{In the presence of atoms, choice can be problematic, and in some cases one cannot choose a basis that is invariant under atom automorphisms. Hence, it is more formally correct to talk about orbit-finitely spanned spaces.}. A prototypical example is the space $\Lin X$, which consists of formal linear combinations of elements from some orbit-finite set~$X$. The original application for these spaces was in automata theory, but they have also found applications in the study of  orbit-finite linear programming~\cite{ghosh2023orbit}, function spaces for orbit-finite sets~\cite{functionSpaces2024}, or  the analysis of two-party communication protocols over infinite alphabets~\cite{aliceBob}.



To be useful, the theory of orbit-finitely spanned vector spaces should have certain properties. For example, one would like to be able to represent these spaces in a finite way, or solve algorithmic problems such as solving systems of linear equations. One rather modest requirement is that the spaces be closed under taking subspaces (here, a subspace must be closed under both linear combinations and atom automorphisms): a subspace of an orbit-finitely spanned vector space $V$ should itself be orbit-finitely spanned. We do not know if this closure property holds in general, and we think that it is an important open problem. This problem can be equivalently phrased  in terms of ascending chains: is it true that every orbit-finitely spanned vector space has the  \emph{ascending chain property}, which means that  one cannot find an infinite ascending chain of its subspaces? 
To the best of our knowledge, this question was first recognised by Camina and Evans~\cite[Q.~2]{CaminaEvans_91}, % who %asked (not using the orbit-finite terminology):
%\begin{description}\item[Question.]
%Which orbit-finitely spanned vector spaces over an oligomorphic structure  have the ascending chain property?
%\end{description}
who identified a sufficient condition for this, namely the existence of an Ahlbrandt-Ziegler enumeration. Using this condition they showed the ascending chain property for some vector spaces, notably $\Lin {\binom \A d}$ for the equality atoms (the first row of Figure~\ref{fig:fraisse-limits}), and also for a similar space over the bit-vector atoms (the last row of Figure~\ref{fig:fraisse-limits}).

In~\cite[p.~21]{BFKM24} it is conjectured that 
%the answer to the above question is ``all of them'', i.e.~
{\em every} orbit-finitely spanned vector space over an oligomorphic structure has the ascending chain property. This conjecture has been confirmed for some oligomorphic structures; indeed, in some cases we can prove a stronger property, which we call the \emph{finite length property}: for an orbit-finitely spanned vector space, there is some finite upper bound on the length of chains of its subspaces. By the Jordan-Holder Theorem, the finite length property is equivalent to saying that both ascending and descending chains of subspaces are finite.
\begin{itemize}
    \item \emph{The equality atoms.} If the atoms have equality only, then all orbit-finitely spanned vector spaces have the finite length property.  This was proved independently in three different contexts:  model theory~\cite[Thm.~3.9]{EvansRashwan_02},  representation theory~\cite[Prop.~6.1.6]{SamSnowden_15} and orbit-finite sets~\cite[Thm.~4.10]{BFKM24}. The result from~\cite{SamSnowden_15} assumes that the underlying field is the complex numbers, while the other two do not restrict the choice of a field. 
    \item \emph{The order atoms.} In~\cite[Thm.~4.10]{BFKM24}, the finite length property was proved for the order atoms, i.e.~the rational numbers with their order. The weaker ascending chain property for this structure was also proved in~\cite[Thm.~27]{GhoshLasota_24}, using an alternative method based on Hilbert's Basis Theorem. That method can also be extended to the Fraïssé limit of trees. (We do not know if the limit of trees has the finite length property.)
\end{itemize}
The  finite length property is strictly stronger than the ascending chain property, and it fails in some orbit-finitely spanned vector spaces. Examples of this can be exhibited using the bit vector atoms, as found independently in~\cite[Thm.~2.7]{Gray97} and~\cite[Thm.~4.16]{BFKM24}.  It is worth noting that the failure of the finite length property is connected to the  Thomas conjecture~\cite[p.~177]{thomas1991reducts} in model theory. As observed by Evans~\cite[Slide 6]{Evans2025Permutation}, a counterexample to that conjecture would arise from a structure which: (a) is homogeneous over a finite relational vocabulary;  and (b) fails the finite length property over some finite field. The known examples of failure of the finite length property are not good enough, since they use an infinite vocabulary (or functions).

\paragraph*{Our contributions.} Until now, the finite length property has been a theory of two examples: it has only been proved for spaces over equality and order atoms. We substantially improve this state of affairs, using two different techniques:
\begin{itemize}
\item In Theorem~\ref{thm:weak-smooth-approximation-finite-length}, we prove the finite length property for structures $\A$ which admit what we call oligomorphic approximation, a relaxed version of smooth approximation known from model theory. This is under an additional assumption that the underlying field has characteristic $0$.
\item In Theorem~\ref{thm:ordered-free-amalg-has-finite-length}, the finite length property is shown for $\A$ which arise as Fraïssé limits of generically ordered free amalgamation classes, over relational vocabularies of arity at most $2$. Here we do not restrict the underlying field.
\end{itemize}
In particular, a special case of either of these techniques is the Rado graph (row 3 in Figure~\ref{fig:fraisse-limits}), where even the weaker ascending condition property was not known before.

Most of the paper is devoted to introducing the background necessary to understand these results (Sections~\ref{sec:structures}-\ref{sec:spaces}), and to proving them (Sections~\ref{sec:characteristic-zero},~\ref{sec:free-amalg}-\ref{sec:cogs-turn} and the technical Appendix), but in Sections~\ref{sec:duals} and~\ref{sec:equivariant-subspaces} we briefly discuss some additional applications of our techniques to the theory of weighted register automata and solving orbit-finite system of equations.
