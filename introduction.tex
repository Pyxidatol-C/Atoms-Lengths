\section{Introduction}

This paper is rooted in the programme on orbit-finite sets. In this programme, one starts with some relational structure $\A$, which is called the \emph{atom structure}, and whose elements are called \emph{atoms}. Based on this structure, one constructs sets which are called \emph{orbit-finite}. A precise definition will be given later in the paper, but the understanding is that elements of an orbit-finite set are constructed using atoms, and there are finitely many of them up to automorphisms of the atom structure. For the theory to make sense, we must assume that the atom structure is \emph{oligomorphic}, which means that $\A^d$ has finitely many orbits for every $d$. The simplest example of an oligomorphic structure is what we refer to as the \emph{equality atoms}, this is the structure which has a countably infinite underlying set, and no relations except for equality. This structure, like all oligomorphic structures, arises by applying a model theoretic construction (the Fraïssé limit) to a  well-behaved class of finite structures. Figure~\ref{fig:fraisse-limits} contains other examples of such structures. 
\begin{figure}
    \begin{center}
    \begin{tabular}{ll|l}
   &  class of finite structures & its Fraïssé limit\\
    \hline
    1.& finite sets with equality only & infinite set with equality only \\
    2.& finite orders & rational numbers with order \\
    3.& finite graphs & the Rado graph \\ 
    4.& vector spaces of finite & vector space of countably\\  
    & \quad dimension over $\mathbb F_2$ & \quad infinite  dimension over $\mathbb F_2$
\end{tabular}
\end{center}
\caption{\label{fig:fraisse-limits} Examples of Fraïssé limits}
\end{figure}

In the last row, we assume that the underlying field is finite, since we want the finite dimensional vector spaces to be finite structure. In the case of the two-element field, we use the name \emph{bit-vector atoms} for the structure in the last row. 

When the underlying atom structure is oligomorphic, the orbit-finite sets have a robust theory, which resembles in some ways the theory of finite sets. This theory was originally developed to describe regular languages over infinite alphabets, by considering orbit-finite versions of various automata models~\cite{bojanczykNominalMonoids2013,bojanczykAutomataTheoryNominal2014}, but it has developed to cover other models, such as orbit-finite Turing machines~\cite{bojanczykTuringMachinesAtoms2013}, or orbit-finite constraint satisfaction problems~\cite{klin2015locally}.
Also, there are programming languages with data structures that can store  orbit-finite sets~\cite{bojanczyk2012towards,bojanczyk2012imperative}, and even working implementations~\cite{kopczynski2016lois, szynwelski-phd}. For a survey of the orbit-finite programme, we refer to the lecture notes~\cite{bojanczyk_slightly}. 

Some results do not depend on the choice of the atom structure, while others do. Here is an example of the latter case that arises for orbit-finite Turing machines~\cite{bojanczykTuringMachinesAtoms2013}. If we choose the atoms  to be the Fraïssé limit of linear orders, as in row 2 of Figure~\ref{fig:fraisse-limits},  then the orbit-finite version of \textsc{p}$\neq$\textsc{np} has the same answer as the  classical version without atoms.  On the other hand, if we choose any of the other three rows of the table, then one can prove unconditionally that \textsc{p}$\neq$\textsc{np} holds in the orbit-finite setting, by leveraging problems with choice. The dependency on the underlying atom structure will play a prominent role in this paper. 


% An \emph{orbit-finite} set over this atom structure  is one that can be generated from finitely many elements by applying atom automorphisms. For example, if the atom structure has on relations except for equality (such a structure is called the \emph{equality atoms}), then its automorphisms are all permutations, and the set $\A^2$ is orbit-finite, since it can be generated by starting with two elements: a non-equal pair $(a,b)$ and an equal pair $(a,a)$; all other elements of $\A^2$ can be obtained from these two by applying permutations of the atoms.  Orbit-finite sets were introduced in order to define models of computation, such as orbit-finite monoids~\cite{bojanczykNominalMonoids2013}, orbit-finite automata~\cite{bojanczykAutomataTheoryNominal2014}, or orbit-finite Turing machines~\cite{bojanczykTuringMachinesAtoms2013}. There are also strong connections with the theory of nominal sets~\cite{PittsAM:nomsns}.

\paragraph*{Vector spaces.}
A recent extension of the orbit-finite programme, which was motivated by the study of orbit-finite weighted automata,  is to consider vector spaces~\cite{BFKM24}. In these vector spaces, one can take linear combinations and apply  automorphisms of the atom structure.  We are interested in spaces which have orbit-finite dimension, which means that there the entire space can be obtained from some finite subset by using atom automorphisms and linear combinations\footnote{In the presence of atoms, choice can be problematic, and in some cases one cannot choose a basis that is invariant under atom automorphisms. Hence, it is more formally correct to talk about orbit-finitely spanned spaces.}. A prototypical example is the space $\Lin X$, which consists of formal linear combinations of elements in some orbit-finite set~$X$. As mentioned before, the original application for these spaces was in automata theory, but they have also found applications in the study of  orbit-finite linear programming~\cite{ghosh2023orbit}, function spaces for orbit-finite sets~\cite{functionSpaces2024}, or  the analysis of two-party communication protocols over infinite alphabets~\cite{aliceBob}.



In order to be useful, the theory orbit-finitely spanned vector spaces should have certain properties. For example, one would like to be able to represent these spaces in a finite way, or solve algorithmic problems such as solving systems of linear equalities. One rather modest requirement is that the spaces are closed under taking subspaces (here, a subspace must be closed under both linear combinations and atom automorphisms): if we take an orbit-finitely spanned vector space $V$, and we restrict it to a subspace, then the result should also be orbit-finitely spanned. We do not know if this closure property holds in general, which we think is an important open problem. This problem can be phrased  in terms of ascending chains: is it true that every orbit-finitely spanned vector space has the  \emph{ascending chain condition}, which means that  one cannot find an infinite ascending chain of subspaces. 
To the best of our knowledge, this question was first recognised by Camina and Evans, who asked (not using the orbit-finite terminology):
\begin{description}\item[Question.]{\cite[Question 2]{CaminaEvans_91}}
Let $\A$ be an oligomorphic structure. Which orbit-finitely spanned vector spaces over $\A$ have the ascending chain condition?
\end{description}
In~\cite{CaminaEvans_91}, the authors identify a sufficient condition for the ascending chain condition, namely the existence of an Ahlbrandt-Ziegler enumeration. Using this sufficient condition, they show the  ascending chain condition for some vector spaces, notably $\Lin {\A \choose d}$ for the equality atoms (the first row of Figure~\ref{fig:fraisse-limits}), and also for a similar space in the bit-vector atoms (the last row of Figure~\ref{fig:fraisse-limits}).

In~\cite[p.~21]{BFKM24}, it is conjectured that the answer to the above question is ``all of them'', for all  oligomorphic structures. This conjecture has been confirmed for some oligomorphic structures, notably the equality atoms: over the equality atoms, all orbit-finitely spanned vector spaces satisfy the ascending chain condition. In fact, the case of the equality atoms was proved independently in three different contexts:  model theory~\cite[Lemma 3.23]{Gray97},  representation theory by~\cite[Proposition 6.1.6]{SamSnowden_15} and orbit-finite sets~\cite{BFKM24}. The second cited result, from representation theory, assumes that the field is the complex numbers, while the other two do not restrict the choice of field. In fact, all three cited results prove a stronger property, which we call the \emph{finite length property}: for every orbit-finitely spanned vector space, there is some finite upper bound on the length of chains of subspaces. Using the Jordan-Holder Theorem, the finite length property is equivalent to saying that both ascending and descending chains of subspaces are finite. This raises the following question: 

\begin{description}\item[Question.]{\cite[Question 2]{CaminaEvans_91}}
Let $\A$ be an oligomorphic structure. Which orbit-finitely spanned vector spaces over $\A$ have the ascending chain condition?
\end{description}


A canonical example of such a set is
\begin{align}\label{eq:lin-first-appearance}
\Lin_\field \A^d,
\end{align}
which is the vector space with basis $\A^d$, over some field $\field$. Assuming that the basis $\A^d$ is finitely generated by using atom automorphisms only (this assumption will always hold in the paper), then this vector space is finitely generated by using both atom automorphisms and linear combinations.  For the spaces in question, we will be interested in subspaces that preserve both kinds of structure, i.e.~they are closed under atom automorphisms and linear combinations. An example of such a subspace in~\eqref{eq:lin-first-appearance} is ``for basis vectors (i.e.~atom tuples) where all atoms are distinct, the coefficients sum up to zero''. This is because membership in this subspace will not be affected by taking linear combinations, or applying a permutation of the atoms. 



(So far I have only included some motivation. More to come later. We should also refer more to the work of Evans.)
\paragraph*{Motivation.}
The original motivation for vector spaces of orbit-finite dimension comes from automata theory. 
 The finite length property was introduced in order to decide equivalence of orbit-finite weighted automata~\cite[Section 3]{BFKM24}.

 Another motivation for studying the finite length property comes from the  Thomas conjecture~\cite[p.~177]{thomas1991reducts} in model theory. As observed by Evans~\cite[Slide 6]{Evans2025Permutation}, a counterexample to the Thomas conjecture would arise if one could find an example of a structure which: (a) is homogeneous over a finite relational vocabulary;  and (b) fails the finite length property over some finite field. We do know examples where (a) is relaxed by allowing functions (while remaining oligomorphic), see~\cite[Section 4.4]{BFKM24}.

 Vector spaces of orbit-finite dimension are also a solution to an important limitation in the theory of orbit-finite sets, which is the lack of function spaces. If we take an orbit-finite set $X$, then the family of finitely supported subsets (even finite subsets) will not be orbit-finite. These problems go away, however, if we go from orbit-finite sets to vector spaces of orbit-finite dimension. Indeed, if we consider the space of  from $X$ to a field (say the two-element field), 
 then this  (which is now endowed with the structure of a vector space) has orbit-finite dimension~\cite[Theorem 6.7]{BFKM24}, at least in the case of some important structures. This issue is discussed in~\cite[Section 7]{functionSpaces2024}, phrased in the language of monoidal closed categories. A concrete application of function spaces can be found in~\cite{aliceBob}, where they are applied to obtain a characterization of two-party communication protocols over infinite alphabets.