\section{Introduction}

(So far I have only included some motivation. More to come later. We should also refer more to the work of Evans.)
\paragraph*{Motivation.}
The original motivation for vector spaces of orbit-finite dimension comes from automata theory. 
 The finite length property was introduced in order to decide equivalence of orbit-finite weighted automata~\cite[Section 3]{BFKM24}.

 Another motivation for studying the finite length property comes from the  Thomas conjecture~\cite[p.~177]{thomas1991reducts} in model theory. As observed by Evans~\cite[Slide 6]{Evans2025Permutation}, a counterexample to the Thomas conjecture would arise if one could find an example of a structure which: (a) is homogeneous over a finite relational vocabulary;  and (b) fails the finite length property over some finite field. We do know examples where (a) is relaxed by allowing functions (while remaining oligomorphic), see~\cite[Section 4.4]{BFKM24}.

 Vector spaces of orbit-finite dimension are also a solution to an important limitation in the theory of orbit-finite sets, which is the lack of function spaces. If we take an orbit-finite set $X$, then the family of finitely supported subsets (even finite subsets) will not be orbit-finite. These problems go away, however, if we go from orbit-finite sets to vector spaces of orbit-finite dimension. Indeed, if we consider the space of  from $X$ to a field (say the two-element field), 
 then this  (which is now endowed with the structure of a vector space) has orbit-finite dimension~\cite[Theorem 6.7]{BFKM24}, at least in the case of some important structures. This issue is discussed in~\cite[Section 7]{functionSpaces2024}, phrased in the language of monoidal closed categories. A concrete application of function spaces can be found in~\cite{aliceBob}, where they are applied to obtain a characterization of two-party communication protocols over infinite alphabets.