\section{Introduction}

This paper is part of a research programme focused on orbit-finite sets and structures. In this programme, one starts with an infinite relational structure $\A$ whose elements are called \emph{atoms}. Based on these, one constructs sets which are called \emph{orbit-finite}. Precise definitions will follow, but the understanding is that elements of an orbit-finite set are constructed using atoms, and there are only finitely many of them up to automorphisms of $\A$. For the theory to make sense, we must assume that $\A$ is \emph{oligomorphic}, which means that $\A^d$ has finitely many orbits for every $d$. The simplest example of an oligomorphic atom structure is what we refer to as the \emph{equality atoms}; this is the structure which has a countably infinite underlying set, and no relations except for equality. This structure, like all oligomorphic structures, arises by applying a model-theoretic construction (the Fraïssé limit) to a  well-behaved class of finite structures. Figure~\ref{fig:fraisse-limits} shows other examples of such structures. 
\begin{figure}
    \begin{center}
    \begin{tabular}{lll}
   &  class of finite structures & its Fraïssé limit\\
    \toprule
    1.& finite sets with equality only & $(\N, =)$ \\
    2.& finite orders & $(\Q, \leq)$ \\
    3.& finite graphs & $(\text{Rado graph}, E)$ \\ 
    4.& finite $\FF_2$-vector spaces & $\FF_2 \oplus \FF_2 \oplus \FF_2 \oplus \cdots$ 
    \\
    \bottomrule  
\end{tabular}
\end{center}
\caption{\label{fig:fraisse-limits} Examples of Fraïssé limits}
\Description{Fully described in text.}
\end{figure}

When the underlying atom structure is oligomorphic, the orbit-finite sets have a robust theory, which resembles in some ways the theory of finite sets. This theory was originally developed to describe regular languages over infinite alphabets, by considering orbit-finite versions of various automata models~\cite{bojanczykNominalMonoids2013,bojanczykAutomataTheoryNominal2014}, but it has since developed to cover other models, such as orbit-finite Turing machines~\cite{bojanczykTuringMachinesAtoms2013} or orbit-finite constraint satisfaction problems~\cite{klin2015locally}.
Also, there are programming languages with data structures that can store  orbit-finite sets~\cite{bojanczyk2012towards,bojanczyk2012imperative} with working implementations~\cite{kopczynski2016lois, szynwelski-phd}. For a survey of the orbit-finite programme, we refer to the lecture notes~\cite{bojanczyk_slightly}. 

Some results do not depend on the choice of the atom structure, while others do. Here is an example of the latter case that arises for orbit-finite Turing machines~\cite{bojanczykTuringMachinesAtoms2013}. If we choose the atoms  to be the Fraïssé limit of linear orders, as in row~2 of Figure~\ref{fig:fraisse-limits},  then the orbit-finite version of \textsc{p}$\neq$\textsc{np} has the same answer as the  classical version without atoms.  On the other hand, if we choose any of the other three rows of the table, then one can prove unconditionally that \textsc{p}$\neq$\textsc{np} holds in the orbit-finite setting, by leveraging problems with choice. The dependency on the underlying atom structure will play a prominent role in this paper. 


\paragraph*{Vector spaces.}
One direction of the orbit-finite programme, motivated by the study of orbit-finite weighted automata,  is focused on vector spaces~\cite{BFKM24}. In these spaces (taken over some fixed field), one can take linear combinations and apply  automorphisms of the atom structure.  We are interested in spaces which have an orbit-finite spanning set, which means that the entire space can be obtained from some finite subset by using atom automorphisms and linear combinations.
A prototypical example is the space $\Lin X$, which consists of formal linear combinations of elements from some orbit-finite set~$X$. The original application of these spaces was in automata theory, but they have also found applications in the study of  orbit-finite linear programming~\cite{ghosh2023orbit}, function spaces for orbit-finite sets~\cite{functionSpaces2024}, or  the analysis of two-party communication protocols over infinite alphabets~\cite{aliceBob}.

To be useful, the theory of orbit-finitely spanned vector spaces should have certain properties. For example, one would like to be able to represent these spaces in a finite way, or solve algorithmic problems such as solving systems of linear equations. One rather modest requirement is that the spaces be closed under taking equivariant subspaces (that is, it must be closed under both linear combinations and atom automorphisms): 
an equivariant subspace of an orbit-finitely spanned vector space $V$ should itself be orbit-finitely spanned. We do not know if this closure property holds in general, and we think that it is an important open problem. This problem can be equivalently phrased  in terms of ascending chains: is it true that every orbit-finitely spanned vector space is \emph{Noetherian}, which means that  one cannot find an infinite ascending chain of its equivariant subspaces? 
To the best of our knowledge, this question was first recognised by Camina and Evans~\cite[Q.~2]{CaminaEvans_91} who identified a sufficient condition for this, namely the existence of an Ahlbrandt--Ziegler enumeration. Using this condition they showed certain vector spaces are Noetherian, notably $\Lin \A$ over the ordered atoms (row~2 of Figure~\ref{fig:fraisse-limits}), $\Lin {\binom \A d}$ over the equality atoms (row~1 of Figure~\ref{fig:fraisse-limits}), and a similar space over the bit-vector atoms (row~4 of Figure~\ref{fig:fraisse-limits}). 

The above question is independently considered in~\cite[p.~21]{BFKM24} where it is conjectured that every oligomorphic structure has the \emph{ascending chain property}, meaning that all orbit-finitely spanned vector spaces over that structure are Noetherian.
In such a vector space, if descending chains as well as ascending chains of equivariant subspaces are all finite, by the Jordan--Hölder Theorem, there is some finite upper bound on the length of chains of its equivariant subspaces.
In that case, we say that the structure has the \emph{finite length property}.
This stronger property has been confirmed for:
\begin{itemize}
    \item \emph{The equality atoms.} This was proved in three different contexts independently:  
    model theory~\cite[Thm.~3.9]{EvansRashwan_02},  representation theory~\cite[Prop.~6.1.6]{SamSnowden_15}, 
    and orbit-finite sets~\cite[Thm.~4.10]{BFKM24}. 
    The result from~\cite{SamSnowden_15} assumes that the underlying field is the complex numbers, while the other two do not restrict the choice of a field. 
    \item \emph{The ordered atoms.} Here, the finite length property was proved in~\cite[Thm.~4.10]{BFKM24}. The weaker ascending chain property for this structure was proved in~\cite[Thm.~27]{GhoshLasota_24}, using an alternative method based on Hilbert's Basis Theorem. That method can also be extended to the Fraïssé limit of trees. (We do not know if the limit of trees has the finite length property.)
\end{itemize}
The finite length property is strictly stronger than the ascending chain property. 
Examples of this can be exhibited using the bit-vector atoms, as found independently in~\cite[Thm.~2.7]{Gray97} and~\cite[Thm.~4.16]{BFKM24}.  
It is worth noting that the failure of the finite length property is connected to the Thomas Conjecture~\cite[p.~177]{thomas1991reducts} in model theory. As observed by Evans~\cite[sld.~6]{Evans2025Permutation}, a counterexample to that conjecture would arise from a structure which: (a) is homogeneous over a finite relational vocabulary;  and (b) fails the finite length property over some finite field. The known examples of failure of the finite length property are not good enough, since they use an infinite vocabulary (or functions).

\paragraph*{Our contributions.} Until now, the finite length property has been a theory of two examples: the  equality and ordered atoms. We substantially improve this state of affairs, using two different techniques:
\begin{itemize}
\item In Theorem~\ref{thm:weak-smooth-approximation-finite-length}, we prove the finite length property for structures $\A$ which admit what we call oligomorphic approximation, a relaxed version of smooth approximation known from model theory --- e.g.~the equality atoms. This is under an additional assumption that the underlying field has characteristic $0$.
\item In Theorem~\ref{thm:ordered-free-amalg-has-finite-length}, the finite length property is shown for $\A$ which arise as Fraïssé limits of generically ordered free amalgamation classes, over relational vocabularies of arity at most $2$ --- e.g.~the ordered atoms. Here we do not restrict the underlying field.
\end{itemize}
In particular, a special case of either of these techniques is the Rado atoms (row~3 in Figure~\ref{fig:fraisse-limits}), where even the weaker ascending condition property was not known before.

Most of the paper is devoted to introducing the background necessary to understand these results (Sections~\ref{sec:structures}-\ref{sec:spaces}), and to proving them (Sections~\ref{sec:characteristic-zero},~\ref{sec:free-amalg}-\ref{sec:cogs-turn} and the technical Appendix). In Section~\ref{sec:duals} we discuss some connections with function spaces and weighted automata, and in Section~\ref{sec:equivariant-subspaces} we  briefly discuss  applications to solving orbit-finite system of equations.
