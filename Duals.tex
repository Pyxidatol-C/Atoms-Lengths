\section{Application: weighted register automata}\label{sec:duals}
Consider the equivariant set
\[
    \fsfun \A \field
\]
of \textcolor{red}{finitely supported (did we ever define this?)} functions from the atom structure $\A$ to a field $\field$.
Here is one way they arise:
given some finite set $S \subseteq \A$, its characteristic function --- which we also denote by $S$ --- is a function with values in $\{0, 1\} \subseteq \field$.
For instance, $\A : \A \to \field$ is the constant function outputting $1$.

Under pointwise addition and scalar multiplication, these $\field$-valued finitely supported functions form a vector space.
It makes sense to ask whether this \emph{dual space} is orbit-finitely spanned.
In \cite[Thm.~6.7 and Ex.~6.9]{BFKM24}, a positive answer was given for the equality atoms and the ordered atoms, whilst a negative answer was given for the Rado graph when $\field$ is the two-element field.
We now show how to drop the assumption on the field,
and we spell out some negative consequences for computation models over the graph atoms.

For the rest of this section, assume that $\A$ is the Rado graph or the {\color{red}{$K_n$-free Henson graph} (never introduced!)}. For any $S \subseteq \A$, let $N_G \subseteq \A$ denote the common neighbours of all atoms in $S$. 

\begin{proposition} The space
    \[
        \langle N_S : \A \to \field \mid S \subseteq \A \text{ is finite} \rangle
    \]
    is not orbit-finitely spanned, for any field $\field$.
\end{proposition}
\begin{proof}
    Suppose towards a contradiction that $\{ N_S \mid S \in \mathcal{S} \}$ is a spanning set for some orbit-finite set $\mathcal{S}$ of finite subsets of $\A$.
    (If $\A$ is the $K_n$-free Henson graph, we may assume that no $S \in \mathcal{S}$ contains a clique on $n - 1$ vertices: otherwise $N_S$ is the constantly zero function, so we may as well remove it from the spanning set.)
    Let $T \subseteq \A$ be an anti-clique on more than $\max_{S \in \mathcal{S}}|S|$ vertices.
    Then we can write
    \[
        N_T = \lambda_1 \cdot N_{S_1} + \cdots + \lambda_n \cdot N_{S_m}
    \]
    for $S_i \in \mathcal{S}$, where we may assume $|S_1| \leq \cdots \leq |S_m| < |H|$.
    
    Let us prove $\lambda_i = 0$ by induction on $i = 1, \dots, n$.
    We may find some $a \in \A$ that is adjacent to all of $S_i$ but to none of 
    \[
    (S_{i+1} \cup \dots \cup S_m \cup T) \setminus S_i.
    \]
    As $\lambda_1, \dots, \lambda_{i-1} = 0$ and each of $S_{i+1}, \dots, S_m, T$ contains some vertex outside of $S_i$,
    we see that
    \[
        0 = N_T(a) = \sum_j \lambda_j \cdot N_{S_j}(a) = 0 + \lambda_i + 0.
    \]

    It follows that $N_S$ is the zero function, which is impossible:
    any finite star certainly embeds in $\A$.
\end{proof}


\begin{corollary}
    Orbit-finitely spanned vector spaces are not closed under taking function spaces.
\end{corollary}
\begin{proof}
As each $N_S$ is finitely supported (by $S$), we have exhibited an equivariant subspace of 
\[
    (\fsfun \A \field)
    \cong (\linfsfun {\Lin \A } {\Lin \{*\}})
\]
that fails to be orbit-finitely spanned.
So the superspace does not have finite length, and therefore it cannot be orbit-finitely spanned by {\color{red}Theorem~\ref{thm:ordered-free-amalg-has-finite-length} -- why 7.3?}.    
\end{proof}


\begin{corollary}
{\color{red}This looks very isolated, with automata etc. never introduced}
    An unambiguous automaton with guessing can recognise the language
    \[
        \text{``the last letter (exists and) is adjacent to every previous letter''}.
    \]
    But viewed as a weighted language, it cannot be  recognised by orbit-finitely spanned automata (unlike its reversal).
\end{corollary}
\begin{proof}
    We use a Myhill-Nerode style argument.
    The derivative ${w -} : \A^* \to \field$ of a word $w \in \A^*$ in this weighted language is given by
    \[
        w v a = N_{w v}(a),
    \]
    Then $\langle {w -} \mid w \in \A^* \rangle$
    cannot be orbit-finitely spanned:
    \[
        {w-} = \sum_i \lambda_i \cdot {w_i-}
    \]
    gives $N_{w} = \sum_i \lambda_i \cdot N_{w_i}$.
\end{proof}