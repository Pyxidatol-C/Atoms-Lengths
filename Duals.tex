\section{Duals}
It was shown in \cite[Example~6.9]{BFKM24} that orbit-finitely generated vector spaces over the Rado graph are not closed under duals over the two-element field.
We can lift the assumption on the field, and we will extend this result to the $K_n$-free Henson graphs.
\begin{theorem}
    Assume $\A$ is the Rado graph or the $K_n$-free Henson graph.
    Then \[
        \fsfun \A \field
    \] is not orbit-finitely generated for any $\field$.
\end{theorem}
\begin{proof}
    By Theorem~\ref{thm:ordered-free-amalg-has-finite-length}, 
    if the space of finitely supported functions from $\A$ to $\field$ is orbit-finitely generated, then it would have finite length.
    However, we will exhibit an infinite ascending chain of equivariant subspaces.

    We can find anti-cliques $I_0, I_1, I_2, \dots$ of sizes $0, 1, 2, \dots$ inside $\A$.
    Let $f_n : \A \to \{0, 1\} \subseteq \field$ be the characteristic function for the common neighbours of all vertices in $I_n$. 
    (Of course, each $f_n$ is supported by $I_n$.)
    We claim that
    \[
        \langle f_0 \rangle 
        \subset \langle f_0, f_1 \rangle
        \subset \langle f_0, f_1, f_2 \rangle
        \subset \cdots
    \]
    is strictly increasing.
    To see this, suppose otherwise that
    \[
        f_{n+1} = \sum_{I \in \mathbb{I}} \lambda_I \cdot f_I
    \]
    where every $I \in \mathbb{I}$ is an anti-clique on at most $n$ vertices, whose common neighbours have characteristic function $f_I$.

    \begin{claim}
        $\lambda_I = 0$ for all $I \in \mathbb{I}$.
    \end{claim}
    \begin{proof}
        Suppose that $\lambda_I = 0$ whenever $|I| < d$. (The base case $d = 0$ is vacuous.)
        Take $I \in \mathbb{I}$ with $|I| = d$.
        Writing $\mathbb{J} = \{J \in \mathbb{I} \setminus \{I\} \mid |J| \geq d\}$, 
        we have
        \[
            f_{n+1} = \lambda_I \cdot f_I 
            + \sum_{J \in \mathbb{J}} \lambda_J \cdot f_J.
        \]
        We may pick a vertex $a \in \A$ that is adjacent to all of $I$ but to none of $\bigcup_{J \in \mathbb{J}} J \cup I_{n+1} \setminus I$.
        Then $f_I(a) = 1$. 
        On the other hand, since $I$ cannot contain $I_{n+1}$ or any $J \in \mathbb{J}$, we must have $f_{n+1}(a) = 0 = f_J(a)$.
        Evaluating the equation above at $a$ yields
        \[
            0 = \lambda_I \cdot 1 + \sum_{J \in \mathbb{J}} \lambda_J \cdot 0;
        \]
        that is, we conclude $\lambda_I = 0$ for any $I \in \mathbb{I}$ with $|I| = d$.
    \end{proof}

    So $f_{n+1}$ is the constant zero function, which is impossible:
    in $\A$ we can certainly find some vertex adjacent to all of $I_{n+1}$.
    This shows that 
    \[
        \fsfun \A \field
    \]
    fails to be Noetherian,
    so it cannot have finite length or be orbit-finitely generated. 
\end{proof}