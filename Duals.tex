\section{Function spaces and  weighted automata}\label{sec:duals}

The original motivation to inroduce orbit-finitely spanned vector spaces in~\cite{BFKM24} was the study of orbit-finite weighted automata. In this section, we recall this motivation, and discuss how it is relates to our new results. This discussion also involves  the issue of function spaces, which is arguably more important, so we begin with that.
 
\paragraph*{Function spaces.} If we have two orbit-finitely spanned vector spaces $V$ and $W$ over the same atoms, then there are two natural ideas for a function space: the space of all linear maps from $V \to W$, and the subspace which consists of equivariant linear maps. As it turns out, the most relevant function space lies between them, as formalised in the following definition.


\begin{definition}[Finitely supported function space]\label{def:function-space}
    For two orbit-finitely spanned vector spaces $V$ and $W$, we define their \emph{function space}, denoted by 
    \begin{align*}
     \linfsfun{V}{W},
    \end{align*}
    to be the space of linear maps $f$ wich satisfy the following {finite support condition}: there is some finite set of atoms $S \subseteq \A$ such that for every atom automorphism $\pi$ that fixes all atoms in $S$, we have 
    \begin{align*}
    \pi(f(v)) = f(\pi(v)) \quad \text{for all $v \in V$.}
    \end{align*}
\end{definition}

The notion of finite supports in the above definition is the same one as used in Section~\ref{sec:orbit-finite-sets}, except that it is applied to the space of linear maps from $V$ to $W$. As argued in~\cite[Section 8.3]{bojanczyk_slightly}, the finitely supported function space is the most relevant kind of function space, since it ensures that the corresponding category is becomes monoidal closed. For this reason, we when talking about function spaces, we mean the finitely-supported function spaces from the above definition. 
\begin{description}
    \item[Question.]  Consider an atom structure. Are orbit-finitely spanned vector spaces closed under taking function spaces?
\end{description}
The answer to the above question  is  ``yes'' for the equality and ordered atoms, which was shown in~\cite[Corollary 6.8]{BFKM24} for a special case of function spaces, namely duals, with the  general case of function spaces being treated in~\cite[Section 8.3]{bojanczyk_slightly}. 
On the other hand, the answer  is ``no'' for the  Rado graph, as shown in the following example. 
\begin{example}
    Assume that the atoms are the Rado graph. (The example would also work for the Hensen graph that will be defined later.) A variant of this example for the two-element field was shown in~\cite[Example 6.9]{BFKM24}, but here we can use any field, in particular a field of characteristic zero as treated in the previous section. 
    
    Consider the space 
    \begin{align}\label{eq:dual-space}
    \fsfun \A \field
    \end{align}
    which consists of functions  from atoms to the field that are finitely supported in the sense of Definition~\ref{def:function-space}. The space above does not talk about linear maps, since there is no vector space structure on the domain $\A$. However, it is  isomorphic to the function space 
    \begin{align}\label{eq:dual-space-two}
    \linfsfun{\Lin_\field \A}{\field}.
    \end{align}
    We will show that~\eqref{eq:dual-space} is not orbit-finitely spanned, and hence the same is true for the isomorphic function space~\eqref{eq:dual-space-two}. 
    
    For a finite set $S \subseteq \A$ of atoms, define a function  $    f_S : \A \to \field$  by
    \begin{align*}
    \quad 
    a \mapsto
    \begin{cases}
        1 & \text{if $a$ is a neighbour of all atoms in $S$}\\
        0 & \text{otherwise}.
    \end{cases}
    \end{align*}
    Define $V$ to be the subspace of~\eqref{eq:dual-space} that is spanned 
    to be the functions $f_S$, where $S$ ranges over finite sets of atoms. Being orbit-finitely spanned is closed under taking subspaces, which follows from   Theorem~\ref{thm:ACP} and the fact that the Rado graph has the ascending chain property (which was already proved in Theorem~\ref{thm:weak-smooth-approximation-finite-length }  for characteristic zero and will be proved later for any field). Therefore, if we show that $V$ is not orbit-finitely spanned, then the same will follow for~\eqref{eq:dual-space}. 
    
    Suppose towards a contradiction that $\{ f_S \mid S \in \mathcal{S} \}$ is a spanning set for some orbit-finite family $\mathcal{S}$ of finite subsets of $\A$.
    Let $T \subseteq \A$ be an independent set (i.e.~no edges inside the set) on more than $\max_{S \in \mathcal{S}}|S|$ vertices.
    Then we can write
    \[
    f_T = \lambda_1 \cdot f_{S_1} + \cdots + \lambda_n \cdot f_{S_n}
    \]
    for $S_i \in \mathcal{S}$, where we may assume $|S_1| \leq \cdots \leq |S_m| < |H|$.
    Let us prove $\lambda_i = 0$ by induction on $i = 1, \dots, n$.
    We may find some $a \in \A$ that is adjacent to all of $S_i$ but to none of 
    \[
    (S_{i+1} \cup \dots \cup S_m \cup T) \setminus S_i.
    \]
    As $\lambda_1, \dots, \lambda_{i-1} = 0$ and each of $S_{i+1}, \dots, S_m, T$ contains some vertex outside of $S_i$,
    we see that
    \[
        0 = f_T(a) = \sum_j \lambda_j \cdot f_{S_j}(a) = 0 + \lambda_i + 0.
    \]
    It follows that $f_T$ is the zero function, which cannot be true, since one can find an atom that is a common neighbour of all atoms in $T$.
\end{example}
 
Since we prove that the Rado graph has the finite length property, and it does not have orbit-finitely spanned function spaces, it follows that one property does not imply the other.


% \begin{theorem}
%     Assume that the atoms $\A$ are such that orbit-finitely spanned vector spaces are closed under taking function spaces. Then two views of orbit-finite weighted automata, deterministic and nondeterministic,  are equivalent. 
% \end{theorem}
% \begin{proof}
%     To go 
% \end{proof}
% A special case of a function space is the dual space 
% \begin{align*}
% \linfsfun V \field.
% \end{align*}
% If $V$ has a basis $X$, then this space is isomorphic to the space 
% \begin{align}\label{eq:fsfun}
% \fsfun X \field,
% \end{align}
% which consists of functions from $X$ to the field that are finitely supported in the sense described in Definition~\ref{def:function-space}. Note that there is a terminology clash, since the expression ``finitely supported'', when applied to functions of type $X \to \field$, could also be understood as ``zero almost everywhere''. 


\paragraph*{Weighted automata.} We now describe the orbit-finite version of weighted automata, and explain how the issues with function spaces have an impact on the theory of these automata. 

There are several ways of defining weighted automata, we choose one that  views them as deterministic automata, in which the states are endowed with a vector space structure.  An \emph{orbit-finite weighted automaton} has an input alphabet $\Sigma$, which is an orbit-finite set, and a state space $Q$, which is an orbit-finitely spanned vector space $Q$. In the state space, there is a distinguished  initial vector $q_0$ that is equivariant, an equivariant transition function 
\begin{align*}
\delta : Q \times \Sigma \to Q
\end{align*}
which becomes a linear map once any input letter $a \in \Sigma$ is fixed, and a  final function which is an equivariant linear map from $Q$ to the field. The output of the automaton is defined by applying the final map to the last state in the run. 


\begin{corollary}
{\color{red}This looks very isolated, with automata etc. never introduced}
    An unambiguous automaton with guessing can recognise the language
    \[
        \text{``the last letter (exists and) is adjacent to every previous letter''}.
    \]
    But viewed as a weighted language, it cannot be  recognised by orbit-finitely spanned automata (unlike its reversal).
\end{corollary}
\begin{proof}
    We use a Myhill-Nerode style argument.
    The derivative ${w -} : \A^* \to \field$ of a word $w \in \A^*$ in this weighted language is given by
    \[
        w v a = N_{w v}(a),
    \]
    Then $\langle {w -} \mid w \in \A^* \rangle$
    cannot be orbit-finitely spanned:
    \[
        {w-} = \sum_i \lambda_i \cdot {w_i-}
    \]
    gives $N_{w} = \sum_i \lambda_i \cdot N_{w_i}$.
\end{proof}