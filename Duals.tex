\section{Dual spaces}
Consider the equivariant set
\[
    \fsfun \A \field
\]
of \textcolor{red}{finitely supported (did we ever define this?)} functions from the atoms $\A$ to the field $\field$.
Here is one way they arise:
given some finitely supported set $S \subseteq \A$, its characteristic function --- which we also denote by $S$ --- is a function with values in $\{0, 1\} \subseteq \field$.
For instance, $\A : \A \to \field$ is the constant function outputting $1$.

Under pointwise addition and scalar multiplication, these $\field$-valued finitely supported functions form a vector space.
It makes sense to ask whether this \emph{dual space} is orbit-finitely spanned.
In \cite[Theorem~6.7 and Example~6.9]{BFKM24}, a positive answer was given for the equality atoms and the ordered atoms, whilst a negative answer was given for the Rado graph when $\field$ is the two-element field.
We show how to lift the assumption on the field,
and we spell out some negative consequences for computation models over the graph atoms.

Hereafter, assume $\A$ is the Rado graph or the $K_n$-free Henson graph.
Let $N_G \subseteq \A$ denote the common neighbours of all atoms in $G \subseteq \A$. 

\begin{proposition}
    \[
        \langle N_G : \A \to \field \mid G \subseteq \A \text{ is finite} \rangle
    \]
    is not orbit-finitely spanned for any field $\field$.
\end{proposition}
\begin{proof}
    Suppose towards a contradiction that $\{ N_G \mid G \in \mathcal{G} \}$ is a spanning set for some orbit-finite $\mathcal{G}$.
    (If $\A$ is the $K_n$-free Henson graph, we may assume that no $G \in \mathcal{G}$ contains a clique on $n - 1$ vertices: otherwise $N_G$ is the constant zero function, so we may as well remove it from the spanning set.)
    Let $H \subseteq \A$ be an anti-clique on more than $\max_{G \in \mathcal{G}}$ vertices.
    Then we can write
    \[
        N_H = \lambda_1 \cdot N_{G_1} + \cdots + \lambda_n \cdot N_{G_m}
    \]
    for $G_i \in \mathcal{G}$, where we may assume $|G_1| \leq \cdots \leq |G_m| < |H|$.
    
    Let us prove $\lambda_i = 0$ for $i = 1, \dots, n$.
    We may find some $a \in \A$ that is adjacent to all of $G_i$ but to none of $G_{i+1} \cup \dots \cup G_m \cup H \setminus G_i$.
    As $\lambda_1, \dots, \lambda_{i-1} = 0$ and $G_{i+1}, \dots, G_m, H$ each contains some vertex outside $G_i$,
    we see that
    \[
        0 = N_H(a) = \sum_j \lambda_j \cdot N_{G_j}(a) = 0 + \lambda_i + 0.
    \]

    It follows that $N_H$ is the zero function, which is impossible:
    any finite star certainly embeds in $\A$.
\end{proof}


\begin{corollary}
    Orbit-finitely spanned vector spaces do not admit function spaces.
\end{corollary}
\begin{proof}
As each $N_G$ is finitely supported (by $G$), we have exhibited an equivariant subspace of 
\[
    (\fsfun \A \field)
    \cong (\linfsfun {\Lin \A } {\Lin \{*\}})
\]
that fails to be orbit-finitely spanned.
So the superspace does not have finite length, and therefore it cannot be orbit-finitely spanned by Theorem~\ref{thm:ordered-free-amalg-has-finite-length}.    
\end{proof}


\begin{corollary}
    An unambiguous automaton with guessing can recognise the language
    \[
        \text{``the last letter (exists and) is adjacent to every previous letter''}.
    \]
    But viewed as a weighted language, it cannot be  recognised by orbit-finitely spanned automata (unlike its reversal).
\end{corollary}
\begin{proof}
    We use a Myhill-Nerode style argument.
    The derivative ${w -} : \A^* \to \field$ of a word $w \in \A^*$ in this weighted language is given by
    \[
        w v a = N_{w v}(a),
    \]
    Then $\langle {w -} \mid w \in \A^* \rangle$
    cannot be orbit-finitely spanned:
    \[
        {w-} = \sum_i \lambda_i \cdot {w_i-}
    \]
    gives $N_{w} = \sum_i \lambda_i \cdot N_{w_i}$.
\end{proof}