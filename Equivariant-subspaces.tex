\section{Application: solving orbit-finite linear equations}
We will now show an important corollary of Theorem~\ref{thm:cog-span-generally}.
Let $\mathcal{O} \subseteq \A^{I}$ be an $S$-ordered orbit.
Then $\len(\Lin_\FF\mathcal{O}) = 2^{|I_1|}$;
in fact, we can characterise all the $\Aut(\A)_{(S)}$-equivariant subspaces of $\Lin_\FF\mathcal{O}$.

%\paragraph{Local coefficients}
First we set up some useful notation.
Consider the $2^{|I|}$ projected $S$-ordered orbits $\mathcal{O}|^{J}$ for $J \subseteq I$.
Suppose that
\[
    f : \mathcal{O}|^J \to \mathcal{O}|^{J'}
\]
is an $\Aut(\A)_{(S)}$-equivariant bijection.
Take any $a \in \mathcal{O}|^J$,
and enumerate its entries as $a_1 < \dots < a_{|J|}$.
Similarly, enumerate the entries of $f(a)$ as $b_1 < \dots < b_{|J'|}$.
Then $\{a_1, \dots, a_{|J|}\} = \{ b_1, \dots, b_{|J'|} \}$ because $\A$ has no algebraicity; 
since the orbits are ordered, we must have $|J| = |J'|$ and $a_1 = b_1, \dots, a_{|J|} = b_{|J'|}$.
That is, $f$ must be the obvious function that reindexes a $J$-tuple to a $J'$-tuple
--- hence we will write $a^{/J'}$ instead of $f(a)$, leaving $f$ implicit.

Now, let $\mathcal{Q}_{1} = \mathcal{O}|^{J_1}, \dots, \mathcal{Q}_{t} = \mathcal{O}|^{J_t}$ be distinct projected $S$-ordered orbits up to $\Aut(\A)_{(S)}$-equivariant bijections,
enumerated in such a way that $|J_1| \geq |J_2| \geq \dots \geq |J_t|$. (In particular, $J_1=I$ and $J_t=\emptyset$.)

%\begin{definition}
    For $i = 1, \dots, t$, let $P_i$ consist of all sets $J$ such that $\mathcal{O}|^J$ is in $\Aut(\A)_{(S)}$-equivariant bijection with $\mathcal{Q}_i$.
    Assemble all $|P_i|$ projections into a single map
    \[
        (-){\restriction_i} : \Lin_\FF\mathcal{O} \to \Lin_{\FF^{P_i}} \mathcal{Q}_i.
    \]
    More precisely, $v{\restriction_i}(a)$ (for $a\in{\cal Q}_i$) is the $P_i$-tuple whose $J$-th entry is $v|^J(a^{/J}) \in \FF$.
    It is straightforward to check that $(-){\restriction_i}$ is $\Aut(\A)_{(S)}$-equivariant and linear.
%\end{definition}

Let $W \subseteq \Lin_\FF \mathcal{O}$ be an $\Aut(\A)_{(S)}$-equivariant subspace. Let $W {\restriction_i} (\mathcal{Q}_i) \subseteq \FF^{P_i}$ be the set of all $P_i$-tuples that can be obtained as $w {\restriction_i} (a)$ for some $w\in W$ and $a\in{\cal Q}_i$. This is a finite-dimensional vector space.
Now define $\widetilde W$, which consists of all vectors $v \in \Lin_\FF \mathcal{O}$ such that
\[
    v {\restriction_i} (\mathcal{Q}_i) \subseteq W {\restriction_i} (\mathcal{Q}_i) \qquad \text{for all } i=1,\ldots, t.
\]
Then $\widetilde W$ is an $\Aut(\A)_{(S)}$-equivariant subspace that contains $W$.
It turns out these two are equal:

\begin{lemma}\label{lem:coeff-approximation}
For every $i=0,\ldots, t$,
\[
        \widetilde W \cap \ker(\restriction_{i+1}) \cap \cdots \cap \ker(\restriction_t) 
        \subseteq W \cap \ker(\restriction_{i+1}) \cap \cdots \cap \ker(\restriction_t).
\]
    In particular $\widetilde W \subseteq W$ when $i = t$.
\end{lemma}

{\sc Proof.}
We proceed by induction on $i$.
The base case $i = 0$ is trivial, since $\ker(\restriction_1) = \{0\}$. Indeed, $P_1=\{I\}$, so $v{\restriction_1}$ is the identity map.

To prove the containment for some $i>0$, 
we allow $v {\restriction_i}$ to be non-zero.
But $v {\restriction_i}$ satisfies the next best property:
\begin{claim}
    The image of \[
        \widetilde W \cap \ker(\restriction_{i+1}) \cap \cdots \cap \ker(\restriction_t)
    \] under $\restriction_{i}$ 
    is contained in $\Ker_{W {\restriction}_{i} (\mathcal{Q}_{i})} \mathcal{Q}_{i}$. 
\end{claim}
\begin{proof}
    Take any $v \in \widetilde W$ such that  $v {\restriction_{i'}} = 0$ for all $i'>i$.
    That $v {\restriction_{i}} \in \Lin_{W{\restriction_{i}}(\mathcal{Q}_{i})} \mathcal{Q}_{i}$ is clear from the definition of $\widetilde W$.
    Recall that $\mathcal{Q}_{i}$ = $\mathcal{O}|^{J_{i}}$.
    For every $j \in J_{i}$,
    we need to prove that $v {\restriction_{i}} |^{-j} = 0$. In more elementary terms, given any $a\in\mathcal{O}|^{J_{i}\setminus\{j\}}$, we need that the $J$-th entry of $v {\restriction_{i}} |^{-j}(a)$ is $0$, for every $J\in P_i$.
    
    So take any $J \in P_{i}$.
    The unique monotone bijection between $J_{i}$ and $J$ restricts to one between $J_{i} \setminus \{j\}$ and $J \setminus \{j'\}$, for some $j'\in J$. Denote $J'=J \setminus \{j'\}$.
    Then $J'$ belongs to some $P_{i'}$ with $i' > i $,
    so $v {\restriction_{i'}} = 0$. Now calculate (with $a\in\mathcal{O}|^{J_{i}\setminus\{j\}}$, $b\in{\cal Q}_i$, $c\in {\cal O}|^J$ and $d\in\cal O$):
\begin{align*}
	(v{\restriction_i}|^{-j}(a))_J &= \sum_{b|^{-j}=a}(v{\restriction_i}(b))_J
	= \sum_{b|^{-j}=a}v|^J(b^{/J}) 
	= \sum_{c|^{-j'}=a^{/J'}}v|^J(c) \\
	&= \sum_{d|^{J'}=a^{/J'}}v(d)
	\,\, = v|^{J'}(a^{/J'}) = (v{\restriction_{i'}}(a))_{J'} = 0.
\end{align*}    
\end{proof}

From Theorem~\ref{thm:cog-span-generally} we get: 
\[\Ker_{W {\restriction_{i}} (\mathcal{Q}_{i})} \mathcal{Q}_{i} \subseteq \Cog_{W {\restriction_{i}} (\mathcal{Q}_{i})} \mathcal{Q}_{i}.\]
Furthermore:
\begin{claim}
    $\Cog_{W {\restriction}_{i} (\mathcal{Q}_{i})} \mathcal{Q}_{i}$ is contained in
    the image of \[
        W \cap \ker(\restriction_{i+1}) \cap \cdots \cap \ker(\restriction_t)
    \] under $\restriction_{i}$. 
\end{claim}
\begin{proof}
    Consider any ${\cal Q}_i$-cog with a coefficient $\lambda\in W {\restriction_{i}} (\mathcal{Q}_{i})$, and let $w\in W$ and $a\in{\cal Q}_i$ be such that $\lambda = w{\restriction_{i}}(a)$.
    Let $S'$ consist of $S$ together with every atom appearing in $w$ but not in $a$.
    We generalise the construction used in the proof of Theorem~\ref{thm:cogs-arise-everywhere}.
    
    Apply Proposition~\ref{prop:cog-fresh-full} and Remark~\ref{rem:duo} to get automorphisms $\pi_j$ for $j \in J_i$ such that $a \parallel \prod_{j \in J_i} \pi_j \cdot a$ is an $\mathcal{Q}_i$-duo, 
    where each $\pi_j$ fixes $S'$ and all $a_{j'}$ and $\pi_{j'}(a_{j'})$ for $j'\neq j$. Since all $\mathcal{Q}_i$-duos are in the same orbit, it is enough to show that the cog corresponding to this particular duo, with the coefficient $\lambda$, belongs to the $\restriction_i$-image as in the statement of the Claim.
    
    
    Put \[
        w' = \prod_{j \in J_{i}} (1 - \pi_j) \cdot w \in W.
    \]
    For any $1 \leq i' \leq t$, note that as no more atoms can appear in $w{\restriction_{i'}}$ than in $w$, we have
    \begin{align*}
        &w' {\restriction_{i'}}
        = \prod_{j \in J_{i}} (1 - \pi_j) \cdot w {\restriction_{i'}} 
        = \sum_{c \in C_i} \sum_{J' \subseteq J_i} (-1)^{|J'|}  
            w {\restriction_{i'}}(c) \left(\prod_{j \in J'} \pi_j \cdot c\right)
%        &{}= \sum_{b_\bullet \in \mathcal{Q}_{i'}, \{b_j \mid j\} \supseteq \{a_j \mid j\}} \sum_{J' \subseteq J} (-1)^{|J'|}  
%            w {\restriction_{i'}}(b_\bullet) \prod_{j \in J'} \pi_j \cdot b_\bullet.
    \end{align*}
    where
        \[C_i = \{ c\in{\cal Q}_{i'} \mid \{c_j\mid j\in J_{i'}\} \supseteq \{a_j \mid j\in J_i\} \}.
    \]
    (The formula used in the proof of Theorem~\ref{thm:cogs-arise-everywhere} is a special case of this for $i'=1$ so that $J_{i'}=I$ and ${\cal Q}_{i'}={\cal O}$.) Now, if $i'>i$ then $C_i$ is empty, and so $w' {\restriction_{i+1}} = \cdots = w' {\restriction_{t}} = 0$. Moreover, if $i'=i$ then $C_i = \{a\}$ and we obtain the cog from before:
    \[
        w' {\restriction_{i}} = \lambda \left(a \between \prod_{j \in J_{i}} \pi_j \cdot a\right).
    \]
So $w'$ is a witness for the inclusion from the Claim.
\end{proof}

This is enough to establish Lemma~\ref{lem:coeff-approximation} for $i$, assuming it for $i-1$.
Indeed, given $v \in \widetilde W \cap \ker(\restriction_{i+1}) \cap \dots \cap \ker(\restriction_t)$, by the preceding claims
we can find $w \in W \cap \ker(\restriction_{i+1}) \cap \dots \cap \ker(\restriction_t) \subseteq \widetilde W$ such that 
$v {\restriction_i} = w {\restriction_i}$.
But then $(v - w) {\restriction_{i}} = 0$, so $v - w$ lies in $\ker(\restriction_{i})$ as well as $\widetilde W \cap \ker(\restriction_{i+1}) \cap \cdots \cap \ker(\restriction_t)$.
It follows from the inductive hypothesis that \[
    v - w \in W \cap \ker(\restriction_{i+1}) \cap \dots \cap \ker(\restriction_t),
\]
so $v = (v - w) + w$ is in $W \cap \ker(\restriction_{i+1}) \cap \dots \cap \ker(\restriction_t)$ as well.

This completes the proof of Lemma~\ref{lem:coeff-approximation}. \qed

\paragraph{Lengths}
Let $W, W'$ be two $\Aut(\A)_{(S)}$-equivariant subspaces of $\Lin_\FF(\mathcal{O})$.
If we have $W {\restriction_i}(\mathcal{Q}_i) = W' {\restriction_i}(\mathcal{Q}_i)$ for all $i=1,\ldots, t$,
then $W = \widetilde W = \widetilde{W'} = W'$ by Lemma~\ref{lem:coeff-approximation}.
As a consquence:

\begin{proposition}\label{prop:length-upper-bound}
    Let $W_0 \subsetneq W_1 \subsetneq \cdots \subsetneq W_l$ be a chain of $\Aut(\A)_{(S)}$-equivariant subspaces of $\Lin_\FF(\mathcal{O})$.
    Then $l \leq 2^{|I|}$. 
\end{proposition}
\begin{proof}
    We obtain $t$ chains
    \begin{align*}
        W_0 {\restriction_1} (\mathcal{Q}_1) \subseteq W_1 {\restriction_1} (\mathcal{Q}_1) \subseteq \cdots \subseteq W_l {\restriction_1} (\mathcal{Q}_1) \subseteq \FF^{P_1}, \\
        W_0 {\restriction_2} (\mathcal{Q}_2) \subseteq W_1 {\restriction_2} (\mathcal{Q}_2) \subseteq \cdots \subseteq W_l {\restriction_2} (\mathcal{Q}_2) \subseteq \FF^{P_2}, \\
        \vdots \\
        W_0 {\restriction_t} (\mathcal{Q}_t) \subseteq W_1 {\restriction_t} (\mathcal{Q}_t) \subseteq \cdots \subseteq W_l {\restriction_t} (\mathcal{Q}_t) \subseteq \FF^{P_t}.
    \end{align*}
    At each of the $l$ steps, one of the $t$ containments must be strict.
    Hence $l \leq |P_1| + |P_2| + \dots + |P_t| = 2^{|I|}$. 
\end{proof}

It follows that every $\Aut(\A)_{(S)}$-equivariant subspace of $\Lin_\FF(\mathcal{O})$ is finitely generated.
We can compute the local coefficients of such subspaces easily:
\begin{remark}
    For $v \in \Lin_\FF(\mathcal{O})$, 
    let $\langle v \rangle$ denote the $\Aut(\A)_{(S)}$-equivariant subspace it generates.
    Then:
    \begin{enumerate}
        \item 
        $\langle v \rangle {\restriction_i}(\mathcal{Q}_i)$ is the subspace of $\FF^{P_i}$ generated by vectors of the form $v {\restriction_i} (a)$,
        which is zero unless every atom appearing in $a$ appears in $v$
        --- there are only finitely many such $a$'s;

        \item 
        $\langle v, v' \rangle {\restriction_i}(\mathcal{Q}_i) = \langle v \rangle {\restriction_i}(\mathcal{Q}_i) + \langle v' \rangle {\restriction_i}(\mathcal{Q}_i)$.
    \end{enumerate}
\end{remark}

To complement the upper bound from Prop.~\ref{prop:length-upper-bound}, we now exhibit a chain of $\Aut(\A)_{(S)}$-equivariant subspaces whose length is precisely $\sum_{i=1}^t 2^{|P_i|}$, generalising \cite[Corollary~4.12]{BFKM24}.
Take any $J \in P_i$. 
Pick some $a\in \mathcal{O}$,
and let $\pi_j$ (for $j \in I$) be the automorphisms from Prop.~\ref{prop:cog-fresh-full}.
Define a vector
\begin{align*}
    v^i_{J} =  \prod_{j \in J} (1 - \pi_j) \cdot a \in \Lin_\FF(\mathcal{O}).
\end{align*}
Then 
\[
    (\langle v^i_{J} \rangle {\restriction_{i'}} (\mathcal{Q}_{i'}))_{J'} =
    \begin{cases}
        \FF & \text{if $J \subseteq J'$}, \\
        \{0\} & \text{otherwise}.
    \end{cases}
\]
Enumerating each $P_i$ in any order as $J^i_1, J^i_2, \dots, J^i_{|P_i|}$,
we obtain a chain
\begin{align*}
    \langle  \rangle 
    & \subsetneq{} \langle v^t_{J^t_1} \rangle 
    \subsetneq{} \langle v^t_{J^t_1}, v^t_{J^t_2} \rangle  
    \subsetneq{} \cdots 
    \subsetneq{} \langle v^t_{J^t_1}, v^t_{J^t_2}, \dots, v^t_{J^t_{|P_t|}} \rangle \\
   & \subsetneq{} \langle v^t_{J^t_1}, v^t_{J^t_2}, \dots, v^t_{J^t_{|P_t|}}, v^{t-1}_{J^{t-1}_1} \rangle 
     \subsetneq{} \cdots
\end{align*}
of length $|P_t| + |P_{t-1}| + \cdots + |P_1| = 2^{|I|}$.
Together with the upper bound from Proposition~\ref{prop:length-upper-bound}, we conclude:
\begin{theorem}
    $\len(\Lin_\FF(\mathcal{O}) = 2^{|I|}$.
\end{theorem}
Along the lines of Proposition~\ref{prop:reduction-to-one-orbit}, this generalises to multi-orbit sets:
\[
\len(\Lin_\FF(\mathcal{O}_1 \uplus \dots \uplus \mathcal{O}_n)) = 2^{|I_1|} + \dots + 2^{|I_n|}.
\]
