\section{All those equivariant subspaces}\label{sec:equivariant-subspaces}
We continue working in the setting of Section~\ref{sec:free-amalg} so that we can demonstrate an important application of Theorem~\ref{thm:cog-span-generally}:
we can enumerate all the equivariant subspaces of an orbit-finitely spanned vector space.

\begin{theorem}
    Assume Lemma~\ref{lem:cog-fresh-full} and Theorem~\ref{thm:cog-span-generally} about cogs.
    Given $d \in \{1, 2, \dots\}$, there exist a finite family of equivariant linear maps $\restriction_i : \Lin_\FF \A^d \to \Lin_{\FF^{n_i}} \cal O_i$ such that:
    \begin{enumerate}
        \item 
        every equivariant subspace of $\Lin_\FF \A^d$ is of the form
        \[
            \{v \in \Lin_\FF \A^d \mid \forall i, \forall a \in \cal{O}_i : v{\restriction_i}(a) \in \EE_i \}
        \]
        where $\EE_i \subseteq F^{n_i}$ are finite-dimensional subspaces;

        \item
        if $W \subseteq \Lin_\FF \A^d$ is an equivariant subspace, 
        then $W$ is precisely the space above with $\EE_i = \{w {\restriction_i}(b) \mid w \in W, b \in \cal O_i \}$.
    \end{enumerate}
    
\end{theorem}

\begin{corollary}
    The length of $Lin_\FF \cal O$, where $\cal O \subseteq \A^d$ is an ordered orbit, is precisely $2^d$.
\end{corollary}

We supply the full proofs in Section~\ref{sec:appendix-eqsubsp} of the appendix,
but we spend the rest of this section explaining the situation when $\A$ is the order atoms and $d = 2$.