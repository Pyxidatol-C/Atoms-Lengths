\section{All those equivariant subspaces}\label{sec:equivariant-subspaces}
We continue working in the setting of Section~\ref{sec:free-amalg} so that we can demonstrate an important application of Theorem~\ref{thm:cog-span-generally}:
not only do we know that an orbit-finitely spanned vector space has finite length,
but we can also enumerate all its equivariant subspaces by going through certain finite-dimensional subspaces.

\begin{theorem}
    Fix $d \in \{1, 2, \dots\}$. 
    There are equivariant linear maps $\restriction_i : \Lin_\FF \A^d \to \Lin_{\FF^{n_i}} \cal O_i$ such that every equivariant subspace of $\Lin_\FF \A^d$ is of the form
    \begin{equation}\label{eq:equiv-subsp}
        \{v \in \Lin_\FF \A^d \mid \forall i, \forall a \in \cal{O}_i : v{\restriction_i}(a) \in \EE_i \}
    \end{equation}
    where $\EE_i \subseteq F^{n_i}$ are finite-dimensional subspaces for $i = 0, 1, \dots, t$.
    More precisely: these spaces are defined, given an equivariant subspace $W \subseteq \Lin_\FF \A^d$, 
    by \[
        \EE_i = \{w {\restriction_i}(b) \mid w \in W, b \in \cal O_i \}.
    \]
\end{theorem}
\textcolor{red}{$\EE$ vs $\FF$ business} Assume Theorem~\ref{thm:cogs-arise-everywhere} and Theorem~\ref{thm:cog-span-generally} about cogs.

\textcolor{red}{Thm: These ideas are in some capacity present in \cite[Cor.~3.17]{Gray97}, \cite{HofmanRozycki_2022}, \cite{ghosh2023orbit}, and in Arka's notes. Cor: this tightens the bound in \cite{BFKM24}.}

\begin{corollary}
    The length of $Lin_\FF \cal O$, where $\cal O \subseteq \A^d$ is an ordered orbit, is precisely $2^d$.
\end{corollary}

\textcolor{red}{Solving orbit-finite system of equations}

We supply the full proofs in Section~\ref{sec:appendix-eqsubsp} of the appendix,
but we spend the rest of this section explaining the situation when $\A = (\Q, \leq)$ is the order atoms and $d = 2$.

First, let us name the equivariant maps $\restriction_1, \restriction_2, \restriction_3$ out of $\Lin_\FF \Q^2$.
They are:
\begin{align*}
    v {\restriction_1} (a < b) &= ( v(a, b),\ v(b, a) ); \\
    v {\restriction_2} (c) &= (\begin{aligned}
        &\sum_{{-} < c} v({-}, c), \sum_{c < {\sim}} v(c, {\sim}), v(c, c), \\
        &\sum_{{-} < c} v({-}, c), \sum_{c < {\sim}} v({\sim}, c)
    \end{aligned}); \\
    v {\restriction_3} () &= ( \sum_{{-} < {\sim}} v({-}, {\sim}), \sum_{{-} < {\sim}} v({\sim}, {-}), \sum_{=} v(=, =)).
\end{align*}
for $(a < b) \in \cal O_0$, $(c) \in \cal O_1$, and $() \in \cal O_2$.
The idea is that, for instance, $v {\restriction_1}(c)$ reveals local information about $v$ around $c \in \Q$.
If $v {\restriction_1}(c)$ is zero on the third coordinate for all $c$, 
then no tuple of the form $(c, c)$ appears in $v$;
nor will they appear in vectors obtained through atom automorphisms and linear combinations from $v$.

Another way to say this is, given an equivariant subspace $W \subseteq \Lin_\FF \Q^2$, the space in \eqref{eq:equiv-subsp} which we denote by $\widetilde W$ is a superspace of $W$.
But we will show that they are equal.

\begin{claim}
    $\begin{aligned}[t]
    &\widetilde{W} \cap \ker(\restriction_0) \cap \ker(\restriction_1) \cap \ker(\restriction_2) \\
    \subseteq{} &W \cap \ker(\restriction_0) \cap \ker(\restriction_1) \cap \ker(\restriction_2).
    \end{aligned}$
\end{claim}
\begin{proof}
    Suppose that $v \in \Lin_\FF \Q^2$ is sent to zero under both ${\restriction_1}$ and ${\restriction_2}$.
\end{proof}

\begin{claim}
    $\widetilde{W} \cap \ker(\restriction_0) \cap \ker(\restriction_1) 
    \subseteq W \cap \ker(\restriction_0) \cap \ker(\restriction_1)$.
\end{claim}
\begin{proof}
    cogs.
\end{proof}

\begin{claim}
    $\widetilde{W} \cap \ker(\restriction_0) \subseteq W \cap \ker(\restriction_0)$.
\end{claim}
\begin{proof}
    Lookalike -> project: (=cog -> =vector) -> IH
\end{proof}

\begin{claim}
    $\widetilde{W} \subseteq W$.
\end{claim}

\begin{claim}
    \begin{align*}
        V_0 &= \{0\} \\
        V_1 &= V_0 + \langle (5, 9) - (5.01, 9) - (5, 9.1) + (5.01, 9.1) \rangle \\
        V_2 &= V_1 + \langle (5, 9) - (5.01, 9) \rangle \\
        V_3 &= V_2 + \langle (5, 9) - (5, 9.1) \rangle \\
        V_4 &= V_3 + \langle (5, 9) \rangle        
    \end{align*}
\end{claim}