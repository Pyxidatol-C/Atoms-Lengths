\section{All those equivariant subspaces}\label{sec:equivariant-subspaces}
We continue working in the setting of Section~\ref{sec:free-amalg} so that we can demonstrate an important application of Theorem~\ref{thm:cog-span-generally}:
given an orbit-finite set $X$ over $\A$, we can find \textcolor{red}{a family of equivariant maps out of $\Lin_\FF X$, indexed by finite-dimensional vector spaces, whose kernels} exhaust all the equivariant subspaces of $\Lin_\FF X$.
In particular, we will show that for a single $S$-ordered orbit $\cal O \subseteq \A^I$, the length of $\Lin_\FF \cal O$ is precisely $2^{|I|}$.


\subsection{Local coefficients}
Fix an orbit-finite set $X$ over $\A$.
Then it is in an equivariant bijection with a finite disjoint union $\biguplus_k \cal O_k$ of equivariant ordered orbits $\cal O_k \subseteq \A^{I_k}$ \textcolor{red}{(since the local symmetries are rigid)}. 
Given $J \subseteq I_k$, we can project a tuple $a \in \cal O_k$ to the $J$ coordinates and then standardise the indices:
\[
    \cal O_k \to \cal O_k|^J \to \A^{|J|}.
\]
This process is equivariant, and the image of $\cal O_k$ is an ordered orbit $\cal Q \subseteq \A^{|J|}$.
For now, let us fix $\cal Q$.
We want to consider every $k$ and $J \subseteq I_k$ that yields the same image $\cal Q$ together:
let $\Jclass$ consist of these pairs $(J, k)$.
We can assemble the projection maps as follows:
\begin{align*}
    {\restriction} : \Lin_\FF X &\rightarrow \Lin_{\FF^\Jclass} \cal Q \\
    v {\restriction} (q)_{J, k} &= v_k |^J (q^J)
\end{align*}
where $v_k$ is the projection of $v$ onto $\Lin_\FF \cal O_k$ and $q^J$ is the reindexed version of $q \in \cal Q$.
(This map is equivariant and linear; note $\Lin_\EE \cal Q$ is a vector space over $\FF$ for any vector space $\EE$.)
In words, $v {\restriction}$ allows us to probe local information around $q$ about the coefficients in $v$.
An extremal example is when $\cal Q = \{()\}$,
in which case $v {\restriction} ()$ tells us what coefficients sum to in each of the $v_k$'s.
Also, given an equivariant subspace $V \subseteq \Lin_\FF X$, we obtain a finite-dimensinoal vector space 
\[
    V {\restriction} (\cal Q)
    = \{ v {\restriction} (q) : v \in V, q \in \cal Q \} \subseteq \FF^\Jclass.
\]
For a vector $w \in \Lin_\FF X$ to belong to $V$, a necessary condition is that $w {\restriction}$ must belong to $\Lin_{V {\restriction} (\cal Q)} \cal Q$.

Let us now consider all the orbits $\cal Q_i$ that arise from the $\sum_k 2^{|I_k|}$ projections $\cal O_k|^J$.
Given subspaces $E_i \subseteq \FF^{\Jclass_i}$, we can define 
\begin{equation}
    \{ v \in \Lin X : v{\restriction_i}(q) \in E_i \text{ for all $i$ and $q \in \cal Q_i$} \}.
\end{equation}
This is an equivariant subspace.

\begin{theorem}
    Every equivariant subspace $V \subseteq \Lin_\FF X$ is of this form.
    (Put $E_i = V {\restriction_i} (\cal Q_i)$.)
\end{theorem}


Using this we can solve orbit-finite systems of equations. 
By this we mean the decision procedure: given $\mathbf{A}, \mathbf{b}$, does there exist a solution to $\mathbf{A} \mathbf{x} = \mathbf{b}$?
Here the rows and columns are indexed by orbit-finite setes, the matrix is equivariant, and in every column there are only finitely many non-zero entries.
(The column-finite restriction does not lose generality: see Arka's justification in Lics'24. Also, equivariance can be relaxed to finitely supported: $(\Q, \leq, {=}c)$ interprets in the ordered atoms.)

\subsection{Lengths}
Consider now the case when $X = \cal O$ is a single ordered orbit in $\A^d$.

\begin{theorem}
    $\len(\Lin_\FF\mathcal{O}) = 2^d$.
\end{theorem}

