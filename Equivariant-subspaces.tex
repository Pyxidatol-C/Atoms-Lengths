\section{All those equivariant subspaces}\label{sec:equivariant-subspaces}
We continue working in the setting of Section~\ref{sec:free-amalg} so that we can demonstrate an important application of Theorem~\ref{thm:cog-span-generally}:
given an orbit-finite set $X$, we can describe every equivariant subspace $\Lin_\FF X$ through finite-dimensional vector spaces.
In particular, we will show that $\Lin_\FF \cal O$ has length $2^{|I|}$.

\subsection{Local coefficients}
First we set up some useful notation.
Consider the $2^{|I|}$ projected $S$-ordered orbits $\mathcal{O}|^{J}$ for $J \subseteq I$.
Suppose that
\[
    f : \mathcal{O}|^J \to \mathcal{O}|^{J'}
\]
is an $\Aut(\A)_{(S)}$-equivariant bijection.
Take any $a \in \mathcal{O}|^J$,
and enumerate its entries as $a_1 < \dots < a_{|J|}$.
Similarly, enumerate the entries of $f(a)$ as $b_1 < \dots < b_{|J'|}$.
Then $\{a_1, \dots, a_{|J|}\} = \{ b_1, \dots, b_{|J'|} \}$ because $\A$ has no algebraicity; 
since the orbits are ordered, we must have $|J| = |J'|$ and $a_1 = b_1, \dots, a_{|J|} = b_{|J'|}$.
That is, $f$ must be the obvious function that reindexes a $J$-tuple to a $J'$-tuple
--- hence we will write $a^{/J'}$ instead of $f(a)$, leaving $f$ implicit.

Now, let $\mathcal{Q}_{1} = \mathcal{O}|^{J_1}, \dots, \mathcal{Q}_{t} = \mathcal{O}|^{J_t}$ be distinct projected $S$-ordered orbits up to $\Aut(\A)_{(S)}$-equivariant bijections,
enumerated in such a way that $|J_1| \geq |J_2| \geq \dots \geq |J_t|$. (In particular, $J_1=I$ and $J_t=\emptyset$.)

%\begin{definition}
    For $i = 1, \dots, t$, let $\Jclass_i$ consist of all sets $J$ such that $\mathcal{O}|^J$ is in $\Aut(\A)_{(S)}$-equivariant bijection with $\mathcal{Q}_i$.
    Assemble all $|\Jclass_i|$ projections into a single map
    \[
        (-){\restriction_i} : \Lin_\FF\mathcal{O} \to \Lin_{\FF^{\Jclass_i}} \mathcal{Q}_i.
    \]
    To be more precise $v{\restriction_i}(a)$ is, for $a\in{\cal Q}_i$, the $\Jclass_i$-tuple whose $J$-th entry is $v|^J(a^{/J}) \in \FF$.
    It is straightforward to check that $(-){\restriction_i}$ is $\Aut(\A)_{(S)}$-equivariant and linear.
%\end{definition}

\begin{remark}\label{rem:FF-EE}
We must tread carefully here, as $\FF^{\Jclass_i}$ is not a field, so notation such as $\Lin_{\FF^{\Jclass_i}} \mathcal{O}$ may seem suspicious. However, $\FF^{\Jclass_i}$ is a vector space over $\FF$, and for any such space $\EE$ one can naturally see $\Lin_\EE\cal O$ and $\Ker_{\EE}\cal O$ as vector spaces over $\FF$. 

A little more care is needed when we want to define a vector in $\Lin_\EE \cal O$:
as $\lambda \in \EE$ is not a scalar in general, we need to understand
\[
    \lambda \cdot a
\]
as a formal expression.
Similarly, We redefine $\Cog_\EE\cal O$ to be spanned by formal expressions $\lambda\cdot(a^+ \between a^-)$ for $\lambda\in\EE$. 

But this changes little: in our proof of Theorem~\ref{thm:cog-span-generally} we only ever added or substracted vectors and cogs from one another, so that theorem holds even if $\EE$ is an arbitrary abelian group. This is not so for Theorem~\ref{thm:cogs-arise-everywhere}, whose proof involves scalar division, which is why in Claim~\ref{claim:cogs-arise-everywhere} below we need to prove a more subtle version of it. 
\end{remark}

Let $W \subseteq \Lin_\FF \mathcal{O}$ be an $\Aut(\A)_{(S)}$-equivariant subspace. Let $W {\restriction_i} (\mathcal{Q}_i) \subseteq \FF^{\Jclass_i}$ be the set of all $\Jclass_i$-tuples that can be obtained as $w {\restriction_i} (a)$ for some $w\in W$ and $a\in{\cal Q}_i$. This is a finite-dimensional vector space.
Now define $\widetilde W$, which consists of all vectors $v \in \Lin_\FF \mathcal{O}$ such that
\[
    v {\restriction_i} (\mathcal{Q}_i) \subseteq W {\restriction_i} (\mathcal{Q}_i) \qquad \text{for all } i=1,\ldots, t.
\]
Then $\widetilde W$ is an $\Aut(\A)_{(S)}$-equivariant subspace that contains $W$.
It turns out these two are equal:

\begin{lemma}\label{lem:coeff-approximation}
For every $i=0,\ldots, t$,
\[
        \widetilde W \cap \ker(\restriction_{i+1}) \cap \cdots \cap \ker(\restriction_t) 
        \subseteq W \cap \ker(\restriction_{i+1}) \cap \cdots \cap \ker(\restriction_t).
\]
    In particular $\widetilde W \subseteq W$ when $i = t$.
\end{lemma}


\paragraph{Lengths}
Let $W, W'$ be two $\Aut(\A)_{(S)}$-equivariant subspaces of $\Lin_\FF(\mathcal{O})$.
If we have $W {\restriction_i}(\mathcal{Q}_i) = W' {\restriction_i}(\mathcal{Q}_i)$ for all $i=1,\ldots, t$,
then $W = \widetilde W = \widetilde{W'} = W'$ by Lemma~\ref{lem:coeff-approximation}.
As a consquence:

\begin{proposition}\label{prop:length-upper-bound}
    Let $W_0 \subsetneq W_1 \subsetneq \cdots \subsetneq W_l$ be a chain of $\Aut(\A)_{(S)}$-equivariant subspaces of $\Lin_\FF(\mathcal{O})$.
    Then $l \leq 2^{|I|}$. 
\end{proposition}
\begin{proof}
    We obtain $t$ chains
    \begin{align*}
        W_0 {\restriction_1} (\mathcal{Q}_1) \subseteq W_1 {\restriction_1} (\mathcal{Q}_1) \subseteq \cdots \subseteq W_l {\restriction_1} (\mathcal{Q}_1) \subseteq \FF^{\Jclass_1}, \\
        W_0 {\restriction_2} (\mathcal{Q}_2) \subseteq W_1 {\restriction_2} (\mathcal{Q}_2) \subseteq \cdots \subseteq W_l {\restriction_2} (\mathcal{Q}_2) \subseteq \FF^{\Jclass_2}, \\
        \vdots \\
        W_0 {\restriction_t} (\mathcal{Q}_t) \subseteq W_1 {\restriction_t} (\mathcal{Q}_t) \subseteq \cdots \subseteq W_l {\restriction_t} (\mathcal{Q}_t) \subseteq \FF^{\Jclass_t}.
    \end{align*}
    At each of the $l$ steps, one of the $t$ containments must be strict.
    Hence $l \leq |\Jclass_1| + |\Jclass_2| + \dots + |\Jclass_t| = 2^{|I|}$. 
\end{proof}

It follows that every $\Aut(\A)_{(S)}$-equivariant subspace of $\Lin_\FF(\mathcal{O})$ is finitely generated.
We can compute the local coefficients of such subspaces easily:
\begin{remark}
    For $v \in \Lin_\FF(\mathcal{O})$, 
    let $\langle v \rangle$ denote the $\Aut(\A)_{(S)}$-equivariant subspace it generates.
    Then:
    \begin{enumerate}
        \item 
        $\langle v \rangle {\restriction_i}(\mathcal{Q}_i)$ is the subspace of $\FF^{\Jclass_i}$ generated by vectors of the form $v {\restriction_i} (a)$,
        which is zero unless every atom appearing in $a$ appears in $v$
        --- there are only finitely many such $a$'s;

        \item 
        $\langle v, v' \rangle {\restriction_i}(\mathcal{Q}_i) = \langle v \rangle {\restriction_i}(\mathcal{Q}_i) + \langle v' \rangle {\restriction_i}(\mathcal{Q}_i)$.
    \end{enumerate}
\end{remark}

To complement the upper bound from Prop.~\ref{prop:length-upper-bound}, we now exhibit a chain of $\Aut(\A)_{(S)}$-equivariant subspaces whose length is precisely $\sum_{i=1}^t |\Jclass_i|$, generalising \cite[Corollary~4.12]{BFKM24}.
Take any $J \in \Jclass_i$. 
Pick some $a\in \mathcal{O}$,
and let $\pi_j$ (for $j \in I$) be the automorphisms from Prop.~\ref{claim:cog-fresh-full}.
Define a vector
\begin{align*}
    v_{J} =  \prod_{j \in J} (1 - \pi_j) a \in \Lin_\FF(\mathcal{O}).
\end{align*}
Then 
\[
    (\langle v_{J} \rangle {\restriction_{i'}} (\mathcal{Q}_{i'}))_{J'} =
    \begin{cases}
        \FF & \text{if $J \subseteq J'$}, \\
        \{0\} & \text{otherwise}.
    \end{cases}
\]
Enumerating each $\Jclass_i$ in any order as $J_i^1, J_i^2, \dots, J_i^{|\Jclass_i|}$,
we obtain a chain
\begin{align*}
    \langle  \rangle 
    & \subsetneq{} \langle v_{J_t^1} \rangle 
    \subsetneq{} \langle v_{J_t^1}, v_{J_t^2} \rangle  
    \subsetneq{} \cdots 
    \subsetneq{} \langle v_{J_t^1}, v_{J_t^2}, \dots, v_{J_t^{|\Jclass_t|}} \rangle \\
   & \subsetneq{} \langle v_{J_t^1}, v_{J_t^2}, \dots, v_{J_t^{|\Jclass_t|}}, v_{J_{t-1}^1} \rangle 
     \subsetneq{} \cdots
\end{align*}
of length $|\Jclass_t| + |\Jclass_{t-1}| + \cdots + |\Jclass_1| = 2^{|I|}$.
Together with the upper bound from Proposition~\ref{prop:length-upper-bound}, we conclude:
\begin{theorem}
    $\len(\Lin_\FF\mathcal{O}) = 2^{|I|}$.
\end{theorem}
Along the lines of Proposition~\ref{claim:reduction-to-one-orbit}, this generalises to multi-orbit sets:
\[
\len(\Lin_\FF(\mathcal{O}_1 \uplus \dots \uplus \mathcal{O}_n)) = 2^{|I_1|} + \dots + 2^{|I_n|}.
\]
