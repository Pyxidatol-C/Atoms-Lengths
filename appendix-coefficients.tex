\begin{proof}
We proceed by induction on $i$.
The base case $i = 0$ is trivial, since $\ker(\restriction_1) = \{0\}$. Indeed, $\Jclass_1=\{I\}$, so $v{\restriction_1}$ is the identity map.

To prove the containment for some $i>0$, 
we allow $v {\restriction_i}$ to be non-zero.
But $v {\restriction_i}$ satisfies the next best property:
\begin{claim}
  The image of \[
        \widetilde W \cap \ker(\restriction_{i+1}) \cap \cdots \cap \ker(\restriction_t)
    \] under $\restriction_{i}$ 
    is contained in $\Ker_{W {\restriction}_{i} (\mathcal{Q}_{i})} \mathcal{Q}_{i}$. 
\end{claim}
\begin{proof}
    Take any $v \in \widetilde W$ such that  $v {\restriction_{i'}} = 0$ for all $i'>i$.
    That $v {\restriction_{i}} \in \Lin_{W{\restriction_{i}}(\mathcal{Q}_{i})} \mathcal{Q}_{i}$ is clear from the definition of $\widetilde W$.
    Recall that $\mathcal{Q}_{i}$ = $\mathcal{O}|^{J_{i}}$.
    For every $j \in J_{i}$,
    we need to prove that $v {\restriction_{i}} |^{-j} = 0$. In more elementary terms, given any $a\in\mathcal{O}|^{J_{i}\setminus\{j\}}$, we need the $J$-th entry of $v {\restriction_{i}} |^{-j}(a)$ to be $0$, for every $J\in \Jclass_i$.
    
    So take any $J \in \Jclass_{i}$.
    The unique ordered bijection between $J_{i}$ and $J$ restricts to one between $J_{i} \setminus \{j\}$ and $J \setminus \{j'\}$, for some $j'\in J$. Denote $J'=J \setminus \{j'\}$.
    Then $J'$ belongs to some $\Jclass_{i'}$ with $i' > i $,
    so $v {\restriction_{i'}} = 0$. Now calculate (with $a\in\mathcal{O}|^{J_{i}\setminus\{j\}}$, $b\in{\cal Q}_i$, $c\in {\cal O}|^J$ and $d\in\cal O$):
\begin{align*}
	(v{\restriction_i}|^{-j}(a))_J &= \sum_{b|^{-j}=a}(v{\restriction_i}(b))_J
	= \sum_{b|^{-j}=a}v|^J(b^{/J}) 
	= \sum_{c|^{-j'}=a^{/J'}}v|^J(c) \\
	&= \sum_{d|^{J'}=a^{/J'}}v(d)
	\,\, = v|^{J'}(a^{/J'}) = (v{\restriction_{i'}}(a))_{J'} = 0.
    \qedhere
\end{align*}    
\end{proof}

In light of Remark~\ref{rem:FF-EE}, from Theorem~\ref{thm:cog-span-generally} we get: 
\[\Ker_{W {\restriction_{i}} (\mathcal{Q}_{i})} \mathcal{Q}_{i} \subseteq \Cog_{W {\restriction_{i}} (\mathcal{Q}_{i})} \mathcal{Q}_{i}.\]
Furthermore:
\begin{claim}\label{claim:cogs-arise-everywhere}
    $\Cog_{W {\restriction}_{i} (\mathcal{Q}_{i})} \mathcal{Q}_{i}$ is contained in
    the image of \[
        W \cap \ker(\restriction_{i+1}) \cap \cdots \cap \ker(\restriction_t)
    \] under $\restriction_{i}$. 
\end{claim}
\begin{proof}
    Consider any ${\cal Q}_i$-cog with a coefficient $\lambda\in W {\restriction_{i}} (\mathcal{Q}_{i})$, and let $w\in W$ and $a\in{\cal Q}_i$ be such that $\lambda = w{\restriction_{i}}(a)$.
    Let $S'$ consist of $S$ together with every atom appearing in $w$ but not in $a$.
    We generalise the construction used in the proof of Theorem~\ref{thm:cogs-arise-everywhere}.
    
    Apply Proposition~\ref{claim:cog-fresh-full} and Remark~\ref{rem:duo} to get automorphisms $\pi_j$ for $j \in J_i$ such that $a \parallel \prod_{j \in J_i} \pi_j a$ is an $\mathcal{Q}_i$-duo, 
    where each $\pi_j$ fixes $S'$ and all $a_{j'}$ and $\pi_{j'}(a_{j'})$ for $j'\neq j$. Since all $\mathcal{Q}_i$-duos are in the same orbit, it is enough to show that the cog corresponding to this particular duo, with the coefficient $\lambda$, belongs to the $\restriction_i$-image as in the statement of the claim.
    
    
    Put: \[
        w' = \prod_{j \in J_{i}} (\mathrm{id} - \pi_j) w \in W.
    \]
    For any $1 \leq i' \leq t$, noting that as no more atoms can appear in $w{\restriction_{i'}}$ than in $w$, we have
    \begin{align*}
        &w' {\restriction_{i'}}
        = \prod_{j \in J_{i}} (\mathrm{id} - \pi_j) w {\restriction_{i'}} 
        = \sum_{c \in C_i} \sum_{J' \subseteq J_i} (-1)^{|J'|}  
            w {\restriction_{i'}}(c) \cdot \left(\prod_{j \in J'} \pi_j c\right)
%        &{}= \sum_{b_\bullet \in \mathcal{Q}_{i'}, \{b_j \mid j\} \supseteq \{a_j \mid j\}} \sum_{J' \subseteq J} (-1)^{|J'|}  
%            w {\restriction_{i'}}(b_\bullet) \prod_{j \in J'} \pi_j \cdot b_\bullet.
    \end{align*}
    where
        \[C_i = \{ c\in{\cal Q}_{i'} : \{c_j\mid j\in J_{i'}\} \supseteq \{a_j \mid j\in J_i\} \}.
    \]
    (The formula used in the proof of Theorem~\ref{thm:cogs-arise-everywhere} is a special case of this for $i'=1$ so that $J_{i'}=I$ and ${\cal Q}_{i'}={\cal O}$.) Now, if $i'>i$ then $C_i$ is empty, and so $w' {\restriction_{i+1}} = \cdots = w' {\restriction_{t}} = 0$. Moreover, if $i'=i$ then $C_i = \{a\}$ and we obtain the cog from before:
    \[
        w' {\restriction_{i}} = \lambda \cdot \left(a \between \prod_{j \in J_{i}} \pi_j a\right).
    \]
So $w'$ is a witness for the inclusion from the claim.
\end{proof}

This is enough to establish Lemma~\ref{lem:coeff-approximation} for $i$, assuming it for $i-1$.
Indeed, given $v \in \widetilde W \cap \ker(\restriction_{i+1}) \cap \dots \cap \ker(\restriction_t)$, by the preceding claims
we can find $w \in W \cap \ker(\restriction_{i+1}) \cap \dots \cap \ker(\restriction_t) \subseteq \widetilde W$ such that 
$v {\restriction_i} = w {\restriction_i}$.
But then $(v - w) {\restriction_{i}} = 0$, so $v - w$ lies in $\ker(\restriction_{i})$ as well as $\widetilde W \cap \ker(\restriction_{i+1}) \cap \cdots \cap \ker(\restriction_t)$.
It follows from the inductive hypothesis that \[
    v - w \in W \cap \ker(\restriction_i) \cap \ker(\restriction_{i+1}) \cap \dots \cap \ker(\restriction_t),
\]
so $v = (v - w) + w$ is in $W \cap \ker(\restriction_{i+1}) \cap \dots \cap \ker(\restriction_t)$ as well.

This completes the proof of Lemma~\ref{lem:coeff-approximation}.
\end{proof}