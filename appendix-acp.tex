\section{Proof of Theorem~\ref{thm:ACP}}

\begin{proof}
    The equivalence of (3) and (1) is a classical result in module theory. Let us recall the proof. For (3) $\Rightarrow$ (1), we build an orbit-finite spanning set for the subspace greedily, by adding orbit after orbit of vectors. Thanks to (3), this must stabilise. For  (1) $\Rightarrow$ (3), we use (1) to get an orbit-finite spanning set for the union of the chain from (3), and then we observe that orbits from this set can only be added finitely often in the chain. 

    Let us now prove the ``furthermore'' part. Take some orbit-finitely spanned vector space $V$.  By Definition~\ref{def:orbit-finite-set},  the spanning set can be obtained from some equivariant subset of $\A^d$, by quotienting under an equivariant equivalence relation. This  gives us a surjective equivariant linear map to $V$ from an equivariant subspace of  $\Lin_\field \A^d$, thus establishing the ``furthermore'' part.

    From the ``furthermore'' part we also get equivalence of (2) and (3). The implication (3) $\Rightarrow$ (2) is immediate, and the converse implication follows from the ``furthermore'' part, since the lack of infinite ascending chains is preserved by taking equivariant subspaces and images under equivariant linear maps. 
\end{proof}