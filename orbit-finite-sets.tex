\section{Orbit-finite sets}
\label{sec:orbit-finite-sets}
As mentioned in the introduction,
the results in this paper are motivated by the research program of orbit-finite sets. This program is described in the lecture notes~\cite{bojanczyk_slightly}, which we have already used above for references. In this section, we briefly explain what orbit-finite sets are. 

The general idea is that we start with some infinite structure $\A$. We think of elements of this structure as \emph{atoms}, which can be used to construct sets, which are finite up to symmetries of the structure, such as
\begin{align*}
\myunderbrace{\A^2}{pairs}
\qquad 
\myunderbrace{\setbuild{ (a,b) \in \A^2}{$a \neq b$}}{non-repeating pairs}
\qquad 
\myunderbrace{\A \choose 2}{unordered pairs}.
\end{align*}
Since we want to be able to construct tuples of arbitrary finite length, and we want them to be finite up to symmetries, we will need to assume that the underlying structure of atoms is oligomorphic. When constructing such sets, we will proceed in three steps: take a finite power of the atoms (as in the first example above), restrict this power to an equivariant subset (as in the second example above), and then take a quotient under an equivariant equivalence relation (as in the third example above, where the equivalence relation identifies two pairs if they differ only in their order). The formal definition is given below (as we explain below, it is the same as a first-order interpretation in the structure $\A$). 

 \begin{definition}[Orbit-finite set]\label{def:orbit-finite-set}
    An \emph{orbit-finite set} over an oligomorphic structure $\A$ is any set that is obtained as follows: 
    \begin{enumerate}
        \item Start with a finite power $\A^d$ for some $d \in \set{0,1,\ldots}$.
        \item Restrict it to an equivariant subset $X \subseteq \A^d$.
        \item Quotient  $X$ under an equivariant equivalence relation. 
    \end{enumerate}
 \end{definition}


 The above definition describes only a set, but this set is also equipped with some structure.
 We think of it as being equipped with the action of the automorphism group of the original structure $\A$, which is inherited from $\A^d$, suitably extended to the quotient. Under this action, the set has finitely many orbits, since $\A^d$ has finitely many orbits, and  the number of orbits can only go down when  restricting to an equivariant subset and quotienting under an equivariant equivalence relation. This explains the name \emph{orbit-finite set}.  
 
  Since we assume the atoms to be oligomorphic, the equivariant relations are exactly those that can be defined in first-order logic. Therefore, the above definition is the same as a first-order interpretation in $\A$, see~\cite[Section 5.3]{hodges1993model}.
Although  definition~\ref{def:orbit-finite-set} is phrased in terms of first-order interpretations,  there are other equivalent descriptions of orbit-finite sets. One of these, see~\cite[Theorem 5.13]{bojanczyk_slightly}, is that an orbit-finite set is a set  equipped with an action of the automorphism group of $\A$, such that: (a) there are finitely many orbits; and (b) every element has finite support.

 \paragraph*{Orbit-finite automata.} The study of orbit-finite sets was originally motivated by automata and regular languages over infinite alphabets~\cite{bojanczykNominalMonoids2013,bojanczykAutomataTheoryNominal2014}. The idea is to use standard models of computation, but to replace finite sets with orbit-finite ones, while keeping all structure equivariant. The standard example is orbit-finite automata, defined as follows.
    An orbit-finite nondeterministic automaton over atoms $\A$ is defined like a nondeterministic finite automaton, except that the states and alphabet are orbit-finite sets over $\A$, instead of finite ones, and all structure (the sets of initial and final states and the transition relation) are equivariant. A deterministic orbit-finite automaton is the special case which has only one initial state, and where the transition relation is a function. 
 \begin{example}[Cycles in the Rado graph]
    Assume that the atoms are the Rado graph. The cycles in the Rado graph can be viewed as a language $L \subseteq \A^*$, which consists of words where every two consecutive letters are related by an edge in the atoms, and furthermore there is an edge between the last and the first letter. This language can be recognised by a deterministic  orbit-finite automaton, which uses its states to remember the first letter of the input word as well as the most recently read letter. The state space of this automaton is the disjoint union
    \begin{align*}
      \myunderbrace{\set{\varepsilon}}{initial \\ state}
      \quad + \quad 
      \myunderbrace{\A^2}{first letter and \\ most recent letter}
        \quad + \quad
        \myunderbrace{\set{\bot}}{rejecting \\ sink state}.
    \end{align*}
    This is an orbit-finite set. This is because the initial and sink states can be seen as two copies of $\A^0$, and  orbit-finite sets are closed under disjoint unions, assuming that the atom structure has at least two elements.
 \end{example}