\section{Structures}\label{sec:structures}

In this section, we briefly recall some basic notions from model theory and describe the main examples of structures that we will consider in this paper.

Let us begin by fixing some notation.
A \emph{vocabulary} is a set of relations, each with a specified arity (we do not use functions in this paper). For example, the vocabulary of graphs will contain one binary relation: the edge relation. (We do not include equality in the vocabulary, since it will be automatically present in all structures.) A \emph{structure} $\A$ over a vocabulary consists of an underlying set, also denoted by $\A$, together with intepretations of relations from the vocabulary as actual relations on that set. An \emph{isomorphism} between two structures over the same vocabulary is a bijection between their underlying sets that preserves and reflects all relations. An \emph{automorphism} of a structure is an isomorphism from the structure to itself; these form a group. 

Automorphisms  of $\A$ act on  tuples in $\A^d$ componentwise.  When we speak of orbits in $\A^d$, we mean the orbits under this action of the automorphism group. For example, if $\A$ is a graph, then two pairs $(a_1,a_2)$ and $(b_1,b_2)$ are in the same orbit if and only if there is some automorphism of the graph that maps $a_1$ to $b_1$ and $a_2$ to $b_2$. In particular, the edge relation must be defined in the same way for both pairs.

All structures considered in this paper will be countable (i.e., \textcolor{blue}{countably infinite}), and we will always want them to have finitely many orbits in every finite dimension, as  in the following definition.

\begin{definition}[Oligomorphic structure]\label{def:oligomorphic-structure}
    A structure $\A$ is \emph{oligomorphic} if $\A^d$ has finitely many orbits  for every $d \in \set{1,2,\ldots}$.
\end{definition}

A relation $R \subseteq \A^d$ on a structure is called \emph{equivariant} if it is invariant under the action of the automorphism group. Equivalently, the relation is a union of orbits.  If the structure is oligomorphic, then there are finitely many orbits to consider once the dimension $d$ is fixed, and therefore  only finitely many equivariant relations. By the Ryll-Nardzewski theorem~\cite[Thm.~7.3.1]{hodges1993model}, if the structure is oligomorphic and countable, then the equivariant relations are exactly those that can be defined in first-order logic --- see for instance \cite[Lem.~5.9]{bojanczyk_slightly}. In fact, the infinite structures that we consider in this paper will satisfy a stronger property: the equivariant relations will be definable not only by first-order formulas, but even by quantifier-free ones. This will be ensured by the additional homogeneity condition defined below. In the condition, a \emph{substructure} of $\A$ is any structure obtained by restricting $\A$ to some subset  of its underlying set; we do not distinguish between substructures and subsets.

\begin{definition}
    [Homogeneous structure] A structure $\A$ is \emph{homogeneous} if every isomorphism between finite substructures of $\A$ extends to an automorphism of $\A$.
\end{definition}

In a homogeneous structure $\A$, an orbit in $\A^d$ consists of tuples that satisfy the same quantifier-free formulas --- see for instance \cite[Thm.~6.3]{bojanczyk_slightly}. 
Every homogeneous structure arises via a construction called the Fraïssé limit~\cite[Sec.~7.4]{hodges1993model}, from classes of finite structures that satisfy certain closure properties (the so-called Fraïssé/amalgamation classes). 
For the Fraïssé limit to be not only homogeneous, but also oligomorphic, we need to assume that the underlying class --- which, up to isomorphism, coincides with the finite substructures of the Fraïssé limit --- has only finitely many non-isomorphic structures of each finite size. 

We will be \textcolor{red}{mainly} interested in countable structures that are both oligomorphic and homogeneous, possibly over an infinite vocabulary. 
Here are some of the important examples that we will consider in this paper.
\begin{example}[Equality only] \label{ex:equality-atoms} In this structure, the underlying set is \textcolor{blue}{countably infinite} and there are no relations other than equality.  Automorphisms are arbitrary permutations, and two tuples are in the same orbit if and only if they have the same equality pattern. Since there are finitely many equality patterns for tuples of fixed length, this structure is oligomorphic. For example, in dimension $d=2$ there are two orbits: $x_1=x_2$ and $x_1 \neq x_2$.
\end{example}

\begin{example}[Order] \label{ex:order-atoms} In this structure, the underlying set is the set of rational numbers, equipped with the usual order --- 
the vocabulary consists of this binary relation only.
Automorphisms are order-preserving permutations, and two tuples are in the same orbit if and only if they have the same order pattern. Since there are finitely many order patterns for tuples of fixed length, this structure is oligomorphic. 
In dimension $d = 2$ there are three orbits: 
$x_1 < x_2$, $x_1 = x_2$, and $x_1 > x_2$.
\end{example}

\begin{example}
    [Vector space] \label{ex:vector-space-atoms} Fix some finite field $\innerfield$, and let $\A$ be the vector space of \textcolor{blue}{countably infinite} dimension over $\innerfield$. This vector space is seen as a structure over an infinite vocabulary, which contains a relation 
    \begin{align*}
    \setbuild{(a_1,\ldots,a_d) \in \A^d}{ $\lambda_1 a_1 + \cdots + \lambda_d a_d = 0$ }
    \end{align*}
    for every $d \in \set{1,2,\ldots}$ and all coefficients $\lambda_1, \ldots, \lambda_d$ in $\innerfield$. The vocabulary is defined so that automorphisms are the same thing as permutations that are linear maps. Two tuples are in the same orbit if and only if they have the same linear dependencies. Over a finite field there are finitely many linear dependency patterns for tuples of fixed length, so this structure is oligomorphic.
    In particular, in dimension $d = 2$, there are three more orbits than there are elements in the field: 
    $x_1 = 0 = x_2$;
    $x_1 = 0 \neq x_2$;
    $0 \neq x_1 = \lambda \cdot x_2$, where $\lambda$ ranges over the field;
    and $x_1, x_2$ being linearly independent. 
    
    This particular example can lead to some confusion, since here $\A$ is itself a vector space, and we will later on consider the vector space (possibly over a different field) with $\A$ as a basis. %Hence, there will be two vector space structures under consideration. 
\end{example}

\begin{example}[Rado graph]\label{ex:rado-graph} The Rado graph is the Fraïssé limit of the class of finite undirected graphs. Here, an undirected graph is viewed as a structure that has one binary relation that is symmetric and irreflexive. 
%An explicit description of the Rado graph is as follows: its vertices are natural numbers, and there is an edge between numbers $n < m$ if the binary representation of $m$ has bit $1$ on the $n$-th digit.    
A famous characterisation of the Rado graph is that if one randomly selects a graph with a \textcolor{blue}{countably infinite} set of vertices by independently including each possible edge with probability $1/2$, then with probability $1$ the resulting graph is isomorphic to the Rado graph. By homogeneity, two tuples in the Rado graph  are in the same orbit if and only if they have the same equality and adjacency patterns, hence there are finitely many orbits in every dimension.
For instance, in $d = 2$ there are three orbits:
the two coordinates can be equal, adjacent hence distinct, or distinct but non-adjacent.
\end{example}
