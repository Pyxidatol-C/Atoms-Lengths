\section{Structures}

We briefly recall some basic notions from model theory, mainly in order to fix notation.
A language is a set of relation and function symbols, each with a specified arity. A structure $\A$ over a language consists of an underlying set, together with interpretations of the symbols in the language as relations and functions on the underlying set. An automorphism of a structure $\A$ is a bijection from the underlying set of $\A$ to itself that preserves all relations and functions of $\A$. When we talk about orbits in $\A^d$, for some power $d \in \set{1,2,\ldots}$, we mean the orbits under the component-wise action of the automorphism group. In other words, two tuples $(a_1,\ldots,a_d)$ and $(b_1,\ldots,b_d)$ are in the same orbit if there exists an automorphism $\pi$ of $\A$ such that $\pi(a_i) = b_i$ for all $i \in \set{1,\ldots,d}$. 

\begin{definition}[Oligomorphic structure]\label{def:oligomorphic-structure}
    A structure $\A$ is \emph{oligomorphic} if $\A^d$ has finitely many orbits  for every $d \in \set{1,2,\ldots}$.
\end{definition}

\begin{definition}
    [Homogeneous structure] A structure $\A$ is \emph{homogeneous} if every isomorphism between finite substructures of $\A$ extends to an automorphism of $\A$.
\end{definition}


We will be mainly interested in countable structures that are both oligomorphic and homogeneous.  Every homogeneous structure arises via construction called the Fraïssé limit, from classes of finite structures that satisfy certain closure properties (the so-called Fraïssé classes). For the Fraïssé limit to be not only homogeneous, but also oligomorphic, we need to assume that the Fraïssé class has only finitely many non-isomorphic structures of each finite size. Here are some of the important examples of oligomorphic structures that we will consider in this paper. These  are Fraïssé limits of the following classes of finite structures: (a) sets with equality only; (b) linear orders; (c) vector spaces over a finite field; and (d) graphs.
\begin{example}[Equality only] \label{ex:equality-atoms} In this structure, the underlying set is countably infinite and there are no relations or functions.  Automorphisms are arbitrary permutations, and two tuples are in the same orbit if and only if they have the same equality pattern. Since there are finitely many equality patterns for tuples of fixed length, this structure is oligomorphic. For example, in dimension $d=2$ there are two orbits: $x_1=x_2$ and $x_1 \neq x_2$.
\end{example}

\begin{example}[Order] \label{ex:order-atoms} In this structure, the underlying set is the set of rational numbers, equipped with the usual order. Automorphisms are order-preserving permutations, and two tuples are in the same orbit if and only if they have the same order pattern. Since there are finitely many order patterns for tuples of fixed length, this structure is oligomorphic. 
\end{example}

\begin{example}
    [Vector space] \label{ex:vector-space-atoms} Fix some finite field, and consider the vector space of countably infinite dimension over this field. This vector space can be seen as a structure, with  functions for vector addition and scalar multiplication. Automorphisms are permutations that are linear maps, and two tuples are in the same orbit if and only if they have the same linear dependencies. Since there are finitely many linear dependency patterns for tuples of fixed length over a finite field, this structure is oligomorphic.
\end{example}

\begin{example}[Rado graph]\label{ex:rado-graph} The Rado graph is the Fraïssé limit of the class of finite undirected graphs. Here, an undirected graph is viewed as a structure that has a binary relation, which is symmetric and irreflexive. It  is not so easy to describe the Rado graph  explicitly, but one of its characterisations is that if one randomly selects a graph with a countably infinite set of vertices, by independently including each possible edge with probability $1/2$, then with probability $1$ the resulting graph is isomorphic to the Rado graph. Two tuples are in the same orbit if and only if they have the same equality and adjacency patterns, hence there are finitely many orbits in every dimension.
\end{example}




% \color{red}
% Definitions: 
% \begin{enumerate}
%     \item oligomorphic
%     \item homogeneous
%     \item smooth approximation by homogeneous substructures \cite{KLM89} (N.B. not the 'smooth approximation' from \cite[Definitino~4]{MP24})
%     \item \emph{oligomorphic approximation} of a homogeneous structure by finite substructures with uniformly few orbits (i.e., types) that cover the age of $\A$
% \end{enumerate}
