\documentclass[sigconf, review, natbib=false, anonymous=true]{acmart}



%% Rights management information.  This information is sent to you
%% when you complete the rights form.  These commands have SAMPLE
%% values in them; it is your responsibility as an author to replace
%% the commands and values with those provided to you when you
%% complete the rights form.
\setcopyright{acmlicensed}
\copyrightyear{2026}
\acmYear{2026}
\acmDOI{}
%% These commands are for a PROCEEDINGS abstract or paper.
\acmConference[LICS'26]{}{}{}
%%
%%  Uncomment \acmBooktitle if the title of the proceedings is different
%%  from ``Proceedings of ...''!
%%
%%\acmBooktitle{Woodstock '18: ACM Symposium on Neural Gaze Detection,
%%  June 03--05, 2018, Woodstock, NY}
\acmISBN{}


%%
%% Submission ID.
%% Use this when submitting an article to a sponsored event. You'll
%% receive a unique submission ID from the organizers
%% of the event, and this ID should be used as the parameter to this command.
%%\acmSubmissionID{123-A56-BU3}

%%
%% For managing citations, it is recommended to use bibliography
%% files in BibTeX format.
%%
%% You can then either use BibTeX with the ACM-Reference-Format style,
%% or BibLaTeX with the acmnumeric or acmauthoryear sytles, that include
%% support for advanced citation of software artefact from the
%% biblatex-software package, also separately available on CTAN.
%%
%% Look at the sample-*-biblatex.tex files for templates showcasing
%% the biblatex styles.
%%


%%
%% The majority of ACM publications use numbered citations and
%% references, obtained by selecting the acmnumeric BibLaTeX style.
%% The acmauthoryear BibLaTeX style switches to the "author year" style.
%%
%% If you are preparing content for an event
%% sponsored by ACM SIGGRAPH, you must use the acmauthoryear style of
%% citations and references.
%%
%% Bibliography style
\RequirePackage[
  datamodel=acmdatamodel,
  style=alphabetic,
  ]{biblatex}

%% Declare bibliography sources (one \addbibresource command per source)
\addbibresource{atoms.bib}


\usepackage[shortlabels]{enumitem}
\renewcommand{\labelitemi}{-}

\usepackage{mathtools}
\usepackage{stmaryrd}

\usepackage{cancel}

\let\Bbbk\relax
\usepackage{quiver}

\usepackage{tikz}
\usetikzlibrary{automata,calc,backgrounds,arrows.meta,decorations.pathmorphing,positioning,automata}

\usepackage{amsthm}

\usepackage[all]{xy}

\usepackage{booktabs}

% \numberwithin{equation}{section}\newcommand{\COUNTER}{equation}\newtheorem{theorem}[\COUNTER]{Theorem}
\theoremstyle{plain}
\newtheorem{theorem}{Theorem}[section]\newcommand{\COUNTER}{theorem}
\newtheorem{lemma}[\COUNTER]{Lemma}
\newtheorem{proposition}[\COUNTER]{Proposition}
\newtheorem{corollary}[\COUNTER]{Corollary}
\newtheorem{claim}[\COUNTER]{Claim}

\theoremstyle{definition}
\newtheorem{definition}[\COUNTER]{Definition}
\newtheorem{conjecture}[\COUNTER]{Conjecture}
\newtheorem{example}[\COUNTER]{Example}
\newtheorem{non-example}[\COUNTER]{Non-example}

\newtheorem{remark}[\COUNTER]{Remark}


\newcommand{\Bclass}{\mathscr{B}}
\newcommand{\Cclass}{\mathscr{C}}
\newcommand{\Fclass}{\mathscr{F}}
\newcommand{\Jclass}{\mathscr{J}}

\newcommand{\A}{\mathbb{A}}
\newcommand{\B}{\mathbb{B}}
\newcommand{\N}{\mathbb{N}} % naturals / names / equality atoms
\newcommand{\Q}{\mathbb{Q}} % rationals / ordered atoms
\newcommand{\G}{\mathbb{G}} % Rado graph 
\newcommand{\V}{\mathbb{V}} % bit vectors
\newcommand{\W}{\mathbb{W}} % symplectic --- w for omega, or « double v » for the vector pair e, f

\newcommand{\FF}{\mathsf{F}} % field
\newcommand{\ff}{\mathsf{f}} % finite field
\newcommand{\EE}{\mathsf{E}} % finite-dimensional vector space

\newcommand{\Aut}{\operatorname{Aut}}
\newcommand{\Lin}{\operatorname{Lin}}
\newcommand{\len}{\operatorname{len}}
\newcommand{\Ker}{\operatorname{Ker}}
\newcommand{\Cog}{\operatorname{Cog}}
\newcommand{\Forb}{\operatorname{Forb}}

\newcommand{\vsup}[1]{[#1]}

% a field
\newcommand{\field}{\mathsf{k}}

% a set 
\newcommand{\set}[1]{\{#1\}}

% set builder expression, which can be multiline
\newcommand{\setbuild}[2]{\set{#1 \ | 
\begin{tabular}{l}
	#2
\end{tabular}}}


\newcommand{\myunderbrace}[2]{\underbrace{#1}_{\mathclap{\text{\footnotesize 
\begin{tabular}{c}
	#2
\end{tabular} }}}}

% various kinds of function spaces
\newcommand{\lineqfun}[2]{ #1 \underset{\textup{lineq}}{\longrightarrow} #2}
\newcommand{\linfsfun}[2]{ #1 \underset{\textup{linfs}}{\longrightarrow} #2}
\newcommand{\eqfun}[2]{ #1 \underset{\textup{eq}}{\longrightarrow} #2}
\newcommand{\fsfun}[2]{ #1 \underset{\textup{fs}}{\longrightarrow} #2}



\begin{document}

%%
%% The "title" command has an optional parameter,
%% allowing the author to define a "short title" to be used in page headers.
\title{The Finite Length Property of the Rado Graph and Friends}

\author{Jingjie Yang}
\affiliation{%
  \institution{University of Oxford}
  \city{}
  \country{UK}
}

\author{Mikołaj Bojańczyk}
\affiliation{%
  \institution{University of Warsaw}
  \city{}
  \country{Poland}
}

\author{Bartek Klin}
\affiliation{%
  \institution{University of Oxford}
  \city{}
  \country{UK}
}

%%
%% By default, the full list of authors will be used in the page
%% headers. Often, this list is too long, and will overlap
%% other information printed in the page headers. This command allows
%% the author to define a more concise list
%% of authors' names for this purpose.
% \renewcommand{\shortauthors}{Trovato et al.}

%%
%% The abstract is a short summary of the work to be presented in the
%% article.
\begin{abstract}
  An infinite structure has the finite length property (over a given field) if, for each of its finite powers, strict chains of equivariant subspaces in the corresponding free vector space are bounded in length. Prior work showed the countable pure set and the dense linear order without endpoints have this property, where it plays a central role in the theory of orbit-finitely spanned vector spaces. We generalise these two results to (a) structures approximated by finite substructures with few orbits, provided the field is of characteristic zero, and (b) generically ordered expansions of Fraïssé limits with free amalgamation, in vocabularies consisting of unary and binary relations. As a special case, we deduce the finite length property of the Rado graph using both methods. We also describe some applications to weighted register automata and solving orbit-finite systems of linear equations.
\end{abstract}

%%
%% The code below is generated by the tool at http://dl.acm.org/ccs.cfm.
%% Please copy and paste the code instead of the example below.
%%
\begin{CCSXML}
<ccs2012>
<concept>
<concept_id>10003752.10003790</concept_id>
<concept_desc>Theory of computation~Logic</concept_desc>
<concept_significance>500</concept_significance>
</concept>
<concept>
<concept_id>10002950.10003624.10003633.10003638</concept_id>
<concept_desc>Mathematics of computing~Random graphs</concept_desc>
<concept_significance>500</concept_significance>
</concept>
<concept>
<concept_id>10003752.10003766.10003770</concept_id>
<concept_desc>Theory of computation~Automata over infinite objects</concept_desc>
<concept_significance>500</concept_significance>
</concept>
</ccs2012>
\end{CCSXML}

\ccsdesc[500]{Theory of computation~Logic}
\ccsdesc[500]{Mathematics of computing~Random graphs}
\ccsdesc[500]{Theory of computation~Automata over infinite objects}

%%
%% Keywords. The author(s) should pick words that accurately describe
%% the work being presented. Separate the keywords with commas.
\keywords{Rado graph, vector spaces, finite length property}

%\received{20 February 2007}
%\received[revised]{12 March 2009}
%\received[accepted]{5 June 2009}

%%
%% This command processes the author and affiliation and title
%% information and builds the first part of the formatted document.
\maketitle


\section{Introduction}

This paper is rooted in the programme on orbit-finite sets. In this programme, one starts with some relational structure $\A$, which is called the \emph{atom structure}, and whose elements are called \emph{atoms}. Based on this structure, one constructs sets which are called \emph{orbit-finite}. A precise definition will be given later in the paper, but the understanding is that elements of an orbit-finite set are constructed using atoms, and there are finitely many of them up to automorphisms of the atom structure. For the theory to make sense, we must assume that the atom structure is \emph{oligomorphic}, which means that $\A^d$ has finitely many orbits for every $d$. The simplest example of an oligomorphic structure is what we refer to as the \emph{equality atoms}, this is the structure which has a countably infinite underlying set, and no relations except for equality. This structure, like all oligomorphic structures, arises by applying a model theoretic construction (the Fraïssé limit) to a  well-behaved class of finite structures. Figure~\ref{fig:fraisse-limits} contains other examples of such structures. 
\begin{figure}
    \begin{center}
    \begin{tabular}{ll|l}
   &  class of finite structures & its Fraïssé limit\\
    \hline
    1.& finite sets with equality only & infinite set with equality only \\
    2.& finite orders & rational numbers with order \\
    3.& finite graphs & the Rado graph \\ 
    4.& vector spaces of finite & vector space of countably\\  
    & \quad dimension over $\mathbb F_2$ & \quad infinite  dimension over $\mathbb F_2$
\end{tabular}
\end{center}
\caption{\label{fig:fraisse-limits} Examples of Fraïssé limits}
\end{figure}

In the last row, we assume that the underlying field is finite, since we want the finite dimensional vector spaces to be finite structure. In the case of the two-element field, we use the name \emph{bit-vector atoms} for the structure in the last row. 

When the underlying atom structure is oligomorphic, the orbit-finite sets have a robust theory, which resembles in some ways the theory of finite sets. This theory was originally developed to describe regular languages over infinite alphabets, by considering orbit-finite versions of various automata models~\cite{bojanczykNominalMonoids2013,bojanczykAutomataTheoryNominal2014}, but it has developed to cover other models, such as orbit-finite Turing machines~\cite{bojanczykTuringMachinesAtoms2013}, or orbit-finite constraint satisfaction problems~\cite{klin2015locally}.
Also, there are programming languages with data structures that can store  orbit-finite sets~\cite{bojanczyk2012towards,bojanczyk2012imperative}, and even working implementations~\cite{kopczynski2016lois, szynwelski-phd}. For a survey of the orbit-finite programme, we refer to the lecture notes~\cite{bojanczyk_slightly}. 

Some results do not depend on the choice of the atom structure, while others do. Here is an example of the latter case that arises for orbit-finite Turing machines~\cite{bojanczykTuringMachinesAtoms2013}. If we choose the atoms  to be the Fraïssé limit of linear orders, as in row 2 of Figure~\ref{fig:fraisse-limits},  then the orbit-finite version of \textsc{p}$\neq$\textsc{np} has the same answer as the  classical version without atoms.  On the other hand, if we choose any of the other three rows of the table, then one can prove unconditionally that \textsc{p}$\neq$\textsc{np} holds in the orbit-finite setting, by leveraging problems with choice. The dependency on the underlying atom structure will play a prominent role in this paper. 


% An \emph{orbit-finite} set over this atom structure  is one that can be generated from finitely many elements by applying atom automorphisms. For example, if the atom structure has on relations except for equality (such a structure is called the \emph{equality atoms}), then its automorphisms are all permutations, and the set $\A^2$ is orbit-finite, since it can be generated by starting with two elements: a non-equal pair $(a,b)$ and an equal pair $(a,a)$; all other elements of $\A^2$ can be obtained from these two by applying permutations of the atoms.  Orbit-finite sets were introduced in order to define models of computation, such as orbit-finite monoids~\cite{bojanczykNominalMonoids2013}, orbit-finite automata~\cite{bojanczykAutomataTheoryNominal2014}, or orbit-finite Turing machines~\cite{bojanczykTuringMachinesAtoms2013}. There are also strong connections with the theory of nominal sets~\cite{PittsAM:nomsns}.

\paragraph*{Vector spaces.}
A recent extension of the orbit-finite programme, which was motivated by the study of orbit-finite weighted automata,  is to consider vector spaces~\cite{BFKM24}. In these vector spaces, one can take linear combinations and apply  automorphisms of the atom structure.  We are interested in spaces which have orbit-finite dimension, which means that there the entire space can be obtained from some finite subset by using atom automorphisms and linear combinations\footnote{In the presence of atoms, choice can be problematic, and in some cases one cannot choose a basis that is invariant under atom automorphisms. Hence, it is more formally correct to talk about orbit-finitely spanned spaces.}. A prototypical example is the space $\Lin X$, which consists of formal linear combinations of elements in some orbit-finite set~$X$. As mentioned before, the original application for these spaces was in automata theory, but they have also found applications in the study of  orbit-finite linear programming~\cite{ghosh2023orbit}, function spaces for orbit-finite sets~\cite{functionSpaces2024}, or  the analysis of two-party communication protocols over infinite alphabets~\cite{aliceBob}.



In order to be useful, the theory orbit-finitely spanned vector spaces should have certain properties. For example, one would like to be able to represent these spaces in a finite way, or solve algorithmic problems such as solving systems of linear equalities. One rather modest requirement is that the spaces are closed under taking subspaces (here, a subspace must be closed under both linear combinations and atom automorphisms): if we take an orbit-finitely spanned vector space $V$, and we restrict it to a subspace, then the result should also be orbit-finitely spanned. We do not know if this closure property holds in general, which we think is an important open problem. This problem can be phrased  in terms of ascending chains: is it true that every orbit-finitely spanned vector space has the  \emph{ascending chain condition}, which means that  one cannot find an infinite ascending chain of subspaces. 
To the best of our knowledge, this question was first recognised by Camina and Evans, who asked (not using the orbit-finite terminology):
\begin{description}\item[Question.]{\cite[Question 2]{CaminaEvans_91}}
Let $\A$ be an oligomorphic structure. Which orbit-finitely spanned vector spaces over $\A$ have the ascending chain condition?
\end{description}
In~\cite{CaminaEvans_91}, the authors identify a sufficient condition for the ascending chain condition, namely the existence of an Ahlbrandt-Ziegler enumeration. Using this sufficient condition, they show the  ascending chain condition for some vector spaces, notably $\Lin {\A \choose d}$ for the equality atoms (the first row of Figure~\ref{fig:fraisse-limits}), and also for a similar space in the bit-vector atoms (the last row of Figure~\ref{fig:fraisse-limits}).

In~\cite[p.~21]{BFKM24}, it is conjectured that the answer to the above question is ``all of them'', for all  oligomorphic structures. This conjecture has been confirmed for some oligomorphic structures, notably the equality atoms: over the equality atoms, all orbit-finitely spanned vector spaces satisfy the ascending chain condition. In fact, the case of the equality atoms was proved independently in three different contexts:  model theory~\cite[Lemma 3.23]{Gray97},  representation theory by~\cite[Proposition 6.1.6]{SamSnowden_15} and orbit-finite sets~\cite{BFKM24}. The second cited result, from representation theory, assumes that the field is the complex numbers, while the other two do not restrict the choice of field. In fact, all three cited results prove a stronger property, which we call the \emph{finite length property}: for every orbit-finitely spanned vector space, there is some finite upper bound on the length of chains of subspaces. Using the Jordan-Holder Theorem, the finite length property is equivalent to saying that both ascending and descending chains of subspaces are finite. This raises the following question: 

\begin{description}\item[Question.]{\cite[Question 2]{CaminaEvans_91}}
Let $\A$ be an oligomorphic structure. Which orbit-finitely spanned vector spaces over $\A$ have the ascending chain condition?
\end{description}


A canonical example of such a set is
\begin{align}\label{eq:lin-first-appearance}
\Lin_\field \A^d,
\end{align}
which is the vector space with basis $\A^d$, over some field $\field$. Assuming that the basis $\A^d$ is finitely generated by using atom automorphisms only (this assumption will always hold in the paper), then this vector space is finitely generated by using both atom automorphisms and linear combinations.  For the spaces in question, we will be interested in subspaces that preserve both kinds of structure, i.e.~they are closed under atom automorphisms and linear combinations. An example of such a subspace in~\eqref{eq:lin-first-appearance} is ``for basis vectors (i.e.~atom tuples) where all atoms are distinct, the coefficients sum up to zero''. This is because membership in this subspace will not be affected by taking linear combinations, or applying a permutation of the atoms. 



(So far I have only included some motivation. More to come later. We should also refer more to the work of Evans.)
\paragraph*{Motivation.}
The original motivation for vector spaces of orbit-finite dimension comes from automata theory. 
 The finite length property was introduced in order to decide equivalence of orbit-finite weighted automata~\cite[Section 3]{BFKM24}.

 Another motivation for studying the finite length property comes from the  Thomas conjecture~\cite[p.~177]{thomas1991reducts} in model theory. As observed by Evans~\cite[Slide 6]{Evans2025Permutation}, a counterexample to the Thomas conjecture would arise if one could find an example of a structure which: (a) is homogeneous over a finite relational vocabulary;  and (b) fails the finite length property over some finite field. We do know examples where (a) is relaxed by allowing functions (while remaining oligomorphic), see~\cite[Section 4.4]{BFKM24}.

 Vector spaces of orbit-finite dimension are also a solution to an important limitation in the theory of orbit-finite sets, which is the lack of function spaces. If we take an orbit-finite set $X$, then the family of finitely supported subsets (even finite subsets) will not be orbit-finite. These problems go away, however, if we go from orbit-finite sets to vector spaces of orbit-finite dimension. Indeed, if we consider the space of  from $X$ to a field (say the two-element field), 
 then this  (which is now endowed with the structure of a vector space) has orbit-finite dimension~\cite[Theorem 6.7]{BFKM24}, at least in the case of some important structures. This issue is discussed in~\cite[Section 7]{functionSpaces2024}, phrased in the language of monoidal closed categories. A concrete application of function spaces can be found in~\cite{aliceBob}, where they are applied to obtain a characterization of two-party communication protocols over infinite alphabets.
\section{Structures}

In this section, we briefly recall some basic notions from model theory and describe the main examples of structures that we will consider in this paper.

Let us begin by fixing some notation.
A \emph{vocabulary} is a set of relations, each with a specified arity (we do not use functions in this paper). For example, the vocabulary of graphs will contain one binary relation $E(x,y)$, which is meant to describe edges. Similarly, the vocabulary of ordered structures will also contain one binary relation $x \leq y$.   A structure $\A$ over a vocabulary consists of an underlying set, together with interpretations of the symbols in the vocabulary as relations on the underlying set. An \emph{embedding} of two structures over the same vocabulary is an injective map between their underlying sets which is consistent with all relations. An \emph{isomorphism} is a surjective embedding. An \emph{automorphism} of a structure is an isomorphism of the structure with itself; these form a group. 

We will be interested in applying  automorphisms  of $\A$ to  tuples in $\A^d$; they are applied component-wise.  When we talk about orbits in $\A^d$, we mean the orbits under this action of the  automorphism group. For example, if $\A$ is a graph, then two pairs $(a_1,a_2)$ and $(b_1,b_2)$ are in the same orbit if and only if there is some automorphism of the graph that maps $a_1$ to $b_1$ and $a_2$ to $b_2$. In particular, the edge relation must be defined in the same way for both pairs.

All structures considered in this paper will be countable, and we will always want them to have finitely many orbits in every finite dimension, as explained in the following definition.

\begin{definition}[Oligomorphic structure]\label{def:oligomorphic-structure}
    A structure $\A$ is \emph{oligomorphic} if $\A^d$ has finitely many orbits  for every $d \in \set{1,2,\ldots}$.
\end{definition}

A relation $R \subseteq \A^d$ in a structure is called \emph{equivariant} if it is invariant under the action of the automorphism group. This means that the  relation is a union of orbits.  If the structure is oligomorphic, then there are finitely many orbits to consider once the dimension $d$ is fixed, and therefore there  can only be finitely many equivariant relations. By the Ryll-Nardzewski theorem, if the structure is oligomorphic and countable, then the equivariant relations are exactly those that can be defined in first-order logic, see~\cite[Lemma 5.9]{bojanczyk_slightly}. In fact, the infinite structures that we consider in this paper will have a stronger property, namely the equivariant relations will be definable not only in first-order logic, but even by quantifier-free formulas. This will be guaranteed by the homogeneity property defined below. Recall that a substructure of $\A$ is any structure obtained by restricting $\A$ to some subset  of its underlying set. 

\begin{definition}
    [Homogeneous structure] A structure $\A$ is \emph{homogeneous} if every isomorphism between finite substructures of $\A$ extends to an automorphism of $\A$.
\end{definition}

In a homogeneous structure, every equivariant relation is definable by a quantifier-free formula, see~\cite[Theorem 6.3]{bojanczyk_slightly}. We will be mainly interested in countable structures that are both oligomorphic and homogeneous. Every homogeneous structure arises via a construction called the Fraïssé limit, from classes of finite structures that satisfy certain closure properties (the so-called Fraïssé/amalgamation classes). For the Fraïssé limit to be not only homogeneous, but also oligomorphic, we need to assume that the underlying class has only finitely many non-isomorphic structures of each finite size. Here are some of the important examples of oligomorphic structures that we will consider in this paper. These are Fraïssé limits of the following classes of finite structures: (a) sets with equality only; (b) linear orders; (c) vector spaces over a finite field; and (d) graphs.
\begin{example}[Equality only] \label{ex:equality-atoms} In this structure, the underlying set is countably infinite and there are no relations or functions.  Automorphisms are arbitrary permutations, and two tuples are in the same orbit if and only if they have the same equality pattern. Since there are finitely many equality patterns for tuples of fixed length, this structure is oligomorphic. For example, in dimension $d=2$ there are two orbits: $x_1=x_2$ and $x_1 \neq x_2$.
\end{example}

\begin{example}[Order] \label{ex:order-atoms} In this structure, the underlying set is the set of rational numbers, equipped with the usual order. Automorphisms are order-preserving permutations, and two tuples are in the same orbit if and only if they have the same order pattern. Since there are finitely many order patterns for tuples of fixed length, this structure is oligomorphic. 
In dimension $d = 2$ there are three orbits: 
$x_1 < x_2$, $x_1 = x_2$, and $x_1 > x_2$.
\end{example}

\begin{example}
    [Vector space] \label{ex:vector-space-atoms} Fix some finite field, and consider the vector space of countably infinite dimension over this field. Since we use only relational vocabularies in this paper, this vector space is seen as a structure over the infinite vocabulary, which contains a relation 
    \begin{align*}
    \setbuild{(a_1,\ldots,a_d) \in \A^d}{ $\lambda_1 a_1 + \cdots + \lambda_d a_d = 0$ }
    \end{align*}
    for every $d \in \set{1,2,\ldots}$ and every  field coefficients $\lambda_1, \ldots, \lambda_d$. The vocabulary is defined so that automorphisms are the same thing as permutations that are linear maps. Two tuples are in the same orbit if and only if they have the same linear dependencies. Since there are finitely many linear dependency patterns for tuples of fixed length over a finite field, this structure is oligomorphic.
    In particular, in dimension $d = 2$, there are three more orbits than there are elements in the field: 
    $x_1 = 0 = x_2$;
    $x_1 = 0 \neq x_2$;
    $0 \neq x_1 = \lambda \cdot x_2$, where $\lambda$ ranges over the field;
    and $x_1, x_2$ being linearly independent.
\end{example}

\begin{example}[Rado graph]\label{ex:rado-graph} The Rado graph is the Fraïssé limit of the class of finite undirected graphs. Here, an undirected graph is viewed as a structure that has a binary relation, which is symmetric and irreflexive. It  is not so easy to describe the Rado graph  explicitly, but one of its characterisations is that if one randomly selects a graph with a countably infinite set of vertices, by independently including each possible edge with probability $1/2$, then with probability $1$ the resulting graph is isomorphic to the Rado graph. Two tuples are in the same orbit if and only if they have the same equality and adjacency patterns, hence there are finitely many orbits in every dimension.
For instance, in $d = 2$ there are three orbits:
the two coordinates can be equal, adjacent (hence distinct), or distinct but non-adjacent.
\end{example}




% \color{red}
% Definitions: 
% \begin{enumerate}
%     \item oligomorphic
%     \item homogeneous
%     \item smooth approximation by homogeneous substructures \cite{KLM89} (N.B. not the 'smooth approximation' from \cite[Definitino~4]{MP24})
%     \item \emph{oligomorphic approximation} of a homogeneous structure by finite substructures with uniformly few orbits (i.e., types) that cover the age of $\A$
% \end{enumerate}

\section{Orbit-finite sets}
\label{sec:orbit-finite-sets}
In this section, we briefly explain orbit-finite sets.
The general idea is that we start with some infinite structure $\A$. We think of elements of this structure as \emph{atoms}, which can be used to construct sets that are finite up to symmetries of the structure, such as
\begin{align*}
\myunderbrace{\A^2}{pairs}
\qquad 
\myunderbrace{\setbuild{ (a,b) \in \A^2}{$a \neq b$}}{non-repeating pairs}
\qquad 
\myunderbrace{\A \choose 2}{unordered pairs}.
\end{align*}
Since we want to be able to construct tuples of arbitrary finite length, and we want them to be finite up to symmetries, we will need to assume that the underlying structure of atoms is oligomorphic.

There are several equivalent definitions of orbit-finiteness. Among these, we use one that is based on first-order interpretations. To construct an orbit-finite set, we  proceed in three steps: take a finite power of the atoms (as in the first example above), restrict this power to an equivariant subset (as in the second example above), and then take a quotient under an equivariant equivalence relation (as in the third example above, where the equivalence relation identifies two pairs if they differ only in their order). The formal definition is given below.

 \begin{definition}[Orbit-finite set]\label{def:orbit-finite-set}
    An \emph{orbit-finite set} over an oligomorphic structure $\A$ is any set that is obtained as follows: 
    \begin{enumerate}
        \item Start with a finite power $\A^d$ for some $d \in \set{0,1,\ldots}$.
        \item Restrict it to an equivariant subset $X \subseteq \A^d$.
        \item Quotient  $X$ under an equivariant equivalence relation. 
    \end{enumerate}
 \end{definition}


An orbit-finite set is  equipped with an action of the automorphism group of the original structure $\A$, namely the action inherited  from $\A^d$, suitably extended to the quotient. Under this action, the set has finitely many orbits, since $\A^d$ has finitely many orbits by assumption on oligomorphism, and  the number of orbits can only go down when  restricting to an equivariant subset and quotienting under an equivariant equivalence relation. This explains the name \emph{orbit-finite set}.  
 
We can think of an orbit-finite set as a relational structure, by equipping it with some relations that are equivariant with respect to the action of the automorphism group of $\A$. 
  Since we assume the atoms to be oligomorphic, the equivariant relations are exactly those that can be defined in first-order logic. Therefore, the above definition is the same as a first-order interpretation in $\A$, see~\cite[Section 5.3]{hodges1993model}. 
  
  As we mentioned earlier, there are other equivalent definitions of orbit-finite sets. One of these, see~\cite[Theorem 5.13]{bojanczyk_slightly}, is that an orbit-finite set is a set  equipped with an action of the automorphism group of $\A$, such that: (a) there are finitely many orbits; and (b) every element has finite support.
\begin{definition}
   [Orbit-finite sets, abstractly] An \emph{orbit-finite set} over an oligomorphic structure $\A$ is a set $X$ equipped with an action of the automorphism group of $\A$ such that: 
   \begin{enumerate}
      \item there are finitely many orbits under the action;
      \item every element has finite support, which means that for every $x \in X$ there is some finite set of atoms $S \subseteq \A$ such that if an automorphism of $\A$ fixes all atoms in $S$, then it also fixes $x$. 
   \end{enumerate}
\end{definition}
 \paragraph*{Orbit-finite automata.} As mentioned in the introduction,  the study of orbit-finite sets was originally motivated by automata and regular languages over infinite alphabets~\cite{bojanczykNominalMonoids2013,bojanczykAutomataTheoryNominal2014}. The idea is to use standard models of computation, but to replace finite sets with orbit-finite ones, while keeping all structure equivariant. The standard example is orbit-finite automata, defined as follows.
    An orbit-finite nondeterministic automaton over atoms $\A$ is defined like a nondeterministic finite automaton, except that the states and alphabet are orbit-finite sets over $\A$, instead of finite ones, and all structure (the sets of initial and final states and the transition relation) are equivariant. A deterministic orbit-finite automaton is the special case which has only one initial state, and where the transition relation is a function. 
%  \begin{example}[Cycles in the Rado graph]
%     Assume that the atoms are the Rado graph. The cycles in the Rado graph can be viewed as a language $L \subseteq \A^*$, which consists of words where every two consecutive letters are related by an edge in the atoms, and furthermore there is an edge between the last and the first letter. This language can be recognised by a deterministic  orbit-finite automaton, which uses its states to remember the first letter of the input word as well as the most recently read letter. The state space of this automaton is the disjoint union
%     \begin{align*}
%       \myunderbrace{\set{\varepsilon}}{initial \\ state}
%       \quad + \quad 
%       \myunderbrace{\A^2}{first letter and \\ most recent letter}
%         \quad + \quad
%         \myunderbrace{\set{\bot}}{rejecting \\ sink state}.
%     \end{align*}
%     This is an orbit-finite set. This is because the initial and sink states can be seen as two copies of $\A^0$, and  orbit-finite sets are closed under disjoint unions, assuming that the atom structure has at least two elements.
%  \end{example}
\section{Orbit-finitely spanned vector spaces}\label{sec:spaces}
We now introduce the main topic of this paper, which is orbit-finitely spanned  vector spaces. We begin with the special case of spaces that have an orbit-finite basis; this special case has an elementary definition, and yet it will be the  relevant case for almost all results of this paper. 

\begin{definition}[Orbit-finite basis]\label{def:orbit-finite-basis}
    A vector space with an orbit-finite basis is any space of the form $\Lin_\field X$ (where $X$ is an orbit-finite set and $\field$ a field), i.e.~the space of finite formal linear combinations of elements in $X$.
\end{definition}

This definition has two important parameters: the field $\field$ and the oligomorphic structure $\A$ over which $X$ is an orbit-finite set.
The spaces defined here have two kinds of structure: that of a vector space, and the action of the automorphism group of $\A$. We will be interested in subsets that preserve both kinds of structure, i.e.~they are closed under taking linear combinations, and applying automorphisms of $\A$. Such subsets are called \emph{equivariant subspaces}. 


\begin{myexample}\label{ex:length-one-dim-one}
    Let $\A$ be the equality atoms and let $\field$ be any field. As explained in~\cite[Ex.~4.2]{BFKM24}, this corresponding vector space with atoms $\Lin_\field \A$
    has only three equivariant subspaces: the zero subspace, the full space, and  subspace which consists of those vectors where the coefficients add up to zero. 
\end{myexample}

 Unfortunately, there is a price to pay for the elementary character of Definition~\ref{def:orbit-finite-basis}, which is the  failure of certain closure  properties. In particular, the spaces are not closed under taking equivariant subspaces, or quotients under such spaces. The problem with subspaces  is apparent already in Example~\ref{ex:length-one-dim-one},  since the unique nontrivial subspace of $\Lin_\field \A$ does not have any equivariant basis, regardless of the underlying field~\cite[Ex.~6.1]{BFKM24}. For this reason, we use a more general notion of vector space, which is presented below, in a style that emphasises the group action, as was done in Definition~\ref{def:orbit-finite-set-alternative}.

\begin{definition}\label{def:orbit-finitely-spanned-abstract}
    [Orbit-finitely spanned vector space, abstractly] An \emph{orbit-finitely spanned vector space} over an oligomorphic structure  $\A$ is a vector space  equipped with an action of  automorphisms of $\A$ such that: 
    \begin{enumerate}
        \item \label{item:equivariance-of-vector-space-structure} vector addition and scalar multiplication are equivariant;\footnote{Equivalently, the group action $v \mapsto \pi(v)$ is linear.}
        \item every vector is supported by a finite set of atoms; 
        \item \label{item:orbit-finite-spanning-subset} the space is spanned by some subset that  is orbit-finite.
    \end{enumerate}
\end{definition}

In~\eqref{item:equivariance-of-vector-space-structure}, by equivariance of scalar multiplication we mean that for every field element $\lambda$, the operation $v \mapsto \lambda v$ is equivariant. The above definition is easily seen to be closed under taking quotients by equivariant subspaces. Closure under taking equivariant subspaces is less obvious: one could imagine that condition~\eqref{item:orbit-finite-spanning-subset} above is violated by moving to an equivariant subspace. It turns out that closure under equivariant subspaces is intimately related to the ascending chain property discussed in the introduction:

\begin{theorem}\label{thm:ACP}
    For any field $\field$ and oligomorphic structure $\A$, the following conditions are equivalent: 
    \begin{enumerate}
        \item orbit-finitely spanned vector spaces are closed under taking equivariant subspaces; 
        \item for every $d \in \set{0,1,\ldots}$,  the vector space 
        $
        \Lin_\field \A^d
        $ does not have any infinite ascending chain of equivariant subspaces;
        \item for every orbit-finitely spanned vector space $V$, there are no infinite ascending chains of equivariant subspaces in $V$.
    \end{enumerate}
    Furthermore, if these conditions hold, then  every orbit-finitely spanned vector space is isomorphic to one of the form $V/U$, where $V \subseteq U$ are equivariant subspaces of $\Lin_\field \A^d$ for some $d \in \set{0,1,\ldots}$.
\end{theorem}


We say that an atom structure $\A$ has the \emph{ascending chain property} over a field $\field$ if any of the equivalent conditions in Theorem~\ref{thm:ACP} are satisfied. One interpretation of the theorem is that   the ascending chain property is necessary for the theory of vector spaces to be well-behaved.  In particular, thanks to the ``furthermore'' part, we get a similar result to the equivalence of Definitions~\ref{def:orbit-finite-set} and~\ref{def:orbit-finite-set-alternative} in the previous section, i.e.~a concrete characterisation of the orbit-finitely spanned vector spaces that can be used in algorithms (provided we know how to represent equivariant subspaces --- see Section~\ref{sec:equivariant-subspaces}). 


We are therefore interested in atom structures that have the ascending chain property. As it turns out, the techniques used in this paper will yield a stronger property, namely a finite bound on the length of chains: 




\begin{definition}
    [Finite length property]
    \label{def:finite-length-property}
    An oligomorphic structure $\A$ has the \emph{finite length property over a field $\field$} if for every orbit-finite set $X$ over $\A$, there is a finite upper bound on the length of chains of equivariant subspaces 
    of $\Lin_\field X$.%
    \footnote{The supremum of the chain lengths is called the \emph{length} of $\Lin_\FF X$, which is a standard notion in algebra. In Appendix~\ref{sec:appendix-modules} we describe a few properties of length that we will use, and we also supply a proof of Theorem~\ref{thm:ACP}.}
\end{definition}

In light of Theorem~\ref{thm:ACP}, we could have used $\A^d$ instead of $X$ in the above theorem.
As mentioned in the introduction, the finite length property can be strictly stronger than the ascending chain property, as witnessed by the vector atoms from Example~\ref{ex:vector-space-atoms}.  (Note that when we talk about the vector atoms, there  two fields involved, namely the finite field used to define $\A$, and the field $\field$ used to define the vector space with atoms. In the counterexample, both fields are the two-element field.)
The finite length property was studied in~\cite{BFKM24}, where it was shown that the equality atoms (Example~\ref{ex:equality-atoms}) and the ordered atoms (Example~\ref{ex:order-atoms}) have this property over any field. 

The main contribution of this paper is to establish the finite length property for more structures. We will use two different techniques for this purpose.






% \begin{myexample}
%     Assume that the atoms are the Rado graph.
%     \begin{center}
%         (todo: give an interesting example of a weighted automaton over the Rado graph)
%     \end{center}
% %      Consider the function
% %     \begin{align*}
% %     \A^* \to \Lin_{\Q} \A.
% %     \end{align*}
% %     A word $w \in \A^*$ can be seen as a vertex weighted graph, where the vertices are the letters of the word, weighted by the number of occurrences. 
% %     \begin{align*}
% %     a_1 \cdots a_n \quad \mapsto \quad \sum_{i \in \set{1,\ldots,n}} \lambda_i a_i,
% %     \end{align*}
% %     where the coefficient $\lambda_i$ is the sum 
% % \begin{align*}
% % |\setbuild{i \in \set{1,\ldots,n}}{$a_j \to  a_i$ is an edge}
% % \end{align*}
% \end{myexample}


\section{Finite length in characteristic zero}
\label{sec:characteristic-zero}

In this section, we present the first of our two main results, which is a  method for proving the finite length property, assuming that the field has characteristic zero. Under this assumption, we will establish the finite length property for the Rado graph (Example~\ref{ex:rado-graph}) and for the vector space atoms (Example~\ref{ex:vector-space-atoms}). These are new results. We also think that the proof itself, even when applied to get already known results, is of independent interest and arguably simpler than previously known proofs.

The method that we use will work for structures that satisfy the following condition. 
 
\begin{definition}
    [Oligomorphic approximation]
    \label{def:oligomorphic-approximation} We say that a structure  $\A$ has \emph{oligomorphic approximation} if it is homogeneous and,  for every $d \in \set{1,2,\ldots}$, there exists a family $\Bclass$ of finite substructures of $\A$ such that: 
    \begin{enumerate}
        \item \label{item:oligomorphic-approximation-embedding} every finite substructure of $\A$ embeds into some $\B \in \Bclass$; and
        \item \label{item:oligomorphic-approximation-orbits} there is a common finite upper bound on the number of orbits in $\B^d$ for $\B \in \Bclass$.
    \end{enumerate}
\end{definition}

This is a relaxation of a stronger notion of  \emph{smooth approximation} known in model theory. There, the family $\Bclass$ is independent of $d$, and there are other requirements as well~\cite[p.~440]{KLM89}. 

\begin{theorem}\label{thm:have-oligomorphic-approximation}
    The following structures have oligomorphic approximation: 
    \begin{enumerate}
        \item the structure with equality only from Example~\ref{ex:equality-atoms};
        \item the vector space from Example~\ref{ex:vector-space-atoms}, for any finite field $\innerfield$;
        \item the Rado graph from Example~\ref{ex:rado-graph}.
    \end{enumerate}
\end{theorem}
Before proving the above theorem, let us observe that the dense linear order (Example~\ref{ex:order-atoms}) does not have oligomorphic approximation.


\begin{non-example}[rational numbers with order]\label{non-ex:weak-smooth-approximation}
A non-example is the rational numbers with the usual order. The finite substructures in this case are finite linear orders, and already for dimension $d=1$, a finite linear order of size $n$ will have $n$ orbits. Hence, we cannot have a common finite upper bound on the number of orbits in $\B^d$ for $\B \in \Bclass$.
\end{non-example}

\begin{proof}[Proof of Theorem~\ref{thm:have-oligomorphic-approximation}]

\ 
    \begin{enumerate}
        \item For the structure with equality only, we can choose $\Bclass$ so that it is independent of $d$, and this is simply all finite structures with equality only. For sufficiently large $\B \in \Bclass$, the number of orbits in $\B^d$ is the same as the number of orbits in $\A^d$. 
        \item For the vector space structure, as in the previous example, we can choose $\Bclass$ independently of $d$, namely the family of vector spaces of finite dimension. This argument applies to any finite field $\innerfield$.
        \item The most interesting case is the Rado graph. The witness for oligomorphic approximation will be a family of \emph{symplectic graphs}, see~\cite[Sec.~8.11]{GR01}.\footnote{We are grateful to Ehud Hrushovski for drawing our attention to this construction.} For every $n \in \set{1,2,\ldots}$ define a finite graph as follows. The set of vertices is the vector space over the two-element field with basis 
\begin{align*}
\set{e_1,\ldots,e_n, f_1,\ldots,f_n}.
\end{align*}
Since the field has two elements, we can view vertices as subsets of this basis. In this graph, there is an edge between vertices $v$ and $w$ if and only if the sets 
\begin{align*}
\setbuild{ i \in \set{1,\ldots,n}}{$e_i \in v$ and $f_i \in w$}\\
\setbuild{ i \in \set{1,\ldots,n}}{$f_i \in v$ and $e_i \in w$}
\end{align*}
have different sizes modulo two. These graphs satisfy condition~\eqref{item:oligomorphic-approximation-embedding} from Definition~\ref{def:oligomorphic-approximation}, i.e.~every finite graph embeds in some symplectic graph, see~\cite[Thm.~8.11.2]{GR01}. In Appendix~\ref{sec:appendix-symplectic}, we prove condition~\eqref{item:oligomorphic-approximation-orbits}, i.e.~that the number of orbits of $d$-tuples in symplectic graphs is uniformly bounded by a function of $d$ only.
    \end{enumerate}
\end{proof}

The main result of this section is the following theorem.

\begin{theorem}\label{thm:weak-smooth-approximation-finite-length}
    If an oligomorphic structure $\A$ has oligomorphic approximation, then it has the finite length property over any field of characteristic $0$.
\end{theorem}

Combining Theorems~\ref{thm:have-oligomorphic-approximation} and~\ref{thm:weak-smooth-approximation-finite-length}, we can get the following results, both old and new, on the finite length property.

\begin{corollary}\label{cor:weak-finite-length}
    Over any field of characteristic $0$, the following structures have  the finite length property: (a) the equality atoms; (b) the vector atoms; and (c) the Rado graph.
\end{corollary}

As mentioned in the introduction, the finite length property was already known for the equality atoms, for arbitrary fields. The results for the vector atoms and the Rado graph are new. The assumption on characteristic zero is important, at least in case of the  vector atoms, where  the finite length property is known to fail over finite fields~\cite[Sec.~4.4]{BFKM24}. Later on in this paper, we will prove the result for the Rado graph again using a different method that works for any field.

The rest of this section is devoted to proving Theorem~\ref{thm:weak-smooth-approximation-finite-length}.

\begin{proof}
    [Proof of Theorem~\ref{thm:weak-smooth-approximation-finite-length}]
Fix a structure $\A$, which has oligomorphic approximation, and  a field of characteristic zero. Since the field is fixed, we omit the field subscript and write $\Lin X$ for linear combinations of elements in $X$ that use coefficients from that field. Fix some power $d \in \set{1,2,\ldots}$.  Our goal is to show that 
$
\Lin \A^d
$
 has  the finite length property. For technical reasons, we apply the assumption on oligomorphic approximation not to $d$, but to $2d$, yielding  some class $\Bclass$ of finite structures that satisfies Definition~\ref{def:oligomorphic-approximation}.


    \begin{lemma}\label{lem:reduce-to-finite-substructures}
        For every $d \in \set{1,2,\ldots}$ we have 
        \begin{align*}
        \text{length of }\Lin \A^d
        \quad \leq \quad  \sup_{\B \in \Bclass} \text{\ length of }\Lin \B^d.
        \end{align*}
   %     where the supremum ranges over finite sets of atoms.
    \end{lemma}
    \begin{proof}
        Consider some chain of equivariant subspaces 
        \begin{align*}
        V_0 \subset V_1 \subset \ldots \subset V_n = \Lin \A^d,
        \end{align*}
        where equivariance is with respect to automorphisms of $\A$. 
        For each $i \in \set{1,\ldots,n}$, choose some vector  that is in $V_i$ but not in $V_{i-1}$. Let $S$ be the (finite) set of atoms that appear in these chosen vectors, and choose some $\B \in \Bclass$ which contains all atoms from $S$. (We can assume without loss of generality that $\Bclass$ is closed under isomorphism, and therefore every finite subset $S$ of $\A$ is contained in some $\B \in \Bclass$.) Define  
        \begin{align*}
        W_i = V_i \cap \Lin \B^d.
        \end{align*}
        By homogeneity, every automorphism of $\B$ extends to an automorphism of $\A$, and therefore the space $W_i$ is  equivariant with respect to automorphisms of $\B$.  The chain of $W_i$'s continues to be strictly growing, since it contains vectors that witness the growth of the original chain. Hence, the new chain  witnesses that the length of $\Lin \B^d$ is at least $n$. 
    \end{proof}


    Thanks to the above lemma, it remains to show that the length of $\Lin \B^d$ is bounded by some number that depends only on $d$. In our proof, this bound will be the number of orbits in $\A^{2d}$. What we have gained by moving from $\A$ to $\Bclass$ is that our vector spaces now have finite (albeit unbounded) linear dimension, and the group actions use finite (albeit unbounded) groups. This will let us leverage a central result from representation theory, called Maschke's Theorem (see e.g.~\cite[Chap.~8]{JamesLiebeck01}), which decomposes  spaces  into irreducible parts.  (In the theorem below, $\oplus$ stands for the direct sum of vector spaces, i.e.~$V \oplus W$ is the set of pairs $(v,w)$ where $v \in V$ and $w \in W$, with the operations defined coordinate-wise.)

        \begin{theorem}[Maschke's Theorem]
          Let $V$ be a finite dimensional vector space over a field of characteristic zero, equipped with an action of a finite group $G$. Then $V$ can be decomposed as 
            \begin{align*}
            V = V_1 \oplus \cdots \oplus V_n,
            \end{align*}
            where each $V_i$ is an equivariant subspace (with respect to the action of $G$)  and  is  irreducible, i.e.~the only equivariant subspaces of $V_i$ are the zero space  and the full space $V_i$.
        \end{theorem}


    We will use Maschke's Theorem to bound the length of the vector  spaces $\Lin \B^d$. 
    For two vector spaces $V$ and $W$ equipped with an action of the same group $G$, let us write 
    \begin{align*}
    \lineqfun V W
    \end{align*}
    for the set of all those linear maps from $V$ to $W$ which are equivariant with respect to the action of $G$. (The group and its action are  implicit in this notation.) Elements of this set are closed under taking linear combinations, and therefore the set can be seen as a vector space. (We do not care to equip this space with an action of $G$.) In particular, it is meaningful to talk about the dimension of this vector space. 
    \begin{lemma}\label{lem:dim-bounds-length}
        Let $V$ be a finite dimensional vector space over a field of characteristic zero, equipped with an action of a finite group $G$. Then 
        \begin{align*}
        \text{length of\, $V$} \quad \leq  \quad \text{dimension of\, }  \lineqfun V V.
        \end{align*}
    \end{lemma}
    \begin{proof}
        Apply Maschke's Theorem, yielding a decomposition 
        \begin{align*}%\label{eq:maschke-decomposition}
        V = V_1 \oplus \cdots \oplus V_n,
        \end{align*}
        where the subspaces $V_1,\ldots,V_n$ are equivariant and irreducible, with respect to the action of the group $G$. Irreducible spaces have length one by definition, and the length is additive with respect to direct sums, i.e.
        \begin{align*}
            \text{length of $V$} = \text{length of $V_1$} + \cdots + \text{length of $V_n$},
        \end{align*}
        so the length of $V$ is equal to $n$.  We will now show that the dimension is at least $n$ on the right-hand side of the inequality in the statement of the lemma.
        For every $i \in \set{1,\ldots,n}$ we can define an equivariant linear map from $V$ to itself which is the identity on $V_i$ and maps vectors from the other components to zero.  This gives us  at least $n$ equivariant linear maps from $V$ to itself. None of these maps is spanned by the other, and hence the dimension is at least $n$. 
    \end{proof}

    Thanks to Lemmas~\ref{lem:reduce-to-finite-substructures} and~\ref{lem:dim-bounds-length}, the length of the vector space with atoms $\Lin \A^d$ is bounded by the dimensions of the vector spaces
    \begin{align}
        \label{eq:endo-maps-b-d}
 \lineqfun { \Lin \B^d} {\Lin \B^d},
    \end{align}
    where $\B$ ranges over the family $\Bclass$.
    To complete the proof of the theorem, it remains to show that these dimensions  are bounded by some constant that depends only on $d$. This is done in the following lemma, which   completes the    proof of Theorem~\ref{thm:weak-smooth-approximation-finite-length}. 

    \begin{lemma}\label{lem:bound-on-dim-for-finite-set} 
        For every  $\B \in \Bclass$ the dimension of the vector space  in~\eqref{eq:endo-maps-b-d} is at most the number of orbits in $\A^{2d}$.
    \end{lemma}
    \begin{proof}
        A linear map  in the space~\eqref{eq:endo-maps-b-d} is the same thing as a square matrix indexed by $\B^d$, i.e.~a function of type  
        \begin{align*}
        {\B^d \times \B^d} \to \field.
        \end{align*}
        This function must be equivariant with respect to automorphisms of $\B$. This means that inputs in the same orbit must be mapped to the same field element. Therefore, to define  such a function, we need to choose one element of $\field$ for each orbit of the input.   Therefore, the dimension of the space in~\eqref{eq:endo-maps-b-d} is equal to the number of orbits of $\B^{2d}$, under the action of the group of automorphisms of $\B$.  This dimension is bounded by some constant that depends only on $d$, by the assumption of oligomorphic approximation. 
    \end{proof}
This completes the proof of Theorem~\ref{thm:weak-smooth-approximation-finite-length}. For convenience, we summarise the steps  in Figure~\ref{fig:proof-summary-cogless}.
\end{proof}


\begin{figure}
    \[
\begin{tikzcd}[row sep=large]
\text{length of $\Lin \A^d$} 
\ar[d, phantom, "\rotatebox{90}{$\geq$} \text{\quad (Lemma~\ref{lem:reduce-to-finite-substructures})}"']
\\
\displaystyle\sup_{\B \in \Bclass} \text{length of $\Lin \B^d$} 
\ar[d, phantom, "\rotatebox{90}{$\geq$} \text{\quad  (Lemma~\ref{lem:dim-bounds-length})}"']
\\
\displaystyle\sup_{\B \in \Bclass} \text{dimension of } \lineqfun {\Lin \B^d} {\Lin \B^d}
\ar[d, phantom, "\rotatebox{90}{$\geq$} \text{\quad( Lemma~\ref{lem:bound-on-dim-for-finite-set})}"']
\\
\displaystyle\sup_{\B \in \Bclass} \text{number of orbits in $\B^{2d}$}
\ar[d, phantom, "\rotatebox{90}{$>$} \text{\quad(by definition)}"']
\\
\infty.
\end{tikzcd}
\]
\caption{
    \label{fig:proof-summary-cogless}
    Summary of the proof of Theorem~\ref{thm:weak-smooth-approximation-finite-length}.}
\end{figure}

The inequalities shown in Figure~\ref{fig:proof-summary-cogless} give us upper bounds on the length. In the case of the equality atoms, this bound is exponential in $d$. In the case of the vector space atoms from Example~\ref{ex:vector-space-atoms}, the bound is the number of  linear dependency patterns for $2d$-tuples over a finite field. Such a pattern is described by: (a) indicating a subset of the coordinates which is a basis for the tuple; and (b) indicating the basis decompositions for the remaining coordinates. This can be done in at most exponentially many ways in $d$, and therefore the overall bound is  exponential in $d$. \textcolor{red}{In the case of the Rado graph, more exact bounds will be presented later in the paper -- where? }

\section{Function spaces and  weighted automata}\label{sec:duals}

The original motivation to inroduce orbit-finitely spanned vector spaces in~\cite{BFKM24} was the study of orbit-finite weighted automata. In this section, we recall this motivation, and discuss how it is relates to our new results. This discussion also involves  the issue of function spaces, which is arguably more important, so we begin with that.
 
\paragraph*{Function spaces.} If we have two orbit-finitely spanned vector spaces $V$ and $W$ over the same atoms, then there are two natural ideas for a function space: the space of all linear maps from $V \to W$, and the subspace which consists of equivariant linear maps. As it turns out, the most relevant function space lies between them, as formalised in the following definition.


\begin{definition}[Finitely supported function space]\label{def:function-space}
    For two orbit-finitely spanned vector spaces $V$ and $W$, we define their \emph{function space}, denoted by 
    \begin{align*}
     \linfsfun{V}{W},
    \end{align*}
    to be the space of linear maps $f$ wich satisfy the following {finite support condition}: there is some finite set of atoms $S \subseteq \A$ such that for every atom automorphism $\pi$ that fixes all atoms in $S$, we have 
    \begin{align*}
    \pi(f(v)) = f(\pi(v)) \quad \text{for all $v \in V$.}
    \end{align*}
\end{definition}

The notion of finite supports in the above definition is the same one as used in Section~\ref{sec:orbit-finite-sets}, except that it is applied to the space of linear maps from $V$ to $W$. As argued in~\cite[Section 8.3]{bojanczyk_slightly}, the finitely supported function space is the most relevant kind of function space, since it ensures that the corresponding category is becomes monoidal closed. For this reason, we when talking about function spaces, we mean the finitely-supported function spaces from the above definition. 
\begin{description}
    \item[Question.]  Consider an atom structure. Are orbit-finitely spanned vector spaces closed under taking function spaces?
\end{description}
The answer to the above question  is  ``yes'' for the equality and ordered atoms, which was shown in~\cite[Corollary 6.8]{BFKM24} for a special case of function spaces, namely duals, with the  general case of function spaces being treated in~\cite[Section 8.3]{bojanczyk_slightly}. 
On the other hand, the answer  is ``no'' for the  Rado graph, as shown in the following example. 
\begin{example}
    Assume that the atoms are the Rado graph. (The example would also work for the Hensen graph that will be defined later.) A variant of this example for the two-element field was shown in~\cite[Example 6.9]{BFKM24}, but here we can use any field, in particular a field of characteristic zero as treated in the previous section. 
    
    Consider the space 
    \begin{align}\label{eq:dual-space}
    \fsfun \A \field
    \end{align}
    which consists of functions  from atoms to the field that are finitely supported in the sense of Definition~\ref{def:function-space}. The space above does not talk about linear maps, since there is no vector space structure on the domain $\A$. However, it is  isomorphic to the function space 
    \begin{align}\label{eq:dual-space-two}
    \linfsfun{\Lin_\field \A}{\field}.
    \end{align}
    We will show that~\eqref{eq:dual-space} is not orbit-finitely spanned, and hence the same is true for the isomorphic function space~\eqref{eq:dual-space-two}. 
    
    For a finite set $S \subseteq \A$ of atoms, define a function  $    f_S : \A \to \field$  by
    \begin{align*}
    \quad 
    a \mapsto
    \begin{cases}
        1 & \text{if $a$ is a neighbour of all atoms in $S$}\\
        0 & \text{otherwise}.
    \end{cases}
    \end{align*}
    Define $V$ to be the subspace of~\eqref{eq:dual-space} that is spanned 
    to be the functions $f_S$, where $S$ ranges over finite sets of atoms. Being orbit-finitely spanned is closed under taking subspaces, which follows from   Theorem~\ref{thm:ACP} and the fact that the Rado graph has the ascending chain property (which was already proved in Theorem~\ref{thm:weak-smooth-approximation-finite-length }  for characteristic zero and will be proved later for any field). Therefore, if we show that $V$ is not orbit-finitely spanned, then the same will follow for~\eqref{eq:dual-space}. 
    
    Suppose towards a contradiction that $\{ f_S \mid S \in \mathcal{S} \}$ is a spanning set for some orbit-finite family $\mathcal{S}$ of finite subsets of $\A$.
    Let $T \subseteq \A$ be an independent set (i.e.~no edges inside the set) on more than $\max_{S \in \mathcal{S}}|S|$ vertices.
    Then we can write
    \[
    f_T = \lambda_1 \cdot f_{S_1} + \cdots + \lambda_n \cdot f_{S_n}
    \]
    for $S_i \in \mathcal{S}$, where we may assume $|S_1| \leq \cdots \leq |S_m| < |H|$.
    Let us prove $\lambda_i = 0$ by induction on $i = 1, \dots, n$.
    We may find some $a \in \A$ that is adjacent to all of $S_i$ but to none of 
    \[
    (S_{i+1} \cup \dots \cup S_m \cup T) \setminus S_i.
    \]
    As $\lambda_1, \dots, \lambda_{i-1} = 0$ and each of $S_{i+1}, \dots, S_m, T$ contains some vertex outside of $S_i$,
    we see that
    \[
        0 = f_T(a) = \sum_j \lambda_j \cdot f_{S_j}(a) = 0 + \lambda_i + 0.
    \]
    It follows that $f_T$ is the zero function, which cannot be true, since one can find an atom that is a common neighbour of all atoms in $T$.
\end{example}
 
Since we prove that the Rado graph has the finite length property, and it does not have orbit-finitely spanned function spaces, it follows that one property does not imply the other.


% \begin{theorem}
%     Assume that the atoms $\A$ are such that orbit-finitely spanned vector spaces are closed under taking function spaces. Then two views of orbit-finite weighted automata, deterministic and nondeterministic,  are equivalent. 
% \end{theorem}
% \begin{proof}
%     To go 
% \end{proof}
% A special case of a function space is the dual space 
% \begin{align*}
% \linfsfun V \field.
% \end{align*}
% If $V$ has a basis $X$, then this space is isomorphic to the space 
% \begin{align}\label{eq:fsfun}
% \fsfun X \field,
% \end{align}
% which consists of functions from $X$ to the field that are finitely supported in the sense described in Definition~\ref{def:function-space}. Note that there is a terminology clash, since the expression ``finitely supported'', when applied to functions of type $X \to \field$, could also be understood as ``zero almost everywhere''. 


\paragraph*{Weighted automata.} We now describe the orbit-finite version of weighted automata, and explain how the issues with function spaces have an impact on the theory of these automata. 

There are several ways of defining weighted automata, we choose one that  views them as deterministic automata, in which the states are endowed with a vector space structure.  An \emph{orbit-finite weighted automaton} has an input alphabet $\Sigma$, which is an orbit-finite set, and a state space $Q$, which is an orbit-finitely spanned vector space $Q$. In the state space, there is a distinguished  initial vector $q_0$ that is equivariant, an equivariant transition function 
\begin{align*}
\delta : Q \times \Sigma \to Q
\end{align*}
which becomes a linear map once any input letter $a \in \Sigma$ is fixed, and a  final function which is an equivariant linear map from $Q$ to the field. The output of the automaton is defined by applying the final map to the last state in the run. 


\begin{corollary}
{\color{red}This looks very isolated, with automata etc. never introduced}
    An unambiguous automaton with guessing can recognise the language
    \[
        \text{``the last letter (exists and) is adjacent to every previous letter''}.
    \]
    But viewed as a weighted language, it cannot be  recognised by orbit-finitely spanned automata (unlike its reversal).
\end{corollary}
\begin{proof}
    We use a Myhill-Nerode style argument.
    The derivative ${w -} : \A^* \to \field$ of a word $w \in \A^*$ in this weighted language is given by
    \[
        w v a = N_{w v}(a),
    \]
    Then $\langle {w -} \mid w \in \A^* \rangle$
    cannot be orbit-finitely spanned:
    \[
        {w-} = \sum_i \lambda_i \cdot {w_i-}
    \]
    gives $N_{w} = \sum_i \lambda_i \cdot N_{w_i}$.
\end{proof}
\section{Rado graph, with cogs}
In this section we work with the following setting:
\begin{itemize}
    \item 
    $\mathcal{L}_0$ is a (possibly infinite) \textcolor{magenta}{BINARY?} relational language containing a binary symbol $=$;

    \item 
    $\mathcal{C}_0$ is a free amalgamation class of $\mathcal{L}_0$-structures
    where $=$ is interpreted as true equality, but every other $R \in \mathcal{L}_0$ is interpreted irreflexively.\footnote{%
        We can enforce irreflexivity by considering a language $\mathcal{L}'_0$ which consists, 
        for each $R \in \mathcal{L}_0 \setminus \{=\}$ of arity $r$ and each partition $\P$ of $r$ into $k$ parts, 
        of a $k$-ary relation symbol $R_\P$.
        Then $\mathcal{L}_0$-structures may be viewed as $\mathcal{L}'_0$-structures and vice versa, without changing the meaning of embeddings.
        In this way, we get a free amalgamation class $\mathcal{C}'_0$ with a Fraïssé limit which, viewed as an $\mathcal{L}_0$-structrure, is isomorphic to $\A_0$.
    } 

    \item 
    $\mathcal{L}$ consists of $\mathcal{L}_0$ together with a new binary symbol $<$;

    \item 
    $\mathcal{C}$ consists of $\mathcal{L}$-structures obtained from $\mathcal{C}_0$ by expanding with all possible linear orderings;

    \item 
    $\A_0$ and $\A$ are the respective Fraïssé limits of $\mathcal{C}_0$ and $\mathcal{C}$,
    where without loss of generality we assume $\A_0$ and $\A$ share the same domain so that $\Aut(\A_0) \supseteq \Aut(\A)$.
\end{itemize}

\begin{example}\label{ex:N-Q}
    Take $\mathcal{L}_0$ consist of $=$ only and let $\mathcal{C}_0$ to be all finite sets.
    Then $\A_0$ is isomorphic to the pure set $\N$, whereas $\A$ is isomorphic to $\Q$ with the usual order.
\end{example}

\begin{example}\label{ex:Rado-orderedRado}
    Let $\mathcal{L}_0$ consist of $=$ together with a single binary symbol $\sim$ 
    and let $\mathcal{C}_0$ consist of all finite undirected graphs not embedding the complete graph $K_n$,
    where $3 \leq n$ ($\leq \infty$).
    Then $\A_0$ is the $K_n$-free Henson graph (or the Rado graph when $n = \infty$), and $\A$ is its generically ordered counterpart.
    (Allowing $n = 2$ makes these degenerate to $\N$ and $\Q$ above).
\end{example}

We note two technicalities and a triviality.

\begin{lemma}\label{lem:free-forb}
    Let $\mathcal{F}_0$ consist of minimal $\mathcal{L}_0$-structures which do not appear in $\mathcal{C}_0$.
    Then
    \begin{enumerate}
        \item $\mathcal{C}_0$ consists of every $\mathcal{L}_0$-structure that does not embed any $F \in \mathcal{F}_0$.
        \item $\mathcal{C}$ consists of every $\mathcal{L}$-structure whose $\mathcal{L}_0$-reduct does not embed any $F \in \mathcal{F}_0$.
        \item Given $F \in \mathcal{F}_0$, every two distinct $x, y \in F$ are related by some $R \in \mathcal{L}_0$.
    \end{enumerate}
\end{lemma}
\begin{proof}
    As $\mathcal{C}_0$ is closed under substructures, its complement is closed under superstructures and thus 
    --- since there are no infinite strictly descending chain of embedded substructures 
    --- determined by its minimal structures.
    2) follows because an $\mathcal{L}$-structure is in $\mathcal{C}$ precisely when its $\mathcal{L}_0$-reduct is in $\mathcal{C}_0$.
    For 3), notice that $F \setminus \{x\}$, $F \setminus \{y\}$ are in $\mathcal{C}_0$ by minimality; 
    therefore so is their free amalgam over $F \setminus \{x, y\}$, which then cannot agree with $F$.
\end{proof}

\begin{lemma}\label{lem:free-fresh}
    Let $X, Y, \{z\} \subseteq \A$ be disjoint and finite.
    Then there is some automorphism $\tau \in \Aut(\A)$ such that
    \begin{enumerate}
        \item $\tau$ fixes every $x \in X$;
        \item $\tau(z)$ does not appear together with any $y \in Y$ or with $a$ in any tuple $a_\bullet \in (X \cup Y \cup \{z, \tau(z)\})^*$ such that $\A \models R(a_\bullet)$ for some $R \in \mathcal{L}_0$;
        \item $\tau(z) > z$.
    \end{enumerate}
\end{lemma}
\begin{proof}
    In $\A_0$, form the free amalgam
% https://q.uiver.app/#q=WzAsNCxbMCwxLCJYIl0sWzEsMiwiWCBcXGN1cCBcXHt6XFx9Il0sWzEsMCwiWCBcXGN1cCBZIFxcY3VwIFxce3pcXH0iXSxbMiwxLCJYIFxcY3VwIFkgXFxjdXAgXFx7eiwgeidcXH0iXSxbMCwxLCIiLDAseyJzdHlsZSI6eyJ0YWlsIjp7Im5hbWUiOiJob29rIiwic2lkZSI6InRvcCJ9fX1dLFswLDIsIiIsMCx7InN0eWxlIjp7InRhaWwiOnsibmFtZSI6Imhvb2siLCJzaWRlIjoidG9wIn19fV0sWzEsMywieCBcXGluIFggXFxtYXBzdG8geCwgeiBcXG1hcHN0byB6JyIsMSx7InN0eWxlIjp7InRhaWwiOnsibmFtZSI6Imhvb2siLCJzaWRlIjoidG9wIn0sImJvZHkiOnsibmFtZSI6ImRhc2hlZCJ9fX1dLFsyLDMsIlxcc3Vic2V0ZXEiLDEseyJzdHlsZSI6eyJ0YWlsIjp7Im5hbWUiOiJob29rIiwic2lkZSI6InRvcCJ9LCJib2R5Ijp7Im5hbWUiOiJkYXNoZWQifX19XV0=
\[\begin{tikzcd}[cramped]
	& {X \cup Y \cup \{z\}} \\
	X && {X \cup Y \cup \{z, z'\}} \\
	& {X \cup \{z\}}
	\arrow["\subseteq"{description}, dashed, hook, from=1-2, to=2-3]
	\arrow[hook, from=2-1, to=1-2]
	\arrow[hook, from=2-1, to=3-2]
	\arrow["{x \in X \mapsto x, z \mapsto z'}"{description}, dashed, hook, from=3-2, to=2-3]
\end{tikzcd}\]
    so that no element of $Y \cup \{z\}$ is related with $z'$ by any $R \in \mathcal{L}_0$.
    Now we make $X \cup Y \cup \{z, z'\}$ an $\mathcal{L}$-structure: 
    inherit the order on $X \cup Y \cup \{z\}$ from $\A$,
    and declare that $z < z'$ as well as $z' < a$ if $a$, the next element of $X \cup Y$ larger than $z$, exists at all.
    Notice that \[
        x \in X \mapsto x, z \mapsto z'
    \] is still an embedding in presence of the order.
    By homogeneity, we may embed $X \cup Y \cup \{z, z'\}$ into $\A$ via some $f$ which is the identity on $X \cup Y \cup \{z\}$;
    again by homogeneity, we may extend the embedding \[ 
        f(x) = x \in X \mapsto f(x), f(z) \mapsto f(z')
    \] to some automorphism $\tau$ which makes 1), 2), and 3) true.
\end{proof}

\begin{proposition}
    The $S$-supported length of $\Lin {\A_0}^d$ is at most that of $\Lin \A^d$ for any finite $S \subseteq \A_0 = \A$.
\end{proposition}
\begin{proof}
    Any chain of subspaces in $\Lin {\A_0}^d = \Lin \A^d$ that are invariant under $\Aut(\A_0)_{(S)}$ must also be invariant under the subgroup $\Aut(\A)_{(S)}$.
\end{proof}









\subsection{Cogs in an orbit}
An inconvenience of $\A^d$ is that it may have many orbits.
\begin{definition}
    Let $S \subseteq \A$ be finite.
    We say an orbit $\mathcal{O} \subseteq \A^d$ is $S$-orderly 
    if $\mathcal{O} = \Aut(\A)_{(S)} \cdot o_\bullet$ for some/any $o_\bullet \in \mathcal{O}$ 
    where $o_1 < \dots < o_d$ and $o_1, \dots, o_d \not\in S$.
\end{definition}
By removing entries of $o_\bullet$ that repeat or come from $S$ and reordering the rest,
we can always find an $\Aut(\A)_{(S)}$-equivariant bijection to an $S$-orderly orbit.
Moreover, we may focus on a single $S$-orderly orbit at a time:

\begin{proposition}
    The following are equivalent for any finite $S \subseteq \A$:
    \begin{enumerate}
        \item $\A_S$ (that is, $\A$ with constants from $S$ fixed) is $\FF$-oligomorphic;
        \item $\A$ is oligomorphic and for every $S$-orderly orbit $\mathcal{O}$, the $S$-supported length of $\Lin_\FF \mathcal{O}$ is finite.
    \end{enumerate}
\end{proposition}
\begin{proof}
    Indeed we have
    \(
        \len(\Lin_\FF \A_S^d) 
        = \len(\Lin_\FF(\biguplus_i \mathcal{O}_i))
        = \len(\bigoplus_i \Lin_\FF \mathcal{O}_i) 
        = \sum_i \len(\Lin_\FF \mathcal{O}_i)
    \), where the $\mathcal{O}_i$'s are the $S$-orderly counterparts of the $\Aut(\A)_{(S)}$-orbits in $\A^d$.
\end{proof}

We now introduce the workhorse for understanding $\Lin_\FF \mathcal{O}$.
\begin{definition}
    Let $\mathcal{O} \subseteq \A^d$ be an $S$-orderly orbit.
    An \emph{$\mathcal{O}$-cog parallel} $a_\bullet \parallel b_\bullet$ consists of atoms $a_1 < b_1 < a_2 < b_2 < \dots < a_d < b_d$
    satisfying the following: 
    for some/any $o_\bullet \in \mathcal{O}$, 
    for every relation $R \in \mathcal{L}$ of arity $r$,
    and for each $r$-tuple $x_\bullet$ with entries in $\{a_1, \dots, a_d, b_1, \dots, b_d\} \cup S$,
    we have
    \[
        \A \models R(x_\bullet) \leftrightarrow R(x_\bullet [a_i \mapsto o_i, b_i \mapsto o_i \mid 1 \leq i \leq d]).
    \]    
    (By $x_\bullet [ a \mapsto b, c \mapsto d ]$ we mean the $r$-tuple where each entry $x_i$ equal to $a$ is replaced by $b$, and each entry equal to $c$ --- assumed to be distinct from $a$ --- is replaced by $d$).
\end{definition}

In particular, for all $I \subseteq \{1, \dots, d\}$ we may take $x_\bullet$ above to have entries in $\{a_i \mid i \in I\} \cup \{b_j \mid j \not\in I\} \cup S$, showing that
\[
    \begin{cases}
        a_i \mapsto o_i, &i \in I; \\ 
        b_j \mapsto o_j, &j \not\in I; \\
        s \mapsto s & s \in S
    \end{cases}
\]
defines an embedding.
It follows from homogeneity that $a_\bullet [a_i \mapsto b_i \mid i \in I]$ lies in the orbit $\mathcal{O} = \Aut(\A)_{(S)} \cdot o_\bullet$.

\begin{definition}
    The $\mathcal{O}$-cog corresponding to an $\mathcal{O}$-cog parallel $a_\bullet \parallel b_\bullet$ is the vector
    \[
        a_\bullet \between b_\bullet = \sum_{I \subseteq \{1, \dots, d\}} (-1)^{ |I| } a_\bullet [a_i \mapsto b_i \mid i \in I] \in \Lin_\FF \mathcal{O}.
    \]
    The linear span of all $\mathcal{O}$-cogs is denoted by $\Cog_\FF \mathcal{O}$.
\end{definition}

\begin{proposition}
    Let $\mathcal{O}$ be $S$-orderly.
    Then $\Cog_\FF \mathcal{O}$ is an $\Aut(\A)_{(S)}$-equivariant subspace of $\Lin_\FF \mathcal{O}$ generated by any single $\mathcal{O}$-cog.
\end{proposition}
\begin{proof}
    Suppose $a_\bullet \parallel b_\bullet$ is an $\mathcal{O}$-cog parallel.
    The definition completely specifies the $\mathcal{L}$-structure on $\{a_1, b_1, \dots, a_d, b_d\} \cup S$
    and says that
    \[
        a_i \mapsto a'_i, b_i \mapsto b'_i, s \mapsto s
    \]
    is an isomorphism given another $\mathcal{O}$-cog parallel $a'_\bullet \parallel b'_\bullet$.
    Homogeneity then yields an automorphism $\pi \in \Aut(\A)_{(S)}$ satisfying $\pi \cdot (a_\bullet \between b_\bullet) = a'_\bullet \between b'_\bullet$.
\end{proof}

Though the definitions were a mouthful, the example below should explain how cogs arise.
\begin{example}
    Let $\A = \Q$ as described in Example~\ref{ex:N-Q};
    there is a unique $\{\}$-orderly orbit $\mathcal{O}$ in $\A^2$.
    Consider the vector
    \[
        v = (0, 4) + (4, 9) - (9, 10) - (0, 10)
    \]
    in $\Lin \mathcal{O}$.
    We can find $4 < 4+\varepsilon < 9 < 9+\delta < 10$ in $\A$
    together with monotone bijections $\pi_1, \pi_2 \in \Aut(\A)$ such that
    \[
        \pi_1 : 
        \begin{cases}
            0 \mapsto 0, \\
            4 \mapsto 4 + \varepsilon, \\
            9 \mapsto 9, \\
            10 \mapsto 10;
        \end{cases}
        \pi_2 : 
        \begin{cases}
            0 \mapsto 0, \\
            4 \mapsto 4, \\
            9 \mapsto 9 + \delta, \\
            10 \mapsto 10
        \end{cases}        
    \]
    by interpolating linearly for example.
    Then
    \begin{align*}
        v_1 
        = v - \pi_1 \cdot v
        ={}& (0, 4) + (4, 9) - (4, 10) \\
        &-(0, 4+\varepsilon) - (4+\varepsilon, 9) + (4+\varepsilon, 10)
    \end{align*}
    duplicates the tuples with $4$ in it but kills the one without it.
    Similarly 
    \begin{align*}
        v_{1,2}
        = v_1 - \pi_2 \cdot v_1
        ={}& (4, 9) - (4, 9+\delta) \\
        &-(4+\varepsilon, 9) + (4+\varepsilon, 9+\delta)
    \end{align*}
    only leaves and duplicates the tuples with $9$ in it.
    Here $v_{1,2}$ is the parallel for the cog $(4, 9 \parallel 4 + \varepsilon, 9 + \delta)$ in $\mathcal{O}$
    as well as the smaller $\{0, 10\}$-orderly orbit $\mathcal{O}' = \Aut(\A)_{(0, 10)} \cdot (4, 9) \subseteq \mathcal{O}$.
\end{example}

To find cog parallels in general, we iterate the following procedure.
\begin{lemma}\label{lem:cog-building}
    Let $a_\bullet \parallel b_\bullet$ be an $\mathcal{O}$-cog parallel, where $\mathcal{O} \subseteq \A^d$ is $S$-orderly.
    Given $s \in S$ with $a_{j-1} < s < a_j$ (where we treat $a_0$ and $a_{d+1}$ as $\pm \infty$), 
    we write $S' = S \setminus \{s\}$ and let $a_\bullet ;_j s \in \A^{d+1}$ be the tuple obtained by inserting $s$ in $a_\bullet$ as the $j$th entry.
    Then \[
        \mathcal{O}' = \Aut(\A)_{(S')} \cdot a_\bullet ;_j s \subseteq \A^{d+1}
    \]
    is $S'$-orderly.

    Apply Lemma~\ref{lem:free-fresh} with $X \Coloneqq \{a_1, b_1, \dots, a_d, b_d\} \cup S'$, $z \Coloneqq s$, and any finite $Y \subseteq \A$ disjoint from $X \cup \{z\}$ to obtain $\tau \in \Aut(\A)_{(X)}$ and $\tau(z) \Eqqcolon s'$.
    Then $a_\bullet ;_j  s \parallel b_\bullet ;_j s'$ is an $\mathcal{O}'$-cog parallel.
    
\end{lemma}
\begin{proof}
    Already $a_{j-1} < b_{j-1} < s$ (if $j > 1$), $s < s'$, and $s' < a_j < b_j$ (if $j \leq d$).
    Now pick any relation $R \in \mathcal{L}$ of arity $r$ and take any $x_\bullet \in (\{a_1, b_1, \dots, a_d, b_d, s, s'\} \cup S')^r = (X \cup \{s, s'\} \cup Y)^r$.
    We split into three cases.
    
    If $s$ and $s'$ both appear in $x_\bullet$, then we have
    \begin{equation}\label{eq:cog-building-relations}
        \A \models R(x_\bullet [ b_i \mapsto a_i, s' \mapsto s \mid 1 \leq i \leq d ] ) \leftrightarrow R(x_\bullet). \tag{\P}
    \end{equation}
    Indeed, the left is false because $R$ is irreflexive;
    so is the right by the design of $s'$.

    Now if $s$ appears in $x_\bullet$ we may assume $s'$ does not.
    This time around we have \eqref{eq:cog-building-relations} as $x_\bullet \in (\{a_1, b_1, \dots, a_d, b_d\} \cup S)^r$ 
    --- so we can ignore the $[s' \mapsto s]$ substitution ---
    and $a_\bullet \parallel b_\bullet$ is a cog parallel in $\mathcal{O} = \Aut(\A)_{(S)} \cdot a_\bullet$.

    Finally, suppose $s'$ appears in $x_\bullet$ and $s$ does not.
    Then $x_\bullet [s' \mapsto s] = \tau^{-1} \cdot x_\bullet$ and
    \begin{align*}
        x_\bullet [ b_i \mapsto a_i, s' \mapsto s \mid 1 \leq i \leq d ] \\
        = (\tau^{-1} \cdot x_\bullet) [ b_i \mapsto a_i \mid 1 \leq i \leq d ]
    \end{align*}
    where $\tau^{-1} \cdot x_\bullet \in (\{a_1, b_1, \dots, a_d, b_d\} \cup S)^r$.
    On the one hand, as discussed in the case above we get
    \[
        \A \models R((\tau^{-1} \cdot x_\bullet) [ b_i \mapsto a_i \mid 1 \leq i \leq d ]) \leftrightarrow R(\tau^{-1} \cdot x_\bullet).
    \]
    On the other hand, we certainly have
    \[
        \A \models R(\tau^{-1} \cdot x_\bullet) \leftrightarrow R(x_\bullet)
    \]
    since $\tau$ is an automorphism.
    This establishes \eqref{eq:cog-building-relations} again,
    showing that $a_\bullet s \parallel b_\bullet s'$ is an $\mathcal{O}'$-cog parallel.
\end{proof}

\begin{proposition}
    Let $\mathcal{O} \subseteq \A^d$ be $S$-orderly.
    Then, given $a_\bullet \in \mathcal{O}$, there exists $b_\bullet \in \mathcal{O}$ such that $a_\bullet \parallel b_\bullet$ is an $\mathcal{O}$-cog parallel.
    Moreover, for $i = 1, \dots, d$ there is an automorphism $\pi_i \in \Aut(\A)_{(\{a_1, b_1, \dots, a_{i-1}, b_{i-1}, a_{i+1}, b_{i+1}, \dots, a_d, b_d\} \cup S)}$ sending $a_i \mapsto b_i$.
\end{proposition}
\begin{proof}
    For $1 \leq i \leq d+1$, let $S^{(i)} = S \cup \{a_i, \dots, a_d\}$ and let $\mathcal{O}^{(i)} = \Aut(\A)_{(S^{(i)})} \cdot (a_1, \dots, a_{i-1})$;
    then $\mathcal{O}^{(i)}$ is $S^{(i)}$-orderly.
    
    Suppose we have found $b_1, \dots, b_{i-1}$ so that \[
        (a_1, \dots, a_{i-1}) \parallel (b_1, \dots, b_{i-1})
    \] is an $\mathcal{O}^{(i)}$-cog parallel ---
    note that $() \parallel ()$ is trivially a cog parallel in $\mathcal{O}^{(1)} = \{()\}$.
    As $S^{(i+1)} = S^{(i)} \setminus \{a_i\}$, a straightforward application of Lemma~\ref{lem:cog-building} with $Y \Coloneqq \{\}$ gives us an atom $b_i$ such that 
    \[
        (a_1, \dots, a_{i-1}, a_i) \parallel (b_1, \dots, b_{i-1}, b_i)
    \] is a cog parallel in $\mathcal{O}^{(i+1)}$.
    We are done when we reach $S^{(d+1)} = S$ and $\mathcal{O}^{(d+1)} = \mathcal{O}$.
    
    
    The automorphisms $\pi_1, \dots, \pi_d$ now come directly from homogeneity: 
    the map
    \begin{align*}
        a_1 \mapsto a_1, \dots,{} &a_i \mapsto b_i, \dots, a_d \mapsto a_d \\
        b_1 \mapsto b_1, \dots,{} &\phantom{a_i \mapsto b_i}, \dots, b_d \mapsto b_d, s \in S \mapsto s
    \end{align*}
    is an embedding because $a_\bullet \parallel b_\bullet$ is a cog parallel in $\mathcal{O} = \Aut(\A)_{(S)} \cdot a_\bullet [a_i \mapsto b_i] $.
\end{proof}

\begin{theorem}
    Given an $S$-orderly orbit $\mathcal{O}$,
    any non-zero $\Aut(\A)_{(S)}$-equivariant subspace of $\Lin_\FF \mathcal{O}$ contains $\Cog_\FF \mathcal{O}$.
\end{theorem}
\begin{proof}
    Let $V \subseteq \Lin_\FF \mathcal{O}$ be a non-zero $\Aut(\A)_{(S)}$-equivariant subspace and let $v \in V$ be a non-zero vector;
    then $v(a_\bullet) \neq 0$ for some $a_\bullet \in \mathcal{O}$.
    By lemma
    \begin{align*}
        v_1 &= v - \pi_1 \cdot v \\
        v_{1, 2} &= v_1 - \pi_2 \cdot v_1 \\
        &\cdots \\
        v_{1, 2, \dots, d} &= v_{1, 2, \dots, d-1} - \pi_d \cdot v_{1, 2, \dots, d-1}.
    \end{align*}
\end{proof}

\begin{corollary}
    $\Cog_\FF \mathcal{O}$ has length $1$.
\end{corollary}


\subsection{Projecting down}
\[
    (-)|_{I \setminus \{i\}} = (-)|_{-i}
\]

\subsection{Building up}
\[
    a^{(1)}_{i_1},
    a^{(2)}_{i_2},
    \dots,
    a^{(n)}_{i_n},
    b_*,
    s_1,
    \dots,
    s_m 
\]

\[
    o_{i_1},
    o_{i_2},
    \dots,
    o_{i_n},
    o_{N},
    s_1,
    \dots,
    s_m 
\]
\section{All those equivariant subspaces}\label{sec:equivariant-subspaces}
We continue working in the setting of Section~\ref{sec:free-amalg} so that we can demonstrate an important application of Theorem~\ref{thm:cog-span-generally}:
given an orbit-finite set $X$, we can describe every equivariant subspace $\Lin_\FF X$ through finite-dimensional vector spaces.
In particular, we will show that $\Lin_\FF \cal O$ has length $2^{|I|}$.

\subsection{Local coefficients}
First we set up some useful notation.
Consider the $2^{|I|}$ projected $S$-ordered orbits $\mathcal{O}|^{J}$ for $J \subseteq I$.
Suppose that
\[
    f : \mathcal{O}|^J \to \mathcal{O}|^{J'}
\]
is an $\Aut(\A)_{(S)}$-equivariant bijection.
Take any $a \in \mathcal{O}|^J$,
and enumerate its entries as $a_1 < \dots < a_{|J|}$.
Similarly, enumerate the entries of $f(a)$ as $b_1 < \dots < b_{|J'|}$.
Then $\{a_1, \dots, a_{|J|}\} = \{ b_1, \dots, b_{|J'|} \}$ because $\A$ has no algebraicity; 
since the orbits are ordered, we must have $|J| = |J'|$ and $a_1 = b_1, \dots, a_{|J|} = b_{|J'|}$.
That is, $f$ must be the obvious function that reindexes a $J$-tuple to a $J'$-tuple
--- hence we will write $a^{/J'}$ instead of $f(a)$, leaving $f$ implicit.

Now, let $\mathcal{Q}_{1} = \mathcal{O}|^{J_1}, \dots, \mathcal{Q}_{t} = \mathcal{O}|^{J_t}$ be distinct projected $S$-ordered orbits up to $\Aut(\A)_{(S)}$-equivariant bijections,
enumerated in such a way that $|J_1| \geq |J_2| \geq \dots \geq |J_t|$. (In particular, $J_1=I$ and $J_t=\emptyset$.)

%\begin{definition}
    For $i = 1, \dots, t$, let $\Jclass_i$ consist of all sets $J$ such that $\mathcal{O}|^J$ is in $\Aut(\A)_{(S)}$-equivariant bijection with $\mathcal{Q}_i$.
    Assemble all $|\Jclass_i|$ projections into a single map
    \[
        (-){\restriction_i} : \Lin_\FF\mathcal{O} \to \Lin_{\FF^{\Jclass_i}} \mathcal{Q}_i.
    \]
    To be more precise $v{\restriction_i}(a)$ is, for $a\in{\cal Q}_i$, the $\Jclass_i$-tuple whose $J$-th entry is $v|^J(a^{/J}) \in \FF$.
    It is straightforward to check that $(-){\restriction_i}$ is $\Aut(\A)_{(S)}$-equivariant and linear.
%\end{definition}

\begin{remark}\label{rem:FF-EE}
We must tread carefully here, as $\FF^{\Jclass_i}$ is not a field, so notation such as $\Lin_{\FF^{\Jclass_i}} \mathcal{O}$ may seem suspicious. However, $\FF^{\Jclass_i}$ is a vector space over $\FF$, and for any such space $\EE$ one can naturally see $\Lin_\EE\cal O$ and $\Ker_{\EE}\cal O$ as vector spaces over $\FF$. 

A little more care is needed when we want to define a vector in $\Lin_\EE \cal O$:
as $\lambda \in \EE$ is not a scalar in general, we need to understand
\[
    \lambda \cdot a
\]
as a formal expression.
Similarly, We redefine $\Cog_\EE\cal O$ to be spanned by formal expressions $\lambda\cdot(a^+ \between a^-)$ for $\lambda\in\EE$. 

But this changes little: in our proof of Theorem~\ref{thm:cog-span-generally} we only ever added or substracted vectors and cogs from one another, so that theorem holds even if $\EE$ is an arbitrary abelian group. This is not so for Theorem~\ref{thm:cogs-arise-everywhere}, whose proof involves scalar division, which is why in Claim~\ref{claim:cogs-arise-everywhere} below we need to prove a more subtle version of it. 
\end{remark}

Let $W \subseteq \Lin_\FF \mathcal{O}$ be an $\Aut(\A)_{(S)}$-equivariant subspace. Let $W {\restriction_i} (\mathcal{Q}_i) \subseteq \FF^{\Jclass_i}$ be the set of all $\Jclass_i$-tuples that can be obtained as $w {\restriction_i} (a)$ for some $w\in W$ and $a\in{\cal Q}_i$. This is a finite-dimensional vector space.
Now define $\widetilde W$, which consists of all vectors $v \in \Lin_\FF \mathcal{O}$ such that
\[
    v {\restriction_i} (\mathcal{Q}_i) \subseteq W {\restriction_i} (\mathcal{Q}_i) \qquad \text{for all } i=1,\ldots, t.
\]
Then $\widetilde W$ is an $\Aut(\A)_{(S)}$-equivariant subspace that contains $W$.
It turns out these two are equal:

\begin{lemma}\label{lem:coeff-approximation}
For every $i=0,\ldots, t$,
\[
        \widetilde W \cap \ker(\restriction_{i+1}) \cap \cdots \cap \ker(\restriction_t) 
        \subseteq W \cap \ker(\restriction_{i+1}) \cap \cdots \cap \ker(\restriction_t).
\]
    In particular $\widetilde W \subseteq W$ when $i = t$.
\end{lemma}


\paragraph{Lengths}
Let $W, W'$ be two $\Aut(\A)_{(S)}$-equivariant subspaces of $\Lin_\FF(\mathcal{O})$.
If we have $W {\restriction_i}(\mathcal{Q}_i) = W' {\restriction_i}(\mathcal{Q}_i)$ for all $i=1,\ldots, t$,
then $W = \widetilde W = \widetilde{W'} = W'$ by Lemma~\ref{lem:coeff-approximation}.
As a consquence:

\begin{proposition}\label{prop:length-upper-bound}
    Let $W_0 \subsetneq W_1 \subsetneq \cdots \subsetneq W_l$ be a chain of $\Aut(\A)_{(S)}$-equivariant subspaces of $\Lin_\FF(\mathcal{O})$.
    Then $l \leq 2^{|I|}$. 
\end{proposition}
\begin{proof}
    We obtain $t$ chains
    \begin{align*}
        W_0 {\restriction_1} (\mathcal{Q}_1) \subseteq W_1 {\restriction_1} (\mathcal{Q}_1) \subseteq \cdots \subseteq W_l {\restriction_1} (\mathcal{Q}_1) \subseteq \FF^{\Jclass_1}, \\
        W_0 {\restriction_2} (\mathcal{Q}_2) \subseteq W_1 {\restriction_2} (\mathcal{Q}_2) \subseteq \cdots \subseteq W_l {\restriction_2} (\mathcal{Q}_2) \subseteq \FF^{\Jclass_2}, \\
        \vdots \\
        W_0 {\restriction_t} (\mathcal{Q}_t) \subseteq W_1 {\restriction_t} (\mathcal{Q}_t) \subseteq \cdots \subseteq W_l {\restriction_t} (\mathcal{Q}_t) \subseteq \FF^{\Jclass_t}.
    \end{align*}
    At each of the $l$ steps, one of the $t$ containments must be strict.
    Hence $l \leq |\Jclass_1| + |\Jclass_2| + \dots + |\Jclass_t| = 2^{|I|}$. 
\end{proof}

It follows that every $\Aut(\A)_{(S)}$-equivariant subspace of $\Lin_\FF(\mathcal{O})$ is finitely generated.
We can compute the local coefficients of such subspaces easily:
\begin{remark}
    For $v \in \Lin_\FF(\mathcal{O})$, 
    let $\langle v \rangle$ denote the $\Aut(\A)_{(S)}$-equivariant subspace it generates.
    Then:
    \begin{enumerate}
        \item 
        $\langle v \rangle {\restriction_i}(\mathcal{Q}_i)$ is the subspace of $\FF^{\Jclass_i}$ generated by vectors of the form $v {\restriction_i} (a)$,
        which is zero unless every atom appearing in $a$ appears in $v$
        --- there are only finitely many such $a$'s;

        \item 
        $\langle v, v' \rangle {\restriction_i}(\mathcal{Q}_i) = \langle v \rangle {\restriction_i}(\mathcal{Q}_i) + \langle v' \rangle {\restriction_i}(\mathcal{Q}_i)$.
    \end{enumerate}
\end{remark}

To complement the upper bound from Prop.~\ref{prop:length-upper-bound}, we now exhibit a chain of $\Aut(\A)_{(S)}$-equivariant subspaces whose length is precisely $\sum_{i=1}^t |\Jclass_i|$, generalising \cite[Corollary~4.12]{BFKM24}.
Take any $J \in \Jclass_i$. 
Pick some $a\in \mathcal{O}$,
and let $\pi_j$ (for $j \in I$) be the automorphisms from Prop.~\ref{claim:cog-fresh-full}.
Define a vector
\begin{align*}
    v_{J} =  \prod_{j \in J} (1 - \pi_j) a \in \Lin_\FF(\mathcal{O}).
\end{align*}
Then 
\[
    (\langle v_{J} \rangle {\restriction_{i'}} (\mathcal{Q}_{i'}))_{J'} =
    \begin{cases}
        \FF & \text{if $J \subseteq J'$}, \\
        \{0\} & \text{otherwise}.
    \end{cases}
\]
Enumerating each $\Jclass_i$ in any order as $J_i^1, J_i^2, \dots, J_i^{|\Jclass_i|}$,
we obtain a chain
\begin{align*}
    \langle  \rangle 
    & \subsetneq{} \langle v_{J_t^1} \rangle 
    \subsetneq{} \langle v_{J_t^1}, v_{J_t^2} \rangle  
    \subsetneq{} \cdots 
    \subsetneq{} \langle v_{J_t^1}, v_{J_t^2}, \dots, v_{J_t^{|\Jclass_t|}} \rangle \\
   & \subsetneq{} \langle v_{J_t^1}, v_{J_t^2}, \dots, v_{J_t^{|\Jclass_t|}}, v_{J_{t-1}^1} \rangle 
     \subsetneq{} \cdots
\end{align*}
of length $|\Jclass_t| + |\Jclass_{t-1}| + \cdots + |\Jclass_1| = 2^{|I|}$.
Together with the upper bound from Proposition~\ref{prop:length-upper-bound}, we conclude:
\begin{theorem}
    $\len(\Lin_\FF\mathcal{O}) = 2^{|I|}$.
\end{theorem}
Along the lines of Proposition~\ref{claim:reduction-to-one-orbit}, this generalises to multi-orbit sets:
\[
\len(\Lin_\FF(\mathcal{O}_1 \uplus \dots \uplus \mathcal{O}_n)) = 2^{|I_1|} + \dots + 2^{|I_n|}.
\]

\section{The length-scape}
With Theorems~\ref{thm:weak-smooth-approximation-finite-length} and \ref{thm:ordered-free-amalg-has-finite-length} and their corollaries, we have extended the finite length property far beyond equality and ordered atoms, the two examples considered in~\cite{BFKM24}.
One quick way to produce even more examples is by considering an \emph{interpretation} in $\A$, 
which consists of an orbit-finite set $\B$ over $\A$ equipped with $\Aut(\A)$-equivariant relations.
($\B$ is called a \emph{first-order reduct} of $\A$ if its underlying set is just $\A$.)

\begin{lemma}\label{lem:interpretations-preserve-fin-len}
    If $\A$ has the finite length property over a field $\FF$, then so does any interpretation $\B$ in $\A$.
\end{lemma}
\begin{proof}
It is a standard result in model theory (see e.g.~\cite[Chap.~5,7]{hodges1993model}) that $\B$ is oligomorphic if $\A$ is.
%
    Now, consider a chain $V_0 \subset V_1 \subset \cdots \subset V_l \subseteq \Lin_\FF \B^d$ of $\Aut(\B)$-equivariant subspaces. 
    Then each $V_i$ is also $\Aut(\A)$-equivariant,
    and the finite length property of $\A$ yields an upper bound on $l$.
\end{proof}

As an example application:

\begin{theorem}
    Every countable homogeneous undirected graph has the finite length property over any field.
\end{theorem}

\begin{proof}
    We rely on the intricate classification result of Lachlan and Woodrow~\cite{LachlanWoodrow_80} which says that any countable homogeneous graph, or its complement,\footnote{Since a graph $\A$ and its complement share the same automorphism group, 
    when studying the length of $\Lin_\FF \A^d$ we only need to examine the former graph.} is isomorphic to: 
    \begin{enumerate}
        \item the Rado graph;
        \item the $K_n$-free Henson graph, where $n \geq 3$; or
        \item $m$ disjoint copies of the complete graph $K_n$, where at least one of $m$ and $n$ is infinite.
    \end{enumerate}
    
    For (1) and (2), the finite length property follows directly from Corollary~\ref{cor:free-finite-length}.
    The case (3) follows from Lemma~\ref{lem:interpretations-preserve-fin-len}, as those graphs admit interpretations in the equality atoms.
    \end{proof}

% Another recipe for producing more structures from $\A$ is by considering $(\A, {=}c_1, \dots, {=}c_k)$, where $c_1, \dots, c_k \in \A$ are seen as constants.
% Because $\A^{d+k}$ is orbit-finite for every $d$,
% this expansion is again oligomorphic if $\A$ is so. But we do not know whether the finite length property of the expansion can be deduced from the property for $\A$ in general. We only know a few positive examples:
% \begin{itemize}
%     \item 
%     Expansions of $(\N, =)$ or $(\Q, \leq)$ by finitely many constants are interpretable in the original structure~\cite[Lem~2.22]{BodirskyBodorMarimon_25},
%     so they have the finite length property by Theorem~\ref{thm:ordered-free-amalg-has-finite-length} in light of 
%     Lemma~\ref{lem:interpretations-preserve-fin-len}. (See also~\cite[Thm.~4.10]{BFKM24}.)
%     \item 
%     On the other hand, for $\A$ the Rado graph, it was shown in \cite{no-constant-interpreted-in-Rado} that $(\A, {=}c)$ does not interpret in $\A$ for any $c \in \A$. 
%     Instead, we explicitly took the constant into account when we established the finite length property in Corollary~\ref{cor:free-finite-length}.
% \end{itemize}


Let us summarise the scope of our results. In the cog-based approach of Sections~\ref{sec:free-amalg}--\ref{sec:equivariant-subspaces}, we cover the generically ordered Fraïssé limit of any free amalgamation class over a finite relational vocabulary of relational symbols of arity at most $2$. We do not know how to drop the arity restriction from our proofs, so it is interesting that recently, using a different approach, Evans proved~\cite[Prop.~3.10]{Evans_pm1} that $\Lin \A^2$ has finite length for a vocabulary of any arity. We do not know how to combine these results. 

Our main cog-based result is Theorem~\ref{thm:cog-span-generally}, 
which allows us in Section~\ref{sec:equivariant-subspaces} to describe all equivariant subspaces in any orbit-finite-dimensional vector space, and to deduce the finite length property with a tight bound.
However, as we saw in Section~\ref{sec:duals}, these structures can give rise to ill-behaved models of computation:
over the Rado and Henson graphs, orbit-finitely spanned sets do not admit orbit-finitely spanned function spaces, and weighted register automata are not closed under reversal.

The cog-less approach to oligomorphically approximated structures in Section~\ref{sec:characteristic-zero} is almost complementary.
The finite length property is proved there only over fields of characteristic $0$; indeed some structures covered in that setting, notably vector atoms (Ex.~\ref{ex:vector-space-atoms}), do not have the finite length property over finite fields~\cite[Sec.~4.4]{BFKM24}.
But the proof here is quick and elegant, and covers two important subclasses of oligomorphic structures. 
The first is $\omega$-stable structures, which admit orbit-finitely spanned dual spaces according to~\cite[Thm.~3.7]{Przybyłek_2024}.
The second is weakly Lie coordinatisable structures, which are thoroughly studied in \cite{CherlinHrushovski_03} and provide a rich variety of examples.

A few major questions remain open:
\begin{itemize}
\item \cite[Q.~2]{CaminaEvans_91} Does every oligomorphic structure have the ascending chain property? 
\item ~\cite[Q.~1.4]{Evans_pm1} Does every structure homogeneous over a finite vocabulary have the finite length property?
\item Does every oligomorphic structure have the finite length property over fields of characteristic $0$?
\end{itemize}
We do not even know whether the ascending chain property or the finite length property hold for some concrete and well-studied Fraïssé limits, notably the universal partial order or the countable atomless Boolean algebra.


\section*{Acknowledgements}
% The first-named author is partially funded by the NCN grant 2022/45/N/ST6/03242 (and IDUB?)

Hrushovski

Evans

Arka Ghosh

\printbibliography

\newpage
\appendix
\input{appendix-symplectic}
\section{Spanning by cogs: a proof}\label{sec:appendix-cogs}
Recall that setting of Section~\ref{sec:free-amalg} and the notations of Section~\ref{sec:cogs-turn}.
Consider an $S$-ordered orbit $\cal O \subseteq \A^I$.
The goal here is to prove Theorem~\ref{thm:cog-span-generally}.

\subsection{Subvectors, locations, conflicts}
We begin by introducing some additional terminology and notation. First, let us make explicit a view we have tacitly taken:
with $\cal O$ as a standard basis, a vector $v \in \Lin_\FF \cal O$ is just a finite set of pairs in $\FF \times \cal O$.
A \emph{subvector} of $v$ is a subset of these pairs. We write $\vsup{v}\subseteq {\cal O}$ for the set of tuples which are present in $v$. For a finite subset $T\subseteq\cal O$, we write $\sqrt T\subseteq\A$ for the set of atoms present anywhere in $T$.

For any $i \in I$ and $a \in \A$ (which is equal to $b_i$ for some $b \in \cal O$), we write 
\[
    \cal O^{i:a} = \{c \in \cal O \mid c_i = a\};
\]
this is an $\Aut(\A)_{(S a)}$-orbit (containing $b$), and its projection $\cal O^{i:a} |^{-i}$ is $Sa$-ordered.
For a vector $v \in \Lin_\FF \cal O$, by
\[
    v^{i:a} \in \Lin_\FF \cal O^{i:a}
\]
we mean the subvector of $v$ consisting of all pairs in $\FF \times \cal O^{i:a}$.

\begin{lemma}\label{lem:balanced-projected-subvector}
    Let $v \in \Lin_\FF \cal O$ be balanced. 
    Then any projected subvector $v^{i:a}|^{-i} \in \Lin_\FF \cal O^{i:a}|^{-i}$ is also balanced.
\end{lemma}
\begin{proof}
    Let $j \in I \setminus \{i\}$. 
    By assumption we have \[
        0 = v|^{-j} = \sum_a v^{i:a}|^{-j}.
    \]
    This sum is finite: it runs over those atoms $a$ that occur as the $i$-th entries in $\vsup{v}$.
    By looking at $i$-th entries, we see that each $v^{i:a}|^{-j}$ must be the zero vector.
    Hence so is $v^{i:a}|^{-j}|^{-i} = v^{i:a}|^{-i}|^{-j}$,
    which shows that $v^{i:a}|^{-i}$ is balanced.
\end{proof}

For a finite subset $T \subseteq \cal O$, a {\em location} in $T$ is a pair $(i,a)\in I\times\A$ such that $a=c_i$ for some $c\in T$. 
Note that for any fixed $i,j\in I$, for all $c\in\cal O$ the atoms $c_i$ and $c_j$ are related in the same way in $\A_0$ (i.e., with respect to equality and binary relations in $\sigma_0$). We say that two locations $(i,a)$ and $(j,b)$ in $T$ are in:
\begin{itemize}
\item an {\em equational conflict}, if $i\neq j$ but $a=b$, and
\item a {\em relational conflict}, if $a$ and $b$ are related in $\A_0$ but not in the same way as $c_i$ and $c_j$ for $c\in\cal O$.
\end{itemize}
(A situation where $a \neq b$ are not related by any relation in $\sigma_0$ at all does not constitute a conflict, even if $c_i$ and $c_j$ are related.) 
Recalling that $a$ and $b$ are related if they are equal, an equational conflict is a special case of a relational one.
A location in a vector $v$ means a location in the set $[v]$.

The prototypical examples of vectors which are free from any conflicts are cogs (or any subvectors of cogs). Note that the locations in a cog $a^+ \between a^-$ are exactly those in $\{a^+ ,a^-\}$, and these have no conflicts if $a^+ \parallel a^-$ is a duo.

In the following proof we will often manipulate many duos and cogs at once, so we will benefit from a concise notation for them. 
An $\cal O$-duo will be denoted by a single letter as $a^\pm$; its constituent parts will then be denoted by $a^+$ and $a^-$, so that $a^\pm=a^+\parallel a^-$. Sets of duos will be denoted with capital letters such as $A^\pm$, and sometimes we will slightly abuse this notation and write $A^\pm$ to mean $\bigcup_{a^\pm\in A^\pm}\{a^+,a^-\}$,
$A^+$ to mean $\bigcup_{a^\pm\in A^\pm}\{a^+\}$, 
and $A^+ a$ to mean $\bigcup_{a^\pm\in A^\pm}\{a^+ a\}$.

\subsection{Conflict resolution lemmas}
The following sister lemmas, relying on free amalgamation as distilled in Lemma~\ref{lem:free-fresh}, show how to merge conflict-free subsets of $\cal O$ in a way that avoids introducing new conflicts. This will be useful in Sections~\ref{subsec:unobstructed} and~\ref{subsec:unambiguous}.

\begin{lemma}\label{lem:?}
    Let $K, V_0 \subseteq V$ be finite subsets of $\cal O$ such that both $V_0\cup K$ and $V$ are free from equational conflicts. Then there exists a $\pi\in\Aut(\A)$ that, while fixeing all atoms in $S$ and in $\sqrt{V_0}$, makes $V \cup \pi(K)$ free from equational conflicts.
\end{lemma}
\begin{proof}
    Fix $V_0, V$ and induct on the number of equationally conflicting locations in $V \cup K$. 
    Take any such locations $(i,a)$ and $(j,a)$, where $i \neq j$; without loss of generality $(i,a)$ is a location in $K$ and $(j,a)$ a location in $V$. 

    Suppose that $(i', a)$ were a location in $V_0$.
    Since there are no equational conflicts in $V_0\cup K$ or in $V$, we see that $i = i'$ and $i' = j$, which is impossible.
    So $a \not\in \sqrt{V_0}$.
    Also $a\not\in S$, as $\cal O$ is $S$-ordered.
    Put: 
    \[
        X = S \cup \sqrt{K \cup V} \setminus \{a\},
    \]
    and note that $X$ contains $S \cup \sqrt{V_0}$.
    Use Lemma~\ref{lem:free-fresh} (putting $z=a$ and $Y=\emptyset$) to obtain an automorphism $\pi \in \Aut(\A/X)$ such that $\pi(a) \not\in X \cup \{a\}$. 
    
    In the set $V\cup\pi(K)$, the conflicting location $(i,a)$ disappears and no new equational conflicts are created, so the number of equationally conflicting locations drops compared to $V\cup K$.
    Because $V_0 \cup \pi(K)$ is still conflict-free,
    the inductive hypothesis gives us some $\pi' \in \Aut(\A)_{(S \cup \sqrt{V_0})}$ such that $V \cup \pi' \pi(K)$ is free from equational conflicts.
\end{proof}

\begin{lemma}\label{lem:!}
    Let $K, V_0 \subseteq V$ be finite subsets of $\cal O$ such that both $V_0\cup K$ and $V$ are free from relational conflicts. Then there exists a $\pi\in\Aut(\A)$ that, while fixing all atoms in $S$ and in $\sqrt{V_0}$, makes $V \cup \pi(K)$ free from relational conflicts.
\end{lemma}
\begin{proof}
    By \autoref{lem:?} we may assume that $V \cup K$ is free from {\em equational} conflicts.  As before, fix $V_0, V$ and proceed by induction on the number of relationally conflicting locations in $V \cup K$.
    
    Let $(i, a)$ and $(j, b)$ be in a relational conflict; without loss of generality $(i, a)$ is a location in $K$ and $(j, b)$ in $V$. This is not an equational conflict, so $a\neq b$ (but possibly $i=j$). 
    
    Since there are no conflicts in $V_0\cup K$ or in $V$, we see that $(i, a)$ is not a location in $V$ and $(j, b)$ is not a location in $V_0 \cup K$
    --- i.e., since there are no equational conflicts, that $a\not\in \sqrt{V}$ and $b\not\in\sqrt{V_0\cup K}$.
    Also, $a,b\not\in S$.
    Let $Y$ consist of all the atoms $b$ that are in a relational conflict with $(i,a)$ in $V\cup K$; we have just shown that $Y$ does not contain $a$ and is disjoint with $S\cup\sqrt{V_0\cup K}$. Put:
    \[
        X = S \cup \sqrt{K \cup V} \setminus (Y \cup \{a\}).
    \]
    Then $X, Y, \{a\}$ are pairwise disjoint, and $X$ contains $S \cup \sqrt{V_0}$. It also contains all atoms in $\sqrt{K}$ except $a$.
   Using Lemma~\ref{lem:free-fresh}, find some $\pi\in \Aut(\A/X)$ such that $\pi(a) \not\in X \cup Y \cup \{a\}$ and $\pi(a)$ is not related to any atom in $Y$. In $V\cup\pi(K)$ the conflicting location $(i,a)$ disappears and no new conflicts are created, so the conclusion follows from the inductive hypothesis.
\end{proof}

\subsection{Conflict-free vectors}\label{subsec:unobstructed}
\begin{claim}\label{claim:!-free-decomposition}
    If $v \in \Ker_\FF \cal O$ is free from conflicts, then it can be written as a linear combination of $\cal O$-cogs:
    \[
        v = \sum_{a^\pm \in A^\pm} \lambda_{a^\pm} \cdot a^+ \between a^-
    \]
    with $\lambda_{a^\pm} \in \FF$, where moreover $\vsup{v} \cup A^\pm$ is free from conflicts.
\end{claim}
\begin{proof}
We proceed by induction on the dimension $|I|$, 
noting that when $I = \emptyset$ we just have $v = \lambda \cdot () = \lambda \cdot ( \between )$.

So suppose $I$ is non-empty; let $j \in I$ be the greatest element. 
Group the terms in $v$ by their greatest atom so that $v = v^1 + v^2 + \cdots + v^k$.
We now induct on $k$.
If $k \leq 1$, we are done: as $v|^{-j} = 0$ we must have $v = 0$ (and $k = 0$), so the empty sum will do.
Otherwise \[
    v = v^{j:a} + v^{j:b} + v'
\]
for some $a\neq b\in \A$. By Lemma~\ref{lem:balanced-projected-subvector}, $v^{j:a}|^{-j}$ is balanced, and it is conflict-free, as every location in it is also a location in $v$.
By the outer inductive hypothesis, we get \[
    v^{j:a} = (v^{j:a}|^{-j}) a = \sum_{a^\pm\in A^\pm} (\lambda_{a^\pm} \cdot a^+ \between a^-)a
\]
where $[v^{j:a}|^{-j}]\cup A^\pm$ is free from conflicts, which immediately implies that $[v^{j:a}]\cup A^\pm a$ is free from conflicts as well. 
Note that if a $\pi \in \Aut(\A/S)$ fixes every atom in $v^{j:a}$ --- in particular, $a$ ---  then
\[
    v^{j:a} = \pi(v^{j:a}) = \sum_{a^\pm \in A^\pm} \lambda_{a^\pm} \cdot \pi a^+ \between \pi a^-,
\]
so by Lemma~\ref{lem:!} (putting $K=A^\pm a$, $V_0=[v^{j:a}]$, and $V=[v]$) we may assume without loss of generality that
$[v]\cup A^\pm a$ is free from conflicts.

Similarly, we can write \[
    v^{j:b} = \sum_{b^\pm\in B^\pm} (\mu_{b^\pm} \cdot b^+ \between b^-)b
\]
and apply Lemma~\ref{lem:!} again (putting $K=B^\pm b$, $V_0=[v^{j:b}]$, and $V=[v]\cup A^\pm a$) to conclude that
\[
    \vsup{v} \cup A^\pm a \cup B^\pm b
\]
is free from conflicts.

We now invent a new element $z$, on which we impose the following relations with $S \cup \sqrt{A^\pm a \cup B^\pm b} \subseteq \A$: 
\begin{enumerate}
    \item $a, b < z$, and $z < s$ iff $a, b < s$ for any $s \in S$;
    
    \item for any unary relation $P \in \sigma_0$:
    \[
        P(z) \;:\Longleftrightarrow\; P(a) \iff P(b);
    \]
    \item for any binary relation $R \in \sigma_0$ and $s \in S$, $a^\pm \in A^\pm$, $b^\pm \in B^\pm$, $i \in I \setminus \{d\}$:
    \begin{itemize}
        \item $R(z, s) \;:\Longleftrightarrow\; R(a, s) \iff R(b, s)$,
        \item $R(z, a^+_i) \;:\Longleftrightarrow\; R(a, a^+_i)$,
        \item $R(z, b^+_i) \;:\Longleftrightarrow\; R(b, b^+_i)$,
        \item $R(z, a^-_i) \;:\Longleftrightarrow\; R(a, a^-_i)$,
        \item $R(z, b^-_i) \;:\Longleftrightarrow\; R(b, b^-_i)$,
        \item $R(z, a)$ and $R(z,b)$ are both false,
        \item and symmetrically for $R(-, z)$.
    \end{itemize}
    These are consistent as there are no equational conflicts.
    (For instance, if $a^+_i = b^-_{i'}$ then $i = i'$, and $R(a, a^+_i) \iff R(b, b^-_i)$ holds since $a^+ a$ and $b^- b$ are both in $\cal O$.)
\end{enumerate}
To see that the $\sigma$-structure $S \cup \sqrt{A^\pm a \cup B^\pm b} \cup \{z\}$ still embeds into $\A$, 
suppose towards a contradiction that it contains a forbidden $\sigma_0$-substructure $F$.
Then $F$ must contain $z$.
Since any two elements in $F$ are necessarily related, we must have $a, b \not\in F$.
Similarly, whenever $F$ contains an atom $x_i$ for any $x\in A^\pm\cup B^\pm$, it does not contain $y_i$ for any other $y\in A^\pm\cup B^\pm$.
It follows that, fixing any $a^\pm\in A^\pm$,
\[
    s \mapsto s,\quad x_i \mapsto a^+_i,\quad z \mapsto a
\]
defines an injective function $\phi : F \to \A_0$,
which is furthermore an embedding (we only need to check this for pairs!) because $A^\pm \cup B^\pm$ is conflict-free and any $x_i, y_{i'}$ for $i \neq i'$ are related.
This is a contradiction. We may therefore assume that $z \in \A$.

It is now routine to check that $a^+ a \parallel a^- z$ and $b^+ b \parallel b^- z$ are $\cal O$-duos for all $a^\pm \in A^\pm, b^\pm \in B^\pm$,
and that 
\[
A^+ a \cup A^- z \cup B^+ b \cup B^- z
\]
is free from conflicts.
From Lemma~\ref{lem:!} we may assume that 
\[
\vsup{v} \cup A^+ a \cup A^- z \cup B^+ b \cup B^- z
\] 
is also free from conflicts.
(Alternatively, we could have explicitly ensured this when defining $z$.)
Then the vector:
\begin{align*}
    v'' &= v
    - \sum_{a^\pm\in A^\pm} \lambda_{a^\pm} \cdot a^+ a \between a^- z 
    - \sum_{b^\pm\in B^\pm} \mu_{b^\pm} \cdot b^+ b \between b^- z  \\
    &= v^{j:a}|^{-j} z + v^{j:b}|^{-j} z + v',
\end{align*}
when grouped into subvectors by the largest atom in each term, has at least one fewer component than $v$.
By the inner inductive hypothesis, we may write
\[
    v'' = \sum_{c^\pm\in C^\pm} \kappa_{c^\pm} \cdot c^+ \between c^-
\]
with $\vsup{v''} \cup C^\pm$ conflict-free, and one
last application of Lemma~\ref{lem:!} allows us to conclude that
\[
    \vsup{v} \cup A^+ a \cup A^- z \cup B^+ b \cup B^- z \cup C^\pm
\]
is conflict-free as well.
We conclude that
\begin{align*}
    v = 
      \sum_{a^\pm\in A^\pm} \lambda_{a^\pm} \cdot a^+ a \between a^- z 
    + \sum_{b^\pm\in B^\pm} \mu_{b^\pm} \cdot b^+ b \between b^- z \\
    + \sum_{c^\pm\in C^\pm} \kappa_{c^\pm} \cdot c^+ \between c^-
\end{align*}
is a decomposition of $v$ into a linear combination of $\cal O$-cogs as required.
\end{proof}

\subsection{Vectors without equational conflicts}\label{subsec:unambiguous}
\begin{claim}\label{claim:?-free-decomposition}
     If $v \in \Ker_\FF \cal O$ has no equational conflicts, then it can be written as a linear combination of $\cal O$-cogs:
    \[
        v = \sum_{a^\pm \in A^\pm} \lambda_{a^\pm} \cdot a^+ \between a^-
    \]
    with $\lambda_{a^\pm} \in \FF$, where moreover $\vsup{v} \cup A^\pm$ has no equational conflicts.
\end{claim}
\begin{proof}
We proceed again by induction, first on $|I|$ then on the number of relational conflicts in $v$.
The outer base case $I = \emptyset$ is trivial 
--- we have $v = \lambda \cdot ( \between )$, and no conflicts arise
--- and the inner base case is just Claim~\ref{claim:!-free-decomposition}.

Suppose that a location $(i, a)$ is part of a relational conflict in $v$.
Since every location in $v^{i:a}|^{-i}$ is also a location in $v$, we know that $v^{i:a}|^{-i}$ has no equational conflicts, and by Lemma~\ref{lem:balanced-projected-subvector} it is balanced.
By the outer inductive hypothesis, we get:
\[
    v^{i:a} = (v^{i:a}|^{-i}) a = \sum_{a^\pm\in A^\pm} (\lambda_{a^\pm} \cdot a^+ \between a^-) a
\]
where $[v^{i:a}|^{-i}]\cup A^\pm$ has no equational conflicts, which immediately implies that $[v^{i:a}]\cup A^\pm a$ is free from equational conflicts as well. 
By Lemma~\ref{lem:?} (putting $K=A^\pm a$, $V_0=[v^{i:a}]$, and $V=[v]$) we may assume that
$[v]\cup A^\pm a$ has no equational conflicts.

Take any location $(j,b)$ which is a in a relational conflict with $(i,a)$ in $v$. Then $b\not\in\sqrt{A^\pm a}$. To see this, note that $b=a$ would imply $j=i$ (since $v$ has no equational conflicts), but that would mean no conflict. On the other hand, if $b = a^+_k$ for some $a^+ \in A^\pm$ and $k \in I \setminus \{i\}$ (the case of $a^-$ is identical) then $j=k$, but this is not a conflict either, since $a^+a\in\cal O$ and $a=(a^+a)_i$.
Also, $b \not\in S$.

   Let $Y$ consist of all the atoms $b$ that are in a relational conflict with $(i,a)$ in $V$. 
   As $a \not\in S$ and $a \not\in \sqrt{A^\pm}$, we have shown that
   \[
X = S \cup \sqrt{\vsup{v} \cup A^\pm} \setminus (Y \cup \{a\})   
\]
contains $S \cup \sqrt{A^\pm}$ and that $X, Y, \{a\}$ are pairwise disjoint.
Use Lemma~\ref{lem:free-fresh} (putting $z=a$) to find $\tau \in \Aut(\A/X)$ such that $\tau(a) \not\in X \cup Y \cup \{a\}$, is greater than $a$, and is not related to any of $Y \cup \{a\}$. Denote $a'=\tau(a)$.
Then, for any $a^\pm \in A^\pm$, it follows by Lemma~\ref{lem:cog-fresh-single} that $a^+ a \parallel a^- a'$ is an $\cal O$-duo. Moreover, 
$\vsup{v} \cup A^+ a \cup A^- a'$
has no equational conflicts, and the vector
\begin{align*}
    v' = v - \sum_{a^\pm\in A^\pm} \lambda_{a^\pm} \cdot a^+ a \between a^- a'
    &= v -  v^{i:a} + v^{i:a'}%\\
  %  &= v^{i : a_i \mapsto \tau(a_i)}    
\end{align*}
has strictly fewer relationally conflicting locations than $v$, as the location $(i,a)$ disappears from it.
The inner inductive hypothesis tells us 
that we may write
\[
v' = \sum_{b^\pm\in B^\pm} \mu_{B^\pm} \cdot b^+ \between b^-
\]
with $\vsup{v'} \cup B^\pm$ free from equational conflicts.

Since $\vsup{v'} \subseteq \vsup{v} \cup A^+ a \cup A^- a'$, Lemma~\ref{lem:?} allows us to assume that 
\[
    \vsup{v} \cup A^+ a \cup A^-a' \cup B^\pm
\]
is also free from equational conflicts. We conclude that 
\[
    v =\sum_{a^\pm\in A^\pm} \lambda_{a^\pm} \cdot a^+ a \between a^-a'+ \sum_{b^\pm\in B^\pm} \mu_{b^\pm} \cdot b^+ \between b^-
\]
as required.
\end{proof}

\subsection{Arbitrary vectors}
We restate and prove Theorem~\ref{thm:cog-span-generally}:
\begin{theorem}
    Any $v \in \Ker_\FF \cal O$ can be written as
    \[
        v = \sum_{a^\pm \in A^\pm} \lambda_{a^\pm} \cdot a^+ \between a^-
    \]
    with $\lambda_{a^\pm} \in \FF$.
\end{theorem}
\begin{proof}
This is similar to the proof of Claim~\ref{claim:?-free-decomposition}, but simpler.
We proceed again by induction, first on $|I|$ then on the number of equational conflicts in $v$.
The outer base case $I = \emptyset$ is trivial as before, and 
and the inner base case is Claim~\ref{claim:?-free-decomposition}.

Suppose that a location $(i, a)$ is part of an equational conflict in $v$. By the outer inductive hypothesis, we get:
\[
    v^{i:a}
    = (v^{i:a}|^{-i}) a 
    = \sum_{a^\pm\in A^\pm} (\lambda_{a^\pm} \cdot a^+ \between a^-) a.
\]
Then neither $S$ nor $\sqrt{A^\pm}$ contains $a$,
so 
\[
X = S \cup \sqrt{\vsup{v} \cup A^\pm} \setminus \{a\}
\]
contains $S \cup \sqrt{A^\pm}$.
Using Lemma~\ref{lem:free-fresh} (putting $z=a$ and $Y=\emptyset$), find $\pi \in \Aut(\A/X)$ such that $\pi(a)$ is not in $X$, is greater than $a$, and is otherwise unrelated to $a$. Denote $a'=\tau(a)$. 
Then, for any $a^\pm \in A^\pm$, it follos by Lemma~\ref{lem:cog-fresh-single} that $a^+ a \parallel a^- a'$ is an $\cal O$-duo. Moreover, 
the vector
\begin{align*}
    v' = v - \sum_{a^\pm\in A^\pm} \lambda_{a^\pm} \cdot a^+ a \between a^- a'
    &= v -  v^{i:a} + v^{i:a'} 
\end{align*}
has strictly fewer equationally conflicting locations than $v$, as the location $(i,a)$ disappears from it.
It follows from the inner inductive hypothesis that we can write
\[
    v' = \sum_{b^\pm \in B^\pm} \mu_{b^\pm} \cdot b^+ \between b^-,
\]
which gives a decomposition of $v = \sum_{a^\pm \in A^\pm} \lambda_{a^\pm} \cdot a^+ a \between a^- a'$ as required.
\end{proof}
\begin{proof}
We proceed by induction on $i$.
The base case $i = 0$ is trivial, since $\ker(\restriction_1) = \{0\}$. Indeed, $\Jclass_1=\{I\}$, so $v{\restriction_1}$ is the identity map.

To prove the containment for some $i>0$, 
we allow $v {\restriction_i}$ to be non-zero.
But $v {\restriction_i}$ satisfies the next best property:
\begin{claim}
  The image of \[
        \widetilde W \cap \ker(\restriction_{i+1}) \cap \cdots \cap \ker(\restriction_t)
    \] under $\restriction_{i}$ 
    is contained in $\Ker_{W {\restriction}_{i} (\mathcal{Q}_{i})} \mathcal{Q}_{i}$. 
\end{claim}
\begin{proof}
    Take any $v \in \widetilde W$ such that  $v {\restriction_{i'}} = 0$ for all $i'>i$.
    That $v {\restriction_{i}} \in \Lin_{W{\restriction_{i}}(\mathcal{Q}_{i})} \mathcal{Q}_{i}$ is clear from the definition of $\widetilde W$.
    Recall that $\mathcal{Q}_{i}$ = $\mathcal{O}|^{J_{i}}$.
    For every $j \in J_{i}$,
    we need to prove that $v {\restriction_{i}} |^{-j} = 0$. In more elementary terms, given any $a\in\mathcal{O}|^{J_{i}\setminus\{j\}}$, we need the $J$-th entry of $v {\restriction_{i}} |^{-j}(a)$ to be $0$, for every $J\in \Jclass_i$.
    
    So take any $J \in \Jclass_{i}$.
    The unique ordered bijection between $J_{i}$ and $J$ restricts to one between $J_{i} \setminus \{j\}$ and $J \setminus \{j'\}$, for some $j'\in J$. Denote $J'=J \setminus \{j'\}$.
    Then $J'$ belongs to some $\Jclass_{i'}$ with $i' > i $,
    so $v {\restriction_{i'}} = 0$. Now calculate (with $a\in\mathcal{O}|^{J_{i}\setminus\{j\}}$, $b\in{\cal Q}_i$, $c\in {\cal O}|^J$ and $d\in\cal O$):
\begin{align*}
	(v{\restriction_i}|^{-j}(a))_J &= \sum_{b|^{-j}=a}(v{\restriction_i}(b))_J
	= \sum_{b|^{-j}=a}v|^J(b^{/J}) 
	= \sum_{c|^{-j'}=a^{/J'}}v|^J(c) \\
	&= \sum_{d|^{J'}=a^{/J'}}v(d)
	\,\, = v|^{J'}(a^{/J'}) = (v{\restriction_{i'}}(a))_{J'} = 0.
    \qedhere
\end{align*}    
\end{proof}

In light of Remark~\ref{rem:FF-EE}, from Theorem~\ref{thm:cog-span-generally} we get: 
\[\Ker_{W {\restriction_{i}} (\mathcal{Q}_{i})} \mathcal{Q}_{i} \subseteq \Cog_{W {\restriction_{i}} (\mathcal{Q}_{i})} \mathcal{Q}_{i}.\]
Furthermore:
\begin{claim}\label{claim:cogs-arise-everywhere}
    $\Cog_{W {\restriction}_{i} (\mathcal{Q}_{i})} \mathcal{Q}_{i}$ is contained in
    the image of \[
        W \cap \ker(\restriction_{i+1}) \cap \cdots \cap \ker(\restriction_t)
    \] under $\restriction_{i}$. 
\end{claim}
\begin{proof}
    Consider any ${\cal Q}_i$-cog with a coefficient $\lambda\in W {\restriction_{i}} (\mathcal{Q}_{i})$, and let $w\in W$ and $a\in{\cal Q}_i$ be such that $\lambda = w{\restriction_{i}}(a)$.
    Let $S'$ consist of $S$ together with every atom appearing in $w$ but not in $a$.
    We generalise the construction used in the proof of Theorem~\ref{thm:cogs-arise-everywhere}.
    
    Apply Proposition~\ref{claim:cog-fresh-full} and Remark~\ref{rem:duo} to get automorphisms $\pi_j$ for $j \in J_i$ such that $a \parallel \prod_{j \in J_i} \pi_j a$ is an $\mathcal{Q}_i$-duo, 
    where each $\pi_j$ fixes $S'$ and all $a_{j'}$ and $\pi_{j'}(a_{j'})$ for $j'\neq j$. Since all $\mathcal{Q}_i$-duos are in the same orbit, it is enough to show that the cog corresponding to this particular duo, with the coefficient $\lambda$, belongs to the $\restriction_i$-image as in the statement of the claim.
    
    
    Put: \[
        w' = \prod_{j \in J_{i}} (\mathrm{id} - \pi_j) w \in W.
    \]
    For any $1 \leq i' \leq t$, noting that as no more atoms can appear in $w{\restriction_{i'}}$ than in $w$, we have
    \begin{align*}
        &w' {\restriction_{i'}}
        = \prod_{j \in J_{i}} (\mathrm{id} - \pi_j) w {\restriction_{i'}} 
        = \sum_{c \in C_i} \sum_{J' \subseteq J_i} (-1)^{|J'|}  
            w {\restriction_{i'}}(c) \cdot \left(\prod_{j \in J'} \pi_j c\right)
%        &{}= \sum_{b_\bullet \in \mathcal{Q}_{i'}, \{b_j \mid j\} \supseteq \{a_j \mid j\}} \sum_{J' \subseteq J} (-1)^{|J'|}  
%            w {\restriction_{i'}}(b_\bullet) \prod_{j \in J'} \pi_j \cdot b_\bullet.
    \end{align*}
    where
        \[C_i = \{ c\in{\cal Q}_{i'} : \{c_j\mid j\in J_{i'}\} \supseteq \{a_j \mid j\in J_i\} \}.
    \]
    (The formula used in the proof of Theorem~\ref{thm:cogs-arise-everywhere} is a special case of this for $i'=1$ so that $J_{i'}=I$ and ${\cal Q}_{i'}={\cal O}$.) Now, if $i'>i$ then $C_i$ is empty, and so $w' {\restriction_{i+1}} = \cdots = w' {\restriction_{t}} = 0$. Moreover, if $i'=i$ then $C_i = \{a\}$ and we obtain the cog from before:
    \[
        w' {\restriction_{i}} = \lambda \cdot \left(a \between \prod_{j \in J_{i}} \pi_j a\right).
    \]
So $w'$ is a witness for the inclusion from the claim.
\end{proof}

This is enough to establish Lemma~\ref{lem:coeff-approximation} for $i$, assuming it for $i-1$.
Indeed, given $v \in \widetilde W \cap \ker(\restriction_{i+1}) \cap \dots \cap \ker(\restriction_t)$, by the preceding claims
we can find $w \in W \cap \ker(\restriction_{i+1}) \cap \dots \cap \ker(\restriction_t) \subseteq \widetilde W$ such that 
$v {\restriction_i} = w {\restriction_i}$.
But then $(v - w) {\restriction_{i}} = 0$, so $v - w$ lies in $\ker(\restriction_{i})$ as well as $\widetilde W \cap \ker(\restriction_{i+1}) \cap \cdots \cap \ker(\restriction_t)$.
It follows from the inductive hypothesis that \[
    v - w \in W \cap \ker(\restriction_i) \cap \ker(\restriction_{i+1}) \cap \dots \cap \ker(\restriction_t),
\]
so $v = (v - w) + w$ is in $W \cap \ker(\restriction_{i+1}) \cap \dots \cap \ker(\restriction_t)$ as well.

This completes the proof of Lemma~\ref{lem:coeff-approximation}.
\end{proof}


\end{document}