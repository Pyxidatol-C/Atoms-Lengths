\documentclass[conference]{IEEEtran}
% \IEEEoverridecommandlockouts
% The preceding line is only needed to identify funding in the first footnote. If that is unneeded, please comment it out.
%Template version as of 6/27/2024

% \usepackage{cite}
\usepackage{amsmath,amssymb,amsfonts}
\usepackage{algorithmic}
\usepackage{graphicx}
\usepackage{textcomp}
\usepackage{xcolor}
\def\BibTeX{{\rm B\kern-.05em{\sc i\kern-.025em b}\kern-.08em
    T\kern-.1667em\lower.7ex\hbox{E}\kern-.125emX}}

%##############################################################################
\usepackage{biblatex}
\addbibresource{atoms.bib}
\usepackage[hidelinks]{hyperref}

\usepackage{mathtools}
\usepackage{stmaryrd}

\usepackage{cancel}

\usepackage{amssymb}
\usepackage{amsthm}
% \numberwithin{equation}{section}\newcommand{\COUNTER}{equation}\newtheorem{theorem}[\COUNTER]{Theorem}
\newtheorem{theorem}{Theorem}[section]\newcommand{\COUNTER}{theorem}
\newtheorem{lemma}[\COUNTER]{Lemma}
\newtheorem{proposition}[\COUNTER]{Proposition}
\newtheorem{corollary}[\COUNTER]{Corollary}

\theoremstyle{definition}
\newtheorem{definition}[\COUNTER]{Definition}
\newtheorem{conjecture}[\COUNTER]{Conjecture}
\newtheorem{example}[\COUNTER]{Example}

\theoremstyle{remark}
\newtheorem*{remark}{Remark}

\usepackage{etoolbox}
\AtBeginEnvironment{proof}{\color{gray}}


\newcommand{\A}{\mathbb{A}}
\newcommand{\N}{\mathbb{N}} % naturals / names / equality atoms
\newcommand{\Q}{\mathbb{Q}} % rationals / ordered atoms
\newcommand{\G}{\mathbb{G}} % Rado graph 
\newcommand{\V}{\mathbb{V}} % bit vectors
\newcommand{\W}{\mathbb{W}} % symplectic --- w for omega, or « double v » for the vector pair e, f

\newcommand{\FF}{\mathfrak{F}} % field
\newcommand{\ff}{\mathfrak{f}} % finite field

\newcommand{\Aut}{\operatorname{Aut}}
\newcommand{\Lin}{\operatorname{Lin}}
\newcommand{\len}{\operatorname{len}}
\newcommand{\Cog}{\operatorname{Cog}}

%##############################################################################


\begin{document}





\title{More Vector Spaces with Atoms of Finite Lengths
}

\author{
    \IEEEauthorblockN{Jingjie Yang}
    \IEEEauthorblockA{University of Oxford %\\ Email: 
    }

\and
    \IEEEauthorblockN{Mikołaj Bojańczyk}
    \IEEEauthorblockA{University of Warsaw}

\and
    \IEEEauthorblockN{Bartek Klin}
    \IEEEauthorblockA{University of Oxford}
}

\maketitle

\begin{abstract}
*CRITICAL: Do Not Use Symbols, Special Characters, Footnotes, or Math in Paper Title or Abstract.
\end{abstract}


\section{Introduction}

\section{Rado graph, sans cogs}

$\A$ is: 
\begin{itemize}
    \item oligomorphic if, for $d = 0, 1, 2, \ldots$, $\A^d$ only has finitely many orbits;
    \item $\FF$-oligomorphic if, for $d = 0, 1, 2, \ldots$, $\Lin_\FF \A^d$ only has finitely long chains.
\end{itemize}
\textcolor{red}{
Of note: with Stirling numbers of the second kind and Gaussian $2$-binomial coefficients,
the orbit counts are given by
\begin{align*}
    \# \N^d &= \sum_{k=0}^d \begin{Bmatrix} d \\ k \end{Bmatrix} \\
    \# \Q^d &= \sum_{k=0}^d \begin{Bmatrix} d \\ k \end{Bmatrix} k! \\
    \# \G^d &= \sum_{k=0}^d \begin{Bmatrix} d \\ k \end{Bmatrix} 2^{\binom{k}{2}} \\
    \# \V_\infty^d &= \sum_{k=0}^d \begin{bmatrix} d \\ k \end{bmatrix}_2 \\
    \# \W_\infty^d &= \sum_{k=0}^d \begin{bmatrix} d \\ k \end{bmatrix}_2 2^{\binom{kW}{2}}
\end{align*}
To introduce: 
\begin{itemize}
    \item 
    \emph{smooth approximation} by \emph{homogeneous substructures} \cite{KLM89} (N.B. `smooth approximation' from \cite[Definition~4]{MP24} seems to be entirely different)
    \item 
    \emph{rough approximation} of a homogeneous structure by finite substructures with few orbits (i.e., types)
    that cover the age of $\A$
\end{itemize}}

\subsection{Symplectic vector spaces}
Throughout this subsection let $\ff$ denote a finite field.
\begin{definition}
    A \emph{symplectic vector space} is an $\ff$-vector space $\W$ 
    equipped with a bilinear form $\omega: \W \times \W \to \ff$ that is
    \begin{itemize}
        \item alternating: $\omega(v, v) = 0$ for all $v$; and
        \item non-degenerate: if $\omega(v, w) = 0$ for all $w$ then $v = 0$.
    \end{itemize}
\end{definition}

\begin{example}
    Let $\W_n$ be the $\ff$-vector space with basis $e_1  , \ldots, e_n, f_1, \ldots, f_n$.
    Define $\omega$ by bilinearly extending
    \begin{equation}\label{eq:symplectic-basis}
        \omega(e_i, f_i) = 1 = -\omega(f_i, e_i),\quad
        \omega(-, *) = 0 \text{ elsewhere;}
        \tag{\S}
    \end{equation}
    one may straightforwardly check that $\omega$ is alternating and non-degenerate.
    Moreover, noticing that $\W_0 \subseteq \W_1 \subseteq \W_2 \subseteq \cdots$,
    we obtain a countable-dimensional symplectic vector space $\W_\infty = \bigcup_n \W_n$.
\end{example}

We will take a straight-line path to prove the following in a self-contained manner.
Detailed expositions can be found in \cite[{\S}III.3]{Ar57}.
\begin{theorem}\label{thm:symplectic-smooth-approximation}
    The symplectic vector space $\W_\infty$ is smoothly approximated by 
    $\W_0 \subseteq \W_1 \subseteq \W_2 \subseteq \cdots$.
\end{theorem}

To begin with, we will refer to vectors satisfying \eqref{eq:symplectic-basis} as a \emph{symplectic basis} 
--- indeed, they must be linearly independent.
Such bases behave very much like the usual bases.

\begin{lemma}\label{lem:symplectic-basis}
    Assume that $\W$ is a symplectic vector space that is at most countable.
    Then any finite symplectic basis $e_1, \ldots, e_n, f_1, \ldots, f_n$ 
    can be extended to a symplectic basis that spans the whole $\W$.
\end{lemma}
\begin{proof}
    Suppose that $e_1, \ldots, e_n, f_1, \ldots, f_n$ does not already span $\W$;
    take $v$ to be a witness (that is least according to some fixed enumeration of $\W$ 
    in the case it is infinite).
    Put
    \[
        e_{n+1} = v - \sum_{i=1}^n \omega(e_i, v) f_i + \sum_{i=1}^n \omega(f_i, v) e_i
    \]
    so that $\omega(e_i, e_{n+1}) = 0 = \omega(f_i, e_{n+1})$.
    By the non-degeneracy of $\omega$, there is --- rescaling if necessary 
    --- some $w$ such that $\omega(e_{n+1}, w) = 1$. 
    Now define
    \[
        f_{n+1} = w - \sum_{i=1^n} \omega(e_i, w) f_i + \sum_{i=1}^n \omega(f_i, w) e_i
    \]
    in a similar manner, 
    making $e_1, \ldots, e_n, e_{n+1}, f_1, \ldots, f_n, f_{n+1}$ a symplectic basis that spans $v$.
    We go through every element of $\W$ by continuing this way.
\end{proof}

Given two symplectic vector spaces $\W$ and $\W'$,
we call a function $f$ between $X \subseteq \W$ and $X' \subseteq \W'$ \emph{isometric} if 
$\omega(f(x_1), f(x_2)) = \omega(x_1, x_2)$ for all $x_1, x_2 \in X$.

\begin{proposition}\label{prop:symplectic-unique-by-dimension}
    $\W_0, \W_1, \W_2, \ldots, \W_\infty$ are all the countable symplectic vector spaces 
    up to linear isometric isomorphisms.
\end{proposition}
\begin{proof}
    Let $\W$ be a countable symplectic vector space.
    By Lemma~\ref{lem:symplectic-basis}, we can extend the empty symplectic basis to one that spans $\W$ 
    --- call the basis vectors $e'_i, f'_i$.
    Then $e'_i \mapsto e_i, f'_i \mapsto f_i$ exhibits a linear isomorphism from $\W$ 
    to $\W_n$ or $\W_\infty$, which one easily checks to be isometric.
\end{proof}

In particular $\W_\infty$ is the essentially unique countable symplectic vector space;
by Ryll-Nardzewski it is oligomorphic.
We need one last fact about its finite counterparts.
\begin{proposition}[Witt Extension] \label{prop:symplectic-witt-extension}
    Any isometric injective linear map $f : X \subseteq \W_n \to \W_n$ 
    can be extended to an isometric linear isomorphism $\W_n \to \W_n$.
\end{proposition}
\begin{proof}
    We proceed by induction on the dimension of $X^{\perp X}$,
    where we write
    \[
        Y^{\perp Z} = \{y \in Y \mid \forall z \in Z : \omega(y, z) = 0\}
    \]
    for subspaces $Y, Z \subseteq \W_n$.
    
    Suppose first that $X^{\perp X} = X \cap \W_n^{\perp X}$ is the zero space.
    Notice that $\dim \W_n^{\perp X} = 2n - \dim X$:
    \begin{itemize}
        \item the map 
            $\W_n \to (\W_n \xrightarrow{\text{lin.}} \ff), v \mapsto \omega(v, -)$ 
        is linear and injective since $\omega$ is bilinear and non-degenrate, 
        so for dimension reasons it is also surjective;

        \item the restriction map 
            $(\W_n \xrightarrow{\text{lin.}} \ff) \to (X \xrightarrow{\text{lin.}} \ff)$ 
        is linear and surjective, 
        since any basis for $X \subseteq \W_n$ can be extended to one for $\W_n$;
        
        \item their composition 
            $v \mapsto \omega(v, -)|_X$ 
        is therefore linear, surjective, and has kernel $\W_n^{\perp X}$.
    \end{itemize}
    It follows that $\W_n^{\perp X}$ is the orthogonal complement of $X$ in $\W_n$: 
    by assumption and the above we have 
        $X \cap \W_n^{\perp X} = \{0\}$, $X + \W_n^{\perp X} = \W_n$, 
    and $\omega$ restricted to $X \times \W_n^{\perp X}$ is the zero function;
    we will use the notation \[
        \W_n = X \ominus \W_n^{\perp X}.
    \]
    On the other hand, as $f$ is isometric, 
    $f(X)^{\perp f(X)} = f(X^{\perp X})$ must also be the zero space,
    meaning $\W_n = f(X) \ominus \W_n^{\perp f(X)}$ by the same analysis.
    But $\dim \W_n^{\perp f(X)} = 2n - \dim f(X) = 2n - \dim X = \dim \W_n^{\perp X}$,
    so by Proposition~\ref{prop:symplectic-unique-by-dimension} there is a isometric 
    linear isomorphism $g : \W_n^{\perp X} \to \W_n^{\perp f(X)}$.
    Combining $f$ and $g$ yields a linear isomorphism 
    \begin{align*}
        \W_n = X \ominus \W_n^{\perp X} &\to f(X) \ominus \W_n^{\perp f(X)} = \W_n \\
        x + y &\mapsto f(x) + g(y)
    \end{align*} 
    which is isometric.
    
    Now suppose $X^{\perp X}$ contains some non-zero vector $x$.
    By extending $x$ to a basis of $X$ we can find a complement $Y$ for $\langle x \rangle$ in $X$;
    then $X = \langle x \rangle \ominus Y$ by the assumption on $x$.
    Writing $Z = \W_n^{\perp Y}$, 
    notice $Y \subseteq \W_n^{\perp Z}$ and $\dim Y = 2n - (2n - \dim Y) = \dim \W_n^{\perp Z}$.
    It follows that $\W_n^{\perp Z} = Y$ does not contain $x$,
    i.e., that some $z \in Z$ satisfies $\omega(x, z) = 1$.
    Consider $X' = \langle x, z \rangle + Y$;
    we must have 
    \[
        X' = \langle x, z \rangle \ominus Y,
    \]
    because if $\lambda x + \mu z \in \langle x, z \rangle$ lies also in $Y \subseteq X$,  
    then $0 = \omega(x, \lambda x + \mu z) = \mu$ and so $\lambda = 0$ too.    
    Similarly, as $f(x)$ is a non-zero vector in 
        $f(X)^{\perp f(X)}$ and $f(X) = \langle f(x) \rangle \ominus f(Y)$,
    we can find a vector $z'$ orthogonal to $\W_n^{\perp f(Y)}$ and satisfying $\omega(f(x), z') = 1$.
    Hence \[
        x \mapsto f(x), z \mapsto z', y \mapsto f(y)
    \]
    defines an isometric linear embedding
    \[
        f' : X' = \langle x, z \rangle \ominus Y 
        \to \langle f(x), z' \rangle \ominus f(Y) \subseteq \W_n
    \]
    extending $f$.
    Finally, we can apply the inductive hypothesis to extend $f'$.
    Indeed, if $v = \lambda x + \mu z + y$ is in $X'^{\perp X'}$ 
    then $\lambda = \omega(v, z) = 0 = \omega(v, x) = \mu$,
    so $v = y$ belongs to $X^{\perp X}$;
    as $x \in X^{\perp X} \setminus X'^{\perp X'}$, 
    we have $\dim X'^{\perp X'} \leq \dim X^{\perp X} - 1$.
\end{proof}

Smooth approximation is now immediate.

\begin{proof}[Proof of Theorem~\ref{thm:symplectic-smooth-approximation}]
    Firstly, observe that the restriction 
        $\Aut(\W_\infty)_{\{\W_n\}} \to \Aut(\W_n)$ 
    is surjective:
    any isometric linear automorphism of $\W_n$ maps the standard symplectic basis to another symplectic basis, 
    both of which can be extended to a symplectic basis of $\W_\infty$ by Lemma~\ref{lem:symplectic-basis}.

    Now suppose $\pi \in \Aut(\W_\infty)$ maps $(x_1, \dots, x_d) \in \W_n^d$ 
    to $(y_1, \dots, y_d) \in \W_n^d$.
    By Proposition~\ref{prop:symplectic-witt-extension}, we may extend 
        $\pi|_{\langle x_1, \dots, x_d \rangle} : \langle x_1, \dots, x_d \rangle \to \W_n$ 
    to some $f \in \Aut(\W_n)$ which still maps $(x_1, \dots, x_d)$ to $(y_1, \dots, y_d)$.
\end{proof}

\begin{corollary}
    Provided $\FF$ is of characteristic $0$, the symplectic $\ff$-vector space $\W_\infty$ is $\FF$-oligomorphic.
\end{corollary}




\subsection{Symplectic graphs}
For this subsection let $\ff$ be the two-element field.
\begin{definition}
    The \emph{symplectic graph} $\widetilde \W_n$ has vertices $\W_n$ and edges
    \[
        v_1 \sim v_2 \iff \omega(v_1, v_2) = 1.
    \]
    This is indeed an undirected graph: as $\omega$ is alternating, we have $\omega(v_1, v_2) = -\omega(v_2, v_1) = \omega(v_2, v_1)$.    
\end{definition}

\begin{proposition}\label{prop:symplectic-vs-graph}
    $\Aut(\widetilde \W_n) = \Aut(\W_n)$.
\end{proposition}
\begin{proof}
    Clearly any isometric linear automorphism of $\W_n$ is a graph automorphism of $\widetilde \W_n$.
    Conversely, any $f \in \widetilde \W_n$ is evidently isometric.
    To show that $f$ is linear, take $\lambda_1, \lambda_2 \in \ff$ and $v_1, v_2 \in \W$.
    We calculate:
    \begin{align*}
        &\omega\Bigl( f(\sum_i \lambda_i v_i) - \sum_i \lambda_i f(v_i) , f(w) \Bigr) \\
        ={}& \omega\Bigl( f(\sum_i \lambda_i v_i), f(w)\Bigr) - \sum_i \lambda_i \omega\bigl(f(v_i), f(w) \bigr) \\
        ={}& \omega\Bigl( \sum_i \lambda_i v_i, w \Bigr) - \sum_i \lambda_i \omega( v_i, w ) \\
        ={}& \omega(0, w) = 0
    \end{align*}
    for all $f(w) \in f(\W_n) = \W_n$;
    since $\omega$ is non-degenerate, 
    we conclude that $f(\sum_i \lambda_i v_i) = \sum_i \lambda_i f(v_i)$.
\end{proof}

\begin{proposition}
    The number of orbits in $\widetilde \W_n^d$ is at most 
        $\prod_{i=1}^{d} (2^{i-1} + 1) = O(2^{d (d-1) / 2})$ 
    for all $n$.
\end{proposition}
\begin{proof}
    By Proposition~\ref{prop:symplectic-vs-graph} and Theorem~\ref{thm:symplectic-smooth-approximation},
    the number of orbits in $\W_\infty^d$ is an upper bound;
    \textcolor{red}{this number is the OEIS sequence A028361.}
\end{proof}

\begin{proposition}[{\cite[Theorem~8.11.2]{GR01}}]
    Every graph on at most $2n$ vertices embeds into $\widetilde \W_n$.
\end{proposition}
\begin{proof}
    Let $G$ be a graph on at most $2n$ vertices. 
    The conclusion is trivial when $n = 0$.
    Also, if $G$ contains no edges, we can choose any $2n$ of the $2^n$ vectors in 
    $\langle e_1, \ldots, e_n \rangle \subseteq \widetilde \W_n$.
    
    So suppose $n \geq 1$ and $G$ has an edge $s \sim t$.
    Let $G_{s,t}$ be the graph on vertices $G \setminus \{s, t\}$ with edges which we will specify later.
    By induction, some embedding $f : G_{s, t} \to \widetilde \W_{n-1}$ exists.
    Define $f' : G \to \widetilde \W_n$ by
    \begin{align*}
        x \in G_{s, t} &\mapsto f(x) - \llbracket x \sim s \rrbracket f_n + \llbracket x \sim t \rrbracket e_n \\
        s &\mapsto e_n \\
        t &\mapsto f_n
    \end{align*}
    where $\llbracket \phi \rrbracket$ is $1$ if $\phi$ holds and $0$ otherwise.
    Then we have $\omega(f'(x), f'(s)) = \llbracket x \sim s \rrbracket$ 
    and $\omega(f'(x), f'(t)) = \llbracket x \sim t \rrbracket$ as desired, on one hand.
    On the other,
    \begin{align*}
        \omega( f'(x_1), f'(x_2) ) 
        = \llbracket x_1 \sim x_2 \rrbracket 
        &+ \llbracket x_1 \sim s \rrbracket \llbracket x_2 \sim t \rrbracket \\
        &+ \llbracket x_1 \sim t \rrbracket \llbracket x_2 \sim s \rrbracket
    \end{align*}
    tells us how we should define the edge relation in $G_{s,t}$ for $f'$ to be an embedding of graphs.
\end{proof}

\begin{theorem}
    The Rado graph is roughly approximated by $\widetilde \W_0 \subseteq \widetilde \W_1 \subseteq \widetilde \W_2 \subseteq \cdots$.
\end{theorem}

\begin{corollary}
    Provided $\FF$ is of characteristic $0$, the Rado graph is $\FF$-oligomorphic.
\end{corollary}

\section{Rado graph, with cogs}
In this section we work with the following setting:
\begin{itemize}
    \item 
    $\mathcal{L}_0$ is a (possibly infinite) \textcolor{magenta}{BINARY?} relational language containing a binary symbol $=$;

    \item 
    $\mathcal{C}_0$ is a free amalgamation class of $\mathcal{L}_0$-structures
    where $=$ is interpreted as true equality, but every other $R \in \mathcal{L}_0$ is interpreted irreflexively.\footnote{%
        We can enforce irreflexivity by considering a language $\mathcal{L}'_0$ which consists, 
        for each $R \in \mathcal{L}_0 \setminus \{=\}$ of arity $r$ and each partition $\P$ of $r$ into $k$ parts, 
        of a $k$-ary relation symbol $R_\P$.
        Then $\mathcal{L}_0$-structures may be viewed as $\mathcal{L}'_0$-structures and vice versa, without changing the meaning of embeddings.
        In this way, we get a free amalgamation class $\mathcal{C}'_0$ with a Fraïssé limit which, viewed as an $\mathcal{L}_0$-structrure, is isomorphic to $\A_0$.
    } 

    \item 
    $\mathcal{L}$ consists of $\mathcal{L}_0$ together with a new binary symbol $<$;

    \item 
    $\mathcal{C}$ consists of $\mathcal{L}$-structures obtained from $\mathcal{C}_0$ by expanding with all possible linear orderings;

    \item 
    $\A_0$ and $\A$ are the respective Fraïssé limits of $\mathcal{C}_0$ and $\mathcal{C}$,
    where without loss of generality we assume $\A_0$ and $\A$ share the same domain so that $\Aut(\A_0) \supseteq \Aut(\A)$.
\end{itemize}

\begin{example}\label{ex:N-Q}
    Take $\mathcal{L}_0$ consist of $=$ only and let $\mathcal{C}_0$ to be all finite sets.
    Then $\A_0$ is isomorphic to the pure set $\N$, whereas $\A$ is isomorphic to $\Q$ with the usual order.
\end{example}

\begin{example}\label{ex:Rado-orderedRado}
    Let $\mathcal{L}_0$ consist of $=$ together with a single binary symbol $\sim$ 
    and let $\mathcal{C}_0$ consist of all finite undirected graphs not embedding the complete graph $K_n$,
    where $3 \leq n$ ($\leq \infty$).
    Then $\A_0$ is the $K_n$-free Henson graph (or the Rado graph when $n = \infty$), and $\A$ is its generically ordered counterpart.
    (Allowing $n = 2$ makes these degenerate to $\N$ and $\Q$ above).
\end{example}

We note two technicalities and a triviality.

\begin{lemma}\label{lem:free-forb}
    Let $\mathcal{F}_0$ consist of minimal $\mathcal{L}_0$-structures which do not appear in $\mathcal{C}_0$.
    Then
    \begin{enumerate}
        \item $\mathcal{C}_0$ consists of every $\mathcal{L}_0$-structure that does not embed any $F \in \mathcal{F}_0$.
        \item $\mathcal{C}$ consists of every $\mathcal{L}$-structure whose $\mathcal{L}_0$-reduct does not embed any $F \in \mathcal{F}_0$.
        \item Given $F \in \mathcal{F}_0$, every two distinct $x, y \in F$ are related by some $R \in \mathcal{L}_0$.
    \end{enumerate}
\end{lemma}
\begin{proof}
    As $\mathcal{C}_0$ is closed under substructures, its complement is closed under superstructures and thus 
    --- since there are no infinite strictly descending chain of embedded substructures 
    --- determined by its minimal structures.
    2) follows because an $\mathcal{L}$-structure is in $\mathcal{C}$ precisely when its $\mathcal{L}_0$-reduct is in $\mathcal{C}_0$.
    For 3), notice that $F \setminus \{x\}$, $F \setminus \{y\}$ are in $\mathcal{C}_0$ by minimality; 
    therefore so is their free amalgam over $F \setminus \{x, y\}$, which then cannot agree with $F$.
\end{proof}

\begin{lemma}\label{lem:free-fresh}
    Let $X, Y, \{z\} \subseteq \A$ be disjoint and finite.
    Then there is some automorphism $\tau \in \Aut(\A)$ such that
    \begin{enumerate}
        \item $\tau$ fixes every $x \in X$;
        \item $\tau(z)$ does not appear together with any $y \in Y$ or with $a$ in any tuple $a_\bullet \in (X \cup Y \cup \{z, \tau(z)\})^*$ such that $\A \models R(a_\bullet)$ for some $R \in \mathcal{L}_0$;
        \item $\tau(z) > z$.
    \end{enumerate}
\end{lemma}
\begin{proof}
    In $\A_0$, form the free amalgam
% https://q.uiver.app/#q=WzAsNCxbMCwxLCJYIl0sWzEsMiwiWCBcXGN1cCBcXHt6XFx9Il0sWzEsMCwiWCBcXGN1cCBZIFxcY3VwIFxce3pcXH0iXSxbMiwxLCJYIFxcY3VwIFkgXFxjdXAgXFx7eiwgeidcXH0iXSxbMCwxLCIiLDAseyJzdHlsZSI6eyJ0YWlsIjp7Im5hbWUiOiJob29rIiwic2lkZSI6InRvcCJ9fX1dLFswLDIsIiIsMCx7InN0eWxlIjp7InRhaWwiOnsibmFtZSI6Imhvb2siLCJzaWRlIjoidG9wIn19fV0sWzEsMywieCBcXGluIFggXFxtYXBzdG8geCwgeiBcXG1hcHN0byB6JyIsMSx7InN0eWxlIjp7InRhaWwiOnsibmFtZSI6Imhvb2siLCJzaWRlIjoidG9wIn0sImJvZHkiOnsibmFtZSI6ImRhc2hlZCJ9fX1dLFsyLDMsIlxcc3Vic2V0ZXEiLDEseyJzdHlsZSI6eyJ0YWlsIjp7Im5hbWUiOiJob29rIiwic2lkZSI6InRvcCJ9LCJib2R5Ijp7Im5hbWUiOiJkYXNoZWQifX19XV0=
\[\begin{tikzcd}[cramped]
	& {X \cup Y \cup \{z\}} \\
	X && {X \cup Y \cup \{z, z'\}} \\
	& {X \cup \{z\}}
	\arrow["\subseteq"{description}, dashed, hook, from=1-2, to=2-3]
	\arrow[hook, from=2-1, to=1-2]
	\arrow[hook, from=2-1, to=3-2]
	\arrow["{x \in X \mapsto x, z \mapsto z'}"{description}, dashed, hook, from=3-2, to=2-3]
\end{tikzcd}\]
    so that no element of $Y \cup \{z\}$ is related with $z'$ by any $R \in \mathcal{L}_0$.
    Now we make $X \cup Y \cup \{z, z'\}$ an $\mathcal{L}$-structure: 
    inherit the order on $X \cup Y \cup \{z\}$ from $\A$,
    and declare that $z < z'$ as well as $z' < a$ if $a$, the next element of $X \cup Y$ larger than $z$, exists at all.
    Notice that \[
        x \in X \mapsto x, z \mapsto z'
    \] is still an embedding in presence of the order.
    By homogeneity, we may embed $X \cup Y \cup \{z, z'\}$ into $\A$ via some $f$ which is the identity on $X \cup Y \cup \{z\}$;
    again by homogeneity, we may extend the embedding \[ 
        f(x) = x \in X \mapsto f(x), f(z) \mapsto f(z')
    \] to some automorphism $\tau$ which makes 1), 2), and 3) true.
\end{proof}

\begin{proposition}
    The $S$-supported length of $\Lin {\A_0}^d$ is at most that of $\Lin \A^d$ for any finite $S \subseteq \A_0 = \A$.
\end{proposition}
\begin{proof}
    Any chain of subspaces in $\Lin {\A_0}^d = \Lin \A^d$ that are invariant under $\Aut(\A_0)_{(S)}$ must also be invariant under the subgroup $\Aut(\A)_{(S)}$.
\end{proof}









\subsection{Cogs in an orbit}
An inconvenience of $\A^d$ is that it may have many orbits.
\begin{definition}
    Let $S \subseteq \A$ be finite.
    We say an orbit $\mathcal{O} \subseteq \A^d$ is $S$-orderly 
    if $\mathcal{O} = \Aut(\A)_{(S)} \cdot o_\bullet$ for some/any $o_\bullet \in \mathcal{O}$ 
    where $o_1 < \dots < o_d$ and $o_1, \dots, o_d \not\in S$.
\end{definition}
By removing entries of $o_\bullet$ that repeat or come from $S$ and reordering the rest,
we can always find an $\Aut(\A)_{(S)}$-equivariant bijection to an $S$-orderly orbit.
Moreover, we may focus on a single $S$-orderly orbit at a time:

\begin{proposition}
    The following are equivalent for any finite $S \subseteq \A$:
    \begin{enumerate}
        \item $\A_S$ (that is, $\A$ with constants from $S$ fixed) is $\FF$-oligomorphic;
        \item $\A$ is oligomorphic and for every $S$-orderly orbit $\mathcal{O}$, the $S$-supported length of $\Lin_\FF \mathcal{O}$ is finite.
    \end{enumerate}
\end{proposition}
\begin{proof}
    Indeed we have
    \(
        \len(\Lin_\FF \A_S^d) 
        = \len(\Lin_\FF(\biguplus_i \mathcal{O}_i))
        = \len(\bigoplus_i \Lin_\FF \mathcal{O}_i) 
        = \sum_i \len(\Lin_\FF \mathcal{O}_i)
    \), where the $\mathcal{O}_i$'s are the $S$-orderly counterparts of the $\Aut(\A)_{(S)}$-orbits in $\A^d$.
\end{proof}

We now introduce the workhorse for understanding $\Lin_\FF \mathcal{O}$.
\begin{definition}
    Let $\mathcal{O} \subseteq \A^d$ be an $S$-orderly orbit.
    An \emph{$\mathcal{O}$-cog parallel} $a_\bullet \parallel b_\bullet$ consists of atoms $a_1 < b_1 < a_2 < b_2 < \dots < a_d < b_d$
    satisfying the following: 
    for some/any $o_\bullet \in \mathcal{O}$, 
    for every relation $R \in \mathcal{L}$ of arity $r$,
    and for each $r$-tuple $x_\bullet$ with entries in $\{a_1, \dots, a_d, b_1, \dots, b_d\} \cup S$,
    we have
    \[
        \A \models R(x_\bullet) \leftrightarrow R(x_\bullet [a_i \mapsto o_i, b_i \mapsto o_i \mid 1 \leq i \leq d]).
    \]    
    (By $x_\bullet [ a \mapsto b, c \mapsto d ]$ we mean the $r$-tuple where each entry $x_i$ equal to $a$ is replaced by $b$, and each entry equal to $c$ --- assumed to be distinct from $a$ --- is replaced by $d$).
\end{definition}

In particular, for all $I \subseteq \{1, \dots, d\}$ we may take $x_\bullet$ above to have entries in $\{a_i \mid i \in I\} \cup \{b_j \mid j \not\in I\} \cup S$, showing that
\[
    \begin{cases}
        a_i \mapsto o_i, &i \in I; \\ 
        b_j \mapsto o_j, &j \not\in I; \\
        s \mapsto s & s \in S
    \end{cases}
\]
defines an embedding.
It follows from homogeneity that $a_\bullet [a_i \mapsto b_i \mid i \in I]$ lies in the orbit $\mathcal{O} = \Aut(\A)_{(S)} \cdot o_\bullet$.

\begin{definition}
    The $\mathcal{O}$-cog corresponding to an $\mathcal{O}$-cog parallel $a_\bullet \parallel b_\bullet$ is the vector
    \[
        a_\bullet \between b_\bullet = \sum_{I \subseteq \{1, \dots, d\}} (-1)^{ |I| } a_\bullet [a_i \mapsto b_i \mid i \in I] \in \Lin_\FF \mathcal{O}.
    \]
    The linear span of all $\mathcal{O}$-cogs is denoted by $\Cog_\FF \mathcal{O}$.
\end{definition}

\begin{proposition}\label{prop:cog-equivariant}
    Let $\mathcal{O}$ be $S$-orderly.
    Then $\Cog_\FF \mathcal{O}$ is an $\Aut(\A)_{(S)}$-equivariant subspace of $\Lin_\FF \mathcal{O}$ generated by any single $\mathcal{O}$-cog.
\end{proposition}
\begin{proof}
    Suppose $a_\bullet \parallel b_\bullet$ is an $\mathcal{O}$-cog parallel.
    The definition completely specifies the $\mathcal{L}$-structure on $\{a_1, b_1, \dots, a_d, b_d\} \cup S$
    and says that
    \[
        a_i \mapsto a'_i, b_i \mapsto b'_i, s \mapsto s
    \]
    is an isomorphism given another $\mathcal{O}$-cog parallel $a'_\bullet \parallel b'_\bullet$.
    Homogeneity then yields an automorphism $\pi \in \Aut(\A)_{(S)}$ satisfying $\pi \cdot (a_\bullet \between b_\bullet) = a'_\bullet \between b'_\bullet$.
\end{proof}

Though the definitions were a mouthful, the example below should explain how cogs arise.
\begin{example}
    Let $\A = \Q$ as described in Example~\ref{ex:N-Q};
    there is a unique $\{\}$-orderly orbit $\mathcal{O}$ in $\A^2$.
    Consider the vector
    \[
        v = (0, 4) + (4, 9) - (9, 10) - (0, 10)
    \]
    in $\Lin \mathcal{O}$.
    We can find $4 < 4+\varepsilon < 9 < 9+\delta < 10$ in $\A$
    together with monotone bijections $\pi_1, \pi_2 \in \Aut(\A)$ such that
    \[
        \pi_1 : 
        \begin{cases}
            0 \mapsto 0, \\
            4 \mapsto 4 + \varepsilon, \\
            9 \mapsto 9, \\
            10 \mapsto 10;
        \end{cases}
        \pi_2 : 
        \begin{cases}
            0 \mapsto 0, \\
            4 \mapsto 4, \\
            9 \mapsto 9 + \delta, \\
            10 \mapsto 10
        \end{cases}        
    \]
    by interpolating linearly for example.
    Then
    \begin{align*}
        v_1 
        = v - \pi_1 \cdot v
        ={}& (0, 4) + (4, 9) - (4, 10) \\
        &-(0, 4+\varepsilon) - (4+\varepsilon, 9) + (4+\varepsilon, 10)
    \end{align*}
    duplicates the tuples with $4$ in it but kills the one without it.
    Similarly 
    \begin{align*}
        v_{1,2}
        = v_1 - \pi_2 \cdot v_1
        ={}& (4, 9) - (4, 9+\delta) \\
        &-(4+\varepsilon, 9) + (4+\varepsilon, 9+\delta)
    \end{align*}
    only leaves and duplicates the tuples with $9$ in it.
    Here $v_{1,2}$ is the parallel for the cog $(4, 9 \parallel 4 + \varepsilon, 9 + \delta)$ in $\mathcal{O}$
    as well as the smaller $\{0, 10\}$-orderly orbit $\mathcal{O}' = \Aut(\A)_{(0, 10)} \cdot (4, 9) \subseteq \mathcal{O}$.
\end{example}

To find cog parallels in general, we iterate the following procedure.
\begin{lemma}\label{lem:cog-building-one}
    Let $a_\bullet \parallel b_\bullet$ be an $\mathcal{O}$-cog parallel, where $\mathcal{O} \subseteq \A^d$ is $S$-orderly.
    Given $s \in S$ with $a_{j-1} < s < a_j$ (where we treat $a_0$ and $a_{d+1}$ as $\pm \infty$), 
    we write $S' = S \setminus \{s\}$ and let $a_\bullet ;_j s \in \A^{d+1}$ be the tuple obtained by inserting $s$ in $a_\bullet$ as the $j$th entry.
    Then \[
        \mathcal{O}' = \Aut(\A)_{(S')} \cdot a_\bullet ;_j s \subseteq \A^{d+1}
    \]
    is $S'$-orderly.

    Apply Lemma~\ref{lem:free-fresh} with $X \Coloneqq \{a_1, b_1, \dots, a_d, b_d\} \cup S'$, $z \Coloneqq s$, and any finite $Y \subseteq \A$ disjoint from $X \cup \{z\}$ to obtain $\tau \in \Aut(\A)_{(X)}$ and $\tau(z) \Eqqcolon s'$.
    Then $a_\bullet ;_j  s \parallel b_\bullet ;_j s'$ is an $\mathcal{O}'$-cog parallel.
    
\end{lemma}
\begin{proof}
    Already $a_{j-1} < b_{j-1} < s$ (if $j > 1$), $s < s'$, and $s' < a_j < b_j$ (if $j \leq d$).
    Now pick any relation $R \in \mathcal{L}$ of arity $r$ and take any $x_\bullet \in (\{a_1, b_1, \dots, a_d, b_d, s, s'\} \cup S')^r = (X \cup \{s, s'\} \cup Y)^r$.
    We split into three cases.
    
    If $s$ and $s'$ both appear in $x_\bullet$, then we have
    \begin{equation}\label{eq:cog-building-relations}
        \A \models R(x_\bullet [ b_i \mapsto a_i, s' \mapsto s \mid 1 \leq i \leq d ] ) \leftrightarrow R(x_\bullet). \tag{\P}
    \end{equation}
    Indeed, the left is false because $R$ is irreflexive;
    so is the right by the design of $s'$.

    Now if $s$ appears in $x_\bullet$ we may assume $s'$ does not.
    This time around we have \eqref{eq:cog-building-relations} as $x_\bullet \in (\{a_1, b_1, \dots, a_d, b_d\} \cup S)^r$ 
    --- so we can ignore the $[s' \mapsto s]$ substitution ---
    and $a_\bullet \parallel b_\bullet$ is a cog parallel in $\mathcal{O} = \Aut(\A)_{(S)} \cdot a_\bullet$.

    Finally, suppose $s'$ appears in $x_\bullet$ and $s$ does not.
    Then $x_\bullet [s' \mapsto s] = \tau^{-1} \cdot x_\bullet$ and
    \begin{align*}
        x_\bullet [ b_i \mapsto a_i, s' \mapsto s \mid 1 \leq i \leq d ] \\
        = (\tau^{-1} \cdot x_\bullet) [ b_i \mapsto a_i \mid 1 \leq i \leq d ]
    \end{align*}
    where $\tau^{-1} \cdot x_\bullet \in (\{a_1, b_1, \dots, a_d, b_d\} \cup S)^r$.
    On the one hand, as discussed in the case above we get
    \[
        \A \models R((\tau^{-1} \cdot x_\bullet) [ b_i \mapsto a_i \mid 1 \leq i \leq d ]) \leftrightarrow R(\tau^{-1} \cdot x_\bullet).
    \]
    On the other hand, we certainly have
    \[
        \A \models R(\tau^{-1} \cdot x_\bullet) \leftrightarrow R(x_\bullet)
    \]
    since $\tau$ is an automorphism.
    This establishes \eqref{eq:cog-building-relations} again,
    showing that $a_\bullet s \parallel b_\bullet s'$ is an $\mathcal{O}'$-cog parallel.
\end{proof}

\begin{proposition}\label{prop:cog-building-whole}
    Let $\mathcal{O} \subseteq \A^d$ be $S$-orderly.
    Then, given $a_\bullet \in \mathcal{O}$, there exists $b_\bullet \in \mathcal{O}$ such that $a_\bullet \parallel b_\bullet$ is an $\mathcal{O}$-cog parallel.
    Moreover, for $i = 1, \dots, d$ there is an automorphism $\pi_i \in \Aut(\A)_{(\{a_1, b_1, \dots, a_{i-1}, b_{i-1}, a_{i+1}, b_{i+1}, \dots, a_d, b_d\} \cup S)}$ sending $a_i \mapsto b_i$.
\end{proposition}
\begin{proof}
    For $1 \leq i \leq d+1$, let $S^{(i)} = S \cup \{a_i, \dots, a_d\}$ and let $\mathcal{O}^{(i)} = \Aut(\A)_{(S^{(i)})} \cdot (a_1, \dots, a_{i-1})$;
    then $\mathcal{O}^{(i)}$ is $S^{(i)}$-orderly.
    
    Suppose we have found $b_1, \dots, b_{i-1}$ so that \[
        (a_1, \dots, a_{i-1}) \parallel (b_1, \dots, b_{i-1})
    \] is an $\mathcal{O}^{(i)}$-cog parallel ---
    note that $() \parallel ()$ is trivially a cog parallel in $\mathcal{O}^{(1)} = \{()\}$.
    As $S^{(i+1)} = S^{(i)} \setminus \{a_i\}$, a straightforward application of Lemma~\ref{lem:cog-building-one} with $Y \Coloneqq \{\}$ gives us an atom $b_i$ such that 
    \[
        (a_1, \dots, a_{i-1}, a_i) \parallel (b_1, \dots, b_{i-1}, b_i)
    \] is a cog parallel in $\mathcal{O}^{(i+1)}$.
    We are done when we reach $S^{(d+1)} = S$ and $\mathcal{O}^{(d+1)} = \mathcal{O}$.
    
    
    The automorphisms $\pi_1, \dots, \pi_d$ now come directly from homogeneity and the definition of an $\mathcal{O}$-cog parallel: 
    the map
    \begin{align*}
        a_1 \mapsto a_1, \dots,{} &a_i \mapsto b_i, \dots, a_d \mapsto a_d \\
        b_1 \mapsto b_1, \dots,{} &\phantom{a_i \mapsto b_i}, \dots, b_d \mapsto b_d, s \in S \mapsto s
    \end{align*}
    is an embedding.
\end{proof}

\begin{theorem}\label{thm:cogs-arise-everywhere}
    Given an $S$-orderly orbit $\mathcal{O}$,
    any non-zero $\Aut(\A)_{(S)}$-equivariant subspace of $\Lin_\FF \mathcal{O}$ contains $\Cog_\FF \mathcal{O}$.
\end{theorem}
\begin{proof}
    Let $V \subseteq \Lin_\FF \mathcal{O}$ be a non-zero $\Aut(\A)_{(S)}$-equivariant subspace and let $v \in V$ be a non-zero vector;
    then $v(a_\bullet) \neq 0$ for some $a_\bullet \in \mathcal{O}$.
    Put \begin{align*}
        \mathcal{O}_v &= \{ o_\bullet \in \mathcal{O} \mid v(o_\bullet) \neq 0 \}, \\
        S' &= S \cup \{ o_i \mid o_\bullet \in \mathcal{O}_v, 1 \leq i \leq d \} \setminus \{a_1, \dots, a_d\}, \\
        \mathcal{O}' &= \Aut(\A)_{(S')} \cdot a_\bullet \subseteq \mathcal{O'}
    \end{align*}
    and use Proposistion~\ref{prop:cog-building-whole} to find the $\mathcal{O}'$-cog parallel $a_\bullet \parallel b_\bullet$ --- which is \textit{a fortiori} an $\mathcal{O}$-cog --- and the automorphisms $\pi_1, \dots, \pi_d$.
    Now define $v_0 = v$ and
    \[
        v_{i+1} = v_{i} - \pi_{i+1} \cdot v_i.
    \]
    We can check inductively that for $i = 1, \dots, d$, with $\mathcal{O}_v^{(i)} = \{o_\bullet \in \mathcal{O}_v \mid \{o_1, \dots, o_d\} \supseteq \{a_1, \dots, a_i\}\}$ we have
    \[
        v_i =  \sum_{o_\bullet \in \mathcal{O}_v^{(i)}} \sum_{J \subseteq \{1, \dots, i\}} (-1)^{ |J| } v(o_\bullet) \cdot o_\bullet [a_j \mapsto b_j \mid j \in J].
    \]
    But $\{o_1, \dots, o_d\} \supseteq \{a_1, \dots, a_d\}$ means that $o_\bullet = a_\bullet$, so at the end we get the $\mathcal{O}$-cog $v_d = a_\bullet \between b_\bullet$.
    We conclude by Proposition~\ref{prop:cog-equivariant} that $V$ contains $\Cog_\FF \mathcal{O}$.
\end{proof}

\begin{corollary}
    $\Cog_\FF \mathcal{O}$ has length $1$.
\end{corollary}
\begin{proof}
    By Theorem~\ref{thm:cogs-arise-everywhere}, an $\Aut(\A)_{(S)}$-equivariant subspace $V \subseteq \Cog_\FF \mathcal{O}$ is either $\{0\}$ or $\Cog_\FF \mathcal{O}$ itself.
\end{proof}


\subsection{Projecting down}
\[
    (-)|_{I \setminus \{i\}} = (-)|_{-i}
\]

\subsection{Building up}
\[
    a^{(1)}_{i_1},
    a^{(2)}_{i_2},
    \dots,
    a^{(n)}_{i_n},
    b_*,
    s_1,
    \dots,
    s_m 
\]

\[
    o_{i_1},
    o_{i_2},
    \dots,
    o_{i_n},
    o_{N},
    s_1,
    \dots,
    s_m 
\]

\section*{Acknowledgements}
Hrushovski

Evans

\printbibliography

\end{document}
