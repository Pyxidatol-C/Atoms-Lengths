\section{Rado graph, with cogs}
In this section we work with the following setting:
\begin{itemize}
    \item 
    $\mathcal{L}_0$ is a (possibly infinite) relational language containing a binary symbol $=$;

    \item 
    $\mathcal{C}_0$ is a free amalgamation class of $\mathcal{L}_0$-structures
    where $=$ is interpreted as true equality, but every other $R \in \mathcal{L}_0$ is interpreted irreflexively.\footnote{%
        We can enforce irreflexivity by considering a language $\mathcal{L}'_0$ which consists, 
        for each $R \in \mathcal{L}_0 \setminus \{=\}$ of arity $r$ and each partition $\P$ of $r$ into $k$ parts, 
        of a $k$-ary relation symbol $R_\P$.
        Then $\mathcal{L}_0$-structures may be viewed as $\mathcal{L}'_0$-structures and vice versa, without changing the meaning of embeddings.
        In this way, we get a free amalgamation class $\mathcal{C}'_0$ with a Fraïssé limit which, viewed as an $\mathcal{L}_0$-structrure, is isomorphic to $\A_0$.
    } 

    \item 
    $\mathcal{L}$ consists of $\mathcal{L}_0$ together with a new binary symbol $<$;

    \item 
    $\mathcal{C}$ consists of $\mathcal{L}$-structures obtained from $\mathcal{C}_0$ by expanding with all possible linear orderings;

    \item 
    $\A_0$ and $\A$ are the respective Fraïssé limits of $\mathcal{C}_0$ and $\mathcal{C}$,
    where without loss of generality we assume $\A_0$ and $\A$ share the same domain so that $\Aut(\A_0) \supseteq \Aut(\A)$.
\end{itemize}

\begin{example}\label{ex:N-Q}
    Take $\mathcal{L}_0$ consist of $=$ only and let $\mathcal{C}_0$ to be all finite sets.
    Then $\A_0$ is isomorphic to the pure set $\N$, whereas $\A$ is isomorphic to $\Q$ with the usual order.
\end{example}

\begin{example}\label{ex:Rado-orderedRado}
    Let $\mathcal{L}_0$ consist of $=$ and a single binary symbol $\sim$ 
    and let $\mathcal{C}_0$ consist of all finite undirected graphs not embedding the complete graph $K_n$,
    where $3 \leq n$ ($\leq \infty$).
    Then $\A_0$ is the $K_n$-free Henson graph (or the Rado graph when $n = \infty$), and $\A$ is its generically ordered counterpart.
    (Allowing $n = 2$ makes these degenerate to $\N$ and $\Q$ above).
\end{example}

We note two technicalities and a triviality.

\begin{lemma}\label{lem:free-forb}
    Let $\mathcal{F}_0$ consist of minimal $\mathcal{L}_0$-structures which do not appear in $\mathcal{C}_0$.
    Then
    \begin{enumerate}
        \item $\mathcal{C}_0$ consists of every $\mathcal{L}_0$-structure that does not embed any $F \in \mathcal{F}_0$.
        \item $\mathcal{C}$ consists of every $\mathcal{L}$-structure whose $\mathcal{L}_0$-reduct does not embed any $F \in \mathcal{F}_0$.
        \item Given $F \in \mathcal{F}_0$, every two distinct $x, y \in F$ are related by some $R \in \mathcal{L}_0$.
    \end{enumerate}
\end{lemma}
\begin{proof}
    As $\mathcal{C}_0$ is closed under substructures, its complement is closed under superstructures and thus 
    --- since there are no infinite strictly descending chain of embedded substructures 
    --- determined by its minimal structures.
    2) follows because an $\mathcal{L}$-structure is in $\mathcal{C}$ precisely when its $\mathcal{L}_0$-reduct is in $\mathcal{C}_0$.
    For 3), notice that $F \setminus \{x\}$, $F \setminus \{y\}$ are in $\mathcal{C}_0$ by 1); 
    therefore so is their free amalgam over $F \setminus \{x, y\}$, which then cannot agree with $F$.
\end{proof}

\begin{lemma}\label{lem:free-fresh}
    Let $X, Y, \{a\} \subseteq \A$ be disjoint.
    Then there is some automorphism $\tau \in \Aut(\A)$ such that
    \begin{enumerate}
        \item $\tau$ fixes every $x \in X$;
        \item $\tau(a)$ is not related with any $y \in Y \cup \{a\}$ by any $R \in \mathcal{L}_0$;
        \item $\tau(a) > a$.
    \end{enumerate}
\end{lemma}
\begin{proof}
    In $\A_0$, form the free amalgam
% https://q.uiver.app/#q=WzAsNCxbMCwxLCJYIl0sWzEsMiwiWCBcXGN1cCBcXHthXFx9Il0sWzEsMCwiWCBcXGN1cCBZIFxcY3VwIFxce2FcXH0iXSxbMiwxLCJYIFxcY3VwIFkgXFxjdXAgXFx7YSwgYSdcXH0iXSxbMCwxXSxbMCwyXSxbMSwzLCIiLDAseyJzdHlsZSI6eyJib2R5Ijp7Im5hbWUiOiJkYXNoZWQifX19XSxbMiwzLCJcXHN1YnNldGVxIiwwLHsic3R5bGUiOnsiYm9keSI6eyJuYW1lIjoiZGFzaGVkIn19fV1d
\[\begin{tikzcd}[cramped]
	& {X \cup Y \cup \{a\}} \\
	X && {X \cup Y \cup \{a, a'\}} \\
	& {X \cup \{a\}}
	\arrow["\subseteq", dashed, from=1-2, to=2-3]
	\arrow[from=2-1, to=1-2]
	\arrow[from=2-1, to=3-2]
	\arrow[dashed, from=3-2, to=2-3]
\end{tikzcd}\]
    so that $x \mapsto x, a \mapsto a'$ is an embedding,
    and no relation in $\mathcal{L}_0$ is satisfied by any tuple in which both $a'$ and an element of $Y \cup \{a\}$ appear.
    Now we make $X \cup Y \cup \{a, a'\}$ an $\mathcal{L}$-structure: 
    inherit the order on $X \cup Y \cup \{a\}$ from $\A$,
    and declare that $a < a'$ as well as $a' < z$ if $z$, the next element of $X \cup Y$ larger than $a$, exists at all.
    Notice that $x \mapsto x, a \mapsto a'$ is still an embedding in presence of the order.
    By homogeneity, we may embed $X \cup Y \cup \{a, a'\}$ into $\A$ via some $f$ which is the identity on $X \cup Y \cup \{a\}$;
    again by homogeneity, we may extend the embedding $f(x) \mapsto f(x), f(a) \mapsto f(a')$ to some automorphism of $\A$ which makes 1), 2), and 3) true.
\end{proof}

\begin{proposition}
    The $S$-supported length of $\Lin {\A_0}^d$ is at most that of $\Lin \A^d$ for any finite $S \subseteq \A_0 = \A$.
\end{proposition}
\begin{proof}
    Any chain of subspaces in $\Lin {\A_0}^d = \Lin \A^d$ that are invariant under $\Aut(\A_0)_{(S)}$ must also be invariant under the subgroup $\Aut(\A)_{(S)}$.
\end{proof}
