\section{Finite length from free amalgamation with a generic order}
\newcommand{\cal}{\mathcal}

As we saw in Non-example~\ref{non-ex:weak-smooth-approximation}, the ordered atoms do not have oligomorphic approximation. 
Nontheless they do have the finite length property over any field \cite[Lemma~4.5]{BFKM24}, not just those of characteristic zero.
We give an adaptation of the proof there in this section to richer ordered structures.
In particular, we will deduce the finite length property for all homogeneous undirected graphs and uncountably many non-isomorphic homogeneous directed graphs.
These results are new.

Throughout this section, we will work with the following setting.

\paragraph*{Graph vocabulary}
Consider a finite relational language $\cal L_0$ consisting of unary and binary symbols.
(This allows us to talk about graphs with coloured vertices and edges, but in the running example we will consider a single edge relation.)

\paragraph*{Free amalgamation class}
Let $\cal C_0$ be a \emph{free} amalgamation class of finite $\cal L_0$-structures. 
This is a standard notion in model theory --- see \cite[Section~2.1]{Macpherson11}.
Informally, it means that when we perform amalgamation, we do not need to glue together more elements than necessary or introduce new relations.

There is a more useful characterisation \cite[Lemma~4.5]{SS20} of the amalgamation class $\cal C_0$ being free.
Since $\cal C_0$ is closed under substructures and isomorphisms, 
we know that $\cal C_0$ consists of all the finite $\cal L_0$-structures which do not embed any structure from $\cal F$, 
where $\cal F$ consists of all the minimal finite $\cal L_0$-structures that do not belong to $\cal C_0$;
we write \[
    \cal C_0 = \Forb(\cal F).
\]
That $\cal C_0$ is close under free amalgamation means that in each $F \in \cal F$, 
any two elements $x, y$ are either equal or satisfy at least one of $R(x, y)$ and $R(y, x)$ for some binary relation $R \in \cal L_0$
--- from here on, we will just say $x, y$ are \emph{related}.
Conversely, given a family $\cal F$ of finite $\cal L_0$-structures in each of which any two elements are related,
the class $\Forb(\cal F)$ of finite $\cal L_0$-structures form a \emph{free} amalgamation class.

Here are a few free amalgamation classes.

\begin{example}\label{ex:free-amalg-equality}
    Let $\cal L_0$ be empty. 
    Then $\cal C_0 = \Forb(\{\})$ is a free amalgamation class consisting of all finite pure sets.
\end{example}

\begin{example}\label{ex:free-amalg-graphs}
    Let $\cal L_0$ consist of a single binary symbol $\sim$. 
    Consider $\cal L_0$-structures ${\circlearrowright} = \{x\}$ and ${\rightarrow} = \{y, z\}$, where the relation $\sim$ is interpreted as $\{(x, x)\}$ and $\{(y, z)\}$.
    Then $\cal C_0 = \Forb(\{{\circlearrowright}, {\rightarrow}\})$ is a free amalgamation class consisting of all finite undirected graphs.

    In addition, let $K_n$ be the $\cal L_0$-structure representing a complete graph on $n$ vertices.
    Then 
    \begin{align*}
        \Forb(\{{\circlearrowright}, {\rightarrow}, K_3\}) 
        \subseteq \Forb(\{{\circlearrowright}, {\rightarrow}, K_4\})
        \subseteq \Forb(\{{\circlearrowright}, {\rightarrow}, K_5\})
        \subseteq \cdots \\
        \subseteq \Forb(\{{\circlearrowright}, {\rightarrow}\})
    \end{align*}    
    are all free amalagamation classes.
\end{example}



\paragraph*{Generically ordered expansion}
Now, let $\cal L$ consist of $\cal L_0$ together with a new binary symbol $<$.
Consider the class $\cal C$ of $\cal L$-structures obtained from $\cal C_0$, 
by interpreting $<$ in any $\cal L_0$-structure there as any total order.

Observe that $\cal C$ is an amalgamation class.
Indeed, let $X, Y_1, Y_2$ be $\cal L$-structures in $\cal C$ with $X \subseteq Y_1 \cap Y_2$.
Then we can amalgamate $Y_1, Y_2$ over $X$ as $\cal L_0$-structures and as $\{<\}$-structures, 
both using the disjoint union of $Y_1, Y_2$ over $X$ as the underlying set. 
Superposing these relations will give an $\cal L$-structure in $\cal C$, which is the desired amalgamation.
(Because of the total order, this amalgamation in $\cal C$ is not free unless it is trivial.)
Denote the Fraïssé limit of $\cal C$ by $\A$.

\begin{theorem}\label{thm:ordered-free-amalg-has-finite-length}
    The ordered structure $\A$, even with finitely many constants fixed, has the finite length property over any field.
\end{theorem}

The Fraïssé limit $\A_0$ of $\cal C$ is a reduct of $\A$.
To see this, notice that $\A$, when viewed as an $\cal L_0$-structure, shares the same age as $\A_0$ --- namely, $\cal C_0$.
Moreover, it follows from a back-and-forth argument \cite[Lemma~7.1.4]{Hodges93} that the $\cal L_0$-structure $\A$ is also homogeneous and, therefore, that it is isomorphic to $\A_0$.
So we may assume that $\A$ and $\A_0$ have the same universe; we then have $\Aut(\A) \subseteq \Aut(\A_0)$.

\begin{corollary}
    The reduct $\A_0$ of $\A$, even with finitely many constants fixed, has the finite length property over any field.
\end{corollary}

Conversely, we call $\A$ the \emph{generically ordered expansion} of $\A_0$ or simply the ``ordered $\A_0$''.

\begin{example}\label{ex:geneircally-ordered-equality}
    Continuing from Example~\ref{ex:free-amalg-equality},
    the ordered atoms (Example~\ref{ex:order-atoms}) is the generically ordered expansion of the equality atoms (Example~\ref{ex:equality-atoms}).
\end{example}

\begin{example}\label{ex:geneircally-ordered-graphs}
    Continuing from Example~\ref{ex:free-amalg-graphs}, we obtain generically ordered expansions of the Rado graph and of the $K_n$-free Henson graphs.
    The ordered Rado graph was studied in \cite{BPP15}.
    In the examples below, we will work with the ordered triangle-free Henson graph.
    \textcolor{red}{(Unfortunately we do not know an easy, explicit description of the structure, even without the order.)}
\end{example}

\section{How the cogs turn: proof of Theorem~\ref{thm:ordered-free-amalg-has-finite-length}}


\subsection{Orbits and projections}

To start with, let us view $\A^d$ as $\A^{\{1, \dots, d\}}$. More generally, it will be convenient to consider $\A^I$ for a finite totally ordered indexing set $I \subseteq \Q$.
Fix a finite support $S \subseteq \A$.
We shall say that a tuple $a\in\A^I$ is {\em ($S$-)ordered} if $a_i \not\in S$ for all $i$, and $a_i < a_j$ whenever $i < j$. Then the orbit
$\mathcal{O} = \Aut(\A)_{(S)} \cdot a$ only contains $S$-ordered tuples, and we will call the orbit $S$-ordered as well.
If $a$ is not $S$-ordered, by removing the entries that repeat or come from $S$ and reordering the rest, we can always find an $\Aut(\A)_{(S)}$-equivariant bijection from $\cal O$ to an $S$-ordered orbit. (Here and in the following, $\Aut(\A)_{(S)}$ is the group of those automorphisms of $\A$ that fix every element of $S$.)

To study the lengths of orbit-finitely spanned spaces, we may focus on a single ordered orbit at a time:
\begin{proposition}\label{prop:reduction-to-one-orbit}
    The following are equivalent:
    \begin{enumerate}
        \item For any $d$ and any finite $S \subseteq \A$, chains of $\Aut(\A)_{(S)}$-equivariant subspaces in $\Lin_\FF \A^d$ are bounded in length;
        \item $\Lin_\FF \mathcal{O}$ has finite length for any ordered orbit $\mathcal{O}$.
    \end{enumerate}
\end{proposition}
\begin{proof}
    Since $\A$ is oligomorphic, given any $d$ and $S$, 
    we know that $\A^d$ is in an $\Aut(\A)_{(S)}$-equivariant bijection with a finite disjoint union $\biguplus_i \cal O_i$ of $S$-ordered orbits.
    Hence the length of $\Lin_\FF \A^d$, under the action of $\Aut(\A)_{(S)}$, equals \[
        \len(\Lin_\FF(\biguplus_i \mathcal{O}_i)) = \len(\bigoplus_i \Lin_\FF \mathcal{O}_i) = \sum_i \len(\Lin_\FF \mathcal{O}_i),
    \]
    which is finite if and only if each summands is finite.
\end{proof}

So fix an ordered orbit $\mathcal{O} = \Aut(\A)_{(S)} \cdot o \subseteq \A^I$.
From here we take an inductive approach.
By $o |^J$ we mean the restriction of $o : I \to \A$ to $J \subseteq I$;
particularly, we will often write $o |^{-i}$ instead of $o |^{ I \setminus \{i\} }$.
The image $\mathcal{O}|^J$ of $\mathcal{O}$ under this projection agrees with $\Aut(\A)_{(S)} \cdot o |^J$ and is still ordered.

The function $(-)|^J$ lifts to a linear $\Aut(\A)_{(S)}$-equivariant map
\begin{align*}
    (-)|^J : \Lin_\FF \mathcal{O} &\to \Lin_\FF \mathcal{O}|^J.
\end{align*}
Many cancellations can occur under $(-)|^J$; 
the \emph{projection kernel} is the $\Aut(\A)_{(S)}$-equivariant subspace
\[
    \Ker_\FF \mathcal{O} = \bigcap_{i \in I} \ker{ (-)|^{-i} }
\]
of $\Lin_\FF \mathcal{O}$.

\begin{proposition}\label{prop:reduction-to-kernel}
    The following are equivalent:
    \begin{enumerate}
        \item $\Lin_\FF \mathcal{O}$ has finite length for every ordered orbit $\mathcal{O}$;
        \item $\Ker_\FF \mathcal{O}$ has finite length for every ordered orbit $\mathcal{O}$.
    \end{enumerate}
\end{proposition}
\begin{proof}
    That (1) implies (2) is clear as $\Ker_\FF \mathcal{O} \subseteq \Lin_\FF \mathcal{O}$.
    
    To prove the other implication, assume (2) and let $\mathcal{O} \subseteq \A^I$.
    We proceed by induction on $|I|$.
    If $I = \emptyset$, then $\mathcal{O}$ must be the entire singleton $\A^\emptyset = \{ () \}$; 
    as $\Lin_\FF \mathcal{O}$ has no non-trivial subspaces (let alone finitely supported ones), it has length $1$.
    Now if $|I| \geq 1$, assemble all $|I|$ projection maps into a single map
    \begin{align*}
        \Lin_\FF \mathcal{O} &\to \bigoplus_{i \in I} \mathcal{O}|^{-i} \\
        v &\mapsto ( v|^{-i} )_{i \in I}
    \end{align*}
    whose kernel is precisely $\Ker_\FF \mathcal{O}$.
    We have
    \[
        \len( \Lin_\FF \mathcal{O} ) - \len( \Ker_\FF \mathcal{O} )
        \leq \sum_{i \in I} \len( \Lin_\FF \mathcal{O}|^{-i} )
    \]
    which shows that $\len( \Lin_\FF \mathcal{O} )$ is finite from the assumptions.
\end{proof}

Following the terminology used in \cite[Equation (4)]{BFKM24}, we shall call a vector from the projection kernel \emph{balanced}.
As we will see shortly, there exist balanced vectors other than $0$.

\subsection{Cogs}

From now on we will use a lightweight notation for combining tuples of atoms: for disjoint indexing sets $I$ and $J$, if $a\in\A^I$ and $b\in\A^{J}$ are both ordered, then $ab\in \A^{I\cup J}$ will denote their obvious combination. We will only use this notation if the combined tuple $ab$ is ordered. For an obvious example, for any $S$-ordered $a\in\A^I$ and $J\subseteq I$, we have $a|^{I\setminus J}a|^J=a$.

\begin{definition}\label{def:duo}
    Let $\mathcal{O} \subseteq \A^I$ be an $S$-ordered orbit. 
    An \emph{$\mathcal{O}$-duo} $a \parallel b$ consists of tuples $a, b \in \cal O$ such that:
    \begin{enumerate}
        \item 
        $a_i < b_i$ for all $i\in I$;
        
        \item 
        $b_i < a_j$ for all $i<j\in I$;

        \item 
        for any binary $R$ in $\mathcal{L}_0$ (except for $=$) and $i,j\in I$:
        \begin{align*}
            R(a_i,b_j) \iff R(&a_i,a_j) \\
            &\Updownarrow \text{ as $a, b \in \mathcal{O}$} \\
            R(b_i,a_j) \iff R(&b_i,b_j).
        \end{align*}
     \end{enumerate}
\end{definition}
\begin{remark}\label{rem:duo}
Conditions (1) and (2) specify a total order on the $2|I|$ atoms in a duo.
Moreover, thanks to irreflexivity \textcolor{red}{TODO: explain somewhere that we may assume this!}, 
each $a_i$ is unrelated to its counterpart $b_i$.
Further, given any $J \subseteq I$, the combined tuple $a|^{I\setminus J} b|^J$ satisfies the same relations as $a$ (and $b$), so it lies in $\mathcal{O}$. In particular, taking $J=\{i\}$, there is an automorphism $\pi_i$ that sends $a_i$ to $b_i$ and fixes all the other elements of $a$, $b$ and $S$.
\end{remark}

\begin{definition}
    Given a $\mathcal{O}$-duo $a \parallel b$, the corresponding \emph{$\mathcal{O}$-cog} is the vector
    \[
        a \between b =
        \sum_{J \subseteq I} (-1)^{|J|}
        (a|^{I \setminus J} b|^{J})
    \]
    in $\Lin_\FF \mathcal{O}$.
    The linear span of all $\mathcal{O}$-cogs is denoted by $\Cog_\FF \mathcal{O}$.
\end{definition}
Note that, for a fixed $S$-ordered orbit $\cal O$, all $\mathcal{O}$-duos (hence all $\cal O$-cogs) are in the same $\Aut(\A)_{(S)}$-equivariant orbit.
As a result, $\Cog_\FF \mathcal{O}$ is an $\Aut(\A)_{(S)}$-equivariant subspace of $\Lin_\FF \mathcal{O}$ and it is generated by any single cog.

\begin{proposition}\label{prop:cogs-are-balanced}
    $\Cog_\FF \mathcal{O}\subseteq \Ker_\FF \mathcal{O}$.
\end{proposition}
\begin{proof}
    Let $\mathcal{O} \subseteq \A^I$, let $a\parallel b$ be an $\mathcal{O}$-duo, and take any $i \in I$.
    The subsets of $I$ come in pairs of $J$ and $J \cup \{i\}$, where $J\subseteq I \setminus \{i\}$.
    The two tuples $a|^{I\setminus J} b|^{J}$ and $a|^{I \setminus (J \cup \{i\})}b|^{J \cup \{i\}}$ are present in 
    $a\between b$ with the opposite coefficients, and they differ only on the $i$-th entry.
    Therefore they cancel out under $(-)|^{-i}$; hence $(a \between b )|^{-i} = 0$.
\end{proof}

So far, we have not shown that $\cal O$-cogs even exist.
Let us rectify this by showing that they can be found in all but one $\Aut(\A)_{(S)}$-equivariant subspaces of $\Lin_\FF{\cal O}$.


\subsection{Finding cogs}
We begin with a technical lemma, which combines the free amalgamation in $\cal C_0$ and the generic order of $\A$.
\begin{lemma}\label{lem:free-fresh}
    Let $X, Y, \{z\} \subseteq \A$ be disjoint and finite.
    Then there is an automorphism $\tau \in \Aut(\A)$ such that
    \begin{enumerate}
        \item $\tau$ fixes every $x \in X$;
        \item $\tau(z)$ is unrelated to all $y \in Y$ and to $z$;
        \item $\tau(z) > z$.
    \end{enumerate}
\end{lemma}
\begin{proof}
    Form the free amalgam of structures in ${\cal C}_0$:
% https://q.uiver.app/#q=WzAsNCxbMCwxLCJYIl0sWzEsMiwiWCBcXGN1cCBcXHt6XFx9Il0sWzEsMCwiWCBcXGN1cCBZIFxcY3VwIFxce3pcXH0iXSxbMiwxLCJYIFxcY3VwIFkgXFxjdXAgXFx7eiwgeidcXH0iXSxbMCwxXSxbMCwyXSxbMSwzLCJ4IFxcaW4gWCBcXG1hcHN0byB4LFxcIHogXFxtYXBzdG8geiciLDEseyJzdHlsZSI6eyJib2R5Ijp7Im5hbWUiOiJkb3R0ZWQifX19XSxbMiwzLCJcXHN1YnNldGVxIiwxLHsic3R5bGUiOnsiYm9keSI6eyJuYW1lIjoiZG90dGVkIn19fV1d
\[\begin{tikzcd}
	& {X \cup Y \cup \{z\}} \\
	X && {X \cup Y \cup \{z, z'\}} \\
	& {X \cup \{z\}}
	\arrow["\subseteq"{description}, dotted, from=1-2, to=2-3]
	\arrow[from=2-1, to=1-2]
	\arrow[from=2-1, to=3-2]
	\arrow["{x \in X \mapsto x,\ z \mapsto z'}"{description}, dotted, from=3-2, to=2-3]
\end{tikzcd}\]
    so that no element of $Y \cup \{z\}$ is related to $z'$.
    To make $X \cup Y \cup \{z, z'\}$ an $\mathcal{L}$-structure, 
    inherit the order on $X \cup Y \cup \{z\}$ from $\A$,
    and declare that $z < z'$, as well as $z' < a$ whenever $z<a$ for $a\in X \cup Y$. This makes the above a diagram of embeddings in the presence of the order.
    By homogeneity, $X \cup Y \cup \{z, z'\}$ embeds into $\A$ via some $f$ which is the identity on $X \cup Y \cup \{z\}$;
    again by homogeneity, we may extend the embedding \[ 
        x \in X \mapsto x, \quad z \mapsto f(z')
    \] to some automorphism $\tau$ that satisfies (1), (2), and (3).
\end{proof}

\begin{lemma}\label{lem:cog-fresh-single}
    Suppose $a\parallel b$ is an $\mathcal{O}$-duo, where $\mathcal{O} \subseteq \A^I$ is $S$-ordered.
    Given $z \in S$, 
    \begin{itemize}
        \item write $S' = S \setminus \{z\}$;
        \item let $j \not\in I$ be such that $a z \in \A^{I \cup \{j\}}$ --- thus ${\cal O'}=\Aut(\A)_{(S')} \cdot az \subseteq \A^{I \cup \{j\}}$ --- is $S'$-ordered;
        \item let $X \subseteq \A$ be any finite set containing $\{a_i, b_i \mid i \in I\} \cup S'$ but not $z$.
 %       \item let $Y \subseteq \A$ be any finite set disjoint from $X \cup \{z\}$;
    \end{itemize}
    Denote $z'=\tau(z)$, where $\tau \in \Aut(\A)_{(X)}$ is afforded by Lemma~\ref{lem:free-fresh} (with an arbitrary $Y$). Then $az \parallel bz'$ is an $\mathcal{O}'$-duo.
\end{lemma}
\begin{proof}
    First, notice that $bz'\in {\cal O}'$ and that we have the required order relations with $z$ and $z'$. The remaining condition (3) of Definition~\ref{def:duo}, for any binary $R$ in $\mathcal{L}_0$, splits into the following cases (and their symmetric counterparts):
\begin{itemize}
\item $R(a_i,b_{i'})\iff R(a_i,a_{i'})$ since $a \parallel b$ is an $\mathcal{O}$-duo;
\item $R(a_i,z')\iff R(a_i,z)$ since $\tau$ is an automorphism that fixes all $a_i$;
\item $R(a_i,z)\iff R(b_i,z)$ since $a, b \in \mathcal{O}$ and $z\in S$;
\item $R(z,z')$ and $R(z,z)$ are both false: $z'$ is unrelated to $z$ by Lemma~\ref{lem:free-fresh}, and $R$ is irreflexive. \qedhere
\end{itemize}
\end{proof}

Starting from an empty duo, we may apply Lemma~\ref{lem:cog-fresh-single} inductively.
\begin{proposition}\label{prop:cog-fresh-full}
    Let $\mathcal{O} \subseteq \A^I$ be an $S$-ordered orbit.
    Then any $a \in \mathcal{O}$ can be extended to an ${\cal O}$-duo $a\parallel b$.
\end{proposition}
\begin{proof}
    Enumerate the indices of $I$ as $i_1, \dots, i_d$.
    Suppose that we have found $b_{i_1}, \dots, b_{i_k}$ such that 
    \[
        a|^{\{i_1, \dots, i_k\}} \parallel (i_1 \mapsto b_{i_1}, \dots, i_k \mapsto b_{i_k})
    \] 
    is a duo for $\mathcal{O}_k = \Aut(\A)_{(S \cup \{a_{i_{k+1}}, \dots, a_{i_d}\})} \cdot a|^{\{i_1, \dots, i_k\}}$ 
    --- note that $() \parallel ()$ is certainly a duo for $\mathcal{O}_0$ at the start.
    If $k < d$, putting $z = a_{i_{k+1}}$, $X = \{a_{i_1}, b_{i_1}, \dots, a_{i_k}, b_{i_k}\} \cup S \cup \{a_{i_{k+2}}, \dots, a_{i_d}\}$, and $Y = \emptyset$, 
    Lemma~\ref{lem:cog-fresh-single} yields an atom $b_{i_{k+1}}$ that makes
    \[
        a|^{\{i_1, \dots, i_k, i_{k+1}\}} \parallel (i_1 \mapsto b_{i_1}, \dots, i_k \mapsto b_{i_k}, i_{k+1} \mapsto b_{i_{k+1}})
    \] a duo for $\mathcal{O}_{k+1}$.
    For $k=d$ this gives the desired duo for $\mathcal{O}_d = \mathcal{O}$.
\end{proof}

As some $a \in \cal O$ always exists, it follows that Proposition~\ref{prop:cogs-are-balanced} was not vacuous:
we now know $\Cog_\FF{\cal O}$, and hence $\Ker_\FF{\cal O}$, is non-trivial --- but barely so.
Indeed, as the result below shows, $\Cog_\FF{\cal O}$ admits no non-trivial $\Aut(\A)_{(S)}$-equivariant subspaces.
\begin{theorem} \label{thm:cogs-arise-everywhere}
    Any non-trivial $\Aut(\A)_{(S)}$-equivariant subspace $V\subseteq \Lin_\FF \mathcal{O}$ contains $\Cog_\FF \mathcal{O}$.
\end{theorem}
\begin{proof}
    Pick any $v \in V$ and $a \in \mathcal{O}$ with $v(a)\neq 0$;
    it is enough to show that $V$ contains $a \between b$ for some $b\in \mathcal{O}$.
    Define:
    \[
        S' = S \cup \{c_i \mid v(c) \neq 0, i \in I\} \setminus \{a_i \mid i \in I\} \supseteq S
    \]
    and put $\mathcal{O}' = \Aut(\A)_{(S')} \cdot a \subseteq \mathcal{O}$ --- then $\mathcal{O}'$ is $S'$-ordered.
    By Proposition~\ref{prop:cog-fresh-full}, we can find $b \in \mathcal{O}'$ such that $a \parallel b$ is an $\mathcal{O}'$-duo and \emph{a fortiori} an $\mathcal{O}$-duo.
    Take the automorphisms $\pi_{i_1}, \dots, \pi_{i_d}$ from Remark~\ref{rem:duo}, where $i_1, \dots, i_d$ enumerate $I$.
    Now define $v^{(0)} = v$ and \[
        v^{(k)} = v^{(k-1)} - \pi_{i_k} \cdot v^{(k-1)}.
    \]
    Then each $v^{(k)}$ is in $V$.
   By induction on $k$, we have:
    \[
        v^{(k)} = \sum_{c \in C_k} \sum_{J \subseteq \{i_1, \dots, i_k\}} (-1)^{|J|} v(c) \left(\prod_{j \in J} \pi_j \cdot c\right),
    \]
    where
    \[C_k = \{ c \mid v(c) \neq 0, \{c_{i_1}, \dots, c_{i_k}, \dots, c_{i_d}\} \supseteq \{a_{i_1}, \dots, a_{i_k}\} \}.
    \]
    But $\{c_{i_1}, \dots, c_{i_d}\} \supseteq \{a_{i_1}, \dots, a_{i_d}\}$ means that $c = a$, so $C_d=\{a\}$ and $\frac{1}{v(a)}v^{(d)}=a\between b$.
\end{proof}

\begin{corollary}
    $\Cog_\FF \mathcal{O}$ has length $1$.
\end{corollary}

In light of Propositions~\ref{prop:reduction-to-one-orbit},~\ref{prop:reduction-to-kernel} and~\ref{prop:cogs-are-balanced}, for the finite length property it is enough to prove that $\Ker_\FF \mathcal{O} \subseteq \Cog_\FF \mathcal{O}$. In words, we need to show that every balanced vector in $\Lin_\FF{\cal O}$ is a linear combination of $\cal O$-cogs. Before we show a proof of this, let us illustrate its key ideas on an example.

\subsection{Spanning by cogs: an example}

Let $\A_0$ be the universal triangle-free (undirected) graph, and $\A$ its totally ordered version. Consider nine atoms $\{a,\ldots,i\}$, ordered by~$<$ alphabetically, with the edge relation as shown here:
\[
\rotatebox{2}{
\xymatrix@R=6pt@C=15pt{
\rotatebox{-2}{$h$} & & & & & & \rotatebox{-2}{$i$} \\
\\
& & & \rotatebox{-2}{$g$}\ar@{-}[uulll]\ar@{-}[uurrr] \\
& \rotatebox{-2}{$e$} & & & & \rotatebox{-2}{$f$} \\
& & \rotatebox{-2}{$c$}\ar@{-}[ul]\ar@{-}[uur] & & \rotatebox{-2}{$d$}\ar@{-}[uul]\ar@{-}[ur] \\
\rotatebox{-2}{$a$}\ar@{-}[rrrrrr]\ar@{-}[uuuuu]\ar@{-}[uur] & & & & & & \rotatebox{-2}{$b$}\ar@{-}[uul]\ar@{-}[uuuuu]
}}\]
This graph is drawn so that the total order of the atoms corresponds to the vertical order.

Putting $S=\emptyset$ and $|I|=2$, let ${\cal O}$ be the ordered orbit of pairs of atoms which are connected by an edge. Consider the following vector:
\[
v = ah - ae + ce - cg + dg - df + bf - bi + gi - gh \in \Lin_\FF{\cal O}.
\]
This can be graphically presented as the following graph:
\[\rotatebox{2}{
\xymatrix@R=6pt@C=15pt{
\rotatebox{-2}{$h$} & & & & & & \rotatebox{-2}{$i$} \\
\\
& & & \rotatebox{-2}{$g$}\ar@[red][uulll]\ar@[blue][uurrr] \\
& \rotatebox{-2}{$e$} & & & & \rotatebox{-2}{$f$} \\
& & \rotatebox{-2}{$c$}\ar@[blue][ul]\ar@[red][uur] & & \rotatebox{-2}{$d$}\ar@[blue][uul]\ar@[red][ur] \\
\rotatebox{-2}{$a$}\ar@{-}[rrrrrr]\ar@[blue][uuuuu]\ar@[red][uur] & & & & & & \rotatebox{-2}{$b$}\ar@[blue][uul]\ar@[red][uuuuu]
}}
\]
where edges with coefficient $+1$ are marked as blue, and with $-1$ as red. The arrows on the chosen edges remind us that the pairs in $\cal O$ are ordered, but this is mere decoration: the definition of $\cal O$ means that all arrows must point upwards.

Note that $v$ is balanced. Graphically, this means that every atom has as many outgoing red edges as outgoing blue edges, and as many incoming red edges as incoming blue edges.

It is easy to draw $\cal O$-cogs in this way. Assuming some additional atom $z>h$ which is connected by edges to $a$ and $g$ but not to $h$, the $\cal O$-cog $ah\between gz$ can be drawn as:
\[\rotatebox{5}{
\xymatrix{
\rotatebox{-5}{$h$} & \rotatebox{-5}{$z$} \\
\rotatebox{-5}{$a$}\ar@[blue][u]\ar@[red][ur] & \rotatebox{-5}{$g$}\ar@[red][ul]\ar@[blue][u]  
}}
\]
We would like to present $v$ as a sum of such ${\cal O}$-cogs. Some additional atoms must be used for that, as no four atoms among the original nine form an $\cal O$-duo \textcolor{red}{--- this is not the case for the ordered atoms; see the proof of \cite[Claim~4.7]{BFKM24}.}
It would be very convenient to have an atom $z$, larger than every atom in $v$, and connected by edges to every atom which is a source of a directed edge in $v$ (equivalently: which occurs as the first component of a pair in $v$). However, such a $z$ does not exist in the triangle-free graph $\A$. There are two problems:
\begin{itemize}
\item The atom $g$ occurs both as the first and as the second component in pairs present in $v$. In particular, $z$ as prescribed would create a triangle $dgz$ in $\A$, which is forbidden.
\item Atoms $a$ and $b$ both occur as first components in $v$, and they are connected by an edge in $\A$. As a result, an atom $z$ as prescribed would create a triangle $abz$ in $\A$.
\end{itemize}
We get rid of such {\em conflicts} by considering auxiliary atoms $g'>g$ and $b'>b$, with just enough edges to make $gh\parallel g'i$ and $bf\parallel b'i$ valid ${\cal O}$-duos. Specifically, we postulate edges \textcolor{red}{$g'$---$h$, $g'$---$i$, $b'$---$f$ and $b'$---$i$} and no more. Such atoms exist by the homogeneity of $\A$. We then define:
\[
v' = v - (gh \between g'i) - (bf \between b'i)
\]
which can be drawn as:
\[
\rotatebox{2}{
\xymatrix@R=6pt@C=15pt{
\rotatebox{-2}{$h$} & & & & & & \rotatebox{-2}{$i$} \\
\\
& & & \rotatebox{-2}{$g$}\ar@{-}[uulll]\ar@{-}[uurrr] & \rotatebox{-2}{$g'$}\ar@[red][uullll]\ar@[blue][uurr] \\
& \rotatebox{-2}{$e$} & & & & \rotatebox{-2}{$f$} \\
& & \rotatebox{-2}{$c$}\ar@[blue][ul]\ar@[red][uur] & & \rotatebox{-2}{$d$}\ar@[blue][uul]\ar@[red][ur] \\
\rotatebox{-2}{$a$}\ar@{-}[rrrrrr]\ar@[blue][uuuuu]\ar@[red][uur] & & & & & & \rotatebox{-2}{$b$}\ar@{-}[uul]\ar@{-}[uuuuu] & \rotatebox{-2}{$b'$}\ar@[blue][uull]\ar@[red][uuuuul]
}}
\]
Now an atom $z$ as postulated above does not create any triangles:
\[
\rotatebox{2}{
\xymatrix@R=6pt@C=15pt{
& & & \rotatebox{-2}{$z$}\ar@{-}@/_4ex/[llldddddd]\ar@{-}[lddddd]\ar@{-}[rddddd]\ar@{-}[rddd]\ar@{-}[rrrrdddddd] \\
\rotatebox{-2}{$h$} & & & & & & \rotatebox{-2}{$i$} \\
\\
& & & \rotatebox{-2}{$g$}\ar@{-}[uulll]\ar@{-}[uurrr] & \rotatebox{-2}{$g'$}\ar@[red][uullll]\ar@[blue][uurr] \\
& \rotatebox{-2}{$e$} & & & & \rotatebox{-2}{$f$} \\
& & \rotatebox{-2}{$c$}\ar@[blue][ul]\ar@[red][uur] & & \rotatebox{-2}{$d$}\ar@[blue][uul]\ar@[red][ur] \\
\rotatebox{-2}{$a$}\ar@{-}[rrrrrr]\ar@[blue][uuuuu]\ar@[red][uur] & & & & & & \rotatebox{-2}{$b$}\ar@{-}[uul]\ar@{-}[uuuuu] & \rotatebox{-2}{$b'$}\ar@[blue][uull]\ar@[red][uuuuul]
}}
\]
and it is easy to calculate:
\[
v' = (ah \between g'z)  - (ae \between cz) - (cg \between dz) + (b'f \between dz) -(b'i \between g'z)
\]
which presents $v$ as a linear combination of $\cal O$-cogs.

\subsection{Spanning by cogs: a proof}

Inspired by the previous example, we shall now prove that $\Ker_\FF \mathcal{O} \subseteq \Cog_\FF \mathcal{O}$ for any $S$-ordered orbit $\cal O\subseteq \A^I$.

\subsubsection{Subvectors, locations, conflicts} \ \\
%
We begin by introducing some additional terminology and notation. First, let us make explicit a view we have tacitly taken:
with $\mathcal{O}$ as a standard basis, a vector $v \in \Lin_\FF \mathcal{O}$ is just a finite set of pairs in $\FF \times \mathcal{O}$.
A \emph{subvector} of $v$ is a subset of these pairs. We write $\vsup{v}\subseteq {\cal O}$ for the set of tuples which are present in $v$. For a finite subset $T\subseteq\cal O$, we write $\sqrt T\subseteq\A$ for the set of atoms present anywhere in $T$, and for a vector $v$, we write $\sqrt v$ to mean $\sqrt{\vsup{v}}$.

For any $i \in I$ and $a \in \A$, we write 
\[
    \mathcal{O}^{i:a} = \{b \in \mathcal{O} \mid b_i = a\};
\]
this is an $\Aut(\A)_{(S a)}$-orbit, and its projection $\mathcal{O}^{i:a} |^{-i}$ is $Sa$-ordered.
For a vector $v \in \Lin_\FF \mathcal{O}$, by
\[
    v^{i:a} \in \Lin_\FF \mathcal{O}^{i:a}
\]
we mean the subvector of $v$ consisting of all pairs in $\FF \times \mathcal{O}^{i:a}$.

\begin{lemma}\label{lem:balanced-projected-subvector}
    Let $v \in \Lin_\FF \mathcal{O}$ be balanced. 
    Then any projected subvector $v^{i:a}|^{-i} \in \Lin_\FF \mathcal{O}^{i:a}|^{-i}$ is also balanced.
\end{lemma}
\begin{proof}
    Let $j \in I \setminus \{i\}$. 
    By assumption we have \[
        0 = v|^{-j} = \sum_a v^{i:a}|^{-j} %\in \Lin_\FF \A^{I \setminus \{j\}};
    \]
    \textcolor{red}{(the above sum is finite, as it runs over those atoms $a$ that occur as the $i$-th entries in $\vsup{v}$).}
    By looking at $i$-th entries, we see that each $v^{i:a}|^{-j}$ must be the zero vector.
    Hence so is $v^{i:a}|^{-j}|^{-i} = v^{i:a}|^{-i}|^{-j}$,
    which shows that $v^{i:a}|^{-i}$ is balanced.
\end{proof}

For a finite subset $T \subseteq \mathcal{O}$, a {\em location} in $T$ is a pair $(i,a)\in I\times\A$ such that $a=c_i$ for some $c\in T$. Note that for any fixed $i,j\in I$, for all $c\in\cal O$ the atoms $c_i$ and $c_j$ are related in the same way in $\A$. We say that two locations $(i,a)$ and $(j,b)$ in $T$ are in:
\begin{itemize}
\item an {\em equational conflict}, if $i\neq j$ but $a=b$, and in
\item a {\em relational conflict}, if $a$ and $b$ are related in $\A_0$ but not in the same way as $c_i$ and $c_j$ for $c\in\cal O$.
\end{itemize}
%write
%\[
%    \overline{\sigma} = \{ (i, a_i) \mid i \in I, a \in \sigma \}
%\]
%and define two binary relations 
%\begin{align*}
%    &(i, a_i) {?} (j, b_j) \iff a_i = b_j \text{ but } i \neq j, \\
%    &(i, a_i) {!} (j, b_j) \iff \text{$a_i, b_j$ are related but not in the same way as $a_i, a_j$}
%    % or equivalently $b_i, b_j$ since both $a$ and $b$ are in $\mathcal{O}$
%\end{align*}
%called \emph{ambiguities} and \emph{obstructions}.
(A situation where $a$ and $b$ are not related by any relation in $\A_0$ at all, does not constitute a conflict even if $c_i$ and $c_j$ are related.) Seeing equality as a binary relation in $\A_0$, an equational conflict is a special case of a relational one.
%(Recall that $a_i$ and $b_j$ are \emph{not related} if they are freely amalgamated in the reduct $\A_0$ of $\A$ 
%--- i.e., they are not equal, and for no $R \in \mathcal{L}_0$ does $R(a_i, b_j) \vee R(b_j, a_i)$ hold;
%in that case, we write $a_i \amalgindep b_j$).
%Both relations are symmetric and ${?} \subseteq {!}$;
%denote their images by ${?\overline{\sigma}} \subseteq {!\overline{\sigma}} \subseteq \overline{\sigma}$.
A location in a vector $v$ means a location in the set $[v]$.
%Let $\EConf(T)$ and $\RConf(T)$ denote, respectively, the sets of those locations which take part in, respectively, equational and relational conflicts in $T$. For a vector $v$, we write $\EConf(v)$ and $\RConf(v)$ for $\EConf([v])$ and $\RConf([v])$.

The prototypical examples of vectors which are free from any conflicts are cogs (or any subvectors of cogs). Note that the locations in a cog $a\between b$ are exactly those in $\{a,b\}$, and these have no conflicts if $a\parallel b$ is a duo.

In the following proof we will often manipulate many duos and cogs at once, so we will benefit from a concise notation for them. To denote an $\cal O$-duo with a single letter we will annotate it as $a^\pm$; its constituent parts will then be denoted with $a^+$ and $a^-$, so that $a^\pm=a^+\parallel a^-$. Sets of duos will be denoted with capital letters such as $A^\pm$, and sometimes we will slightly abuse this notation and write $A^\pm$ to mean $\bigcup_{a^\pm\in A^\pm}\{a^+,a^-\}$,
$A^+$ to mean $\bigcup_{a^\pm\in A^\pm}\{a^+\}$, and $A^-$ to mean $\bigcup_{a^\pm\in A^\pm}\{a^-\}$.

\subsubsection{Conflict resolution lemmas}\ \\
%
The following sister lemmas, relying on free amalgamation as distilled in Lemma~\ref{lem:free-fresh}, show how to merge conflict-free subsets of $\cal O$ in a way that avoids introducing new conflicts. This will be useful in Sections~\ref{sec:unobstructed} and~\ref{sec:unambiguous}.
%
%Let $v \in \Ker_\EE \mathcal{O}$ and let $v^{i:a_i}$ be a subvector.
%In the cog decomposition results we are about to prove,
%we will work with the assumption that $V = \vsup{v}$ is unobstructed (resp., unambiguous);
%then so is $V' = \vsup{v^{i:a_i}} \subseteq \vsup{v}$.
%We will be able to write $v^{i:a_i} = \sum_{a^\pm \in A^\pm} (\lambda_{a^\pm} \cdot a^+ \between a^-) a_i$
%where the union of $V'$ and $K = A^\pm a_i = \{a^+ a_i, a^- a_i \mid a^\pm \in A^\pm\}$ is unobstructed (resp., unambiguous).
%But we want to make $V \cup K$ unobstructed (resp., unambiguous);
%we can do so by choosing $K$ more carefully.
%
% Here O is orbit of edges (a, b) with a < b, a ~ b.
% The YELLOW family of edges, which contains the GREEN family, has no obstructions.
% Nor does the union of GREEN and RED. 
% However YELLOW and RED together has an ambiguity (2, b) ? (1, b).
% Lemma-? removes ambiguities but still leaves an obstruction (2, \pi b) ! (2, d).
% Lemma-! removes such obstructions.
%\begin{center}
%\begin{tikzpicture}[font=\sffamily, line cap=round, line join=round, >={Latex[length=3.6mm,width=2.4mm]}]
%\coordinate (a) at (0,0); \coordinate (b) at (3.25,0); \coordinate (d) at (7.25,0); \coordinate (c) at (5.55,2.35); \coordinate (pib) at (3.05,-1.5); \coordinate (taub) at (1.95,-3);
%\begin{scope}[on background layer]
%\draw[yellow!60, opacity=0.55, line width=40pt] (a) to[out=25,in=200] (c); \draw[green!55!black, opacity=0.28, line width=26pt] (a) to[out=25,in=200] (c);
%\draw[yellow!60, opacity=0.55, line width=40pt] (b) -- (d);
%\draw[magenta!65, opacity=0.30, line width=28pt] (a) -- (b);
%\draw[magenta!55, opacity=0.22, line width=30pt] (a) to[out=-55,in=170] (pib);
%\draw[magenta!45, opacity=0.14, line width=34pt] (a) to[out=-80,in=155] (taub);    
%\end{scope}
%\node (A) at (a) {$a$}; \node (B) at (b) {$b$}; \node (C) at (c) {$c$}; \node (D) at (d) {$d$}; \node[blue!70!black] (PIB) at (pib) {$\pi(b)$}; \node[blue!70!black] (TAUB) at (taub) {$\tau(b)$};
%\draw[->, line width=2.2pt] (A) to[out=25,in=200] (C);
%\draw[->, line width=2.2pt] (A) -- (B);
%\draw[->, line width=2.2pt] (B) -- (D);
%\draw[->, line width=2.4pt, blue!70!black] (A) to[out=-55,in=170] (PIB);
%\draw[->, line width=2.4pt, blue!70!black] (A) to[out=-80,in=155] (TAUB);
%\draw[->, line width=2.4pt, blue!70!black] (PIB) to[out=10,in=-140] (D);
%\end{tikzpicture}
%\end{center}

\begin{lemma}\label{lem:?}
    Let $K, V_0, V$ be finite subsets of $\mathcal{O}$ such that $V_0 \subseteq V$ and both $V_0\cup K$ and $V$ are free from equational conflicts. Then there exists a $\pi\in\Aut(\A)$ that fixes all atoms in $S$ and in $\rt{V_0}$, and such that $V \cup \pi(K)$ is free from equational conflicts.
\end{lemma}
\begin{proof}
    Fix $V_0, V$ and induct on the number of equationally conflicting locations in $V \cup K$. Take any such locations $(i,a)$ and $(j,a)$; without loss of generality $(i,a)$ is a location in $K$ and $(j,a)$ a location in $V$. Since there are no equational conflicts in $V_0\cup K$ or in $V$, we see that $a \not\in \sqrt{V_0}$; also $a\not\in S$, as $\mathcal{O}$ is $S$-ordered.
    Put 
    \[
        X = S \cup \sqrt{K \cup V} \setminus \{a\}
    \]
    and note that $X$ contains $S \cup \sqrt{V_0}$.
    Use Lemma~\ref{lem:free-fresh} (putting $z=a$ and $Y=\emptyset$) to obtain an automorphism $\pi \in \Aut(\A)_{(X)}$ such that $\pi(a) \not\in X \cup \{a\}$. In the set $V\cup\pi(K)$, the conflicting location $(i,a)$ disappears and no new equational conflicts are created, so the number of equationally conflicting locations drops compared to $V\cup K$.
    Because $V_0 \cup \pi(K)$ is still conflict-free,
    the inductive hypothesis gives us some $\pi' \in \Aut(\A)_{(S \cup \sqrt{V_0})}$ such that $V \cup \pi' \pi(K)$ is free from equational conflicts.
\end{proof}

\begin{lemma}\label{lem:!}
    Let $K, V_0, V$ be finite subsets of $\mathcal{O}$ such that $V_0 \subseteq V$ and both $V_0\cup K$ and $V$ are free from relational conflicts. Then there exists a $\pi\in\Aut(\A)$ that fixes all atoms in $S$ and in $\rt{V_0}$, and such that $V \cup \pi(K)$ is free from relational conflicts.
\end{lemma}
\begin{proof}
    By \autoref{lem:?} we may assume that $V \cup K$ is free from {\em equational} conflicts.  As before, fix $V_0, V$ and proceed by induction on the number of relationally conflicting locations in $V \cup K$.
    
    Let $(i, a)$ and $(j, b)$ be in a relational conflict; without loss of generality $(i, a)$ is a location in $K$ and $(j, b)$ in $V$. This is not an equational conflict, so $a\neq b$ (but possibly $i=j$). Since there are no conflicts in $V_0\cup K$ or in $V$, we see that $a\not\in \sqrt{V}$ and $b\not\in\sqrt{V_0\cup K}$. Also, $a,b\not\in S$.
    Let $Y$ consist of all the atoms $b$ that are in a relational conflict with $(i,a)$ in $V\cup K$; we have just shown that $Y$ does not contain $a$ and is disjoint with $S\cup\rt{V_0\cup K}$. Put:
    \[
        X = S \cup \sqrt{K \cup V} \setminus (Y \cup \{a\}).
    \]
    Then $X, Y, \{a\}$ are pairwise disjoint, and $X$ contains $S \cup \sqrt{V_0}$. It also contains all atoms in $\rt{K}$ except $a$.
   Using Lemma~\ref{lem:free-fresh}, find some $\pi\in \Aut(\A)_{(X)}$ such that $\pi(a) \not\in X \cup Y \cup \{a\}$ and $\pi(a)$ is not related to any atom in $Y$. In $V\cup\pi(K)$ the conflicting location $(i,a)$ disappears and no new conflicts are created, so the conclusion follows from the inductive hypothesis.
\end{proof}

\subsubsection{Conflict-free vectors}\label{sec:unobstructed}
% https://q.uiver.app/#q=WzAsMTUsWzAsMywiYl4rIl0sWzEsMywiYiJdLFswLDQsImJeLSJdLFswLDAsImFeKyJdLFsxLDEsImEiXSxbMCwxLCJhXi0iXSxbMywyLCJjIl0sWzQsMiwiXFxyaWdodHNxdWlnYXJyb3ciXSxbNSwzLCJiXisiXSxbNSw0LCJiXi0iXSxbNSwxLCJhXi0iXSxbNSwwLCJhXisiXSxbNiwxLCJhIl0sWzYsMywiYiJdLFs4LDIsImMiXSxbMCwxLCIiLDAseyJjb2xvdXIiOlswLDYwLDYwXX1dLFsyLDEsIiIsMix7ImNvbG91ciI6WzI0MCw2MCw2MF19XSxbMyw0LCIiLDIseyJjb2xvdXIiOlswLDYwLDYwXX1dLFs1LDQsIiIsMCx7ImNvbG91ciI6WzI0MCw2MCw2MF19XSxbMyw2LCIiLDAseyJjdXJ2ZSI6LTF9XSxbNSw2XSxbMCw2XSxbMiw2LCIiLDIseyJjdXJ2ZSI6MX1dLFsxMCwxNCwiIiwwLHsiY29sb3VyIjpbMjQwLDYwLDYwXX1dLFs4LDE0LCIiLDIseyJjb2xvdXIiOlswLDYwLDYwXX1dLFs4LDEzXSxbOSwxM10sWzExLDEyXSxbMTAsMTJdLFsxMSwxNCwiIiwxLHsiY3VydmUiOi0xLCJjb2xvdXIiOlswLDYwLDYwXX1dLFs5LDE0LCIiLDEseyJjdXJ2ZSI6MSwiY29sb3VyIjpbMjQwLDYwLDYwXX1dXQ==
%\[\begin{tikzcd}[cramped]
%	{a^+} &&&&& {a^+} \\
%	{a^-} & a &&&& {a^-} & a \\
%	&&& c & \rightsquigarrow &&&& c \\
%	{b^+} & b &&&& {b^+} & b \\
%	{b^-} &&&&& {b^-}
%	\arrow[draw={rgb,255:red,214;green,92;blue,92}, from=1-1, to=2-2]
%	\arrow[curve={height=-6pt}, from=1-1, to=3-4]
%	\arrow[from=1-6, to=2-7]
%	\arrow[color={rgb,255:red,214;green,92;blue,92}, curve={height=-6pt}, from=1-6, to=3-9]
%	\arrow[draw={rgb,255:red,92;green,92;blue,214}, from=2-1, to=2-2]
%	\arrow[from=2-1, to=3-4]
%	\arrow[from=2-6, to=2-7]
%	\arrow[color={rgb,255:red,92;green,92;blue,214}, from=2-6, to=3-9]
%	\arrow[from=4-1, to=3-4]
%	\arrow[draw={rgb,255:red,214;green,92;blue,92}, from=4-1, to=4-2]
%	\arrow[color={rgb,255:red,214;green,92;blue,92}, from=4-6, to=3-9]
%	\arrow[from=4-6, to=4-7]
%	\arrow[curve={height=6pt}, from=5-1, to=3-4]
%	\arrow[draw={rgb,255:red,92;green,92;blue,214}, from=5-1, to=4-2]
%	\arrow[color={rgb,255:red,92;green,92;blue,214}, curve={height=6pt}, from=5-6, to=3-9]
%	\arrow[from=5-6, to=4-7]
%\end{tikzcd}\]
\begin{proposition}\label{prop:!-free-decomposition}
    If $v \in \Ker_\FF \mathcal{O}$ is free from conflicts, then it can be written as a linear combination of $\cal O$-cogs:
    \[
        v = \sum_{a^\pm \in A^\pm} \lambda_{a^\pm} \cdot a^+ \between a^-
    \]
    with $\lambda_{a^\pm} \in \FF$, where moreover $\vsup{v} \cup A^\pm$ is free from conflicts.
\end{proposition}
\begin{proof}
We proceed by induction on the dimension $|I|$, 
noting that when $I = \emptyset$ we just have $v = \lambda \cdot () = \lambda \cdot ( \between )$.

So suppose $I$ is non-empty; let $d \in I$ be the greatest element. 
Group the terms in $v$ by their greatest atom so that $v = v^1 + v^2 + \cdots + v^k$.
We now induct on $k$.
If $k =1$, we are done: as $v|^{-d} = 0$ we must have $v = 0$, so the empty sum will do.
Otherwise \[
    v = v^{d:a} + v^{d:b} + v'
\]
for some $a\neq b\in \A$. By Lemma~\ref{lem:balanced-projected-subvector}, $v^{d:a}|^{-d}$ is balanced, and it is conflict-free, as every location in it is also a location in $v$.
By the outer inductive hypothesis, we get \[
    v^{d:a} = (v^{d:a}|^{-d}) a = \sum_{a^\pm\in A^\pm} (\lambda_{a^\pm} \cdot a^+ \between a^-)a
\]
where $[v^{d:a}|^{-d}]\cup A^\pm$ is free from conflicts, which immediately implies that $[v^{d:a}]\cup A^\pm a$ is free from conflicts as well. 
Note that if a $\pi \in \Aut(\A)_{(S)}$ fixes every atom in $v^{d:a}$ then
\[
    v^{d:a} = \pi(v^{d:a}) = \sum_{a^\pm \in A^\pm} \lambda_{a^\pm} \cdot \pi a^+ \between \pi a^-,
\]
so by Lemma~\ref{lem:!} (putting $K=A^\pm a$, $V_0=[v^{d:a}]$ and $V=[v]$) we may assume without loss of generality that
$[v]\cup A^\pm a$ is free from conflicts.

Similarly, we can write \[
    v^{d:b} = \sum_{b^\pm\in B^\pm} (\lambda_{b^\pm} \cdot b^+ \between b^-)b
\]
and apply Lemma~\ref{lem:!} again (putting $K=B^\pm b$, $V_0=[v^{d:b}]$ and $V=[v]\cup A^\pm a$) to conclude that
\[
    \vsup{v} \cup A^\pm a \cup B^\pm b
\]
is free from conflicts.

We now invent a new element $z$, on which we impose the following relations with $S \cup \sqrt{A^\pm a \cup B^\pm b} \subseteq \A$: 
\begin{enumerate}
    \item $a, b < z$, and $z < s$ iff $a, b < s$ for any $s \in S$;
    
    \item for any unary relation $P \in \mathcal{L}_0$:
    \[
        P(z) \;:\Longleftrightarrow\; P(a) \iff P(b)
    \]
    \item for any binary relation $R \in \mathcal{L}_0$ and $s \in S$, $a^\pm \in A^\pm$, $b^\pm \in B^\pm$, $i \in I \setminus \{d\}$:
    \begin{enumerate}
        \item $R(z, s) \;:\Longleftrightarrow\; R(a, s) \iff R(b, s)$,
        % \item $R(s, z) \;:\Longleftrightarrow\; R(s, a_d) \iff R(s, b_d)$;
        \item $R(z, a^\pm_i) \;:\Longleftrightarrow\; R(a, a^\pm_i)$,
        \item $R(z, b^\pm_i) \;:\Longleftrightarrow\; R(b, b^\pm_i)$;
        % \item $R(c_i, z) \;:\Longleftrightarrow\; R(c_i, a_d) \iff R(c_i, b_d)$;
        \item $R(z, a)$ and $R(z,b)$ are both false;
        % \item $R(a_d, z), R(b_d, z) \;:\Longleftrightarrow\; \bot$;
        \item and symmetrically for $R(-, z)$.
    \end{enumerate}
    These are well-defined because $a^\pm a, b^\pm b \in \mathcal{O}$ and $i = j$ whenever $a^\pm_i = b^\pm_j$.
\end{enumerate}
To see that the $\mathcal{L}$-structure $S \cup \sqrt{A^\pm a \cup B^\pm b} \cup \{z\}$ still embeds into $\A$, 
suppose towards a contradiction that it contains a forbidden $\mathcal{L}_0$-substructure $F$.
Then $F$ must contain $z$.
Since any two elements in $F$ are necessarily related, we must have $a, b \not\in F$.
Similarly, whenever $F$ contains an atom $x_i$ for any $x\in A^\pm\cup B^\pm$, then it does not contain $y_i$ for any other $y\in A^\pm\cup B^\pm$.
It follows that, fixing any $a^\pm\in A^\pm$,
\[
    s \mapsto s,\quad x_i \mapsto a^+_i,\quad z \mapsto a
\]
defines an injective function $\phi : F \to \A_0$,
which is furthermore an embedding (we only need to check this for pairs!) because $A^\pm \cup B^\pm$ is conflict-free and any $x_i, y_{j}$ for $i \neq j$ are related.
This is a contradiction. We may therefore assume that $z \in \A$.

It is now routine to check that $a^+ a \parallel a^- z$ and $b^+ b \parallel b^- z$ are $\mathcal{O}$-duos for all $a^\pm \in A^\pm, b^\pm \in B^\pm$,
and that 
\[
A^+ a \cup A^- z \cup B^+ b \cup B^- z
\]
is free from conflicts.
From Lemma~\ref{lem:!} we may assume %since $!\overline{\vsup{v} \cup A^\pm a_d \cup B^\pm b_d}$ is empty
that 
\[
\vsup{v} \cup A^+ a \cup A^- z \cup B^+ b \cup B^- z
\] 
is also free from conflicts.
(Alternatively, we could have explicitly ensured this when defining $z$.)
Then the vector:
\begin{align*}
    v'' &= v
    - \sum_{a^\pm\in A^\pm} \lambda_{a^\pm} \cdot a^+ a \between a^- z 
    - \sum_{b^\pm\in B^\pm} \lambda_{b^\pm} \cdot b^+ b \between b^- z  \\
    &= v^{d:a}|^{-d} z + v^{d:b}|^{-d} z + v',
\end{align*}
when grouped into subvectors by the largest atom in each term, has at least one fewer component than $v$.
By the inner inductive hypothesis, we may write
\[
    v'' = \sum_{c^\pm\in C^\pm} \lambda_{c^\pm} \cdot c^+ \between c^-
\]
where $\vsup{v''} \cup C^\pm$ is conflict-free, and one
last application of Lemma~\ref{lem:!} allows us to conclude that
\[
    \vsup{v} \cup A^+ a \cup A^- z \cup B^+ b \cup B^- z \cup C^\pm
\]
is conflict-free as well.
We conclude that
\[
    v = 
      \sum_{a^\pm\in A^\pm} \lambda_{a^\pm} \cdot a^+ a \between a^- z 
    + \sum_{b^\pm\in B^\pm} \lambda_{b^\pm} \cdot b^+ b \between b^- z
    + \sum_{c^\pm\in C^\pm} \lambda_{c^\pm} \cdot c^+ \between c^-
\]
is a decomposition of $v$ into a linear combination of $\cal O$-cogs as required.
\end{proof}

\subsubsection{Vectors without equational conflicts}\label{sec:unambiguous}
% https://q.uiver.app/#q=WzAsMTUsWzAsMSwiYSJdLFsyLDAsImIiXSxbMiwyLCJjIl0sWzEsMSwiYSciXSxbMSwzLCJkIl0sWzIsMywiZSJdLFsyLDQsImYiXSxbMywyLCJcXHJpZ2h0c3F1aWdhcnJvdyJdLFs0LDEsImEiXSxbNSwxLCJhJyJdLFs2LDAsImIiXSxbNiwyLCJjIl0sWzUsMywiZCJdLFs2LDMsImUiXSxbNiw0LCJmIl0sWzAsMSwiIiwyLHsiY29sb3VyIjpbMCw2MCw2MF19XSxbMCwyLCIiLDAseyJjb2xvdXIiOlsyNDAsNjAsNjBdfV0sWzMsMV0sWzMsMl0sWzQsNSwiIiwwLHsiY29sb3VyIjpbMCw2MCw2MF19XSxbNCw2LCIiLDIseyJjb2xvdXIiOlsyNDAsNjAsNjBdfV0sWzAsNF0sWzgsMTBdLFs5LDEwLCIiLDIseyJjb2xvdXIiOlswLDYwLDYwXX1dLFs5LDExLCIiLDAseyJjb2xvdXIiOlsyNDAsNjAsNjBdfV0sWzgsMTJdLFsxMiwxMywiIiwyLHsiY29sb3VyIjpbMCw2MCw2MF19XSxbMTIsMTQsIiIsMix7ImNvbG91ciI6WzI0MCw2MCw2MF19XSxbOCwxMV1d
%\[\begin{tikzcd}[cramped]
%	&& b &&&& b \\
%	a & {a'} &&& a & {a'} \\
%	&& c & \rightsquigarrow &&& c \\
%	& d & e &&& d & e \\
%	&& f &&&& f
%	\arrow[color={rgb,255:red,214;green,92;blue,92}, from=2-1, to=1-3]
%	\arrow[color={rgb,255:red,92;green,92;blue,214}, from=2-1, to=3-3]
%	\arrow[from=2-1, to=4-2]
%	\arrow[from=2-2, to=1-3]
%	\arrow[from=2-2, to=3-3]
%	\arrow[from=2-5, to=1-7]
%	\arrow[from=2-5, to=3-7]
%	\arrow[from=2-5, to=4-6]
%	\arrow[color={rgb,255:red,214;green,92;blue,92}, from=2-6, to=1-7]
%	\arrow[color={rgb,255:red,92;green,92;blue,214}, from=2-6, to=3-7]
%	\arrow[color={rgb,255:red,214;green,92;blue,92}, from=4-2, to=4-3]
%	\arrow[color={rgb,255:red,92;green,92;blue,214}, from=4-2, to=5-3]
%	\arrow[color={rgb,255:red,214;green,92;blue,92}, from=4-6, to=4-7]
%	\arrow[color={rgb,255:red,92;green,92;blue,214}, from=4-6, to=5-7]
%\end{tikzcd}\]
\begin{proposition}\label{prop:?-free-decomposition}
     If $v \in \Ker_\FF \mathcal{O}$ has no equational conflicts, then it can be written as a linear combination of $\cal O$-cogs:
    \[
        v = \sum_{a^\pm \in A^\pm} \lambda_{a^\pm} \cdot a^+ \between a^-
    \]
    with $\lambda_{a^\pm} \in \FF$, where moreover $\vsup{v} \cup A^\pm$ has no equational conflicts.
\end{proposition}
\begin{proof}
We proceed again by induction, first on $|I|$ then on number of relational conflicts in $v$.
The outer base case $I = \emptyset$ is trivial 
--- we have $v = \lambda \cdot ( \between )$, and no conflicts arise
--- and the inner base case is just Proposition~\ref{prop:!-free-decomposition}.

Suppose that a location $(i, a)$ is part of a relational conflict in $v$.
Since every location in $v^{i:a}|^{-i}$ is also a location in $v$, we know that $v^{i:a}|^{-i}$ has no equational conflicts, and by Lemma~\ref{lem:balanced-projected-subvector} it is balanced.
By the outer inductive hypothesis, we get:
\[
    v^{i:a} = (v^{i:a}|^{-i}) a = \sum_{a^\pm\in A^\pm} (\lambda_{a^\pm} \cdot a^+ \between a^-) a
\]
where $[v^{i:a}|^{-i}]\cup A^\pm$ has no equational conflicts, which immediately implies that $[v^{i:a}]\cup A^\pm a$ is free from equational conflicts as well. 
By Lemma~\ref{lem:?} (putting $K=A^\pm a$, $V_0=[v^{i:a}]$ and $V=[v]$) we may assume that
$[v]\cup A^\pm a$ has no equational conflicts.

Take any location $(j,b)$ which is a in a relational conflict with $(a,i)$ in $v$. Then $b\not\in\rt{A^\pm a}$. To see this, note that $b=a$ would imply $j=i$ (since $v$ has no equational conflicts), but that would mean no conflict. On the other hand $b = a^+_k$ for some $a^+ \in A^\pm$ and $k \in I \setminus \{i\}$ (the case of $a^-$ is identical) then $j=k$, but this is not a conflict either, since $a^+a\in\cal O$ and $a=(a^+a)_i$.

   Let $Y$ consist of all the atoms $b$ that are in a relational conflict with $(i,a)$ in $V$. We have shown that
   \[
X = S \cup \sqrt{\vsup{v} \cup A^\pm} \setminus (Y \cup \{a\})   
\]
contains $S \cup \sqrt{A^\pm}$ and that $X, Y, \{a\}$ are pairwise disjoint.
Use Lemma~\ref{lem:free-fresh} (putting $z=a$) to find $\tau \in \Aut(\A)_{(X)}$ such that $\tau(a) \not\in X \cup Y \cup \{a\}$, is greater than $a$, and is not related to any of $Y \cup \{a\}$. Denote $a'=\tau(a)$.
Then, for any $a^\pm \in A^\pm$, it is \textcolor{red}{easy (using \ref{lem:cog-fresh-single})} to check that $a^+ a \parallel a^- a'$ is an $\mathcal{O}$-duo. Moreover, 
$\vsup{v} \cup A^+ a \cup A^- a'$
has no equational conflicts, and the vector
\begin{align*}
    v' = v - \sum_{a^\pm\in A^\pm} \lambda_{a^\pm} \cdot a^+ a \between a^- a'
    &= v -  v^{i:a} + v^{i:a'}%\\
  %  &= v^{i : a_i \mapsto \tau(a_i)}    
\end{align*}
has strictly fewer relationally conflicting locations than $v$, as the location $(i,a)$ disappears from it.
The inner inductive hypothesis tells us 
that 
\[
v' = \sum_{b^\pm\in B^\pm} \lambda_{B^\pm} \cdot b^+ \between b^-
\]
with $\vsup{v'} \cup B^\pm$ free from equational conflicts.

Since $\vsup{v'} \subseteq \vsup{v} \cup A^+ a \cup A^- a'$, Lemma~\ref{lem:?} allows us to assume that 
\[
    \vsup{v} \cup A^+ a \cup A^-a' \cup B^\pm
\]
is also free from equational conflicts. We conclude that 
\[
    v =\sum_{a^\pm\in A^\pm} \lambda_{a^\pm} \cdot a^+ a \between a^-a'+ \sum_{b^\pm\in B^\pm} \lambda_{b^\pm} \cdot b^+ \between b^-
\]
as required.
\end{proof}

\subsubsection{Arbitrary vectors}
% https://q.uiver.app/#q=WzAsMTMsWzAsMSwiXFxkb3RzIl0sWzEsMiwiYSJdLFswLDMsIlxcZG90cyJdLFszLDAsImIiXSxbMyw0LCJjIl0sWzIsMiwiYSciXSxbNCwyLCJcXHJpZ2h0c3F1aWdhcnJvdyJdLFs1LDEsIlxcY2RvdHMiXSxbNSwzLCJcXGNkb3RzIl0sWzYsMiwiYSJdLFs4LDAsImIiXSxbOCw0LCJjIl0sWzcsMiwiYSciXSxbMiwxLCIiLDIseyJjb2xvdXIiOlsyNDAsNjAsNjBdfV0sWzEsMywiIiwyLHsiY29sb3VyIjpbMCw2MCw2MF19XSxbMSw0LCIiLDIseyJjb2xvdXIiOlsyNDAsNjAsNjBdfV0sWzUsM10sWzUsNF0sWzAsMSwiIiwwLHsiY29sb3VyIjpbMCw2MCw2MF19XSxbNyw5LCIiLDIseyJjb2xvdXIiOlswLDYwLDYwXX1dLFs4LDksIiIsMCx7ImNvbG91ciI6WzI0MCw2MCw2MF19XSxbOSwxMF0sWzksMTFdLFsxMiwxMCwiIiwyLHsiY29sb3VyIjpbMCw2MCw2MF19XSxbMTIsMTEsIiIsMSx7ImNvbG91ciI6WzI0MCw2MCw2MF19XV0=
%\[\begin{tikzcd}[cramped]
%	&&& b &&&&& b \\
%	\dots &&&&& \cdots \\
%	& a & {a'} && \rightsquigarrow && a & {a'} \\
%	\dots &&&&& \cdots \\
%	&&& c &&&&& c
%	\arrow[color={rgb,255:red,214;green,92;blue,92}, from=2-1, to=3-2]
%	\arrow[color={rgb,255:red,214;green,92;blue,92}, from=2-6, to=3-7]
%	\arrow[color={rgb,255:red,214;green,92;blue,92}, from=3-2, to=1-4]
%	\arrow[color={rgb,255:red,92;green,92;blue,214}, from=3-2, to=5-4]
%	\arrow[from=3-3, to=1-4]
%	\arrow[from=3-3, to=5-4]
%	\arrow[from=3-7, to=1-9]
%	\arrow[from=3-7, to=5-9]
%	\arrow[color={rgb,255:red,214;green,92;blue,92}, from=3-8, to=1-9]
%	\arrow[color={rgb,255:red,92;green,92;blue,214}, from=3-8, to=5-9]
%	\arrow[color={rgb,255:red,92;green,92;blue,214}, from=4-1, to=3-2]
%	\arrow[color={rgb,255:red,92;green,92;blue,214}, from=4-6, to=3-7]
%\end{tikzcd}\]
\begin{theorem}\label{thm:cog-span-generally}
    Any $v \in \Ker_\FF \mathcal{O}$ can be written as
    \[
        v = \sum_{a^\pm \in A^\pm} \lambda_{a^\pm} \cdot a^+ \between a^-
    \]
    with $\lambda_{a^\pm} \in \FF$.
\end{theorem}
\begin{proof}
This is similar to the proof of Proposition~\ref{prop:?-free-decomposition}, but simpler.
We proceed again by induction, first on $|I|$ then on number of equational conflicts in $v$.
The outer base case $I = \emptyset$ is trivial as before, and 
and the inner base case is Proposition~\ref{prop:?-free-decomposition}.

Suppose that a location $(i, a)$ is part of an equational conflict in $v$. By the outer inductive hypothesis, we get:
\[
    v^{i:a}
    = (v^{i:a}|^{-i}) a 
    = \sum_{a^\pm\in A^\pm} (\lambda_{a^\pm} \cdot a^+ \between a^-) a.
\]
Then neither $S$ nor $\sqrt{A^\pm}$ contains $a$,
so 
\[
X = S \cup \sqrt{\vsup{v} \cup A^\pm} \setminus \{a\}
\]
contains $S \cup \sqrt{A^\pm}$.
Using Lemma~\ref{lem:free-fresh} (putting $z=a$ and $Y=\emptyset$), find $\pi \in \Aut(\A)_{(X)}$ such that $\pi(a)$ is not in $X$, is greater than $a$, and is otherwise unrelated to $a$. Denote $a'=\tau(a)$. 

Then, for any $a^\pm \in A^\pm$, it is easy to check that $a^+ a \parallel a^- a'$ is an $\mathcal{O}$-duo. Moreover, 
the vector
\begin{align*}
    v' = v - \sum_{a^\pm\in A^\pm} \lambda_{a^\pm} \cdot a^+ a \between a^- a'
    &= v -  v^{i:a} + v^{i:a'} 
\end{align*}
%<<<<<<< HEAD
has strictly fewer equationally conflicting locations than $v$, as the location $(i,a)$ disappears from it.
%
It follows from the inner inductive hypothesis that we can write
\[
    v' = \sum_{b^\pm \in B^\pm} \lambda_{b^\pm} \cdot b^+ \between b^-,
\]
which gives a decomposition of $v$ as required.
\end{proof}
