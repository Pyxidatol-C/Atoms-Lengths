\section{Rado graph, with cogs}
In this section we work with the following setting:
\begin{itemize}
    \item 
    $\mathcal{L}_0$ is a (possibly infinite) relational language containing a binary symbol $=$;

    \item 
    $\mathcal{C}_0$ is a free amalgamation class of $\mathcal{L}_0$-structures
    where $=$ is interpreted as true equality, but every other $R \in \mathcal{L}_0$ is interpreted irreflexively.\footnote{We may assume irreflexivity with no loss of generality: see \cite[beginning of \S2.4]{SS20}.}

    \item 
    $\mathcal{L}$ consists of $\mathcal{L}_0$ together with a new binary symbol $<$;

    \item 
    $\mathcal{C}$ consists of $\mathcal{L}$-structures obtained from $\mathcal{C}_0$ by expanding with all possible linear orderings;

    \item 
    $\A_0$ and $\A$ are the respective Fraïssé limits of $\mathcal{C}_0$ and $\mathcal{C}$,
    where without loss of generality we assume $\A_0$ and $\A$ share the same domain so that $\Aut(\A_0) \supseteq \Aut(\A)$.
\end{itemize}

\begin{example}\label{ex:N-Q}
    Take $\mathcal{L}_0$ to consist of $=$ only and $\mathcal{C}_0$ to be all finite sets.
    Then $\A_0$ is isomorphic to the pure set $\N$, whereas $\A$ is isomorphic to $\Q$ with the usual order.
\end{example}

\begin{example}\label{ex:Rado-orderedRado}
    Let $\mathcal{L}_0$ consist of $=$ together with a single binary symbol $\sim$ 
    and let $\mathcal{C}_0$ consist of all finite undirected graphs not embedding the complete graph $K_n$,
    where $3 \leq n$ ($\leq \infty$).
    Then $\A_0$ is the $K_n$-free Henson graph (or the Rado graph when $n = \infty$), and $\A$ is its generically ordered counterpart.
    (Allowing $n = 2$ makes these degenerate to $\N$ and $\Q$ above).
\end{example}

Free amalgamation in $\mathcal{C}_0$ allows us to free atoms in $\A$ from undesired relations.
Let us make this more precise.
Given atoms $z_i, z_j \in Z \subseteq \A$, we say $z_i$ is \emph{related} to $z_j$ in $Z$ if $z_i, z_j$ appear together in some tuple $z_\bullet \in Z^*$ such that $\A \models R(a_\bullet)$ for some $R \in \mathcal{L}_0$ (since $z_i$ and $z_j$ are certainly related by $<$).
Otherwise we say $z_i$ is \emph{unrelated} to $z_j$ in $Z$
--- notice it is still possible that $z_i$ becomes related to $z_j$ in some bigger $Z$ unless every relation in $\mathcal{L}_0$ is at most binary.
\begin{lemma}\label{lem:free-fresh}
    Let $X, Y, \{z\} \subseteq \A$ be disjoint and finite.
    Then there is some automorphism $\tau \in \Aut(\A)$ such that
    \begin{enumerate}
        \item $\tau$ fixes every $x \in X$;
        \item $\tau(z)$ is unrelated to any $y \in Y$ and to $z$ in $X \cup Y \cup \{z, \tau(z)\}$;
        \item $\tau(z) > z$.
    \end{enumerate}
\end{lemma}
\begin{proof}
    In $\A_0$, form the free amalgam
% https://q.uiver.app/#q=WzAsNCxbMCwxLCJYIl0sWzEsMiwiWCBcXGN1cCBcXHt6XFx9Il0sWzEsMCwiWCBcXGN1cCBZIFxcY3VwIFxce3pcXH0iXSxbMiwxLCJYIFxcY3VwIFkgXFxjdXAgXFx7eiwgeidcXH0iXSxbMCwxLCIiLDAseyJzdHlsZSI6eyJ0YWlsIjp7Im5hbWUiOiJob29rIiwic2lkZSI6InRvcCJ9fX1dLFswLDIsIiIsMCx7InN0eWxlIjp7InRhaWwiOnsibmFtZSI6Imhvb2siLCJzaWRlIjoidG9wIn19fV0sWzEsMywieCBcXGluIFggXFxtYXBzdG8geCwgeiBcXG1hcHN0byB6JyIsMSx7InN0eWxlIjp7InRhaWwiOnsibmFtZSI6Imhvb2siLCJzaWRlIjoidG9wIn0sImJvZHkiOnsibmFtZSI6ImRhc2hlZCJ9fX1dLFsyLDMsIlxcc3Vic2V0ZXEiLDEseyJzdHlsZSI6eyJ0YWlsIjp7Im5hbWUiOiJob29rIiwic2lkZSI6InRvcCJ9LCJib2R5Ijp7Im5hbWUiOiJkYXNoZWQifX19XV0=
\[\begin{tikzcd}[cramped]
	& {X \cup Y \cup \{z\}} \\
	X && {X \cup Y \cup \{z, z'\}} \\
	& {X \cup \{z\}}
	\arrow["\subseteq"{description}, dashed, hook, from=1-2, to=2-3]
	\arrow[hook, from=2-1, to=1-2]
	\arrow[hook, from=2-1, to=3-2]
	\arrow["{x \in X \mapsto x, z \mapsto z'}"{description}, dashed, hook, from=3-2, to=2-3]
\end{tikzcd}\]
    so that no element of $Y \cup \{z\}$ is related to $z'$ in $X \cup Y \cup \{z, z'\}$.
    Now we make $X \cup Y \cup \{z, z'\}$ an $\mathcal{L}$-structure: 
    inherit the order on $X \cup Y \cup \{z\}$ from $\A$,
    and declare that $z < z'$ as well as $z' < a$ if $a$, the next element of $X \cup Y$ larger than $z$, exists at all.
    Observe that \[
        x \in X \mapsto x, z \mapsto z'
    \] is still an embedding in presence of the order.
    By homogeneity, we may embed $X \cup Y \cup \{z, z'\}$ into $\A$ via some $f$ which is the identity on $X \cup Y \cup \{z\}$;
    again by homogeneity, we may extend the embedding \[ 
        f(x) = x \in X \mapsto f(x), f(z) \mapsto f(z')
    \] to some automorphism $\tau$ which makes 1), 2), and 3) true.
\end{proof}

On the other hand, an $\mathcal{L}$-structure fails to embed into $\A$ precisely when it embeds some forbidden structure, in which every two distinct elements are related:
\begin{lemma}\label{lem:free-forb}
    Let $\mathcal{F}_0$ consist of minimal (with respect to $\subseteq$) $\mathcal{L}_0$-structures which do not appear in $\mathcal{C}_0$.
    Then
    \begin{enumerate}
        \item $\mathcal{C}_0$ consists of every $\mathcal{L}_0$-structure that does not embed any $F \in \mathcal{F}_0$.
        \item $\mathcal{C}$ consists of every $\mathcal{L}$-structure whose $\mathcal{L}_0$-reduct does not embed any $F \in \mathcal{F}_0$.
        \item In any $F \in \mathcal{F}_0$, every two distinct elements $x, y \in F$ are related by some $R \in \mathcal{L}_0$.
    \end{enumerate}
\end{lemma}
\begin{proof}
    As $\mathcal{C}_0$ is closed under substructures, its complement is closed under superstructures and thus is 
    --- since there are no infinite strictly descending chain of embedded substructures 
    --- determined by its minimal structures.
    2) follows because an $\mathcal{L}$-structure is in $\mathcal{C}$ precisely when its $\mathcal{L}_0$-reduct is in $\mathcal{C}_0$.
    For 3), notice that $F \setminus \{x\}$, $F \setminus \{y\}$ are in $\mathcal{C}_0$ by minimality; 
    therefore so is their free amalgam over $F \setminus \{x, y\}$, which then cannot agree with $F$.
\end{proof}

In what follows, we will juggle with Lemma~\ref{lem:free-fresh} just enough so that we avoid the forbidden structures described in Lemma~\ref{lem:free-forb}.
A main result will be:
\begin{theorem}\label{thm:free-bin-oligo}
    Assume each $R \in \mathcal{L}_0$ has arity at most two.
    Then $\A$ is $\FF$-oligomorphic for any field $\FF$ even with finitely many constants fixed, provided that $\A$ is oligomorphic (for instance if $\mathcal{L}_0$ is finite).
\end{theorem}
And a corollary will be that $\A$ from Examples~\ref{ex:N-Q} and \ref{ex:Rado-orderedRado} is $\FF$-oligomorphic;
as is its reduct $\A_0$.

\subsection{Two reductions: orbits and projections}

To start with, let us view $\A^d$ as $\A^{\{1, \dots, d\}}$ and more generally consider $\A^I$ for a finite indexing set $I \subseteq \N$.
Fix a finite support $S \subseteq \A$.
If $\A$ is oligomorphic, the tuples in $\A^I$ split into finitely many $\Aut(\A)_{(S)}$-equivariant orbits.
Let $\mathcal{O} = \Aut(\A)_{(S)} \cdot o_\bullet$ be one such orbit.
We shall call $\mathcal{O}$ \emph{($S$-)orderly} if $o_i \not\in S$ and if $o_i < o_j$ whenever $i < j$.
By removing the entries in $o_\bullet$ that repeat or come from $S$ and reordering the rest, we can always find an $\Aut(\A)_{(S)}$-equivariant bijection to an $S$-orderly orbit.
An easy observation is that we may focus on a single orderly orbit at a time:
\begin{proposition}\label{prop:reduction-to-one-orbit}
    The following are equivalent:
    \begin{enumerate}
        \item For $d = 0, 1, 2, \dots$ and any finite $S \subseteq \A$, chains of $\Aut(\A)_{(S)}$-equivariant subspaces in $\Lin_\FF \A^d$ are bounded in length;
        \item $\A$ is oligomorphic, and $\Lin_\FF \mathcal{O}$ has fintie length for any orderly orbit $\mathcal{O}$.
    \end{enumerate}
\end{proposition}
\begin{proof}
    We have $\len(\Lin_\FF(\biguplus_i \mathcal{O}_i)) = \len(\bigoplus_i \Lin_\FF \mathcal{O}_i) = \sum_i \len(\Lin_\FF \mathcal{O}_i)$.
\end{proof}

So fix an orderly orbit $\mathcal{O} = \Aut(\A)_{(S)} \cdot o_\bullet \subseteq \A^I$.
From here we take an inductive approach.
By $o |^J _\bullet$ we mean the restriction of $o_\bullet : I \to \A$ to $J \subseteq I$;
we will often write $o |^{-i} _\bullet$ instead of $o |^{ I \setminus \{i\} }_\bullet$.
Note the image $\mathcal{O}|^I$ of $\mathcal{O}$ under this projection agrees with $\Aut(\A)_{(S)} \cdot o |^J _\bullet$ and is still orderly.

To anticipate more general statements later, Let $\EE$ be a finite-dimensional $\FF$-vector space --- for instance, $\FF$ itself.
Things become more interesting when we lift $(-)|^J$ to a linear $\Aut(\A)_{(S)}$-equivariant map
\begin{align*}
    (-)|^J : \Lin_\EE \mathcal{O} &\to \Lin_\EE \mathcal{O}|^J \\
    v &\mapsto v|^J.
\end{align*}
Many cancellations can occur under $(-)|^J$; 
the \emph{projection kernel} is the $\Aut(\A)_{(S)}$-equivariant subspace
\[
    \Ker_\EE \mathcal{O} = \bigcap_{i \in I} \ker{ (-)|^{-i} }
\]
of $\Lin_\EE \mathcal{O}$.

\begin{proposition}\label{prop:reduction-to-kernel}
    The following are equivalent:
    \begin{enumerate}
        \item $\Lin_\FF \mathcal{O}$ has finite length for every orderly orbit $\mathcal{O}$;
        \item $\Ker_\FF \mathcal{O}$ has finite length for every orderly orbit $\mathcal{O}$.
    \end{enumerate}
\end{proposition}
\begin{proof}
    That 1) implies 2) is clear as $\Ker_\FF \mathcal{O} \subseteq \Lin_\FF \mathcal{O}$.
    
    To prove the other implication, assume 2) and let $\mathcal{O} \subseteq \A^I$.
    We proceed by induction on $|I|$.
    If $I = \emptyset$, then $\mathcal{O}$ must be the entire singleton $\A^\emptyset = \{ () \}$; 
    as $\Lin_\FF \mathcal{O}$ has no nontrivial subspaces (let alone finitely supported ones), it has length $1$.
    Now if $|I| \geq 1$, assemble all $|I|$ projection maps into a single map
    \begin{align*}
        \Lin_\FF \mathcal{O} &\to \bigoplus_{i \in I} \mathcal{O}|^{-i} \\
        v &\mapsto ( v|^{-i} )_{i \in I}
    \end{align*}
    whose kernel is precisely $\Ker_\FF \mathcal{O}$.
    We have
    \[
        \len( \Lin_\FF \mathcal{O} ) - \len( \Ker_\FF \mathcal{O} )
        \leq \sum_{i \in I} \len( \Lin_\FF \mathcal{O}|^{-i} )
    \]
    which shows that $\len( \Lin_\FF \mathcal{O} )$ is finite from the assumptions.
\end{proof}

We call a vector from the projection kernel \emph{balanced}.
As we will see in the next subsection, cogs are a prominent example.

\subsection{Cogs}
\begin{definition}
    Let $\mathcal{O} = \Aut(\A)_{(S)} \cdot o_\bullet \subseteq \A^I$ be an $S$-orderly orbit. 
    An \emph{$\mathcal{O}$-cog duo} $a_\bullet \parallel b_\bullet$ consists of $2 \cdot |I|$ atoms in $\A$ with the following $\mathcal{L}$-structure on $\{a_i, b_i \mid i \in I\} \cup S$:
    \begin{enumerate}
        \item 
        $a_{i_1} < b_{i_1} < a_{i_2} < b_{i_2} < \dots < a_{i_d} < b_{i_d}$ where $I$ consists of the indices $i_1 < i_2 < \dots < i_d$;
        
        \item 
        $a_i , b_i < s$ if and only if $o_i < s$;        

        \item 
        any relation in $\mathcal{L}_0$ (in particular $=$) holds for a tuple $c_\bullet$ with entries in $a_I \cup b_I \cup S$ if and only if it holds for $c_\bullet$ with each entry equal to $a_i, b_i$ replaced by $o_i$.
    \end{enumerate}
\end{definition}
Three remarks are in order.
First, each $a_i$ is unrelated to its counterpart $b_i$ in $a_I \cup b_I \cup S$.
Second, given any $J \subseteq I$, the combined tuple $a|^J_\bullet ; b|^{I \setminus J}_\bullet$ lies in $\mathcal{O}$ by homogeneity:
observe
\begin{align*}
    a_j &\mapsto o_j, j \in J ; \\
    b_i &\mapsto o_i, i \in I \setminus J ; \\
    s &\mapsto s, s \in S
\end{align*}
defines an embedding.
Third, by homogeneity still, any two cog duos $a_\bullet \parallel b_\bullet$ and $a'_\bullet \parallel b'_\bullet$ belong to the same $\Aut(\A)_{(S)}$-equivariant orbit.

\begin{definition}
    Given $\lambda \in \EE$ and an $\mathcal{O}$-cog duo $a_\bullet \parallel b_\bullet$, the corresponding \emph{$\mathcal{O}$-cog} with coefficient $\lambda$ is the vector
    \[
        \lambda \cdot a_\bullet \between b_\bullet =
        \sum_{J \subseteq I} (-1)^{|J|} \lambda \cdot  a|^J_\bullet ; b^{I \setminus J}_\bullet
    \]
    in $\Lin_\EE \mathcal{O}$.
    The linear span of all $\mathcal{O}$-cogs with coefficients from $\EE$ is denoted by $\Cog_\EE \mathcal{O}$.
\end{definition}
As remarked above, given any two $\mathcal{O}$-cog duos there is some $\pi \in \Aut(\A)_{(S)}$ such that $\pi \cdot (a_\bullet \parallel b_\bullet) = a'_\bullet \parallel b'_\bullet$ and thus $\pi \cdot (\lambda \cdot a_\bullet \between b_\bullet) = \lambda \cdot a'_\bullet \between b'_\bullet$.
Hence $\Cog_\EE \mathcal{O}$ is an $\Aut(\A)_{(S)}$-equivariant subspace of $\Lin_\EE \mathcal{O}$ and it is generated by cogs based on a single duo.


\begin{proposition}\label{prop:cogs-are-balanced}
    $\Cog_\EE \mathcal{O}$ is contained in $\Ker_\EE \mathcal{O}$.
\end{proposition}
\begin{proof}
    Let $\mathcal{O} \subseteq \A^I$, let $a_\bullet \parallel b_\bullet$ be an $\mathcal{O}$-cog duo, and let $i \in I$.
    The subsets of $I$ come in pairs of $J$ and $J \cup \{i\}$, where $J$ is a subset of $I \setminus \{i\}$.
    The two tuples $a|^{J}_\bullet ; b|^{I \setminus J}_\bullet$ and $a|^{J \cup \{i\}}_\bullet ; b|^{I \setminus (J \cup \{i\})}_\bullet$ differ only on the $i$th entry.
    But this difference gets erased under $(-)|^{-i}$, so the two corresponding terms in $\lambda \cdot a_\bullet \between b_\bullet$ will cancel out and hence $( \lambda \cdot a_\bullet \between b_\bullet )|^{-i} = 0$ overall.
\end{proof}

In fact, cogs arise anywhere.
\begin{lemma}\label{lem:cog-fresh-single}
    Suppose $a_\bullet \parallel b_\bullet$ is an $\mathcal{O}$-cog duo, where $\mathcal{O} \subseteq \A^I$ is $S$-orderly.
    Given $z \in S$, 
    \begin{itemize}
        \item write $S' = S \setminus \{z\}$;
        \item let $j \not\in I$ be such that $\mathcal{O}' = \Aut(\A)_{(S')} \cdot (a_\bullet ; z) \subseteq \A^{I \cup \{j\}}$ is orderly,
        \item let $X \subseteq \A$ be a finite set containing $\{a_i, b_i \mid i \in I\} \cup S'$ but not $z$;
        \item let $Y \subseteq \A$ be any finite set disjoint from $X \cup \{z\}$;
    \end{itemize}
    then the $\tau \in \Aut(\A)_{(X)}$ afforded by Lemma~\ref{lem:free-fresh} gives us an $\mathcal{O}'$-cog duo $(a_\bullet ; z) \parallel (b_\bullet ; \tau \cdot z)$.
\end{lemma}
\begin{proof}
    To start with, notice that we have the required order relations with $z$ and $\tau(z)$.
    Now let $R \in \mathcal{L}_0$ and let $c_\bullet$ be a tuple with entries in $a_{I \cup \{j\}} \cup b_{I \cup \{j\}} \cup S' = a_I \cup b_I \cup S \cup \{\tau \cdot z\}$;
    we want to show that \begin{equation}\label{eq:cog-building-relation}
        \A \models R(c_\bullet) \text{ if and only if } 
        \A \models R(d_\bullet), \tag{\P}
    \end{equation} 
    where $d_\bullet$ is obtained from $c_\bullet$ by replacing every $b_i$ with $a_i$ and $\tau \cdot z$ with $z$.
    We split into three cases.
    \begin{enumerate}
        \item 
        Suppose $z$ and $\tau \cdot z$ both appear in $c_\bullet$. 
        Then $\A \not\models R(c_\bullet)$ because $\tau \cdot z$ is unrelated to $z$ by design, 
        and $\A \not\models R(d_\bullet)$ because $R$ is irreflexive and $z$ appears twice.
        
        \item 
        So if $z$ appears in $c_\bullet$, we may assume that $\tau \cdot z$ does not --- that is, $c_\bullet$ has entries in $a_I \cup b_I \cup S$.
        In this case \eqref{eq:cog-building-relation} follows from the assumption that $a_\bullet \parallel b_\bullet$ is a cog duo in $\mathcal{O} = \Aut(\A)_{(S)} \cdot a_\bullet$.

        \item
        Finally, suppose that $\tau \cdot z$ appears in $c_\bullet$ but not $z$.
        Then only $z$ appears in $\tau^{-1} \cdot c_\bullet$, and $d_\bullet$ is obtained by replacing every $b_i$ with $a_i$ in $\tau^{-1} \cdot c_\bullet$.
        But $\A \models R(c_\bullet)$ if and only if $\A \models R(\tau^{-1} \cdot c_\bullet)$, and now \eqref{eq:cog-building-relation} holds as we discussed for the case above. 
        \qedhere
    \end{enumerate}
\end{proof}

Starting from an empty cog duo, we may apply the previous lemma inductively.
\begin{proposition}\label{prop:cog-fresh-full}
    Let $\mathcal{O} \subseteq \A^I$ be an $S$-orderly orbit.
    Then for any $a_\bullet \in \mathcal{O}$, there is: 
    \begin{enumerate}
        \item some $b_\bullet \in \mathcal{O}$ such that $a_\bullet \parallel b_\bullet$ is an $\mathcal{O}$-cog duo; and
        \item for each $i \in I$, some $\pi_i \in \Aut(\A)_{(S \cup \{a_j, b_j \mid j \in I, j \neq i \})}$ sending $a_i \mapsto b_i$.
    \end{enumerate}
\end{proposition}
\begin{proof}
    Enumerate the indices of $I$ as $i_1, \dots, i_d$.
    Suppose that we have found $b_{i_1}, \dots, b_{i_k}$ such that 
    \[
        a|^{\{i_1, \dots, i_k\}}_\bullet \parallel (i_1 \mapsto b_{i_1}, \dots, i_k \mapsto b_{i_k})
    \] 
    is a cog duo in $\mathcal{O}_k = \Aut(\A)_{(S \cup \{a_{i_{k+1}}, \dots, a_{i_d}\})} \cdot a|^{\{i_1, \dots, i_k\}}_\bullet$ 
    --- note that $() \parallel ()$ is certainly a cog duo in $\mathcal{O}_0$ at the start.
    If $k < d$, with $z = a_{i_{k+1}}$, $X = \{a_{i_1}, b_{i_1}, \dots, a_{i_k}, b_{i_k}\} \cup S \cup \{a_{i_{k+2}}, \dots, a_{i_d}\}$, and $Y = \emptyset$, 
    a straightforward application of Lemma~\ref{lem:cog-fresh-single} yields an atom $b_{i_{k+1}}$ that makes
    \[
        a|^{\{i_1, \dots, i_k, i_{k+1}\}}_\bullet \parallel (i_1 \mapsto b_{i_1}, \dots, i_k \mapsto b_{i_k}, i_{k+1} \mapsto b_{i_{k+1}})
    \] a cog duo in $\mathcal{O}_{k+1}$.
    We thus obtain the desired cog duo in $\mathcal{O}_d = \mathcal{O}$ as we reach $k = d$.
    The automorphisms $\pi_{i_k}$ now come directly from homogeneity and the definition of an $\mathcal{O}$-cog duo: 
    the map
    \begin{align*}
        a_{i_1} \mapsto a_{i_1}, \dots,{} &a_{i_k} \mapsto b_{i_k}, \dots, a_{i_d} \mapsto a_{i_d} \\
        b_{i_1} \mapsto b_{i_1}, \dots,{} &\phantom{a_{i_k} \mapsto b_{i_k}}, \dots, b_{i_d} \mapsto b_{i_d}, s \in S \mapsto s
    \end{align*}
    is an embedding.
\end{proof}

The result below substantiates the slogan that cogs are found everywhere. 
\begin{theorem}
    Any $\Aut(\A)_{(S)}$-equivariant subspace $V$ of $\Lin_\EE \mathcal{O}$ contains $\Cog_{\EE'} \mathcal{O}$,
    where $\mathcal{O} \subseteq \A^I$ is $S$-orderly and $\EE'$ is the subspace of $\EE$ spanned by $\{ v(a_\bullet) \mid v \in V, a_\bullet \in \mathcal{O} \}$.
\end{theorem}
\begin{proof}
    Pick any $v \in V$ and $a_\bullet \in \mathcal{O}$;
    it is enough to show that $V$ contains $v(a_\bullet) \cdot a_\bullet \between b_\bullet$ for some $\mathcal{O}$-cog duo $a_\bullet \parallel b_\bullet$.
    Actually, write 
    \[
        S' = S \cup \{c_i \mid v(c_\bullet) \neq 0, i \in I\} \setminus \{a_i \mid i \in I\} \supseteq S
    \]
    and put $\mathcal{O}' = \Aut(\A)_{(S')} \cdot a_\bullet \subseteq \mathcal{O}$ --- then $\mathcal{O}'$ is $S'$-orderly.
    By Proposition~\ref{prop:cog-fresh-full}, we can find $b_\bullet \in \mathcal{O}'$ such that $a_\bullet \parallel b_\bullet$ is a cog duo in $\mathcal{O}'$ and \emph{a fortiori} a cog duo in $\mathcal{O}$.
    Also take the automorphisms $\pi_{i_1}, \dots, \pi_{i_d}$ from there, where $i_1, \dots, i_d$ enumerate $I$.
    Now define $v_0 = v$ and \[
        v_k = v_{k-1} - \pi_{i_k} \cdot v_{k-1}.
    \]
    We can check inductively that for $k = 0, 1, \dots, d$, 
    with $\mathcal{O}_k = \{ c_\bullet \mid v(c_\bullet) \neq 0, \{c_{i_1}, \dots, c_{i_k}, \dots, c_{i_d}\} \supseteq \{a_{i_1}, \dots, a_{i_k}\} \}$ we have
    \[
        v_k = \sum_{c_\bullet \in \mathcal{O}_k} \sum_{J \subseteq \{i_1, \dots, i_k\}} (-1)^{|J|} v(c_\bullet) \cdot c|^J_\bullet ; b|^{I \setminus J}_\bullet.
    \]
    But $\{c_{i_1}, \dots, c_{i_d}\} \supseteq \{a_{i_1}, \dots, a_{i_d}\}$ means that $c_\bullet = a_\bullet$, so at the end $v_d$ is the desired $\mathcal{O}$-cog.
\end{proof}

\begin{corollary}
    $\Cog_\FF \mathcal{O}$ has length $1$.
\end{corollary}
\begin{proof}
    Let $V \subseteq \Cog_\FF \mathcal{O}$ be a non-zero $\Aut(\A)_{(S)}$-equivariant subspace.
    Then $\EE'$ is the entire field $\FF$, and by above $V$ must be $\Cog_\FF \mathcal{O}$ itself.
\end{proof}

In light of Propositions~\ref{prop:reduction-to-one-orbit} and \ref{prop:reduction-to-kernel}, we will be able to prove the finite length property for an oligomorphic structure with free amalgamation over any field and support if we know $\Ker_\FF \mathcal{O} = \Cog_\FF \mathcal{O}$.
Let us now attempt to show that.

\subsection{Subvectors}
This is a good time to recall a view we have tacitly taken:
with $\mathcal{O}$ as a standard basis, a vector $v \in \Lin_\EE \mathcal{O}$ is just a finite set of pairs in $\EE \times \mathcal{O}$.
A \emph{subvector} of $v$ is a subset of these pairs.

Now suppose as usual that $\mathcal{O} \subseteq \A^I$ is $S$-orderly.
Given $i \in I$ and $a = a_i \in \A$ for some $a_\bullet \in \mathcal{O}$, 
we write \[
    \mathcal{O}^{i:a} = \{b_\bullet \in \mathcal{O} \mid b_i = a\};
\]
this is an $\Aut(\A)_{(S \cup \{a\})}$-orbit, and its projection $\mathcal{O}^{i:a} |^{-i} = \Aut(\A)_{(S \cup \{a\})} \cdot a|^{-i}_\bullet$ is orderly.
For a vector $v \in \Lin_\EE \mathcal{O}$, by
\[
    v^{i:a} \in \Lin_\EE \mathcal{O}^{i:a}
\]
we mean the subvector consisting of all pairs in $\EE \times \mathcal{O}^{i:a}$.

\begin{lemma}
    Let $v \in \Lin_\EE \mathcal{O}$ be balanced. 
    Then any projected subvector $v^{i:a}|^{-i} \in \Lin_\EE \mathcal{O}^{i:a}|^{-i}$ is also balanced.
\end{lemma}
\begin{proof}
    Let $j \in I \setminus \{i\}$. 
    By assumption we have \[
        0 = v|^{-j} = \sum_a v^{i:a}|^{-j}
    \] in $\Lin_\EE \A^{I \setminus \{j\}}$,
    so by looking at $i$th entries we see that each $v^{i:a}|^{-j}$ must be the zero vector.
    Hence so is $v^{i:a}|^{-j}|^{-i} = v^{i:a}|^{-i}|^{-j}$,
    which shows that $v^{i:a}|^{-i}$ is in the projection kernel.
\end{proof}

So we can try to prove $\Ker_\EE \mathcal{O} \subseteq \Cog_\EE \mathcal{O}$ for any orderly $\mathcal{O} \subseteq \A^I$ by inducting on $|I|$;
we just need to reassemble the various cogs in $\mathcal{O}^{i:a}|^{-i}$ back into $\mathcal{O}$-cogs.
Unfortunately we were only able to do so under the hypothesis that $\mathcal{L}_0$ is an at most binary language, 
which we henceforth assume.

\subsection{Special case: vectors free enough for binary relations}
We will begin by showing any $v \in \Ker_\EE \mathcal{O}$ lies in $\Cog_\EE \mathcal{O}$ provided that $v$ satisfies an additional condition which, 
as we will explain in the subsection, may be assumed without loss of generality. 
We motivate and introduce this condition now. 

Consider the atoms in the tuples 
\[
    \mathcal{O}[v] = \{ b_\bullet \in \mathcal{O} \mid v(b_\bullet) \neq 0 \}.
\]
that appear in a vector $v \in \Lin_\EE \mathcal{O}$, with $\mathcal{O}$ being $S$-orderly.
Take $a = b_i$ to be one such atom,
and let $R_1, R_2 \in \mathcal{L}_0$ be unary and binary respectively.
Then whether $R_1(a)$, $R_2(a, s)$, $R_2(s, a)$ hold in $\A$ given $s \in S$ is determined by $\mathcal{O}$
--- more preciesly, by whether $R_1(o_i)$, $R_2(o_i, s)$, $R_2(s, o_i)$ hold in $\A$ for any $o_\bullet \in \mathcal{O}$.

Now take another atom $a' = b'_j$ occurring in $\mathcal{O}[v]$.
What can be said about $R_2(a, a')$?
Except in those happy dispositions where $b'_\bullet = b_\bullet$, not much --- whether $\A \models R_2(o_i, o_j)$ need not be a constraint for $a$ and $a'$.
The index $j$ may not even be unique: we may well have $b'_j = b''_k$ for $j \neq k$.


\begin{definition}
    Let $\mathcal{O} = \Aut(\A)_{(S)} \cdot o_\bullet$ be orderly in $\A^I$.
    We say a finite family $\{ b^{(k)}_\bullet \in \mathcal{O} \mid k \in K\}$ is
    \begin{enumerate}
        \item 
        \emph{well-indexed} if $b^{(k)}_i \mapsto i$ for $k \in K$ defines a function;

        \item 
        \emph{free enough} for an $r$-ary relation $R \in \mathcal{L}_0$ if it is well-indexed and for any $r$-tuple $z_\bullet$ with distinct entries in $\{a^{(k)}_i \mid i \in I, k \in K\} \cup S$,
        \begin{enumerate}
            \item either $R$ holds for $z_\bullet$ if and only if $R$ holds for $z_\bullet$ with each entry equal to some $a^{(k)}_i$ being replaced by $o_i$ --- note this $i$ is uniquely defined;
            \item or some two distinct $a^{(k)}_i, a^{(k')}_{i'}$ appearing in $z_\bullet$ are unrelated in $\{a^{(k)}_i \mid i \in I, k \in K\} \cup S$.
        \end{enumerate}
    \end{enumerate}
\end{definition}

\begin{theorem}
    Let $v \in \Ker_\EE \mathcal{O}$ be such that $\mathcal{O}[v]$ is free for any relation in $\mathcal{L}_0$.
    Then 
    \[
        v = \sum_{k \in K} v(a^{(k)}_\bullet) \cdot x^{(k)}_\bullet \between y^{(k)}_\bullet
    \]
    for some finite $K$, where moreover
    \[
        \mathcal{O}[v] \cup \{ x^{(k)}_\bullet, y^{(k)}_\bullet \mid k \in K \}
    \]
    is free for any relation in $\mathcal{L}_0$.
\end{theorem}

\begin{proposition}
    If $\mathcal{O}[v]$ is free enough for $R$,
    then so is $\mathcal{O}^{i:a}|^{-i}[v^{i:a}|^{-i}]$.
\end{proposition}

\begin{proposition}
    We can make $\mathcal{O}[v] \cup \{ x^{(a, k)}_\bullet ; a, y^{(a, k)}_\bullet ; a \mid a, k \}$ free enough for $R$.
\end{proposition}

