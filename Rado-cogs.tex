\section{Rado graph, with cogs}
In this section we work with the following setting:
\begin{itemize}
    \item 
    $\mathcal{L}_0$ is a (possibly infinite) relational language containing a binary symbol $=$;

    \item 
    $\mathcal{C}_0$ is a free amalgamation class of $\mathcal{L}_0$-structures
    where $=$ is interpreted as true equality, but every other $R \in \mathcal{L}_0$ is interpreted irreflexively.\footnote{We may assume irreflexivity with no loss of generality: see \cite[beginning of \S2.4]{SS20}.}

    \item 
    $\mathcal{L}$ consists of $\mathcal{L}_0$ together with a new binary symbol $<$;

    \item 
    $\mathcal{C}$ consists of $\mathcal{L}$-structures obtained from $\mathcal{C}_0$ by expanding with all possible linear orderings;

    \item 
    $\A_0$ and $\A$ are the respective Fraïssé limits of $\mathcal{C}_0$ and $\mathcal{C}$,
    where without loss of generality we assume $\A_0$ and $\A$ share the same domain so that $\Aut(\A_0) \supseteq \Aut(\A)$.
\end{itemize}

\begin{example}\label{ex:N-Q}
    Take $\mathcal{L}_0$ to consist of $=$ only and $\mathcal{C}_0$ to be all finite sets.
    Then $\A_0$ is isomorphic to the pure set $\N$, whereas $\A$ is isomorphic to $\Q$ with the usual order.
\end{example}

\begin{example}\label{ex:Rado-orderedRado}
    Let $\mathcal{L}_0$ consist of $=$ together with a single binary symbol $\sim$ 
    and let $\mathcal{C}_0$ consist of all finite undirected graphs not embedding the complete graph $K_n$,
    where $3 \leq n$ ($\leq \infty$).
    Then $\A_0$ is the $K_n$-free Henson graph (or the Rado graph when $n = \infty$), and $\A$ is its generically ordered counterpart.
    (Allowing $n = 2$ makes these degenerate to $\N$ and $\Q$ above).
\end{example}

Free amalgamation in $\mathcal{C}_0$ allows us to free atoms in $\A$ from undesired relations.
Let us make this more precise.
Given atoms $z_i, z_j \in Z \subseteq \A$, we say $z_i$ is \emph{related} to $z_j$ in $Z$ if $z_i, z_j$ appear together in some tuple $z_\bullet \in Z^*$ such that $\A \models R(a_\bullet)$ for some $R \in \mathcal{L}_0$ (since $z_i$ and $z_j$ are certainly related by $<$).
Otherwise we say $z_i$ is \emph{unrelated} to $z_j$ in $Z$
--- notice it is still possible that $z_i$ becomes related to $z_j$ in some bigger $Z$ unless every relation in $\mathcal{L}_0$ is at most binary.
\begin{lemma}\label{lem:free-fresh}
    Let $X, Y, \{z\} \subseteq \A$ be disjoint and finite.
    Then there is some automorphism $\tau \in \Aut(\A)$ such that
    \begin{enumerate}
        \item $\tau$ fixes every $x \in X$;
        \item $\tau(z)$ is unrelated to any $y \in Y$ and to $z$ in $X \cup Y \cup \{z, \tau(z)\}$;
        \item $\tau(z) > z$.
    \end{enumerate}
\end{lemma}
\begin{proof}
    In $\A_0$, form the free amalgam
% https://q.uiver.app/#q=WzAsNCxbMCwxLCJYIl0sWzEsMiwiWCBcXGN1cCBcXHt6XFx9Il0sWzEsMCwiWCBcXGN1cCBZIFxcY3VwIFxce3pcXH0iXSxbMiwxLCJYIFxcY3VwIFkgXFxjdXAgXFx7eiwgeidcXH0iXSxbMCwxLCIiLDAseyJzdHlsZSI6eyJ0YWlsIjp7Im5hbWUiOiJob29rIiwic2lkZSI6InRvcCJ9fX1dLFswLDIsIiIsMCx7InN0eWxlIjp7InRhaWwiOnsibmFtZSI6Imhvb2siLCJzaWRlIjoidG9wIn19fV0sWzEsMywieCBcXGluIFggXFxtYXBzdG8geCwgeiBcXG1hcHN0byB6JyIsMSx7InN0eWxlIjp7InRhaWwiOnsibmFtZSI6Imhvb2siLCJzaWRlIjoidG9wIn0sImJvZHkiOnsibmFtZSI6ImRhc2hlZCJ9fX1dLFsyLDMsIlxcc3Vic2V0ZXEiLDEseyJzdHlsZSI6eyJ0YWlsIjp7Im5hbWUiOiJob29rIiwic2lkZSI6InRvcCJ9LCJib2R5Ijp7Im5hbWUiOiJkYXNoZWQifX19XV0=
\[\begin{tikzcd}[cramped]
	& {X \cup Y \cup \{z\}} \\
	X && {X \cup Y \cup \{z, z'\}} \\
	& {X \cup \{z\}}
	\arrow["\subseteq"{description}, dashed, hook, from=1-2, to=2-3]
	\arrow[hook, from=2-1, to=1-2]
	\arrow[hook, from=2-1, to=3-2]
	\arrow["{x \in X \mapsto x, z \mapsto z'}"{description}, dashed, hook, from=3-2, to=2-3]
\end{tikzcd}\]
    so that no element of $Y \cup \{z\}$ is related to $z'$ in $X \cup Y \cup \{z, z'\}$.
    Now we make $X \cup Y \cup \{z, z'\}$ an $\mathcal{L}$-structure: 
    inherit the order on $X \cup Y \cup \{z\}$ from $\A$,
    and declare that $z < z'$ as well as $z' < a$ if $a$, the next element of $X \cup Y$ larger than $z$, exists at all.
    Observe that \[
        x \in X \mapsto x, z \mapsto z'
    \] is still an embedding in presence of the order.
    By homogeneity, we may embed $X \cup Y \cup \{z, z'\}$ into $\A$ via some $f$ which is the identity on $X \cup Y \cup \{z\}$;
    again by homogeneity, we may extend the embedding \[ 
        f(x) = x \in X \mapsto f(x), f(z) \mapsto f(z')
    \] to some automorphism $\tau$ which makes 1), 2), and 3) true.
\end{proof}

On the other hand, an $\mathcal{L}$-structure fails to embed into $\A$ precisely when it embeds some forbidden structure, in which every two distinct elements are related:
\begin{lemma}\label{lem:free-forb}
    Let $\mathcal{F}_0$ consist of minimal (with respect to $\subseteq$) $\mathcal{L}_0$-structures which do not appear in $\mathcal{C}_0$.
    Then
    \begin{enumerate}
        \item $\mathcal{C}_0$ consists of every $\mathcal{L}_0$-structure that does not embed any $F \in \mathcal{F}_0$.
        \item $\mathcal{C}$ consists of every $\mathcal{L}$-structure whose $\mathcal{L}_0$-reduct does not embed any $F \in \mathcal{F}_0$.
        \item In any $F \in \mathcal{F}_0$, every two distinct elements $x, y \in F$ are related by some $R \in \mathcal{L}_0$.
    \end{enumerate}
\end{lemma}
\begin{proof}
    As $\mathcal{C}_0$ is closed under substructures, its complement is closed under superstructures and thus is 
    --- since there are no infinite strictly descending chain of embedded substructures 
    --- determined by its minimal structures.
    2) follows because an $\mathcal{L}$-structure is in $\mathcal{C}$ precisely when its $\mathcal{L}_0$-reduct is in $\mathcal{C}_0$.
    For 3), notice that $F \setminus \{x\}$, $F \setminus \{y\}$ are in $\mathcal{C}_0$ by minimality; 
    therefore so is their free amalgam over $F \setminus \{x, y\}$, which then cannot agree with $F$.
\end{proof}

In what follows, we will juggle with Lemma~\ref{lem:free-fresh} just enough so that we avoid the forbidden structures described in Lemma~\ref{lem:free-forb}.
A main result will be:
\begin{theorem}\label{thm:free-bin-oligo}
    Assume each $R \in \mathcal{L}_0$ has arity at most two.
    Then $\A$ is $\FF$-oligomorphic for any field $\FF$ even with finitely many constants fixed, provided that $\A$ is oligomorphic (for instance if $\mathcal{L}_0$ is finite).
\end{theorem}
And a corollary will be that $\A$ from Examples~\ref{ex:N-Q} and \ref{ex:Rado-orderedRado} is $\FF$-oligomorphic;
as is its reduct $\A_0$.

\subsection{Two reductions: orbits and projections}

To start with, let us view $\A^d$ as $\A^{\{1, \dots, d\}}$ and more generally consider $\A^I$ for a finite indexing set $I \subseteq \N$.
Fix a finite support $S \subseteq \A$.
If $\A$ is oligomorphic, the tuples in $\A^I$ split into finitely many $\Aut(\A)_{(S)}$-equivariant orbits.
Let $\mathcal{O} = \Aut(\A)_{(S)} \cdot o_\bullet$ be one such orbit.
We shall call $\mathcal{O}$ \emph{orderly} if $o_i \not\in S$ and if $o_i < o_j$ whenever $i < j$.
By removing the entries in $o_\bullet$ that repeat or come from $S$ and reordering the rest, we can always find an $\Aut(\A)_{(S)}$-equivariant bijection to an orderly orbit.
An easy observation is that we may focus on a single orderly orbit at a time:
\begin{proposition}
    The following are equivalent:
    \begin{enumerate}
        \item For $d = 0, 1, 2, \dots$ and any finite $S \subseteq \A$, chains of $\Aut(\A)_{(S)}$-equivariant subspaces in $\Lin_\FF \A^d$ are bounded in length;
        \item $\A$ is oligomorphic, and $\Lin_\FF \mathcal{O}$ has fintie length for any orderly orbit $\mathcal{O}$.
    \end{enumerate}
\end{proposition}
\begin{proof}
    We have $\len(\Lin_\FF(\biguplus_i \mathcal{O}_i)) = \len(\bigoplus_i \Lin_\FF \mathcal{O}_i) = \sum_i \len(\Lin_\FF \mathcal{O}_i)$.
\end{proof}

So fix an orderly orbit $\mathcal{O} = \Aut(\A)_{(S)} \cdot o_\bullet \subseteq \A^I$.
From here we take an inductive approach.
By $o |^J _\bullet$ we mean the restriction of $o_\bullet : I \to \A$ to $J \subseteq I$;
we will often write $o |^{-i} _\bullet$ instead of $o |^{ I \setminus \{i\} }_\bullet$.
Note the image $\mathcal{O}|^I$ of $\mathcal{O}$ under this projection agrees with $\Aut(\A)_{(S)} \cdot o |^J _\bullet$ and is still orderly.

Things become more interesting when we lift $(-)|^J$ to an additive $\Aut(\A)_{(S)}$-equivariant map
\begin{align*}
    (-)|^J : \Lin_\AA \mathcal{O} &\to \Lin_\AA \mathcal{O}|^J \\
    v &\mapsto v|^J
\end{align*}
where $\AA$ is an Abelian group.
(Of course if $\AA$ is an $\FF$-vector space, this map is moreover linear.)
Many cancellations can occur under $(-)|^J$; 
the \emph{projection kernel} is the $\Aut(\A)_{(S)}$-equivariant subset
\[
    \Ker_\AA \mathcal{O} = \bigcap_{i \in I} \ker{ (-)|^{-i} }
\]
of $\Lin_\AA \mathcal{O}$ that is closed under addition (and $\FF$-linear combinations if $\AA$ has an $\FF$-linear structure).

\begin{proposition}
    The following are equivalent:
    \begin{enumerate}
        \item $\Lin_\FF \mathcal{O}$ has finite length for every orderly orbit $\mathcal{O}$;
        \item $\Ker_\FF \mathcal{O}$ has finite length for every orderly orbit $\mathcal{O}$.
    \end{enumerate}
\end{proposition}
\begin{proof}
    That 1) implies 2) is clear as $\Ker_\FF \mathcal{O} \subseteq \Lin_\FF \mathcal{O}$.
    
    To prove the other implication, assume 2) and let $\mathcal{O} \subseteq \A^I$.
    We proceed by induction on $|I|$.
    If $I = \emptyset$, then $\mathcal{O}$ must be the entire singleton $\A^\emptyset = \{ () \}$; 
    as $\Lin_\FF \mathcal{O}$ has no nontrivial subspaces (let alone finitely supported ones), it has length $1$.
    Now if $|I| \geq 1$, assemble all $|I|$ projection maps into a single map
    \begin{align*}
        \Lin_\FF \mathcal{O} &\to \bigoplus_{i \in I} \mathcal{O}|^{-i} \\
        v &\mapsto ( v|^{-i} )_{i \in I}
    \end{align*}
    whose kernel is precisely $\Ker_\FF \mathcal{O}$.
    We have
    \[
        \len( \Lin_\FF \mathcal{O} ) - \len( \Ker_\FF \mathcal{O} )
        \leq \sum_{i \in I} \len( \Lin_\FF \mathcal{O}|^{-i} )
    \]
    which shows that $\len( \Lin_\FF \mathcal{O} )$ is finite.
\end{proof}

We call a vector from the projection kernel \emph{balanced}.
As we will see in the next subsection, cogs are a prominent example.

\subsection{Cogs}
\begin{definition}
    Let $\mathcal{O} = \Aut(\A)_{(S)} \cdot o_\bullet \subseteq \A^I$ be an orderly orbit. 
    An \emph{$\mathcal{O}$-cog duo} $a_\bullet \parallel b_\bullet$ consists of $2 \cdot |I|$ atoms in $\A$ with the following $\mathcal{L}$-structure on $\{a_i, b_i \mid i \in I\} \cup S$:
    \begin{enumerate}
        \item 
        $a_{i_1} < b_{i_1} < a_{i_2} < b_{i_2} < \dots < a_{i_d} < b_{i_d}$ where $I$ consists of the indices $i_1 < i_2 < \dots < i_d$;
        
        \item 
        $a_i , b_i < s$ if and only if $o_i < s$;        

        \item 
        any relation in $\mathcal{L}_0$ (in particular $=$) holds for a tuple $c_\bullet$ with entries in $a_I \cup b_I \cup S$ if and only if it holds for $c_\bullet$ with each entry equal to $a_i, b_i$ replaced by $o_i$.
    \end{enumerate}
\end{definition}
Three remarks are in order.
First, each $a_i$ is unrelated to its counterpart $b_i$ in $a_I \cup b_I \cup S$.
Second, given any $J \subseteq I$, the combined tuple $a|^J ; b|^{I \setminus J}$ lies in $\mathcal{O}$ by homogeneity:
observe
\begin{align*}
    a_j &\mapsto o_j, j \in J ; \\
    b_i &\mapsto b_i, i \in I \setminus J ; \\
    s &\mapsto s, s \in S
\end{align*}
defines an embedding.
Third, by homogeneity still, any two cog duos $a_\bullet \parallel b_\bullet$ and $a'_\bullet \parallel b'_\bullet$ belong to the same $\Aut(\A)_{(S)}$-equivariant orbit.

\begin{definition}
    Given $\lambda \in \AA$ and an $\mathcal{O}$-cog duo $a_\bullet \parallel b_\bullet$, the corresponding \emph{$\mathcal{O}$-cog} with coefficient $\lambda$ is the vector
    \[
        \lambda \cdot a_\bullet \between b_\bullet =
        \sum_{J \subseteq I} (-1)^{|J|} \lambda \cdot  a|^J ; b^{I \setminus J}
    \]
    in $\Lin_\AA \mathcal{O}$.
    The collection of all $\mathcal{O}$-cogs with coefficients from $\A$ is denoted by $\Cog_\AA \mathcal{O}$.
\end{definition}
As remarked above, given any two $\mathcal{O}$-cog duos there is some $\pi \in \Aut(\A)_{(S)}$ such that $\pi \cdot (a_\bullet \parallel b_\bullet) = a'_\bullet \parallel b'_\bullet$ and thus $\pi \cdot (\lambda \cdot a_\bullet \between b_\bullet) = \lambda \cdot a'_\bullet \between b'_\bullet$.
Hence $\Cog_\AA \mathcal{O}$ is $\Aut(\A)_{(S)}$-equivariant.
Also, it is clearly closed under addition (and under $\FF$-linear combinations if $\AA$ is so).


\begin{proposition}
    $\Cog_\AA \mathcal{O}$ is contained in $\Ker_\AA \mathcal{O}$.
\end{proposition}
\begin{proof}
    Let $\mathcal{O} \subseteq \A^I$ and let $i \in I$.
    The subsets of $I$ come in pairs of $J$ and $J \cup \{i\}$, where $J$ is a subset of $I \setminus \{i\}$.
    The two tuples $a|^{J} ; b|^{I \setminus J}$ and $a|^{J \cup \{i\}} ; b|^{I \setminus (J \cup \{i\})}$ differ only on the $i$th entry.
    But this difference gets erased under $(-)|^{-i}$, so the two corresponding terms in $\lambda \cdot a_\bullet \between b_\bullet$ will cancel out and hence $( \lambda \cdot a_\bullet \between b_\bullet )|^{-i} = 0$ overall.
\end{proof}

Cogs arise everywhere.

\begin{lemma}
    Let $\mathcal{O} = \Aut(\A)_{(S)} \cdot o_\bullet \subseteq \A^I$ be orderly,
    and suppose $a_\bullet \between b_\bullet$ is an $\mathcal{O}$-cog.
    Given $s \in S$, let $j \not\in I$ be such that $\Aut(\A)_{(S \setminus \{s\})} \cdot (a_\bullet ; s) \in \A^{I \cup \{j\}}$ is orderly.
    
\end{lemma}

\begin{proposition}
    \color{red}
    $\Cog_\AA \mathcal{O}$ has length $1$ if $\AA = \FF$ is a field.
\end{proposition}


\subsection{Building with cogs}
\begin{theorem}
    Assume $|I| \leq 2$ or that each relation in $\mathcal{L}_0$ has arity at most $2$.
    Then $\Ker_\AA \mathcal{O} = \Cog_\AA \mathcal{O}$ for any orderly orbit $\mathcal{O} \subseteq \A^I$. 
\end{theorem}

\begin{definition}
    Let $\mathcal{O} = \Aut(\A)_{(S)} \cdot o_\bullet \subseteq \A^I$ be orderly.
    We say a finite family of tuples $\{ a^{(k)}_\bullet \mid k \in K \} \subseteq \mathcal{O}$ is
    \begin{enumerate}
        \item \emph{coordinated} if $a^{(k)}_i \mapsto i$ defines a function;
        \item \emph{loopless} if it is coordinated and for each $i \in I$, any $a^{(k)}_i, a^{(k')}_i$ are unrelated in $\{ a^{(k)}_i \mid k \in I, i \in I \} \cup S$;
        \item \emph{sufficiently freed} if it is coordinated and whenever 
        \[
            a^{(k_1)}_{i_1} \mapsto o_{i_1}, \dots, a^{(k_n)}_{i_n} \mapsto o_{i_n}, s \in S \mapsto s
        \]
        fails to be an embedding, some $a^{(k_j)}_{i_j} \neq a^{(k_{j'})}_{i_{j'}}$ are unrelated in $\{ a^{(k)}_i \mid k \in I, i \in I \} \cup S$.
    \end{enumerate}
\end{definition}

\textcolor{red}{By adding appropriate cogs, we can make any vector's corresponding family coordinated then loopless, and furthermore sufficiently freed if $\mathcal{L}_0$ is at most binary.}

\begin{remark}
    Our results cover David Evans's Proposition 3.19 which assumes $|I| \leq 2$.
    Indeed, the failure to be an embedding means some relation $R$ is not respected.
    If the family is loopless and $R$ is at least ternary, we will have an unrelated pair.
    Otherwise $R$ must be binary and we know how to free such pairs.
\end{remark}

