\section{Rado graph, with cogs}
In this section we work with the following setting:
\begin{itemize}
    \item 
    $\mathcal{L}_0$ is a (possibly infinite) relational language containing a binary symbol $=$;

    \item 
    $\mathcal{C}_0$ is a monotone, free amalgamation class of $\mathcal{L}_0$-structures 
    where each $R \in \mathcal{L}_0$ is interpreted irreflexively, except $=$ being interpreted as true equality;

    \item 
    $\mathcal{L}$ consists of $\mathcal{L}_0$ together with a new binary symbol $<$;

    \item 
    $\mathcal{C}$ consists of $\mathcal{L}$-structures obtained from $\mathcal{C}_0$ by expanding with all possible linear orderings;

    \item 
    $\A_0$ and $\A$ are the respective Fraïssé limits of $\mathcal{C}_0$ and $\mathcal{C}$,
    where without loss of generality we assume $\A_0$ and $\A$ share the same domain so that $\Aut(\A_0) \supseteq \Aut(\A)$.
\end{itemize}

\begin{example}\label{ex:N-Q}
    Take $\mathcal{L}_0$ to be the empty language and $\mathcal{C}_0$ to be all finite sets.
    Then $\A_0$ is isomorphic to the pure set $\N$, whereas $\A$ is isomorphic to $\Q$ with the usual order.
\end{example}

\begin{example}\label{ex:Rado-orderedRado}
    Let $\mathcal{L}_0$ consist of a single binary symbol $\sim$ 
    and let $\mathcal{C}_0$ consist of all finite undirected graphs not embedding the complete graph $K_n$,
    where $3 \leq n$ ($\leq \infty$).
    Then $\A_0$ is the $K_n$-free Henson graph (or the Rado graph when $n = \infty$), and $\A$ is its generically ordered counterpart.
    (Allowing $n = 2$ makes these degenerate to $\N$ and $\Q$ above).
\end{example}

We note three technicalities and a triviality.

\begin{lemma}
\color{blue}
    \begin{enumerate}
        \item 
        forbidden structures are link structures

        \item 
        free amalgamation with order
    \end{enumerate}
\end{lemma}

\begin{proposition}
    The $S$-supported length of $\Lin {\A_0}^d$ is at most that of $\Lin \A^d$ for any finite $S \subseteq \A_0 = \A$.
\end{proposition}
\begin{proof}
    Any chain of subspaces in $\Lin {\A_0}^d = \Lin \A^d$ that are invariant under $\Aut(\A_0)_{(S)}$ must also be invariant under the subgroup $\Aut(\A)_{(S)}$.
\end{proof}


\subsection{Cogs in an orbit}
One problem with $\A^d$ is that it can have many orbits.
\begin{definition}\label{def:strict-orbit}
    An orbit $\mathcal{O} = \Aut(\A)_{(S)} \cdot o_\bullet \subseteq \A^d$ is \emph{strict} 
    if the atoms in $o_\bullet$ satisfy $o_1 < \cdots < o_d$ and are not from the finite set $S$.
\end{definition}

Any orbit $\Aut(\A)_{(S)} \cdot a_\bullet$ is in an $\Aut(\A)_{(S)}$-equivariant bijection with a strict one
--- it suffices to remove entries in $a_\bullet$ that repeat or come from $S$ and reorder the rest.

\begin{proposition}
    $\A$ is \textcolor{red}{supportively} $\FF$-oligomorphic
    if and only if $\A$ is oligomorphic and $\Lin_\FF \mathcal{O}$ has finite length for every strict orbit $\mathcal{O}$.
\end{proposition}
\begin{proof}
    Indeed, we have \(
        \len(\Lin_\FF \A^d) 
        = \len(\Lin_\FF(\biguplus_i \mathcal{O}_i)) 
        = \len(\bigoplus_i \Lin_\FF \mathcal{O}_i)
        = \sum_i \len(\Lin_\FF \mathcal{O}_i)
    \)
    given a finite support.
\end{proof}

Thus fix a strict orbit $\mathcal{O}$ with support $S$.

\begin{remark}
    As $\A$ has no algebraicity, the strict orbit $\mathcal{O}$ uniquely determines its support $S$: 
    we have $\A \setminus S = \{a_1 \mid a_\bullet \in \mathcal{O}\}$.
    Also, since $\A$ is homogeneous, a tuple $a_\bullet \in \A^d$ belongs to $\mathcal{O}$ if and only if
    \[
        R(x_\bullet) \iff R(x_\bullet[o_i / a_i \mid i])
    \]
    holds in $\A$ for every symbol $R$ in $\mathcal{L}$ of arity $r$ 
    and every $x_\bullet \in (\{a_1, \ldots, a_d\} \cup S)^r$.
\end{remark}

We now introduce the workhorse we use to understand $\Lin \mathcal{O} \subseteq \Lin \A^d$.
\begin{definition}\label{def:cogs}
    An \emph{$\mathcal{O}$-cog} $a_\bullet | b_\bullet$ consists of atoms $a_1 < b_1 < a_2 < b_2 < \cdots < a_d < b_d$ such that
    for any symbol $R$ in $\mathcal{L}$ of arity $r$,
    given any $x_\bullet \in (\{a_1, b_1, \ldots, a_d, b_d \} \cup S)^r$ we have
    \[
        R(x_\bullet) \iff R(x_\bullet[o_i / a_i, o_i / b_i \mid i ])
    \]
    in $\A$.
    In particular, for each $I \subseteq \{1, \ldots, d\}$ we see that $a_\bullet [b_i / a_i \mid i \in I]$ is in the strict orbit $\mathcal{O}$ from the remark above;
    we define the corresponding \emph{$\mathcal{O}$-cogwheel} to be the vector
    \[
        a_\bullet {\between} b_\bullet =
        \sum_{I \subseteq \{1, \ldots, d\}} (-1)^{|I|} a_\bullet[b_i / a_i \mid i \in I]
    \]
    in $\Lin \mathcal{O}$.
\end{definition}

It is actually easier to find cogs than to define them.
\begin{example}
    Let $\A = \Q$ as described in Example~\ref{ex:N-Q};
    there is a unique equivariant strict orbit $\mathcal{O}$ in $\A^2$.
    Consider the vector
    \[
        v = (0, 5) + (5, 6) - (5, 10) - (0, 10)
    \]
    in $\Lin \mathcal{O}$.
    We can find $5 < 5+\varepsilon < 6 < 6+\delta < 10$ in $\A$
    together with monotone bijections $\pi_1, \pi_2 \in \Aut(\A)$ such that
    \[
        \pi_1 : 
        \begin{cases}
            0 \mapsto 0, \\
            5 \mapsto 5 + \varepsilon, \\
            6 \mapsto 6, \\
            10 \mapsto 10;
        \end{cases}
        \pi_2 : 
        \begin{cases}
            0 \mapsto 0, \\
            5 \mapsto 5, \\
            6 \mapsto 6 + \delta, \\
            10 \mapsto 10
        \end{cases}        
    \]
    by interpolating linearly for example.
    Then
    \begin{align*}
        v_1 
        = v - \pi_1 \cdot v
        ={}& (0, 5) + (5, 6) - (5, 10) \\
        &-(0, 5+\varepsilon) - (5+\varepsilon, 6) + (5+\varepsilon, 10)
    \end{align*}
    duplicates the tuples with $5$ in it but kills the one without it.
    Similarly 
    \begin{align*}
        v_{1,2}
        = v_1 - \pi_2 \cdot v_1
        ={}& (5, 6) - (5, 6+\delta) \\
        &-(5+\varepsilon, 6) + (5+\varepsilon, 6+\delta)
    \end{align*}
    only leaves and duplicates the tuples with $6$ in it.
    Here $v_{1,2}$ is the cogwheel for the cog $(5, 6|5 + \varepsilon, 6 + \delta)$ in $\mathcal{O}$
    as well as the smaller strict orbit $\mathcal{O}' = \Aut(\A)_{(-2, 10)} \cdot (5, 6) \subseteq \mathcal{O}$.
\end{example}

In general, we may build cogs and cogwheels as follows.
\begin{lemma}\label{lem:finding-cogs}
    Take any $(a_1, \ldots, a_d)$ from a strict orbit $\mathcal{O}'$ with support $S'$.
    Then: 
    \begin{enumerate}
        \item 
        we can find $(b_1, \ldots, b_d)$ so that $a_\bullet | b_\bullet$ forms an $\mathcal{O}'$-cog;
        
        \item 
        for $i = 1, \ldots, d$ there is some $\pi_i \in \Aut(\A)_{(S')}$ that sends $a_i \mapsto b_i$ but fixes $a_{i'}$ for $i' \neq i$. 
    \end{enumerate}
\end{lemma}

\begin{proof}
    We find $b_1, b_2, \ldots$ inductively.
    Write
    \[
        X_i = \{a_1, \ldots, a_d, b_1, \ldots, b_{i-1}\} \cup S,
    \]
    and form the $\mathcal{L}_0$-amalgam 
\end{proof}

