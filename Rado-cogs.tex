\section{Finite length from ordered free amalgamation}
\newcommand{\cal}{\mathcal}

As we saw in Non-example~\ref{non-ex:weak-smooth-approximation}, the ordered atoms do not have oligomorphic approximation. 
Nontheless they do have the finite length property over any field \cite[Lemma~4.5]{BFKM24}, not just those of characteristic zero.
We give an adaptation of the proof there in this section to richer ordered structures.
In particular, we will deduce the finite length property for all homogeneous undirected graphs and uncountably many non-isomorphic homogeneous directed graphs.
These results are new.

Throughout this section, we will work with the following setting.

\paragraph*{Graph vocabulary}
Consider a finite relational language $\cal L_0$ consisting of unary and binary symbols.
(This allows us to talk about graphs with coloured vertices and edges, but in the running example we will consider a single edge relation.)

\paragraph*{Free amalgamation class}
Let $\cal C_0$ be a \emph{free} amalgamation class of finite $\cal L_0$-structures. 
This is a standard notion in model theory --- see \cite[Section~2.1]{Macpherson11}.
Informally, it means that when we perform amalgamation, we do not need to glue together more elements than necessary or introduce new relations.

There is a more useful characterisation \cite[Lemma~4.5]{SS20} of the amalgamation class $\cal C_0$ being free.
Since $\cal C_0$ is closed under substructures and isomorphisms, 
we know that $\cal C_0$ consists of all the finite $\cal L_0$-structures which do not embed any structure from $\cal F$, 
where $\cal F$ consists of all the minimal finite $\cal L_0$-structures that do not belong to $\cal C_0$;
we write \[
    \cal C_0 = \Forb(\cal F).
\]
That $\cal C_0$ is close under free amalgamation means that in each $F \in \cal F$, 
any two elements $x, y$ are either equal or satisfy at least one of $R(x, y)$ and $R(y, x)$ for some binary relation $R \in \cal L_0$
--- from here on, we will just say $x, y$ are \emph{related}.
Conversely, given a family $\cal F$ of finite $\cal L_0$-structures in each of which any two elements are related,
the class $\Forb(\cal F)$ of finite $\cal L_0$-structures form a \emph{free} amalgamation class.

Here are a few free amalgamation classes.

\begin{example}\label{ex:free-amalg-equality}
    Let $\cal L_0$ be empty. 
    Then $\cal C_0 = \Forb(\{\})$ is a free amalgamation class consisting of all finite pure sets.
\end{example}

\begin{example}\label{ex:free-amalg-graphs}
    Let $\cal L_0$ consist of a single binary symbol $-$. 
    Consider $\cal L_0$-structures ${\circlearrowright} = \{x\}$ and ${\rightarrow} = \{y, z\}$, where the relation $-$ is interpreted as $\{(x, x)\}$ and $\{(y, z)\}$.
    Then $\cal C_0 = \Forb(\{{\circlearrowright}, {\rightarrow}\})$ is a free amalgamation class consisting of all finite undirected graphs.

    In addition, let $K_n$ be the $\cal L_0$-structure representing a complete graph on $n$ vertices.
    Then 
    \begin{align*}
        \Forb(\{{\circlearrowright}, {\rightarrow}, K_3\}) 
        \subseteq \Forb(\{{\circlearrowright}, {\rightarrow}, K_4\})
        \subseteq \Forb(\{{\circlearrowright}, {\rightarrow}, K_5\})
        \subseteq \cdots \\
        \subseteq \Forb(\{{\circlearrowright}, {\rightarrow}\})
    \end{align*}    
    are all free amalagamation classes.
\end{example}



\paragraph*{Generically ordered expansion}
Now, let $\cal L$ consist of $\cal L_0$ together with a new binary symbol $<$.
Consider the class $\cal C$ of $\cal L$-structures obtained from $\cal C_0$, 
by interpreting $<$ in any $\cal L_0$-structure there as any total order.

Observe that $\cal C$ is an amalgamation class.
Indeed, let $X, Y_1, Y_2$ be $\cal L$-structures in $\cal C$ with $X \subseteq Y_1 \cap Y_2$.
Then we can amalgamate $Y_1, Y_2$ over $X$ as $\cal L_0$-structures and as $\{<\}$-structures, 
both using the disjoint union of $Y_1, Y_2$ over $X$ as the underlying set. 
Superposing these relations will give an $\cal L$-structure in $\cal C$, which is the desired amalgamation.
(Because of the total order, this amalgamation in $\cal C$ is not free unless it is trivial.)
Denote the Fraïssé limit of $\cal C$ by $\A$.

\begin{theorem}\label{thm:ordered-free-amalg-has-finite-length}
    The ordered structure $\A$, even with finitely many constants fixed, has the finite length property over any field.
\end{theorem}

The Fraïssé limit $\A_0$ of $\cal C$ is a reduct of $\A$.
To see this, notice that $\A$, when viewed as an $\cal L_0$-structure, shares the same age as $\A_0$ --- namely, $\cal C_0$.
Moreover, it follows from a back-and-forth argument \cite[Lemma~7.1.4]{Hodges93} that the $\cal L_0$-structure $\A$ is also homogeneous and, therefore, that it is isomorphic to $\A_0$.
So we may assume that $\A$ and $\A_0$ have the same universe; we then have $\Aut(\A) \subseteq \Aut(\A_0)$.

\begin{corollary}
    The reduct $\A_0$ of $\A$, even with finitely many constants fixed, has the finite length property over any field.
\end{corollary}

Conversely, we call $\A$ the \emph{generically ordered expansion} of $\A_0$ or simply the ``ordered $\A_0$''.

\begin{example}\label{ex:geneircally-ordered-equality}
    Continuing from Example~\ref{ex:free-amalg-equality},
    the ordered atoms (Example~\ref{ex:order-atoms}) is the generically ordered expansion of the equality atoms (Example~\ref{ex:equality-atoms}).
\end{example}

\begin{example}\label{ex:geneircally-ordered-graphs}
    Continuing from Example~\ref{ex:free-amalg-graphs}, we obtain generically ordered expansions of the Rado graph and of the $K_n$-free Henson graphs.
    The ordered Rado graph was studied in \cite{BPP15}.
    In the examples below, we will work with the ordered triangle-free Henson graph.
    (We do not know an easy, explicit description of the structure, even without the order.)
\end{example}

\textcolor{red}{Introduce Lachlan's classification of homogeneous graphs. 
The Rado graph and the Henson $K_n$-free graphs are directly treated; 
the disjoint union of cliques interpret in the ordered atoms.}

\textcolor{red}{Introduce tournaments and the Henson digraphs. 
Mention that the universal homogeneous tournament is a reduct of the ordered Rado graph \cite[Section~2.2]{BPP15}.}

\section{Proof of Theorem~\ref{thm:ordered-free-amalg-has-finite-length}}


\subsection{Orbits and projections}

To start with, let us view $\A^d$ as $\A^{\{1, \dots, d\}}$. More generally, it will be convenient to consider $\A^I$ for a finite totally ordered indexing set $I \subseteq \Q$.
Fix a finite support $S \subseteq \A$.
%As $\A$ is oligomorphic, the tuples in $\A^I$ split into finitely many $\Aut(\A)_{(S)}$-equivariant orbits.
%Let $\mathcal{O} = \Aut(\A)_{(S)} \cdot o$ be one such orbit.
We shall say than an $a\in\A^I$ is {\em ($S$-)ordered} if $a_i \not\in S$ for all $i$, and if $a_i < a_j$ whenever $i < j$. Then the orbit
$\mathcal{O} = \Aut(\A)_{(S)} \cdot a$ only contains $S$-ordered elements, and we will call the orbit $S$-ordered as well.
If $a$ is not $S$-ordered, by removing the entries that repeat or come from $S$ and reordering the rest, we can always find an $\Aut(\A)_{(S)}$-equivariant bijection from $\cal O$ to an $S$-ordered orbit.

To study length of orbit-finitely spanned spaces, we may focus on a single ordered orbit at a time:
\begin{proposition}\label{prop:reduction-to-one-orbit}
    The following are equivalent:
    \begin{enumerate}
        \item For any $d$ and any finite $S \subseteq \A$, chains of $\Aut(\A)_{(S)}$-equivariant subspaces in $\Lin_\FF \A^d$ are bounded in length;
        \item $\Lin_\FF \mathcal{O}$ has finite length for any ordered orbit $\mathcal{O}$.
    \end{enumerate}
\end{proposition}
\begin{proof}
    Since $\A$ is oligomorphic, given any $d$ and $S$, 
    we know that $\A^d$ is in an $\Aut(\A)_{(S)}$-equivariant bijection with a finite disjoint union of $S$-ordered orbits $\cal O_i$.
    So the $S$-length of $\A^d$ equals \[
        \len(\Lin_\FF(\biguplus_i \mathcal{O}_i)) = \len(\bigoplus_i \Lin_\FF \mathcal{O}_i) = \sum_i \len(\Lin_\FF \mathcal{O}_i).
        \qedhere
    \]
\end{proof}

So fix an ordered orbit $\mathcal{O} = \Aut(\A)_{(S)} \cdot o \subseteq \A^I$.
From here we take an inductive approach.
By $o |^J$ we mean the restriction of $o : I \to \A$ to $J \subseteq I$;
we will often write $o |^{-i}$ instead of $o |^{ I \setminus \{i\} }$.
The image $\mathcal{O}|^J$ of $\mathcal{O}$ under this projection agrees with $\Aut(\A)_{(S)} \cdot o |^J$ and is still ordered.

%To anticipate more general statements later, Let $\EE$ be a finite-dimensional $\FF$-vector space --- for instance, $\FF$ itself.
The function $(-)|^J$ lifts to a linear $\Aut(\A)_{(S)}$-equivariant map
\begin{align*}
    (-)|^J : \Lin_\FF \mathcal{O} &\to \Lin_\FF \mathcal{O}|^J.
\end{align*}
Many cancellations can occur under $(-)|^J$; 
the \emph{projection kernel} is the $\Aut(\A)_{(S)}$-equivariant subspace
\[
    \Ker_\FF \mathcal{O} = \bigcap_{i \in I} \ker{ (-)|^{-i} }
\]
of $\Lin_\FF \mathcal{O}$.

\begin{proposition}\label{prop:reduction-to-kernel}
    The following are equivalent:
    \begin{enumerate}
        \item $\Lin_\FF \mathcal{O}$ has finite length for every ordered orbit $\mathcal{O}$;
        \item $\Ker_\FF \mathcal{O}$ has finite length for every ordered orbit $\mathcal{O}$.
    \end{enumerate}
\end{proposition}
\begin{proof}
    That 1) implies 2) is clear as $\Ker_\FF \mathcal{O} \subseteq \Lin_\FF \mathcal{O}$.
    
    To prove the other implication, assume 2) and let $\mathcal{O} \subseteq \A^I$.
    We proceed by induction on $|I|$.
    If $I = \emptyset$, then $\mathcal{O}$ must be the entire singleton $\A^\emptyset = \{ () \}$; 
    as $\Lin_\FF \mathcal{O}$ has no nontrivial subspaces (let alone finitely supported ones), it has length $1$.
    Now if $|I| \geq 1$, assemble all $|I|$ projection maps into a single map
    \begin{align*}
        \Lin_\FF \mathcal{O} &\to \bigoplus_{i \in I} \mathcal{O}|^{-i} \\
        v &\mapsto ( v|^{-i} )_{i \in I}
    \end{align*}
    whose kernel is precisely $\Ker_\FF \mathcal{O}$.
    We have
    \[
        \len( \Lin_\FF \mathcal{O} ) - \len( \Ker_\FF \mathcal{O} )
        \leq \sum_{i \in I} \len( \Lin_\FF \mathcal{O}|^{-i} )
    \]
    which shows that $\len( \Lin_\FF \mathcal{O} )$ is finite from the assumptions.
\end{proof}

We call a vector from the projection kernel \emph{balanced}.
As we will see in the next subsection, cogs are a prominent example.

\subsection{Cogs}

From now on we will use a lightweight notation for combining tuples of atoms: for disjoint indexing sets $I$ and $J$, if $a\in\A^I$ and $b\in\A^{J}$ are both ordered, then $ab\in \A^{I\cup J}$ will denote their obvious combination. We will only use this notation if the combined tuple $ab$ is ordered. For an obvious example, for any $S$-ordered $a\in\A^I$ and $J\subseteq I$, we have $a|^{I\setminus J}a|^J=a$.

\begin{definition}\label{def:duo}
    Let $\mathcal{O} \subseteq \A^I$ be an $S$-ordered orbit. 
    An \emph{$\mathcal{O}$-duo} $a \parallel b$ consists of tuples $a, b \in \cal O$ such that:
 %   $2|I|$ atoms in $\A\setminus S$ such that: with the following $\mathcal{L}$-structure on $\{a_i, b_i \mid i \in I\} \cup S$:
    \begin{enumerate}
        \item 
        $a_i < b_i$ for all $i\in I$;
        
        \item 
        $b_i < a_j$ for all $i<j\in I$;
        
       % \item
       % $a_{i_1} < b_{i_1} < a_{i_2} < b_{i_2} < \dots < a_{i_d} < b_{i_d}$ where $I$ consists of the indices $i_1 < i_2 < \dots < i_d$;
        
        %\item 
        %$a_i , b_i < s$ if and only if $o_i < s$;       

        \item 
        for any binary $R$ in $\mathcal{L}_0$ (except for $=$) and $i,j\in I$:
        \[R(a_i,b_j) \iff R(b_i,a_j) \iff R(a_i,a_j).\] 
     \end{enumerate}
\end{definition}
\begin{remark}\label{rem:duo}
Conditions (1) and (2) specify a total order on the $2|I|$ atoms in a duo.
Moreover, thanks to irreflexivity \textcolor{red}{TODO: explain somewhere that we assume this!}, each $a_i$ is unrelated to its counterpart $b_i$.
Further, given any $J \subseteq I$, the combined tuple $a|^{I\setminus J} b|^J$ satisfies the same relations as $a$ (and $b$), so it lies in $\mathcal{O}$. In particular, taking $J=\{i\}$, there is an automorphism $\pi_i$ that sends $a_i$ to $b_i$ and fixes all the other elements of $a$, $b$ and $S$.
\end{remark}

\begin{definition}
    Given a %$\lambda \in \EE$ and an 
    $\mathcal{O}$-duo $a \parallel b$, the corresponding \emph{$\mathcal{O}$-cog} %with coefficient $\lambda$ 
    is the vector
    \[
        %\lambda \cdot 
        a \between b =
        \sum_{J \subseteq I} (-1)^{|J|} %\lambda \cdot  
        (a|^{I \setminus J} b|^{J})
    \]
    in $\Lin_\FF \mathcal{O}$.
    The linear span of all $\mathcal{O}$-cogs 
    %with coefficients from $\EE$ 
    is denoted by $\Cog_\FF \mathcal{O}$.
\end{definition}
%As remarked above, given any two $\mathcal{O}$-duos there is some $\pi \in \Aut(\A)_{(S)}$ such that $\pi \cdot (a \parallel b) = a' \parallel b'$ and thus $\pi \cdot (a \between b) = a' \between b'$.
Note that, for a fixed $S$-ordered orbit $\cal O$, all $\mathcal{O}$-duos (hence all $\cal O$-cogs) are in the same $\Aut(\A)_{(S)}$-equivariant orbit.
As a result, $\Cog_\FF \mathcal{O}$ is an $\Aut(\A)_{(S)}$-equivariant subspace of $\Lin_\FF \mathcal{O}$ and it is generated by %cogs based on a single duo.
a single cog.

\begin{proposition}[Cog calculus]
    \textcolor{red}{$$a x \between b y = (a \between b) x - (a \between b) y$$}
\end{proposition}

\begin{proposition}\label{prop:cogs-are-balanced}
    $\Cog_\FF \mathcal{O}\subseteq \Ker_\FF \mathcal{O}$.
\end{proposition}
\begin{proof}
   % \textcolor{red}{TODO: rewrite using the cog calculus.}
    Let $\mathcal{O} \subseteq \A^I$, let $a\parallel b$ be an $\mathcal{O}$-duo, and take any $i \in I$.
    The subsets of $I$ come in pairs of $J$ and $J \cup \{i\}$, where $J\subseteq I \setminus \{i\}$.
    The two tuples $a|^{I\setminus J} b|^{J}$ and $a|^{I \setminus (J \cup \{i\})}b|^{J \cup \{i\}}$ are present in 
    %$\lambda\cdot 
    $a\between b$ with the opposite coefficients, and they differ only on the $i$-th entry.
    Therefore they cancel out under $(-)|^{-i}$, hence %$( \lambda \cdot 
    $(a \between b )|^{-i} = 0$.
\end{proof}

In fact, cogs arise anywhere.
We combine the free amalgamation in $\cal C_0$ and the generic order of $\A$ into the following lemma.
\begin{lemma}\label{lem:free-fresh}
    Let $X, Y, \{z\} \subseteq \A$ be disjoint and finite.
    Then there is ab automorphism $\tau \in \Aut(\A)$ such that
    \begin{enumerate}
        \item $\tau$ fixes every $x \in X$;
        \item $\tau(z)$ is unrelated to all $y \in Y$ and to $z$;
        \item $\tau(z) > z$.
    \end{enumerate}
\end{lemma}
\begin{proof}
    Form the free amalgam of structures in ${\cal C}_0$:
% https://q.uiver.app/#q=WzAsNCxbMCwxLCJYIl0sWzEsMiwiWCBcXGN1cCBcXHt6XFx9Il0sWzEsMCwiWCBcXGN1cCBZIFxcY3VwIFxce3pcXH0iXSxbMiwxLCJYIFxcY3VwIFkgXFxjdXAgXFx7eiwgeidcXH0iXSxbMCwxLCIiLDAseyJzdHlsZSI6eyJ0YWlsIjp7Im5hbWUiOiJob29rIiwic2lkZSI6InRvcCJ9fX1dLFswLDIsIiIsMCx7InN0eWxlIjp7InRhaWwiOnsibmFtZSI6Imhvb2siLCJzaWRlIjoidG9wIn19fV0sWzEsMywieCBcXGluIFggXFxtYXBzdG8geCwgeiBcXG1hcHN0byB6JyIsMSx7InN0eWxlIjp7InRhaWwiOnsibmFtZSI6Imhvb2siLCJzaWRlIjoidG9wIn0sImJvZHkiOnsibmFtZSI6ImRhc2hlZCJ9fX1dLFsyLDMsIlxcc3Vic2V0ZXEiLDEseyJzdHlsZSI6eyJ0YWlsIjp7Im5hbWUiOiJob29rIiwic2lkZSI6InRvcCJ9LCJib2R5Ijp7Im5hbWUiOiJkYXNoZWQifX19XV0=
\[\begin{tikzcd}[cramped]
	& {X \cup Y \cup \{z\}} \\
	X && {X \cup Y \cup \{z, z'\}} \\
	& {X \cup \{z\}}
	\arrow["\subseteq"{description}, dashed, hook, from=1-2, to=2-3]
	\arrow[hook, from=2-1, to=1-2]
	\arrow[hook, from=2-1, to=3-2]
	\arrow["{x \in X \mapsto x, z \mapsto z'}"{description}, dashed, hook, from=3-2, to=2-3]
\end{tikzcd}\]
    so that no element of $Y \cup \{z\}$ is related to $z'$.
    To make $X \cup Y \cup \{z, z'\}$ an $\mathcal{L}$-structure, 
    inherit the order on $X \cup Y \cup \{z\}$ from $\A$,
    and declare that $z < z'$, as well as $z' < a$ whenever $z<a$ for $a\in X \cup Y$. This makes the above a diagram of embeddings in the presence of the order.
    By homogeneity, $X \cup Y \cup \{z, z'\}$ embeds into $\A$ via some $f$ which is the identity on $X \cup Y \cup \{z\}$;
    again by homogeneity, we may extend the embedding \[ 
        x \in X \mapsto x, \quad z \mapsto f(z')
    \] to some automorphism $\tau$ which makes 1), 2), and 3) true.
\end{proof}

\begin{lemma}\label{lem:cog-fresh-single}
    Suppose $a\parallel b$ is an $\mathcal{O}$-duo, where $\mathcal{O} \subseteq \A^I$ is $S$-ordered.
    Given $z \in S$, 
    \begin{itemize}
        \item write $S' = S \setminus \{z\}$;
        \item let $j \not\in I$ be such that $\mathcal{O}' = \Aut(\A)_{(S')} \cdot (a ; z) \subseteq \A^{I \cup \{j\}}$ is ordered,
        \item let $X \subseteq \A$ be a finite set containing $\{a_i, b_i \mid i \in I\} \cup S'$ but not $z$;
        \item let $Y \subseteq \A$ be any finite set disjoint from $X \cup \{z\}$;
    \end{itemize}
    then the $\tau \in \Aut(\A)_{(X)}$ afforded by Lemma~\ref{lem:free-fresh} gives us an $\mathcal{O}'$-duo $(a ; z) \parallel (b; \tau(z))$.
\end{lemma}
\begin{proof}
    First, notice that $b ; \tau(z)\in {\cal O}'$ and that we have the required order relations with $z$ and $\tau(z)$. The remaining condition of Def.~\ref{def:duo}, for any $R$ in $\mathcal{L}_0$, splits into the following cases (and their symmetric versions):
\begin{itemize}
\item[-] $R(a_i,b_j)\iff R(a_i,a_j)$ since $a \parallel b$ is an $\mathcal{O}$-duo;
\item[-] $R(a_i,\tau(z))\iff R(a_i,z)$ since $\tau$ is an automorphism that fixes all $a_i$;
\item[-] $R(a_i,z)\iff R(b_i,z)$ since $a, b \in \mathcal{O}$ and $z\in S$;
\item[-] $R(z,\tau(z))$ and $R(z,z)$ are both false: $\tau(z)$ is unrelated to $z$ by Lemma~\ref{lem:free-fresh}, and $R$ is irreflexive.
\end{itemize}
\end{proof}

Starting from an empty duo, we may apply the previous lemma inductively.
\begin{proposition}\label{prop:cog-fresh-full}
    Let $\mathcal{O} \subseteq \A^I$ be an $S$-ordered orbit.
    Then any $a \in \mathcal{O}$ can be extended to an ${\cal O}$-duo $a\parallel b$.
\end{proposition}
\begin{proof}
    Enumerate the indices of $I$ as $i_1, \dots, i_d$.
    Suppose that we have found $b_{i_1}, \dots, b_{i_k}$ such that 
    \[
        a|^{\{i_1, \dots, i_k\}} \parallel (i_1 \mapsto b_{i_1}, \dots, i_k \mapsto b_{i_k})
    \] 
    is a duo for $\mathcal{O}_k = \Aut(\A)_{(S \cup \{a_{i_{k+1}}, \dots, a_{i_d}\})} \cdot a|^{\{i_1, \dots, i_k\}}$ 
    --- note that $() \parallel ()$ is certainly a duo for $\mathcal{O}_0$ at the start.
    If $k < d$, with $z = a_{i_{k+1}}$, $X = \{a_{i_1}, b_{i_1}, \dots, a_{i_k}, b_{i_k}\} \cup S \cup \{a_{i_{k+2}}, \dots, a_{i_d}\}$, and $Y = \emptyset$, 
    a straightforward application of Lemma~\ref{lem:cog-fresh-single} yields an atom $b_{i_{k+1}}$ that makes
    \[
        a|^{\{i_1, \dots, i_k, i_{k+1}\}} \parallel (i_1 \mapsto b_{i_1}, \dots, i_k \mapsto b_{i_k}, i_{k+1} \mapsto b_{i_{k+1}})
    \] a duo for $\mathcal{O}_{k+1}$.
    For $k=d$ we thus obtain the desired duo for $\mathcal{O}_d = \mathcal{O}$.
%    The automorphisms $\pi_{i_k}$ now come directly from homogeneity and the definition of an $\mathcal{O}$-duo: 
%    the map
%    \begin{align*}
%        a_{i_1} \mapsto a_{i_1}, \dots,{} &a_{i_k} \mapsto b_{i_k}, \dots, a_{i_d} \mapsto a_{i_d} \\
%        b_{i_1} \mapsto b_{i_1}, \dots,{} &\phantom{a_{i_k} \mapsto b_{i_k}}, \dots, b_{i_d} \mapsto b_{i_d}, s \in S \mapsto s
%    \end{align*}
%    is an embedding.
\end{proof}

The result below substantiates the slogan that cogs are found everywhere. 
\begin{theorem} \label{thm:cogs-arise-everywhere}
    Any $\Aut(\A)_{(S)}$-equivariant subspace $V$ of $\Lin_\EE \mathcal{O}$ contains $\Cog_{\EE(V)} \mathcal{O}$,
    where $\mathcal{O} \subseteq \A^I$ is $S$-ordered and $\EE(V)$ is the subspace spanned by $\{ v(a) \mid v \in V, a \in \mathcal{O} \}$ of $\EE$.
\end{theorem}
\begin{proof}
    Pick any $v \in V$ and $a \in \mathcal{O}$;
    it is enough to show that $V$ contains $v(a) \cdot a \between b$ for some $\mathcal{O}$-duo $a \parallel b$.
    Actually, write 
    \[
        S' = S \cup \{c_i \mid v(c) \neq 0, i \in I\} \setminus \{a_i \mid i \in I\} \supseteq S
    \]
    and put $\mathcal{O}' = \Aut(\A)_{(S')} \cdot a \subseteq \mathcal{O}$ --- then $\mathcal{O}'$ is $S'$-ordered.
    By Proposition~\ref{prop:cog-fresh-full}, we can find $b \in \mathcal{O}'$ such that $a \parallel b$ is an $\mathcal{O}'$-duo and \emph{a fortiori} an $\mathcal{O}$-duo.
    Take the automorphisms $\pi_{i_1}, \dots, \pi_{i_d}$ from Remark~\ref{rem:duo}, where $i_1, \dots, i_d$ enumerate $I$.
    Now define $v^{(0)} = v$ and \[
        v^{(k)} = v^{(k-1)} - \pi_{i_k} \cdot v^{(k-1)}.
    \]
    We can check inductively that for $k = 0, 1, \dots, d$, 
    with $\mathcal{O}^{(k)} = \{ c \mid v(c_\bullet) \neq 0, \{c_{i_1}, \dots, c_{i_k}, \dots, c_{i_d}\} \supseteq \{a_{i_1}, \dots, a_{i_k}\} \}$ we have
    \[
        v^{(k)} = \sum_{c \in \mathcal{O}^{(k)}} \sum_{J \subseteq \{i_1, \dots, i_k\}} (-1)^{|J|} v(c) \prod_{j \in J} \pi_j \cdot c.
    \]
    But $\{c_{i_1}, \dots, c_{i_d}\} \supseteq \{a_{i_1}, \dots, a_{i_d}\}$ means that $c = a$, so at the end $v^{(d)}$ is the desired $\mathcal{O}$-cog.
\end{proof}

\begin{corollary}
    $\Cog_\FF \mathcal{O}$ has length $1$.
\end{corollary}
\begin{proof}
    Let $V \subseteq \Cog_\FF \mathcal{O}$ be a non-zero $\Aut(\A)_{(S)}$-equivariant subspace.
    Then $\{0\} \subsetneq \EE(V) \subseteq \EE = \FF$ so $\EE$ must be the entire field $\FF$, and by above $V$ must be $\Cog_\FF \mathcal{O}$ itself.
\end{proof}

In light of Propositions~\ref{prop:reduction-to-one-orbit},~\ref{prop:reduction-to-kernel} and~\ref{prop:cogs-are-balanced}, for the finite length property for an oligomorphic structure with free amalgamation over any field and support it is enough to prove that $\Ker_\FF \mathcal{O} \subseteq \Cog_\FF \mathcal{O}$. In words, we need to show that every balanced vector in $\Lin_\FF{\cal O}$ is a linear combination of $\cal O$-cogs. Before we show a proof of this, let us illustrate its key ideas on an example.

\subsection{Spanning by cogs: an example}

Let $\A_0$ be the universal triangle-free (undirected) graph, and $\A$ its totally ordered version. Consider nine atoms $\{a,\ldots,i\}$, ordered by~$<$ alphabetically, with the edge relation as shown here:
\[
\rotatebox{2}{
\xymatrix@R=6pt@C=15pt{
\rotatebox{-2}{$h$} & & & & & & \rotatebox{-2}{$i$} \\
\\
& & & \rotatebox{-2}{$g$}\ar@{-}[uulll]\ar@{-}[uurrr] \\
& \rotatebox{-2}{$e$} & & & & \rotatebox{-2}{$f$} \\
& & \rotatebox{-2}{$c$}\ar@{-}[ul]\ar@{-}[uur] & & \rotatebox{-2}{$d$}\ar@{-}[uul]\ar@{-}[ur] \\
\rotatebox{-2}{$a$}\ar@{-}[rrrrrr]\ar@{-}[uuuuu]\ar@{-}[uur] & & & & & & \rotatebox{-2}{$b$}\ar@{-}[uul]\ar@{-}[uuuuu]
}}\]
This graph is drawn so that the total order of the atoms corresponds to the vertical order.

Putting $S=\emptyset$ and $d=2$, let ${\cal O}$ be the ordered orbit of pairs of atoms which are connected by an edge. Consider the following vector:
\[
v = ah - ae + ce - cg + dg - df + bf - bi + gi - gh \in \Lin_\FF{\cal O}.
\]
This can be graphically presented as the following graph:
\[\rotatebox{2}{
\xymatrix@R=6pt@C=15pt{
\rotatebox{-2}{$h$} & & & & & & \rotatebox{-2}{$i$} \\
\\
& & & \rotatebox{-2}{$g$}\ar@[red][uulll]\ar@[blue][uurrr] \\
& \rotatebox{-2}{$e$} & & & & \rotatebox{-2}{$f$} \\
& & \rotatebox{-2}{$c$}\ar@[blue][ul]\ar@[red][uur] & & \rotatebox{-2}{$d$}\ar@[blue][uul]\ar@[red][ur] \\
\rotatebox{-2}{$a$}\ar@{-}[rrrrrr]\ar@[blue][uuuuu]\ar@[red][uur] & & & & & & \rotatebox{-2}{$b$}\ar@[blue][uul]\ar@[red][uuuuu]
}}
\]
where edges with coefficient $+1$ are marked as blue, and with $-1$ as red. The arrows on the chosen edges remind us that the pairs in $\cal O$ are ordered, but this is mere decoration: the definition of $\cal O$ means that all arrows must point upwards.

Note that $v$ is balanced. Graphically, this means that every atom has as many outgoing red edges as outgoing blue edges, and as many incoming red edges as incoming blue edges.

It is easy to draw $\cal O$-cogs in this way. Assuming some additional atom $z>h$ which is connected by edges to $a$ and $g$, the $\cal O$-cog $ah\between gz$ can be drawn as:
\[\rotatebox{5}{
\xymatrix{
\rotatebox{-5}{$h$} & \rotatebox{-5}{$z$} \\
\rotatebox{-5}{$a$}\ar@[blue][u]\ar@[red][ur] & \rotatebox{-5}{$g$}\ar@[red][ul]\ar@[blue][u]  
}}
\]
We would like to present $v$ as a sum of such ${\cal O}$-cogs. Some additional atoms must be used for that, as no four atoms among the original nine form an $\cal O$-duo. It would be very convenient to have an atom $z$, larger than every atom in $v$, and connected by edges to every atom which is a source of a directed edge in $v$ (equivalently: which occurs as the first component of a pair in $v$). However, such a $z$ does not exist in the triangle-free graph $\A$. There are two problems:
\begin{itemize}
\item The atom $g$ occurs both as the first and as the second component in pairs present in $v_0$. In particular, $z$ as prescribed would create a triangle $dgz$ in $\A$, which is forbidden.
\item Atoms $a$ and $b$ both occur as first components in $v$, and they are connected by an edge in $\A$. As a result, an atom $z$ as prescribed would create a triangle $abz$ in $\A$.
\end{itemize}
We get rid of such {\em obstructions} by considering auxiliary atoms $g'>g$ and $b'>b$, with just enough edges to make $gh\parallel g'i$ and $bf\parallel b'i$ valid ${\cal O}$-duos. Specifically, we postulate edges $g'$---$h$, $g'$---$i$, $b'$---$f$ and $b'$---$i$ and no more. Such atoms exist by the homogeneity of $\A$. We then define:
\[
v' = v - (gh \between g'i) - (bf \between b'i)
\]
which can be drawn as:
\[
\rotatebox{2}{
\xymatrix@R=6pt@C=15pt{
\rotatebox{-2}{$h$} & & & & & & \rotatebox{-2}{$i$} \\
\\
& & & \rotatebox{-2}{$g$}\ar@{-}[uulll]\ar@{-}[uurrr] & \rotatebox{-2}{$g'$}\ar@[red][uullll]\ar@[blue][uurr] \\
& \rotatebox{-2}{$e$} & & & & \rotatebox{-2}{$f$} \\
& & \rotatebox{-2}{$c$}\ar@[blue][ul]\ar@[red][uur] & & \rotatebox{-2}{$d$}\ar@[blue][uul]\ar@[red][ur] \\
\rotatebox{-2}{$a$}\ar@{-}[rrrrrr]\ar@[blue][uuuuu]\ar@[red][uur] & & & & & & \rotatebox{-2}{$b$}\ar@{-}[uul]\ar@{-}[uuuuu] & \rotatebox{-2}{$b'$}\ar@[blue][uull]\ar@[red][uuuuul]
}}
\]
Now an atom $z$ as postulated above does not create any triangles:
\[
\rotatebox{2}{
\xymatrix@R=6pt@C=15pt{
& & & \rotatebox{-2}{$z$}\ar@{-}@/_4ex/[llldddddd]\ar@{-}[lddddd]\ar@{-}[rddddd]\ar@{-}[rddd]\ar@{-}[rrrrdddddd] \\
\rotatebox{-2}{$h$} & & & & & & \rotatebox{-2}{$i$} \\
\\
& & & \rotatebox{-2}{$g$}\ar@{-}[uulll]\ar@{-}[uurrr] & \rotatebox{-2}{$g'$}\ar@[red][uullll]\ar@[blue][uurr] \\
& \rotatebox{-2}{$e$} & & & & \rotatebox{-2}{$f$} \\
& & \rotatebox{-2}{$c$}\ar@[blue][ul]\ar@[red][uur] & & \rotatebox{-2}{$d$}\ar@[blue][uul]\ar@[red][ur] \\
\rotatebox{-2}{$a$}\ar@{-}[rrrrrr]\ar@[blue][uuuuu]\ar@[red][uur] & & & & & & \rotatebox{-2}{$b$}\ar@{-}[uul]\ar@{-}[uuuuu] & \rotatebox{-2}{$b'$}\ar@[blue][uull]\ar@[red][uuuuul]
}}
\]
and it is easy to calculate:
\[
v' = (ah \between g'z)  - (ae \between cz) - (cg \between dz) + (b'f \between dz) -(b'i \between g'z)
\]
which presents $v$ as a linear combination of $\cal O$-cogs.

\subsection{Subvectors}
This is a good time to recall a view we have tacitly taken:
with $\mathcal{O}$ as a standard basis, a vector $v \in \Lin_\EE \mathcal{O}$ is just a finite set of pairs in $\EE \times \mathcal{O}$.
A \emph{subvector} of $v$ is a subset of these pairs.

Now suppose as usual that $\mathcal{O} \subseteq \A^I$ is $S$-ordered.
Given $i \in I$ and $a \in \mathcal{O}$, 
we write \[
    \mathcal{O}^{i:a_i} = \{b \in \mathcal{O} \mid b_i = a_i\};
\]
this is an $\Aut(\A)_{S a_i}$-orbit, and its projection $\mathcal{O}^{i:a_i} |^{-i} = \Aut(\A)_{S a_i} \cdot a|^{-i}$ is ordered.
For a vector $v \in \Lin_\EE \mathcal{O}$, by
\[
    v^{i:a_i} \in \Lin_\EE \mathcal{O}^{i:a_i}
\]
we mean the subvector consisting of all pairs in $\EE \times \mathcal{O}^{i:a_i}$.

\begin{lemma}\label{lem:balanced-projected-subvector}
    Let $v \in \Lin_\EE \mathcal{O}$ be balanced. 
    Then any projected subvector $v^{i:a_i}|^{-i} \in \Lin_\EE \mathcal{O}^{i:a_i}|^{-i}$ is also balanced.
\end{lemma}
\begin{proof}
    Let $j \in I \setminus \{i\}$. 
    By assumption we have \[
        0 = v|^{-j} = \sum_a v^{i:a_i}|^{-j}
    \] in $\Lin_\EE \A^{I \setminus \{j\}}$,
    so by looking at $i$th entries we see that each $v^{i:a_i}|^{-j}$ must be the zero vector.
    Hence so is $v^{i:a_i}|^{-j}|^{-i} = v^{i:a_i}|^{-i}|^{-j}$,
    which shows that $v^{i:a_i}|^{-i}$ is in the projection kernel.
\end{proof}

\subsection{Spanning by cogs: a proof}

Given a vector $v \in \Lin_\EE \mathcal{O}$, write $\vsup{v}$ for its set-theoretic support, i.e., the finite subset $v^{-1}(\EE^*) \subseteq \mathcal{O}$.
More generally, given any finite subset $\sigma \subseteq \mathcal{O}$, 
write
\[
    \overline{\sigma} = \{ (i, a_i) \mid i \in I, a \in \sigma \}
\]
and define two binary relations 
\begin{align*}
    &(i, a_i) {?} (j, b_j) \iff a_i = b_j \text{ but } i \neq j, \\
    &(i, a_i) {!} (j, b_j) \iff \text{$a_i, b_j$ are related but not in the same way as $a_i, a_j$}
    % or equivalently $b_i, b_j$ since both $a$ and $b$ are in $\mathcal{O}$
\end{align*}
called \emph{ambiguities} and \emph{obstructions}.
(Recall that $a_i$ and $b_j$ are \emph{not related} if they are freely amalgamated in the reduct $\A_0$ of $\A$ 
--- i.e., they are not equal, and for no $R \in \mathcal{L}_0$ does $R(a_i, b_j) \vee R(b_j, a_i)$ hold;
in that case, we write $a_i \amalgindep b_j$).
Both relations are symmetric and ${?} \subseteq {!}$;
denote their images by ${?\overline{\sigma}} \subseteq {!\overline{\sigma}} \subseteq \overline{\sigma}$.
Lastly, refer to the atoms that appear in $\sigma$ by $\sqrt \sigma \subseteq \A$.

The prototypical example of an unobstructed family is $\overline{\vsup{\lambda \cdot a^+ \between a^-}} = \overline{\{a^+, a^-\}} = \overline{\{a^\pm\}}$,
where $a^+ \parallel a^-$ is an $\mathcal{O}$-dipole.

\paragraph{Two lemmas}
Let $v \in \Ker_\EE \mathcal{O}$ and let $v^{i:a_i}$ be a subvector.
In the cog decomposition results we are about to prove,
we will work with the assumption that $V = \vsup{v}$ is unobstructed (resp., unambiguous);
then so is $V' = \vsup{v^{i:a_i}} \subseteq \vsup{v}$.
We will be able to write $v^{i:a_i} = \sum_{a^\pm \in A^\pm} (\lambda_{a^\pm} \cdot a^+ \between a^-) a_i$
where the union of $V'$ and $K = A^\pm a_i = \{a^+ a_i, a^- a_i \mid a^\pm \in A^\pm\}$ is unobstructed (resp., unambiguous).
But we want to make $V \cup K$ unobstructed (resp., unambiguous);
we can do so by choosing $K$ more carefully.

% Here O is orbit of edges (a, b) with a < b, a ~ b.
% The YELLOW family of edges, which contains the GREEN family, has no obstructions.
% Nor does the union of GREEN and RED. 
% However YELLOW and RED together has an ambiguity (2, b) ? (1, b).
% Lemma-? removes ambiguities but still leaves an obstruction (2, \pi b) ! (2, d).
% Lemma-! removes such obstructions.
\begin{center}
\begin{tikzpicture}[font=\sffamily, line cap=round, line join=round, >={Latex[length=3.6mm,width=2.4mm]}]
\coordinate (a) at (0,0); \coordinate (b) at (3.25,0); \coordinate (d) at (7.25,0); \coordinate (c) at (5.55,2.35); \coordinate (pib) at (3.05,-1.5); \coordinate (taub) at (1.95,-3);
\begin{scope}[on background layer]
\draw[yellow!60, opacity=0.55, line width=40pt] (a) to[out=25,in=200] (c); \draw[green!55!black, opacity=0.28, line width=26pt] (a) to[out=25,in=200] (c);
\draw[yellow!60, opacity=0.55, line width=40pt] (b) -- (d);
\draw[magenta!65, opacity=0.30, line width=28pt] (a) -- (b);
\draw[magenta!55, opacity=0.22, line width=30pt] (a) to[out=-55,in=170] (pib);
\draw[magenta!45, opacity=0.14, line width=34pt] (a) to[out=-80,in=155] (taub);    
\end{scope}
\node (A) at (a) {$a$}; \node (B) at (b) {$b$}; \node (C) at (c) {$c$}; \node (D) at (d) {$d$}; \node[blue!70!black] (PIB) at (pib) {$\pi(b)$}; \node[blue!70!black] (TAUB) at (taub) {$\tau(b)$};
\draw[->, line width=2.2pt] (A) to[out=25,in=200] (C);
\draw[->, line width=2.2pt] (A) -- (B);
\draw[->, line width=2.2pt] (B) -- (D);
\draw[->, line width=2.4pt, blue!70!black] (A) to[out=-55,in=170] (PIB);
\draw[->, line width=2.4pt, blue!70!black] (A) to[out=-80,in=155] (TAUB);
\draw[->, line width=2.4pt, blue!70!black] (PIB) to[out=10,in=-140] (D);
\end{tikzpicture}
\end{center}

\begin{lemma}\label{lem:?}
    Let $K, V', V$ be finite subsets of $\mathcal{O}$ such that $\overline{V'} \subseteq \overline{V}$ and ${?}\overline{V' \cup K} = \emptyset = {?}\overline{V}$.
    Then there exists $\pi \in \Aut(\A / S \cup \sqrt{V'})$ that satisfies
    \[
        ?\overline{V \cup \pi(K)} = \emptyset.
    \]
\end{lemma}
\begin{proof}
    Fix $V', V$ and induct on the size of $? \overline{V \cup K}$.
    
    Let $(i, a_i) ? (j, b_j)$; without loss of generality we may assume $(j, b_j) \in \overline{V}$ and $(i, a_i) \in \overline{K} \setminus \overline{V'}$.
    Since $?\overline{V'} = \emptyset$, we see that $a_i \not\in \sqrt{V'}$; also $a_i \not\in S$, as $\mathcal{O}$ is $S$-ordered.
    Put 
    \[
        X = S \cup \sqrt{K \cup V} \setminus \{a_i\}
    \]
    and note that $X$ contains $S \cup \sqrt{V'}$.
    Use Lemma~\ref{lem:free-fresh} to obtain an automorphism $\pi \in \Aut(\A / X)$ so that $\pi(a_i) \not\in X \cup \{a_i\}$.    
    Now it is straightforward to check that
    \[
        {?}\overline{V \cup \pi(K)} \subseteq {?}\overline{V \cup K} \setminus \{(i, a_i)\}.
    \]
    Because ${?}\overline{V' \cup \pi(K)} = {?}\overline{\pi(V') \cup \pi(K)} = \pi(\emptyset) = \emptyset$ still,
    the inductive hypothesis gives us some $\pi' \in \Aut(\A / S \cup \sqrt{V'})$ such that $?\overline{V \cup \pi' \pi(K)} = \emptyset$.
\end{proof}

\begin{lemma}\label{lem:!}
    Let $K, V', V$ be finite subsets of $\mathcal{O}$ such that $\overline{V'} \subseteq \overline{V}$ and $!\overline{V' \cup K} = \emptyset = {!}\overline{V} \supseteq {!}\overline{V'}$.
    Then
    \[
        !\overline{V \cup \pi(K)} = \emptyset
    \]
    for some $\pi \in \Aut(\A / S \cup \sqrt{V'})$.
\end{lemma}
\begin{proof}
    By \autoref{lem:?} we may assume that $?\overline{V \cup K} = \emptyset$ already.
    As before, fix $V', V$ and proceed by induction on the size of $!\overline{V \cup K}$.
    
    Let $(i, a_i) ! (j, b_j)$; without loss of generality we may assume $(j, b_j) \in \overline{V}$ and $(i, a_i) \in \overline{K}$.
    Now $(i, a_i) \not\in \overline{V'}$, so $a_i \not\in \sqrt{V'}$;
    further, whenever $(i, a_i) ! (k, c_k)$ we observe that $c_k \not\in S \cup \{a_i\} \cup \sqrt{K \cup V'}$.
    Let $Y$ consist of all such $c_k$ and put
    \[
        X = S \cup \sqrt{K \cup V} \setminus (Y \cup \{a_i\}).
    \]
    Then $X, Y, \{a_i\}$ are pairwise disjoint, and we see $X$ contains $S \cup \sqrt{V'}$ as well as $\sqrt{K} \setminus \{a_i\}$.
    With Lemma~\ref{lem:free-fresh}, we may find some $\tau \in \Aut(\A / S \cup \sqrt{V'})$ such that $\tau(a_i) \not\in X \cup Y \cup \{a_i\}$ and $\tau \amalgindep Y$. 
    Again, we can check that
    \[
        {!}\overline{V \cup \tau(K)} \subseteq {!}\overline{V \cup K} \setminus (i, a_i)
    \]
    so the conclusion follows straightforwardly from the inductive hypothesis.
\end{proof}

\subsubsection{Unobstructed vector}
% https://q.uiver.app/#q=WzAsMTUsWzAsMywiYl4rIl0sWzEsMywiYiJdLFswLDQsImJeLSJdLFswLDAsImFeKyJdLFsxLDEsImEiXSxbMCwxLCJhXi0iXSxbMywyLCJjIl0sWzQsMiwiXFxyaWdodHNxdWlnYXJyb3ciXSxbNSwzLCJiXisiXSxbNSw0LCJiXi0iXSxbNSwxLCJhXi0iXSxbNSwwLCJhXisiXSxbNiwxLCJhIl0sWzYsMywiYiJdLFs4LDIsImMiXSxbMCwxLCIiLDAseyJjb2xvdXIiOlswLDYwLDYwXX1dLFsyLDEsIiIsMix7ImNvbG91ciI6WzI0MCw2MCw2MF19XSxbMyw0LCIiLDIseyJjb2xvdXIiOlswLDYwLDYwXX1dLFs1LDQsIiIsMCx7ImNvbG91ciI6WzI0MCw2MCw2MF19XSxbMyw2LCIiLDAseyJjdXJ2ZSI6LTF9XSxbNSw2XSxbMCw2XSxbMiw2LCIiLDIseyJjdXJ2ZSI6MX1dLFsxMCwxNCwiIiwwLHsiY29sb3VyIjpbMjQwLDYwLDYwXX1dLFs4LDE0LCIiLDIseyJjb2xvdXIiOlswLDYwLDYwXX1dLFs4LDEzXSxbOSwxM10sWzExLDEyXSxbMTAsMTJdLFsxMSwxNCwiIiwxLHsiY3VydmUiOi0xLCJjb2xvdXIiOlswLDYwLDYwXX1dLFs5LDE0LCIiLDEseyJjdXJ2ZSI6MSwiY29sb3VyIjpbMjQwLDYwLDYwXX1dXQ==
\[\begin{tikzcd}[cramped]
	{a^+} &&&&& {a^+} \\
	{a^-} & a &&&& {a^-} & a \\
	&&& c & \rightsquigarrow &&&& c \\
	{b^+} & b &&&& {b^+} & b \\
	{b^-} &&&&& {b^-}
	\arrow[draw={rgb,255:red,214;green,92;blue,92}, from=1-1, to=2-2]
	\arrow[curve={height=-6pt}, from=1-1, to=3-4]
	\arrow[from=1-6, to=2-7]
	\arrow[color={rgb,255:red,214;green,92;blue,92}, curve={height=-6pt}, from=1-6, to=3-9]
	\arrow[draw={rgb,255:red,92;green,92;blue,214}, from=2-1, to=2-2]
	\arrow[from=2-1, to=3-4]
	\arrow[from=2-6, to=2-7]
	\arrow[color={rgb,255:red,92;green,92;blue,214}, from=2-6, to=3-9]
	\arrow[from=4-1, to=3-4]
	\arrow[draw={rgb,255:red,214;green,92;blue,92}, from=4-1, to=4-2]
	\arrow[color={rgb,255:red,214;green,92;blue,92}, from=4-6, to=3-9]
	\arrow[from=4-6, to=4-7]
	\arrow[curve={height=6pt}, from=5-1, to=3-4]
	\arrow[draw={rgb,255:red,92;green,92;blue,214}, from=5-1, to=4-2]
	\arrow[color={rgb,255:red,92;green,92;blue,214}, curve={height=6pt}, from=5-6, to=3-9]
	\arrow[from=5-6, to=4-7]
\end{tikzcd}\]
\begin{proposition}\label{prop:!-free-decomposition}
    Let $v \in \Ker_\EE \mathcal{O}$ and suppose that $!\overline{\vsup v} = \emptyset$.
    Then we can write
    \[
        v = \sum_{a^\pm \in A^\pm} \lambda_{a^\pm} \cdot a^+ \between a^-
    \]
    with $!\overline{\vsup{v} \cup A^\pm} = \emptyset$ and $\lambda_{a^\pm} \in v(\mathcal{O})$.
\end{proposition}

We proceed by induction on the dimension $|I|$, 
noting that when $I = \emptyset$ we just have $v = v() \cdot () = v() \cdot ( \between )$ without any possible obstructions.

So suppose $I$ is non-empty; let $d \in I$ be the greatest. 
Group the terms in $v$ by their greatest atom so that $v = v^1 + v^2 + \cdots + v^k$.
We now induct on $k$.
If $k < 2$, we are done: as $v_{-d} = 0$ we must have $v = 0$, so the empty sum will do.
Otherwise \[
    v = v^{d:a_d} + v^{d:b_d} + v'.
\]
By the outer inductive hypothesis, we get \[
    v^{d:a_d} = v^{d:a_d}_{-d} a_d = \sum_{A^\pm} (\lambda_{a^\pm} \cdot a^+ \between a^-)a_d
\]
where we only know $!\overline{\vsup{v^{d:a_d}_{-d}} \cup A^\pm} = \emptyset$ so that $!\overline{\vsup{v^{d:a_d}} \cup A^\pm a_d} = \emptyset$.
But any $\pi \in \Aut(\A / S \cup \sqrt{\vsup{v^{d:a_d}}})$ satisfies 
\[
    v^{d:a_d} = \pi(v^{d:a_d}) = \sum_{a^\pm \in A^\pm} \lambda_{a^\pm} \cdot \pi a^+ \between \pi a^-,
\]
so by \autoref{lem:!} we may assume without loss of generality that \[
    !\overline{\vsup{v} \cup A^\pm a_d} = \emptyset.
\]
Similarly, we can write \[
    v^{d:a_d} = \sum_{B^\pm} (\lambda_{b^\pm} \cdot b^+ \between b^-)b_d
\]
where, in turn, we may upgrade the assumption that $! \overline{\vsup{v^{d:b_d}} \cup B^\pm b_d} = \emptyset$ to \[
    !\overline{\vsup{v} \cup A^\pm a_d \cup B^\pm b_d} = \emptyset.
\]

The key is that we may now invent a new element $z$, on which we impose the following relations with $S \cup \sqrt{A^\pm a_d \cup B^\pm b_d} \subseteq \A$: 
\begin{enumerate}
    \item $a_d, b_d < z$, and $z < s$ if $a_d, b_d < s$ for some $s \in S$ (enough to let $s$ be the least such);
    
    \item for any unary relation $P \in \mathcal{L}_0$:
    \[
        P(z) \;:\Longleftrightarrow\; P(a_d) \iff P(b_d)
    \]
    --- recall that $a, b \in \mathcal{O}$;
    \item for any binary relation $R \in \mathcal{L}_0$ and $s \in S$, $a^\pm \in A^\pm$, $b^\pm \in B^\pm$, $i \in I \setminus \{d\}$:
    \begin{enumerate}
        \item $R(z, s) \;:\Longleftrightarrow\; R(a_d, s) \iff R(b_d, s)$,
        % \item $R(s, z) \;:\Longleftrightarrow\; R(s, a_d) \iff R(s, b_d)$;
        \item $R(z, a^\pm_i) \;:\Longleftrightarrow\; R(a_d, a^\pm_i)$,
        \item $R(z, a_d) \;:\Longleftrightarrow\; \bot$,
        \item $R(z, b^\pm_i) \;:\Longleftrightarrow\; R(b_d, b^\pm_i)$;
        % \item $R(c_i, z) \;:\Longleftrightarrow\; R(c_i, a_d) \iff R(c_i, b_d)$;
        \item $R(z, b_d) \;:\Longleftrightarrow\; \bot$,
        % \item $R(a_d, z), R(b_d, z) \;:\Longleftrightarrow\; \bot$;
        \item and symmetrically for $R(-, z)$.
    \end{enumerate}
    These are well-defined because $a^\pm a_d, b^\pm b_d \in \mathcal{O}$ and $i = j$ whenever $a^\pm_i = b^\pm_j$.
\end{enumerate}
To see that the $\mathcal{L}$-structure $S \cup \sqrt{A^\pm a_d \cup B^\pm b_d} \cup \{z\}$ still embeds into $\A$, 
suppose towards a contradiction that it contains a forbidden $\mathcal{L}_0$-substructure $F$.
Then $F$ must contain $z$.
Since any two elements in $F$ are necessarily related, we must have $a_d, b_d \not\in F$.
Similarly, whenever $F$ contains $x_i$ where $x \in A^\pm \cup B^\pm, i \in I \setminus \{d\}$ it does not contain a distinct atom of the form $x'_i$.
It follows that \[
    s \mapsto s,\quad x_i \mapsto a_i,\quad z \mapsto a_d
\]
defines an injective function $\phi : F \to \A_0$,
which is furthermore an embedding (we only need to check this for pairs!) because $!\overline{A^\pm \cup B^\pm} = \emptyset$ and any $x_i, x'_{i'}$ for $i \neq i'$ are related.
This is impossible --- therefore assume $z \in \A$.

It is now routine to check that $a^+ a_d \parallel a^- z$ and $b^+ a_d \parallel b^- z$ are $\mathcal{O}$-dipoles for $a^\pm \in A^\pm, b^\pm \in B^\pm$
and that $!\overline{A^+ a_d \cup A^- z \cup B^+ b_d \cup B^- z} = \emptyset$.
By \autoref{lem:!} we may assume %since $!\overline{\vsup{v} \cup A^\pm a_d \cup B^\pm b_d}$ is empty
that $!\overline{\vsup{v} \cup A^+ a_d \cup A^- z \cup B^+ b_d \cup B^- z} = \emptyset$. 
(Alternatively, we could have explicitly ensured this when defining $z$).
Then
\begin{align*}
    v'' &= v
    - \sum_{A^\pm} \lambda_{a^\pm} \cdot a^+ a_d \between a^- z 
    - \sum_{B^\pm} \lambda_{b^\pm} \cdot b^+ b_d \between b^- z  \\
    &= v^{d:a_d}_{-d} z + v^{d:b_d}_{-d} z + v',
\end{align*}
when grouped into subvectors by the largest atom in each term, has at least one fewer component than $v$.
By the inner inductive hypothesis, we may write
we may write 
\[
    v'' = \sum_{C^\pm} \lambda_{c^\pm} \cdot c^+ \between c^-
\]
where $!\overline{\vsup{v''} \cup C^\pm} = \emptyset$.
But $\overline{\vsup{v''}} \subseteq \overline{\vsup{v} \cup A^+ a_d \cup A^- z \cup B^+ b_d \cup B^- z}$, 
so one last application of \autoref{lem:!} allows us to assume that
\[
    !\overline{\vsup{v} \cup A^+ a_d \cup A^- z \cup B^+ b_d \cup B^- z \cup C^\pm} = \emptyset.
\]
We conclude that
\[
    v = 
      \sum_{A^\pm} \lambda_{a^\pm} \cdot a^+ a_d \between a^- z 
    + \sum_{B^\pm} \lambda_{b^\pm} \cdot b^+ b_d \between b^- z
    + \sum_{C^\pm} \lambda_{c^\pm} \cdot c^+ \between c^-;
\]
in other words, we have decomposed an unobstructed vector into an unobstructed family of cogs.


\subsubsection{Unambiguous vector}
% https://q.uiver.app/#q=WzAsMTUsWzAsMSwiYSJdLFsyLDAsImIiXSxbMiwyLCJjIl0sWzEsMSwiYSciXSxbMSwzLCJkIl0sWzIsMywiZSJdLFsyLDQsImYiXSxbMywyLCJcXHJpZ2h0c3F1aWdhcnJvdyJdLFs0LDEsImEiXSxbNSwxLCJhJyJdLFs2LDAsImIiXSxbNiwyLCJjIl0sWzUsMywiZCJdLFs2LDMsImUiXSxbNiw0LCJmIl0sWzAsMSwiIiwyLHsiY29sb3VyIjpbMCw2MCw2MF19XSxbMCwyLCIiLDAseyJjb2xvdXIiOlsyNDAsNjAsNjBdfV0sWzMsMV0sWzMsMl0sWzQsNSwiIiwwLHsiY29sb3VyIjpbMCw2MCw2MF19XSxbNCw2LCIiLDIseyJjb2xvdXIiOlsyNDAsNjAsNjBdfV0sWzAsNF0sWzgsMTBdLFs5LDEwLCIiLDIseyJjb2xvdXIiOlswLDYwLDYwXX1dLFs5LDExLCIiLDAseyJjb2xvdXIiOlsyNDAsNjAsNjBdfV0sWzgsMTJdLFsxMiwxMywiIiwyLHsiY29sb3VyIjpbMCw2MCw2MF19XSxbMTIsMTQsIiIsMix7ImNvbG91ciI6WzI0MCw2MCw2MF19XSxbOCwxMV1d
\[\begin{tikzcd}[cramped]
	&& b &&&& b \\
	a & {a'} &&& a & {a'} \\
	&& c & \rightsquigarrow &&& c \\
	& d & e &&& d & e \\
	&& f &&&& f
	\arrow[color={rgb,255:red,214;green,92;blue,92}, from=2-1, to=1-3]
	\arrow[color={rgb,255:red,92;green,92;blue,214}, from=2-1, to=3-3]
	\arrow[from=2-1, to=4-2]
	\arrow[from=2-2, to=1-3]
	\arrow[from=2-2, to=3-3]
	\arrow[from=2-5, to=1-7]
	\arrow[from=2-5, to=3-7]
	\arrow[from=2-5, to=4-6]
	\arrow[color={rgb,255:red,214;green,92;blue,92}, from=2-6, to=1-7]
	\arrow[color={rgb,255:red,92;green,92;blue,214}, from=2-6, to=3-7]
	\arrow[color={rgb,255:red,214;green,92;blue,92}, from=4-2, to=4-3]
	\arrow[color={rgb,255:red,92;green,92;blue,214}, from=4-2, to=5-3]
	\arrow[color={rgb,255:red,214;green,92;blue,92}, from=4-6, to=4-7]
	\arrow[color={rgb,255:red,92;green,92;blue,214}, from=4-6, to=5-7]
\end{tikzcd}\]
\begin{proposition}\label{prop:?-free-decomposition}
    Let $v \in \Ker_\EE \mathcal{O}$ and suppose that $?\overline{\vsup v} = \emptyset$.
    Then we can write
    \[
        v = \sum_{a^\pm \in A^\pm} \lambda_{a^\pm} \cdot a^+ \between a^-
    \]
    with $?\overline{\vsup{v} \cup A^\pm} = \emptyset$ and $\lambda_{a^\pm} \in v(\mathcal{O})$.
\end{proposition}
We proceed again by induction, first on the dimension $|I|$ then on the cardinality of $! \overline{\vsup{v}}$.
The outer base case $I = \emptyset$ is trivial 
--- we have $v = v() \cdot ( \between )$, and no ambiguities may arise
--- whilst the inner base case is just \autoref{prop:!-free-decomposition}.

So suppose that $!\overline{\vsup{v}}$ contains some $(i, a_i)$.
Since $\overline{\vsup{v^{i:a_i}_{-i}}} \subseteq \overline{\vsup{v^{i:a_i}}} \subseteq \overline{\vsup{v}}$, we know that $v^{i:a_i}_{-i}$ is unambiguous.
Applying the outer inductive hypothesis, we may write
\[
    v^{i:a_i} = v^{i:a_i}_{-i} a_i = \sum_{A^\pm} (\lambda_{a^\pm} \cdot a^+ \between a^-) a_i
\]
where $\emptyset = ?\overline{(\vsup{v^{i:a_i}_{-i}} \cup A^\pm) a_i} = ?\overline{\vsup{v^{i:a_i}} \cup A^\pm a_i}$.
Moreover, we may assume by \autoref{lem:?} that \[
    ?\overline{\vsup{v} \cup A^\pm a_i} = \emptyset.
\]
Now we can show that whenever $(j, b_j) ! (i, a_i)$ in $\overline{\vsup{v}}$ we have $b_j \not\in \sqrt{A^\pm a_i}$.
Already $b_j = a_i$ would imply $j = i$ as $?\overline{\vsup{v}} = 0$, which is impossible.
So suppose to the contrary that $b_j = a^\pm_k$ for some $a^\pm \in A^\pm$ and $k \in I \setminus \{i\}$.
By the assumption above, we must have $j = k$;
this is a contradiction: note that $a^\pm a_i \in \mathcal{O}$.

Put $Y = \{b_j \mid (j, b_j) \in \overline{\vsup{v}}, (j, b_j) ! (i, a_i)\}$.
It follows that $X = S \cup \sqrt{\vsup{v} \cup A^\pm} \setminus Y \setminus \{a_i\}$ contains $S \cup \sqrt{A^\pm}$ and that $X, Y, \{a_i\}$ are pairwise disjoint.
Using Lemma~\ref{lem:free-fresh}, we may find $\tau \in \Aut(\A / X)$ such that $\tau(a_i) \not\in X \cup Y \cup \{a_i\}$, is greater than $a_i$, and is not related to any of $Y \cup \{a_i\}$.
Then, given any $a^\pm \in A^\pm$, we can straightforwardly check that $a^+ a_i \parallel a^- \tau(a_i)$ is an $\mathcal{O}$-dipole, 
that $?\overline{\vsup{v} \cup A^+ a_i \cup A^- \tau(a_i)} = \emptyset$,
and that
\begin{align*}
    v - \sum_{A^\pm} \lambda_{a^\pm} \cdot a^+ a_i \between a^- \tau(a_i)
    &= v -  v^{i:a_i} + v^{i:\tau(a_i)}\\
    &= v^{i : a_i \mapsto \tau(a_i)}    
\end{align*}
satisfies ${!}\overline{\vsup{v^{i : a_i \mapsto \tau(a_i)}}} \subseteq {!}\overline{\vsup{v}} \setminus \{(i, a_i)\}$.
The inner inductive hypothesis tells us 
that $v^{i : a_i \mapsto \tau(a_i)} = \sum_{B^\pm} \lambda_{B^\pm} \cdot b^+ \between b^-$ 
with \[
    ?\overline{\vsup{v^{i : a_i \mapsto \tau(a_i)}} \cup B^\pm} = \emptyset.
\]
But $\overline{\vsup{v^{i : a_i \mapsto \tau(a_i)}}} \subseteq \overline{\vsup{v} \subseteq A^+ a_i \cup A^- \tau(a_i)}$,
so \autoref{lem:?} allows us to assume that \[
    ? \overline{\vsup{v} \cup A^+ a_i \cup A^- \tau(a_i) \cup B^\pm} = \emptyset.
\]
We conclude that 
\[
    v =\sum_{A^\pm} \lambda_{a^\pm} \cdot a^+ a_i \between a^- \tau(a_i) + \sum_{B^\pm} \lambda_{b^\pm} \cdot b^+ \between b^-
\]
--- in other words, we have decomposed an unambiguous vector into an unambiguous family of cogs.

\subsubsection{General vector}
% https://q.uiver.app/#q=WzAsMTMsWzAsMSwiXFxkb3RzIl0sWzEsMiwiYSJdLFswLDMsIlxcZG90cyJdLFszLDAsImIiXSxbMyw0LCJjIl0sWzIsMiwiYSciXSxbNCwyLCJcXHJpZ2h0c3F1aWdhcnJvdyJdLFs1LDEsIlxcY2RvdHMiXSxbNSwzLCJcXGNkb3RzIl0sWzYsMiwiYSJdLFs4LDAsImIiXSxbOCw0LCJjIl0sWzcsMiwiYSciXSxbMiwxLCIiLDIseyJjb2xvdXIiOlsyNDAsNjAsNjBdfV0sWzEsMywiIiwyLHsiY29sb3VyIjpbMCw2MCw2MF19XSxbMSw0LCIiLDIseyJjb2xvdXIiOlsyNDAsNjAsNjBdfV0sWzUsM10sWzUsNF0sWzAsMSwiIiwwLHsiY29sb3VyIjpbMCw2MCw2MF19XSxbNyw5LCIiLDIseyJjb2xvdXIiOlswLDYwLDYwXX1dLFs4LDksIiIsMCx7ImNvbG91ciI6WzI0MCw2MCw2MF19XSxbOSwxMF0sWzksMTFdLFsxMiwxMCwiIiwyLHsiY29sb3VyIjpbMCw2MCw2MF19XSxbMTIsMTEsIiIsMSx7ImNvbG91ciI6WzI0MCw2MCw2MF19XV0=
\[\begin{tikzcd}[cramped]
	&&& b &&&&& b \\
	\dots &&&&& \cdots \\
	& a & {a'} && \rightsquigarrow && a & {a'} \\
	\dots &&&&& \cdots \\
	&&& c &&&&& c
	\arrow[color={rgb,255:red,214;green,92;blue,92}, from=2-1, to=3-2]
	\arrow[color={rgb,255:red,214;green,92;blue,92}, from=2-6, to=3-7]
	\arrow[color={rgb,255:red,214;green,92;blue,92}, from=3-2, to=1-4]
	\arrow[color={rgb,255:red,92;green,92;blue,214}, from=3-2, to=5-4]
	\arrow[from=3-3, to=1-4]
	\arrow[from=3-3, to=5-4]
	\arrow[from=3-7, to=1-9]
	\arrow[from=3-7, to=5-9]
	\arrow[color={rgb,255:red,214;green,92;blue,92}, from=3-8, to=1-9]
	\arrow[color={rgb,255:red,92;green,92;blue,214}, from=3-8, to=5-9]
	\arrow[color={rgb,255:red,92;green,92;blue,214}, from=4-1, to=3-2]
	\arrow[color={rgb,255:red,92;green,92;blue,214}, from=4-6, to=3-7]
\end{tikzcd}\]
\begin{theorem}\label{thm:cog-span-generally}
    Let $v \in \Ker_\EE \mathcal{O}$.
    Then we can write
    \[
        v = \sum_{a^\pm \in A^\pm} \lambda_{a^\pm} \cdot a^+ \between a^-
    \]
    with $\lambda_{a^\pm} \in v(\mathcal{O})$.
\end{theorem}
This is an easier induction on $|I|$ then on $?\overline{\vsup{v}}$.
If $I = \emptyset$, the decomposition is trivial; 
if $v$ is unambiguous already, the decomposition comes from \autoref{prop:?-free-decomposition}.

Now suppose $?\overline{\vsup{v}}$ contains some $(i, a_i)$,
and use the outer inductive hypothesis to write 
\[
    v^{i:a_i}
    = v^{i:a_i}_{-i} a_i 
    = \sum_{A^\pm} (\lambda_{a^\pm} \cdot a^+ \between a^-) a_i.
\]
Then neither $S$ nor $\sqrt{A^\pm}$ contains $a_i$,
so $X = S \cup \sqrt{\vsup{v} \cup A^\pm} \setminus \{a_i\}$ contains $S \cup \sqrt{A^\pm}$.
Using Lemma~\ref{lem:free-fresh}, we may find some $\pi \in \Aut(\A / X)$ such that 
$\pi(a_i) \not\in X$, 
$\pi(a_i) > a_i$, 
$\pi(a_i) \amalgindep a_i$. % This 'stationary independence relation' symbol means: 
% \pi(a_i) and a_i are freely amalgamated in A0 over the empty set (and thus over anything).
We can check that
\begin{enumerate}
    \item 
    $a^+ a_i \parallel a^- \pi(a_i)$ is an $\mathcal{O}$-dipole, given any $a^\pm \in A^\pm$;

    \item 
    $v - \sum_{A^\pm} \lambda_{a^\pm} \cdot a^+ a_i \between a^- \pi(a_i) = v - v^{i:a_i} + v^{i:\pi(a_i)} = v^{i:a_i \mapsto \pi(a_i)}$
    satisfies 
    \[
        {?}\overline{\vsup{v^{i:a_i \mapsto \pi(a_i)}}} \subseteq {?}\overline{\vsup{v}} \setminus \{(i, a_i)\}.
    \]
\end{enumerate}
It follows from the inner inductive hypothesis that
\begin{align*}
    v 
    &= \sum_{a^\pm \in A^\pm} \lambda_{a^\pm} \cdot a^+ a_i \between a^- \pi(a_i) + v^{i:a_i \mapsto \pi(a_i)} \\
    &= \sum_{a^\pm \in A^\pm} \lambda_{a^\pm} \cdot a^+ a_i \between a^- \pi(a_i) + \sum_{b^\pm \in B^\pm} \lambda_{b^\pm} \cdot b^+ \between b^-.
\end{align*}



\section{All those equivariant subspaces}
We finish this section with an important corollary of Theorem~\ref{thm:cog-span-generally}.
Let $\mathcal{O}_1 \subseteq \A^{I_1}, \dots, \mathcal{O}_n \subseteq \A^{I_n}$ all be $S$-ordered orbits.
Then $\len(\Lin_\FF(\mathcal{O}_1 \uplus \dots \uplus \mathcal{O}_n)) = 2^{|I_1|} + \dots + 2^{|I_n|}$;
in fact, we know and can characterise all the $\Aut(\A)_{(S)}$-equivariant subspaces of $\Lin_\FF(\mathcal{O}_1 \uplus \dots \uplus \mathcal{O}_n) \simeq \Lin_\FF(\mathcal{O}_1) \oplus \cdots \oplus \Lin_\FF(\mathcal{O}_n)$.

\paragraph{Local coefficients}
First we set up some notations.
Consider the $\sum_k 2^{|I_k|}$ projected $S$-ordered orbits $\mathcal{O}_k|^{J}$ for $1 \leq k \leq n, J \subseteq I_k$.
Suppose
\[
    f : \mathcal{O}_k|^J \to \mathcal{O}_{k'}|^{J'}
\]
is an $\Aut(\A)_{(S)}$-equivariant bijection.
Take any $o_\bullet \in \mathcal{O}_k|^J$,
and enumerate its entries as $o_1 < \dots < o_{|J|}$.
Similarly, enumerate the entries of $f(o_\bullet)$ as $o'_1 < \dots < o'_{|J'|}$.
Then $\{o_1, \dots, o_{|J|}\} = \{ o'_1, \dots, o'_{|J'|} \}$ because $\A$ has no algebraicity; 
since the orbits are ordered, we must have $|J| = |J'|$ and $o_1 = o'_1, \dots, o_{|J|} = o'_{|J'|}$.
That is, $f$ must be the obvious function that reindexes a $J$-tuple to a $J'$-tuple
--- hence we will write $o^{/J'}_\bullet$ instead of $f(o_\bullet)$, leaving $f$ implicit.

Now, let $\mathcal{Q}_{1} = \mathcal{O}_{k_1}|^{J_1}, \dots, \mathcal{Q}_{t} = \mathcal{O}_{k_t}|^{J_t}$ be the distinct $S$-ordered orbits up to $\Aut(\A)_{(S)}$-equivariant bijections,
which we enumerate in such a way that $|J_1| \geq |J_2| \geq \dots \geq |J_t| = 0$.

\begin{definition}
    For $i = 1, \dots, t$, let $P_i$ consist of pairs $(k, J)$ such that $\mathcal{O}_k|^J$ is $\Aut(\A)_{(S)}$-equivariantly isomorphic to $\mathcal{Q}_i$.
    Assemble all $|P_i|$ projections into a single map
    \[
        (-){\restriction_i} : \Lin_\FF(\mathcal{O}_1 \uplus \dots \uplus \mathcal{O}_n) \to \Lin_{\FF^{P_i}} \mathcal{Q}_i.
    \]
    More precisely $(v_1, \dots, v_n){\restriction_i}(a_\bullet)$ is the $P_i$-tuple whose entry at $(k, J)$ is $v_k|^J(a^{/J}_\bullet) \in \FF$.
    It is straightforward to check that $(-){\restriction_i}$ is $\Aut(\A)_{(S)}$-equivariant and linear.
\end{definition}

Let $W \subseteq \Lin_\FF( \mathcal{O}_1 \uplus \dots \uplus \mathcal{O}_n )$ be an $\Aut(\A)_{(S)}$-equivariant subspace.
Using the $t$ finite-dimensional vector spaces $W {\restriction_1} (\mathcal{Q}_1) \subseteq \FF^{P_1}, \dots, W {\restriction_t} (\mathcal{Q}_t) \subseteq \FF^{P_t}$
we define $\widetilde W$, which consists of all vectors $v \in \Lin_\FF( \mathcal{O}_1 \uplus \dots \uplus \mathcal{O}_n )$ such that
\[
    v {\restriction_1} (\mathcal{Q}_1) \subseteq W {\restriction_1} (\mathcal{Q}_1), \dots, v {\restriction_t} (\mathcal{Q}_t) \subseteq W {\restriction_t} (\mathcal{Q}_t).
\]
Then $\widetilde W$ is an $\Aut(\A)_{(S)}$-equivariant subspace that contains $W$.
It turns out these two are equal:

\begin{lemma}\label{lem:coeff-approximation}
    \begin{align*}
        &\widetilde W \cap \ker(\restriction_{i+1}) \cap \cdots \cap \ker(\restriction_t) \\
        \subseteq{}&W \cap \ker(\restriction_{i+1}) \cap \cdots \cap \ker(\restriction_t).
    \end{align*}
    In particular $\widetilde W \subseteq W$ when $i = t$.
\end{lemma}

\paragraph{Proof of the lemma}
by induction on $i$.
When $i = 0$, this containment is trivial:
\begin{claim}
    $\ker(\restriction_1) \cap \ker(\restriction_2) \cdots \cap \ker(\restriction_t) = \{0\}$.
\end{claim}
\begin{proof}
    Let $v = (v_1, \dots, v_n) \in \ker(\restriction_1) \cap \ker(\restriction_2) \cdots \cap \ker(\restriction_t)$. 
    Each $(k, I_k)$ belongs to some $P_i$; 
    that $v {\restriction_i} = 0$ implies $0 = v_k|^{I_k} = v_k$. 
\end{proof}

To prove the containment for $i + 1$, 
we allow $v {\restriction_{i+1}}$ to be non-zero.
But $v {\restriction_{i+1}}$ satisfies the next best property:
\begin{claim}
    The image of \[
        \widetilde W \cap \ker(\restriction_{i+2}) \cap \cdots \cap \ker(\restriction_t)
    \] under $\restriction_{i+1}$ 
    is contained in $\Ker_{W {\restriction}_{i+1} (\mathcal{Q}_{i+1})} \mathcal{Q}_{i+1}$. 
\end{claim}
\begin{proof}
    Take any $v = (v_1, \dots, v_n)$ satisfying \[
        v {\restriction_{i+2}} = 0, \dots, v {\restriction_t} = 0
    \] from $\widetilde W$.
    That $v {\restriction_{i+1}} \in \Lin_{W{\restriction_{i+1}}(\mathcal{Q}_{i+1})} \mathcal{Q}_{i+1}$ is clear from the definition of $\widetilde W$.
    Recall that $\mathcal{Q}_{i+1}$ = $\mathcal{O}_{k_{i+1}}|^{J_{i+1}}$.
    Given $j \in J_{i+1}$,
    we need to prove that $v {\restriction_{i+1}} |^{-j} = 0$.
    
    To do so, take $(k, J) \in P_{i+1}$;
    the unique monotone bijection between $J_{i+1}$ and $J$ restricts to one between $J_{i+1} \setminus \{j\}$ and $J \setminus \{j'\}$.
    Now $(k, J \setminus \{j'\})$ belongs to some $P_{i'}$ with $i' > i + 1$,
    so $v {\restriction_{i'}} = 0$.
    We calculate that $v {\restriction_{i+1}} |^{-j}(a_\bullet)_{k, J}$ is equal to 
    \[
        \sum_{b_\bullet \in \mathcal{O}_k, b|^{J \setminus \{j'\}}_\bullet = a^{/ J \setminus \{j'\}}_\bullet} v_k(b_\bullet),
    \]
    and that so is $0 = v {\restriction_{i'}} (a^{/ J \setminus \{j'\}}_\bullet)_{k, J \setminus \{j'\}}$.
\end{proof}

Now Theorem~\ref{thm:cog-span-generally} tells us that $\Ker_{\widetilde W {\restriction_{i+1}} (\mathcal{Q}_{i+1})} \mathcal{Q}_{i+1} \subseteq \Cog_{\widetilde W {\restriction_{i+1}} (\mathcal{Q}_{i+1})} \mathcal{Q}_{i+1}$,
which is good news:
\begin{claim}
    $\Cog_{W {\restriction}_{i+1} (\mathcal{Q}_{i+1})} \mathcal{Q}_{i+1}$ is contained in
    the image of \[
        W \cap \ker(\restriction_{i+2}) \cap \cdots \cap \ker(\restriction_t)
    \] under $\restriction_{i+1}$. 
\end{claim}
\begin{proof}
    Let $w {\restriction_{i+1}} (a_\bullet) \in W {\restriction_{i+1}} (\mathcal{Q}_{i+1})$.
    Let $S'$ consist of $S$ together with every atom appearing in $w$ but not in $a_\bullet$.
    We generalise the proof of Theorem~\ref{thm:cogs-arise-everywhere}.
    
    Start by applying Proposition~\ref{prop:cog-fresh-full} and Remark~\ref{rem:duo} to get automorphisms $\pi_j$ for $j \in J_{i+1}$ such that $a_\bullet \parallel \prod_{j \in J_{i+1}} \pi_j \cdot a_\bullet$ is an $\mathcal{Q}_{i+1}$-duo, 
    where $\pi_j$ fixes $S'$ and $a_{j'}, \pi_{j'} \cdot a_{j'}$ for $j' \in J \setminus \{j\}$.
    Put \[
        w' = \prod_{j \in J_{i+1}} (1 - \pi_j) \cdot w \in W.
    \]
    Given $1 \leq i' \leq t$, observe that as no more atoms can appear in $w{\restriction_{i'}}$ than in $w$, we have
    \begin{align*}
        &w' {\restriction_{i'}}
        = \prod_{j \in J_{i+1}} (1 - \pi_j) \cdot w {\restriction_{i'}} \\ 
        &{}= \sum_{b_\bullet \in \mathcal{Q}_{i'}, \{b_j \mid j\} \supseteq \{a_j \mid j\}} \sum_{J' \subseteq J} (-1)^{|J'|}  
            w {\restriction_{i'}}(b_\bullet) \prod_{j \in J'} \pi_j \cdot b_\bullet.
    \end{align*}
    Suppose $\{b_j \mid j \in J_{i'}\} \supseteq \{a_j \mid j \in J_{i+1}\}$. 
    Then $i' \leq i + 1$; if $i' = i + 1$, we must have $b_\bullet = a_\bullet$.
    We therefore have \[
        w' {\restriction_{i+1}} = w {\restriction_{i+1}} (a_\bullet) \cdot a_\bullet \between \prod_{j \in J_{i+1}} \pi_j \cdot a_\bullet
    \]
    and $w' {\restriction_{i+2}} = 0, \dots, w' {\restriction_{t}} = 0$.
    This proves that $(W \cap \ker\restriction_{i+2} \cap \cdots \cap \ker\restriction_t) {\restriction_{i+1}}$ contains $\Cog_{W {\restriction_{i+1}}(\mathcal{Q}_{i+1})} \mathcal{Q}_{i+1}$.
\end{proof}

This is enough to establish Lemma~\ref{lem:coeff-approximation} for $i + 1$ assuming the result for $i$.
Indeed, given $v \in \widetilde W \cap \ker(\restriction_{i+2}) \cap \dots \cap \ker(\restriction_t)$,
we can find $w \in W \cap \ker(\restriction_{i+2}) \cap \dots \cap \ker(\restriction_t) \subseteq \widetilde W$ such that 
\[
    v {\restriction_{i+1}} = w {\restriction_{i+1}}
\]
by the preceding claims.
But $(v - w) {\restriction_{i+1}} = 0$ --- 
that is, $v - w$ lies in in $\ker(\restriction_{i+1})$ as well as $\widetilde W \cap \ker(\restriction_{i+2}) \cap \cdots \cap \ker(\restriction_t)$.
It follows from the inductive hypothesis that \[
    v - w \in W \cap \ker(\restriction_{i+2}) \cap \dots \cap \ker(\restriction_t),
\]
so $v = (v - w) + w$ is a member of $W \cap \ker(\restriction_{i+2}) \cap \dots \cap \ker(\restriction_t)$ as well.

\paragraph{Lengths}
Let $W, W'$ be two $\Aut(\A)_{(S)}$-equivariant subspaces of $\Lin_\FF(\mathcal{O}_1 \uplus \dots \uplus \mathcal{O}_n)$.
If we have $W {\restriction_1}(\mathcal{Q}_1) = W' {\restriction_1}(\mathcal{Q}_1), \dots, W {\restriction_t}(\mathcal{Q}_t) = W' {\restriction_t}(\mathcal{Q}_t)$,
then $W = \widetilde W = \widetilde{W'} = W'$ by Lemma~\ref{lem:coeff-approximation}.
An immediate consquence is:

\begin{proposition}\label{prop:length-upper-bound}
    Let $W_0 \subsetneq W_1 \subsetneq \cdots \subsetneq W_l$ be a chain of $\Aut(\A)_{(S)}$-equivariant subspaces in $\Lin_\FF(\mathcal{O}_1 \uplus \dots \uplus \mathcal{O}_n)$.
    Then $l \leq 2^{|I_1|} + \cdots + 2^{|I_n|}$. 
\end{proposition}
\begin{proof}
    We obtain $t$ chains
    \begin{align*}
        W_0 {\restriction_1} (\mathcal{Q}_1) \subseteq W_1 {\restriction_1} (\mathcal{Q}_1) \subseteq \cdots \subseteq W_l {\restriction_1} (\mathcal{Q}_1) \subseteq \FF^{P_1}, \\
        W_0 {\restriction_2} (\mathcal{Q}_2) \subseteq W_1 {\restriction_2} (\mathcal{Q}_2) \subseteq \cdots \subseteq W_l {\restriction_2} (\mathcal{Q}_2) \subseteq \FF^{P_2}, \\
        \vdots \\
        W_0 {\restriction_t} (\mathcal{Q}_t) \subseteq W_1 {\restriction_t} (\mathcal{Q}_t) \subseteq \cdots \subseteq W_l {\restriction_t} (\mathcal{Q}_t) \subseteq \FF^{P_t}.
    \end{align*}
    At each of the $l$ steps, one of the $t$ containments must be strict.
    Hence $l \leq |P_1| + |P_2| + \dots + |P_t| = 2^{|I_1|} + \dots + 2^{|I_t|}$. 
\end{proof}

It follows that any $\Aut(\A)_{(S)}$-equivariant subspace of $\Lin_\FF(\mathcal{O}_1 \uplus \dots \uplus \mathcal{O}_n)$ is finitely generated.
We can compute the local coefficients of such subspaces easily:
\begin{remark}
    For $v \in \Lin_\FF(\mathcal{O}_1 \uplus \dots \uplus \mathcal{O}_n)$, 
    let $\langle v \rangle$ denote the $\Aut(\A)_{(S)}$-equivariant subspace it generates.
    Then:
    \begin{enumerate}
        \item 
        $\langle v \rangle {\restriction_i}(\mathcal{Q}_i)$ is the subspace of $\FF^{P_i}$ generated by vectors of the form $v {\restriction_i} (a_\bullet)$,
        which is zero unless every atom appearing in $a_\bullet$ appears in $v$
        --- there are only finitely many such $a_\bullet$'s;

        \item 
        $\langle v, v' \rangle {\restriction_i}(\mathcal{Q}_i) = \langle v \rangle {\restriction_i}(\mathcal{Q}_i) + \langle v' \rangle {\restriction_i}(\mathcal{Q}_i)$.
    \end{enumerate}
\end{remark}

We may now exhibit a chain of $\Aut(\A)_{(S)}$-equivariant subspaces whose length is precisely $\sum_{i=1}^t 2^{|P_i|}$,
generalising \cite[Corollary~4.12]{BFKM24}.
Take any $(k, J) \in P_i$. 
Pick some $a_\bullet \in \mathcal{O}_k$,
and let $\pi_j, j \in I_k$ be the automorphisms from Proposition~\ref{prop:cog-fresh-full}.
Define a vector
\begin{align*}
    v^i_{k, J}
    &\in \Lin_\FF(\mathcal{O}_1 \uplus \dots \uplus \mathcal{O}_n)\\
    &\simeq \Lin_\FF(\mathcal{O}_1) \oplus \dots \oplus \Lin_\FF(\mathcal{O}_n)
\end{align*}
with $\prod_{j \in J} (1 - \pi_j) \cdot a_\bullet$ as its $k$th component and zero everywhere else.
Then 
\[
    \langle v^i_{k, J} \rangle {\restriction_{i'}} (\mathcal{Q}_{i'})_{k', J'} =
    \begin{cases}
        \FF & \text{if $k = k'$ and $J \subseteq J'$}, \\
        \{0\} & \text{otherwise}.
    \end{cases}
\]
Enumerating each $P_i$ as $(k^i_1, J^i_1), (k^i_2, J^i_2), \dots, (k^i_{|P_i|}, J^i_{|P_i|})$,
we obtain a chain
\begin{align*}
    &\langle  \rangle \\
    \subsetneq{} &\langle v^t_{k^t_1, J^t_1} \rangle \\ 
    \subsetneq{} &\langle v^t_{k^t_1, J^t_1}, v^t_{k^t_2, J^t_2} \rangle \\ 
    \subsetneq{} &\cdots \\
    \subsetneq{} &\langle v^t_{k^t_1, J^t_1}, v^t_{k^t_2, J^t_2}, \dots, v^t_{k^t_{|P_t|}, J^t_{|P_t|}} \rangle \\
    \subsetneq{} &\langle v^t_{k^t_1, J^t_1}, v^t_{k^t_2, J^t_2}, \dots, v^t_{k^t_{|P_t|}, J^t_{|P_t|}}, v^{t-1}_{k^{t-1}_1, J^{t-1}_1} \rangle \\
    \subsetneq{} &\cdots
\end{align*}
of length $|P_t| + |P_{t-1}| + \cdots + |P_1| = 2^{|I_1|} + \cdots + 2^{|I_n|}$.
With the upper bound in Proposition~\ref{prop:length-upper-bound}, we conclude:
\begin{theorem}
    $\len(\Lin_\FF(\mathcal{O}_1 \uplus \dots \uplus \mathcal{O}_n)) = 2^{|I_1|} + \dots + 2^{|I_n|}$.
\end{theorem}
