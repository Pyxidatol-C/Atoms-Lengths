\section{Finite length from free amalgamation with a generic order}\label{sec:free-amalg}
\newcommand{\cal}{\mathcal}

As we saw in Non-example~\ref{non-ex:weak-smooth-approximation}, the ordered atoms do not have oligomorphic approximation. 
Nonetheless they do have the finite length property over any field \cite[Theorem~4.8]{BFKM24}, not just over those of characteristic zero. We shall now generalise that result to a wide class of structures. In particular, we will deduce the finite length property for the Rado graph and its variants, even with finitely many constants fixed.

We shall now state our assumptions and results, with proofs in Section~\ref{sec:cogs-turn}.

\paragraph*{Graph vocabulary}
Consider a finite relational vocabulary $\sigma_0$ consisting of unary and binary relations only.
(This allows us to talk about graphs with coloured vertices and edges, but in the running example we will consider a single edge relation.)

\paragraph*{Free amalgamation class}
Let $\Cclass_0$ be a \emph{free} amalgamation class of finite $\sigma_0$-structures. 
This is a standard notion in model theory --- see \cite[Section~2.1]{Macpherson11}.
Informally, it means that when we perform amalgamation, we do not need to glue together any new elements, or introduce new relations.

There is a useful characterisation \cite[Lemma~4.5]{SS20} of the amalgamation class $\Cclass_0$ being free.
Since $\Cclass_0$ is closed under substructures and isomorphisms, 
we know that it consists of all the finite $\sigma_0$-structures which do not embed any structure from $\Fclass$, 
where $\Fclass$ consists of all the minimal finite $\sigma_0$-structures that do not belong to $\Cclass_0$;
we write \[
    \Cclass_0 = \Forb(\Fclass).
\]
That $\Cclass_0$ is a free amalgamation class means that in each $F \in \Fclass$, 
any two elements $x, y$ are either equal or satisfy at least one of $R(x, y)$ and $R(y, x)$ for some binary relation $R \in \sigma_0$
--- from here on, we will just say $x, y$ are \emph{related}.
Conversely, given a family $\Fclass$ of finite $\sigma_0$-structures where every two elements are related,
the class $\Forb(\Fclass)$ of finite $\sigma_0$-structures form a free amalgamation class.

%Here are a few free amalgamation classes.

\begin{example}\label{ex:free-amalg-equality}
    Let $\sigma_0$ be empty. 
    Then $\Cclass_0 = \Forb(\{\})$ is a free amalgamation class consisting of all finite pure sets.
\end{example}

\begin{example}\label{ex:free-amalg-graphs}
    Let $\sigma_0$ consist of a single binary relation $E$. 
    Consider $\sigma_0$-structures ${\circlearrowright} = \{x\}$ and ${\rightarrow} = \{y, z\}$, where the relation $E$ is interpreted as $\{(x, x)\}$ and $\{(y, z)\}$.
    Then $\Cclass_0 = \Forb(\{{\circlearrowright}, {\rightarrow}\})$ is a free amalgamation class consisting of all finite undirected graphs.

    In addition, let $K_n$ be the $\sigma_0$-structure representing a complete graph on $n$ vertices.
    Then 
    \begin{align*}
        \Forb(\{{\circlearrowright}, {\rightarrow}, K_3\}) 
        \subseteq \Forb(\{{\circlearrowright}, {\rightarrow}, K_4\})
      %  \subseteq \Forb(\{{\circlearrowright}, {\rightarrow}, K_5\})
        \subseteq \cdots 
        \subseteq \Forb(\{{\circlearrowright}, {\rightarrow}\})
    \end{align*}    
    are all free amalagamation classes.
\end{example}

\paragraph*{Irreflexivity} 
For technical reasons, we restrict attention to classes $\Cclass_0$ where all structures are {\em irreflexive}, i.e. such that if $R(x,y)$ then $x\neq y$ for each relation $R$. This does not lose generality: for any free amalgamation class $\Cclass_0$, we may replace every binary relation $R$ in the vocabulary with a new unary predicate $R_=$, interpreted in every structure so that $R_=(x)$ iff $R(x,x)$, and a binary relation interpreted as the irreflexive part of $R$. Then $\Cclass_0$ interpreted over this modified language is still a free amalgamation class, and all our results transport back to the original $\Cclass_0$. This way of encoding arbitrary structures as irreflexive ones is standard; see e.g.~\cite[Section~2.4]{Herwig1998} and \cite[p.~121]{SS20}. All examples considered here are already irreflexive.

\paragraph*{Generically ordered expansion}
Now, let $\sigma$ consist of $\sigma_0$ together with a new binary relation $<$.
Consider the class $\Cclass$ of $\sigma$-structures obtained from $\Cclass_0$, 
by interpreting $<$ in any $\sigma_0$-structure there as any total order.

Observe that $\Cclass$ is an amalgamation class.
Indeed, let $X, Y_1, Y_2$ be $\sigma$-structures in $\Cclass$ with $X \subseteq Y_1 \cap Y_2$.
Then we can amalgamate $Y_1, Y_2$ over $X$ as $\sigma_0$-structures and as $\{<\}$-structures, 
both using the disjoint union of $Y_1, Y_2$ over $X$ as the underlying set. 
Superposing these relations will give an $\sigma$-structure in $\Cclass$, which is the desired amalgamation.
(Because of the total order, this amalgamation in $\Cclass$ is not free unless it is trivial.)
Denote the Fraïssé limit of $\Cclass$ by $\A$.

\begin{theorem}\label{thm:ordered-free-amalg-has-finite-length}
    The ordered structure $\A$, even with finitely many constants fixed, has the finite length property over any field.
\end{theorem}

Before we prove this theorem in Section~\ref{sec:cogs-turn}, let us state an easy consequence of it and give some examples.

The Fraïssé limit $\A_0$ of $\Cclass$ is a reduct of $\A$.
To see this, notice that $\A$, when viewed as an $\sigma_0$-structure, shares the same age as $\A_0$ --- namely, $\Cclass_0$.
Moreover, it follows from a back-and-forth argument \cite[Lemma~7.1.4]{hodges1993model} that the $\sigma_0$-structure $\A$ is also homogeneous and, therefore, that it is isomorphic to $\A_0$.
So we may assume that $\A$ and $\A_0$ have the same underlying set; we then have $\Aut(\A) \subseteq \Aut(\A_0)$, where $\Aut(X)$ denotes the group of automorphisms of $X$.

\begin{corollary}\label{cor:free-finite-length}
    The reduct $\A_0$ of $\A$, even with finitely many constants fixed, has the finite length property over any field.
\end{corollary}
\begin{proof}
    A chain of subspaces in $\Lin_\FF \A_0^d = \Lin_\FF \A^d$ supported by $S \subseteq \A_0$ is also a chain supported by $S \subseteq \A$; 
    the latter has a bounded length by the theorem above.
\end{proof}

Conversely, we call $\A$ the \emph{generically ordered expansion} of $\A_0$ or simply the ``ordered $\A_0$''.

\begin{example}\label{ex:geneircally-ordered-equality}
    Continuing from Example~\ref{ex:free-amalg-equality},
    the ordered atoms (Example~\ref{ex:order-atoms}) is the generically ordered expansion of the equality atoms (Example~\ref{ex:equality-atoms}).
\end{example}

\begin{example}\label{ex:geneircally-ordered-graphs}
    Continuing from Example~\ref{ex:free-amalg-graphs}, we obtain generically ordered expansions of the Rado graph and of the $K_n$-free Henson graphs.
    The ordered Rado graph was studied in \cite{BPP15}.
    In the example below, we will work with the ordered triangle-free Henson graph.
\end{example}


\section{How the cogs turn: proof of Theorem~\ref{thm:ordered-free-amalg-has-finite-length}}\label{sec:cogs-turn}
We will proceed in several steps, with the general idea similar to \cite[Section~4.1]{BFKM24}, but with significant new complications arising from the presence of nontrivial relations in $\A_0$.


\subsection{Orbits and projections}

To start with, let us view $\A^d$ as $\A^{\{1, \dots, d\}}$. More generally, it will be convenient to consider $\A^I$ for a finite totally ordered indexing set $I \subseteq \Q$.
Fix a finite support $S \subseteq \A$.
We shall say that a tuple $a\in\A^I$ is {\em ($S$-)ordered} if $a_i \not\in S$ for all $i$, and $a_i < a_j$ whenever $i < j$. Then the orbit
$\cal O = \Aut(\A/S) \cdot a$ only contains $S$-ordered tuples, and we will call the orbit $S$-ordered as well.
If $a$ is not $S$-ordered, by removing the entries that repeat or come from $S$ and reordering the rest, we can always find an $\Aut(\A/S)$-equivariant bijection from $\cal O$ to an $S$-ordered orbit. (Here and in the following, $\Aut(\A/S)$ is the group of those automorphisms of $\A$ that fix every element of $S$. So ``$\Aut(\A/S)$-equivariant'' is synonymous with ``$S$-supported''.)

To study the lengths of orbit-finitely spanned spaces, we may focus on a single ordered orbit at a time:
\begin{claim}\label{claim:reduction-to-one-orbit}
    The following are equivalent:
    \begin{enumerate}
        \item For any $d$ and finite $S \subseteq \A$, chains of $\Aut(\A/S)$-equivariant subspaces in $\Lin_\FF \A^d$ are bounded in length;
        \item $\Lin_\FF \cal O$ has finite length for any ordered orbit $\cal O$.
    \end{enumerate}
\end{claim}
\begin{proof}
    Since $\A$ is oligomorphic, given any $d$ and $S$, 
    we know that $\A^d$ is in an $\Aut(\A/S)$-equivariant bijection with a finite disjoint union $\biguplus_i \cal O_i$ of $S$-ordered orbits.
    Hence the length of $\Lin_\FF \A^d$, under the action of $\Aut(\A/S)$, equals \[
        \len(\Lin_\FF(\biguplus_i \cal O_i)) = \len(\bigoplus_i \Lin_\FF \cal O_i) = \sum_i \len(\Lin_\FF \cal O_i),
    \]
    which is finite if and only if each summands is finite.
\end{proof}

So fix an ordered orbit $\cal O = \Aut(\A/S) \cdot o \subseteq \A^I$.
From here we take an inductive approach.
By $o |^J$ we mean the restriction of $o : I \to \A$ to $J \subseteq I$;
particularly, we will often write $o |^{-i}$ instead of $o |^{ I \setminus \{i\} }$.
The image $\cal O|^J$ of $\cal O$ under this projection agrees with $\Aut(\A/S) \cdot o |^J$ and is still ordered.

The function $(-)|^J$ lifts to a linear $\Aut(\A/S)$-equivariant map
\begin{align*}
    (-)|^J : \Lin_\FF \cal O &\to \Lin_\FF \cal O|^J.
\end{align*}
Many cancellations can occur under $(-)|^J$; 
the \emph{projection kernel} is the $\Aut(\A/S)$-equivariant subspace
\[
    \Ker_\FF \cal O = \bigcap_{i \in I} \ker{ (-)|^{-i} }
\]
of $\Lin_\FF \cal O$.

\begin{claim}\label{claim:reduction-to-kernel}
    The following are equivalent:
    \begin{enumerate}
        \item $\Lin_\FF \cal O$ has finite length for every ordered orbit $\cal O$;
        \item $\Ker_\FF \cal O$ has finite length for every ordered orbit $\cal O$.
    \end{enumerate}
\end{claim}
\begin{proof}
    That (1) implies (2) is clear as $\Ker_\FF \cal O \subseteq \Lin_\FF \cal O$.
    
    To prove the other implication, assume (2) and let $\cal O \subseteq \A^I$.
    We proceed by induction on $|I|$.
    If $I = \emptyset$, then $\cal O$ must be the entire singleton $\A^\emptyset = \{ () \}$; 
    as $\Lin_\FF \cal O$ has no nontrivial subspaces (let alone finitely supported ones), it has length $1$.
    Now if $|I| \geq 1$, assemble all $|I|$ projection maps into a single map
    \begin{align*}
        \Lin_\FF \cal O &\to \bigoplus_{i \in I} \cal O|^{-i} \\
        v &\mapsto ( v|^{-i} )_{i \in I}
    \end{align*}
    whose kernel is precisely $\Ker_\FF \cal O$.
    We have
    \[
        \len( \Lin_\FF \cal O ) - \len( \Ker_\FF \cal O )
        \leq \sum_{i \in I} \len( \Lin_\FF \cal O|^{-i} )
    \]
    which shows that $\len( \Lin_\FF \cal O )$ is finite from the assumptions.
\end{proof}

Following the terminology used in \cite[Equation (4)]{BFKM24}, we shall call a vector from the projection kernel \emph{balanced}.
As we will see shortly, there exist balanced vectors other than $0$.

\subsection{Cogs}
From now on we will use a lightweight notation for combining tuples of atoms: for disjoint indexing sets $I$ and $J$, if $a\in\A^I$ and $b\in\A^{J}$ are both ordered, then $ab\in \A^{I\cup J}$ will denote their obvious combination. We will only use this notation if this $ab$ is ordered. For an obvious example, for any $S$-ordered $a\in\A^I$ and $J\subseteq I$, we have $a|^{I\setminus J}a|^J=a$.

\begin{definition}\label{def:duo}
    Let $\cal O \subseteq \A^I$ be an $S$-ordered orbit. 
    An \emph{$\cal O$-duo} $a \parallel b$ consists of tuples $a, b \in \cal O$ such that:
    \begin{enumerate}
        \item 
        $a_i < b_i$ for all $i\in I$;
        
        \item 
        $b_i < a_j$ for all $i<j\in I$;

        \item 
        for any binary $R$ in $\sigma_0$ (which we assumed to be irreflexive) and $i,j\in I$:
        \begin{align*}
            R(a_i,b_j) \iff R(&a_i,a_j) \\
            &\Updownarrow \text{ as $a, b \in \cal O$} \\
            R(b_i,a_j) \iff R(&b_i,b_j).
        \end{align*}
     \end{enumerate}
\end{definition}
\begin{remark}\label{rem:duo}
Conditions (1) and (2) specify a total order on the $2|I|$ atoms in a duo.
Moreover, thanks to irreflexivity, 
each $a_i$ is unrelated to its counterpart $b_i$.
Further, given any $J \subseteq I$, the combined tuple $a|^{I\setminus J} b|^J$ satisfies the same relations as $a$ (and $b$), so it lies in $\cal O$. In particular, taking $J=\{i\}$, there is an automorphism $\pi_i$ that sends $a_i$ to $b_i$ and fixes all the other elements of $a$, $b$ and $S$.
\end{remark}

For the special case of the ordered atoms (Ex.~\ref{ex:order-atoms} \&~\ref{ex:geneircally-ordered-equality}) the following construction was studied in~\cite[p.~11]{BFKM24}, and it already appeared earlier in~\cite[p.~125]{Gray97} under the name of ``polytabloids''.

\begin{definition}
    Given a $\cal O$-duo $a \parallel b$, the corresponding \emph{$\cal O$-cog} is the vector
    \[
        a \between b =
        \sum_{J \subseteq I} (-1)^{|J|}
        (a|^{I \setminus J} b|^{J})
    \]
    in $\Lin_\FF \cal O$.
    The linear span of all $\cal O$-cogs is denoted by $\Cog_\FF \cal O$.
\end{definition}
Note that, for a fixed $S$-ordered orbit $\cal O$, all $\cal O$-duos (hence all $\cal O$-cogs) are in the same $\Aut(\A/S)$-equivariant orbit.
As a result, $\Cog_\FF \cal O$ is an $\Aut(\A/S)$-equivariant subspace of $\Lin_\FF \cal O$ and it is generated by any single cog.

\begin{claim}\label{claim:cogs-are-balanced}
    $\Cog_\FF \cal O\subseteq \Ker_\FF \cal O$.
\end{claim}
\begin{proof}
    Let $\cal O \subseteq \A^I$, let $a\parallel b$ be an $\cal O$-duo, and take any $i \in I$.
    The subsets of $I$ come in pairs of $J$ and $J \cup \{i\}$, where $J\subseteq I \setminus \{i\}$.
    The two tuples $a|^{I\setminus J} b|^{J}$ and $a|^{I \setminus (J \cup \{i\})}b|^{J \cup \{i\}}$ are present in 
    $a\between b$ with the opposite coefficients, and they differ only on the $i$-th entry.
    Therefore they cancel out under $(-)|^{-i}$; hence $(a \between b )|^{-i} = 0$.
\end{proof}

So far, we have not shown that $\cal O$-duos and $\cal O$-cogs even exist in general.
Let us rectify this by showing that they can be found in all but one $\Aut(\A/S)$-equivariant subspaces of $\Lin_\FF{\cal O}$.


\subsection{Finding cogs}\label{subsec:finding-cogs}
We begin with a technical lemma, which combines the free amalgamation in $\Cclass_0$ and the generic order of $\A$.
\begin{lemma}\label{lem:free-fresh}
    Let $X, Y, \{z\} \subseteq \A$ be disjoint and finite.
    Then there is an automorphism $\tau \in \Aut(\A)$ such that
    \begin{enumerate}
        \item $\tau$ fixes every $x \in X$;
        \item $\tau(z)$ is unrelated to all $y \in Y$ and to $z$;
        \item $\tau(z) > z$.
    \end{enumerate}
\end{lemma}
\begin{proof}
    Form the free amalgam of structures in ${\Cclass}_0$:
% https://q.uiver.app/#q=WzAsNCxbMCwxLCJYIl0sWzEsMiwiWCBcXGN1cCBcXHt6XFx9Il0sWzEsMCwiWCBcXGN1cCBZIFxcY3VwIFxce3pcXH0iXSxbMiwxLCJYIFxcY3VwIFkgXFxjdXAgXFx7eiwgeidcXH0iXSxbMCwxXSxbMCwyXSxbMSwzLCJ4IFxcaW4gWCBcXG1hcHN0byB4LFxcIHogXFxtYXBzdG8geiciLDEseyJzdHlsZSI6eyJib2R5Ijp7Im5hbWUiOiJkb3R0ZWQifX19XSxbMiwzLCJcXHN1YnNldGVxIiwxLHsic3R5bGUiOnsiYm9keSI6eyJuYW1lIjoiZG90dGVkIn19fV1d
\[\begin{tikzcd}
	& {X \cup Y \cup \{z\}} \\
	X && {X \cup Y \cup \{z, z'\}} \\
	& {X \cup \{z\}}
	\arrow["\subseteq"{description}, dotted, from=1-2, to=2-3]
	\arrow[from=2-1, to=1-2]
	\arrow[from=2-1, to=3-2]
	\arrow["{x \in X \mapsto x,\ z \mapsto z'}"{description}, dotted, from=3-2, to=2-3]
\end{tikzcd}\]
    so that no element of $Y \cup \{z\}$ is related to $z'$.
    To make $X \cup Y \cup \{z, z'\}$ an $\sigma$-structure, 
    inherit the order on $X \cup Y \cup \{z\}$ from $\A$,
    and declare that $z < z'$, as well as $z' < a$ whenever $z<a$ for $a\in X \cup Y$. This makes the above a diagram of embeddings in the presence of the order.
    By homogeneity, $X \cup Y \cup \{z, z'\}$ embeds into $\A$ via some $f$ which is the identity on $X \cup Y \cup \{z\}$;
    again by homogeneity, we may extend the embedding \[ 
        x \in X \mapsto x, \quad z \mapsto f(z')
    \] to some automorphism $\tau$ that satisfies (1), (2), and (3).
\end{proof}

\begin{lemma}\label{lem:cog-fresh-single}
    Suppose $a\parallel b$ is an $\cal O$-duo, where $\cal O \subseteq \A^I$ is $S$-ordered.
    Given $z \in S$, 
    \begin{itemize}
        \item write $S' = S \setminus \{z\}$;
        \item let $j \not\in I$ be such that $a z \in \A^{I \cup \{j\}}$ --- thus ${\cal O'}=\Aut(\A)_{(S')} \cdot az \subseteq \A^{I \cup \{j\}}$ --- is $S'$-ordered;
        \item let $X \subseteq \A$ be any finite set containing $\{a_i, b_i \mid i \in I\} \cup S'$ but not $z$.
 %       \item let $Y \subseteq \A$ be any finite set disjoint from $X \cup \{z\}$;
    \end{itemize}
    Denote $z'=\tau(z)$, where $\tau \in \Aut(\A/X)$ is afforded by Lemma~\ref{lem:free-fresh} (with an arbitrary $Y$). Then $az \parallel bz'$ is an $\cal O'$-duo.
\end{lemma}
\begin{proof}
    First, notice that $bz'\in {\cal O}'$ and that we have the required order relations with $z$ and $z'$. The remaining condition (3) of Definition~\ref{def:duo}, for any binary $R$ in $\sigma_0$, splits into the following cases (and their symmetric counterparts):
\begin{itemize}
\item $R(a_i,b_{i'})\iff R(a_i,a_{i'})$ since $a \parallel b$ is an $\cal O$-duo;
\item $R(a_i,z')\iff R(a_i,z)$ since $\tau$ is an automorphism that fixes all $a_i$;
\item $R(a_i,z)\iff R(b_i,z)$ since $a, b \in \cal O$ and $z\in S$;
\item $R(z,z')$ and $R(z,z)$ are both false: $z'$ is unrelated to $z$ by Lemma~\ref{lem:free-fresh}, and $R$ is irreflexive. \qedhere
\end{itemize}
\end{proof}

Starting from an empty duo, we may apply Lemma~\ref{lem:cog-fresh-single} inductively and obtain:
\begin{lemma}\label{lem:cog-fresh-full}
    Let $\cal O \subseteq \A^I$ be an $S$-ordered orbit.
    Then any $a \in \cal O$ can be extended to an ${\cal O}$-duo $a\parallel b$.
\end{lemma}
% \begin{proof}
%     Enumerate the indices of $I$ as $i_1, \dots, i_d$.
%     Suppose that we have found $b_{i_1}, \dots, b_{i_k}$ such that 
%     \[
%         a|^{\{i_1, \dots, i_k\}} \parallel (i_1 \mapsto b_{i_1}, \dots, i_k \mapsto b_{i_k})
%     \] 
%     is a duo for $\cal O_k = \Aut(\A)_{(S \cup \{a_{i_{k+1}}, \dots, a_{i_d}\})} \cdot a|^{\{i_1, \dots, i_k\}}$ 
%     --- note that $() \parallel ()$ is certainly a duo for $\cal O_0$ at the start.
%     If $k < d$, putting $z = a_{i_{k+1}}$, $X = \{a_{i_1}, b_{i_1}, \dots, a_{i_k}, b_{i_k}\} \cup S \cup \{a_{i_{k+2}}, \dots, a_{i_d}\}$, and $Y = \emptyset$, 
%     Lemma~\ref{lem:cog-fresh-single} yields an atom $b_{i_{k+1}}$ that makes
%     \[
%         a|^{\{i_1, \dots, i_k, i_{k+1}\}} \parallel (i_1 \mapsto b_{i_1}, \dots, i_k \mapsto b_{i_k}, i_{k+1} \mapsto b_{i_{k+1}})
%     \] a duo for $\cal O_{k+1}$.
%     For $k=d$ this gives the desired duo for $\cal O_d = \cal O$.
% \end{proof}

As some $a \in \cal O$ always exists, it follows that Claim~\ref{claim:cogs-are-balanced} was not vacuous:
we now know $\Cog_\FF{\cal O}$, and hence $\Ker_\FF{\cal O}$, is nontrivial --- but barely so.
Indeed, as the result below shows, $\Cog_\FF{\cal O}$ admits no nontrivial $\Aut(\A/S)$-equivariant subspaces.
\begin{theorem} \label{thm:cogs-arise-everywhere}
    Any nontrivial $\Aut(\A/S)$-equivariant subspace $V\subseteq \Lin_\FF \cal O$ contains $\Cog_\FF \cal O$.
\end{theorem}
\begin{proof}
    Pick any $v \in V$ and $a \in \cal O$ with $v(a)\neq 0$;
    it is enough to show that $V$ contains $a \between b$ for some $b\in \cal O$.
    Define:
    \[
        S' = S \cup \{c_i \mid v(c) \neq 0, i \in I\} \setminus \{a_i \mid i \in I\} \supseteq S
    \]
    and put $\cal O' = \Aut(\A)_{(S')} \cdot a \subseteq \cal O$ --- then $\cal O'$ is $S'$-ordered.
    By Lemma~\ref{lem:cog-fresh-full}, we can find $b \in \cal O'$ such that $a \parallel b$ is an $\cal O'$-duo and \emph{a fortiori} an $\cal O$-duo.
    Take the automorphisms $\pi_{i_1}, \dots, \pi_{i_d}$ from Remark~\ref{rem:duo}, where $i_1, \dots, i_d$ enumerate $I$.
    Now define $v^{(0)} = v$ and \[
        v^{(k)} = v^{(k-1)} - \pi_{i_k} v^{(k-1)}.
    \]
    Then each $v^{(k)}$ is in $V$.
   By induction on $k$, we have:
    \[
        v^{(k)} = \sum_{c \in C_k} \sum_{J \subseteq \{i_1, \dots, i_k\}} (-1)^{|J|} v(c) \left(\prod_{j \in J} \pi_j c\right),
    \]
    where
    \[C_k = \{ c \mid v(c) \neq 0, \{c_{i_1}, \dots, c_{i_k}, \dots, c_{i_d}\} \supseteq \{a_{i_1}, \dots, a_{i_k}\} \}.
    \]
    But $\{c_{i_1}, \dots, c_{i_d}\} \supseteq \{a_{i_1}, \dots, a_{i_d}\}$ means that $c = a$, so $C_d=\{a\}$ and $\frac{1}{v(a)}v^{(d)}=a\between b$.
\end{proof}

\begin{corollary}\label{cor:cog-simple}
    $\Cog_\FF \cal O$ has length $1$.
\end{corollary}

\subsection{Spanning by cogs}
In light of Claims~\ref{claim:reduction-to-one-orbit},~\ref{claim:reduction-to-kernel},~\ref{claim:cogs-are-balanced}, and Corollary~\ref{cor:cog-simple}, for the finite length property it is enough to prove that $\Ker_\FF \cal O \subseteq \Cog_\FF \cal O$. In words, we need to show that every balanced vector in $\Lin_\FF{\cal O}$ is a linear combination of $\cal O$-cogs.
Before we show a proof of this (stated as Theorem~\ref{thm:cog-span-generally}), let us illustrate its key ideas on an example.

\begin{example}
Let $\A_0$ be the universal triangle-free (undirected) graph, and $\A$ its totally ordered version. Consider nine atoms $\{a,\ldots,i\}$, ordered by~$<$ alphabetically, with the edge relation as shown here:
\[
\rotatebox{2}{
\xymatrix@R=6pt@C=15pt{
\rotatebox{-2}{$h$} & & & & & & \rotatebox{-2}{$i$} \\
\\
& & & \rotatebox{-2}{$g$}\ar@{-}[uulll]\ar@{-}[uurrr] \\
& \rotatebox{-2}{$e$} & & & & \rotatebox{-2}{$f$} \\
& & \rotatebox{-2}{$c$}\ar@{-}[ul]\ar@{-}[uur] & & \rotatebox{-2}{$d$}\ar@{-}[uul]\ar@{-}[ur] \\
\rotatebox{-2}{$a$}\ar@{-}[rrrrrr]\ar@{-}[uuuuu]\ar@{-}[uur] & & & & & & \rotatebox{-2}{$b$}\ar@{-}[uul]\ar@{-}[uuuuu]
}}\]
This graph is drawn so that the total order of the atoms corresponds to the vertical order.

Putting $S=\emptyset$ and $|I|=2$, let ${\cal O}$ be the ordered orbit of pairs of atoms which are adjacent. Consider the following vector:
\[
v = ah - ae + ce - cg + dg - df + bf - bi + gi - gh \in \Lin_\FF{\cal O}.
\]
This can be graphically presented as the following graph:
\[\rotatebox{2}{
\xymatrix@R=6pt@C=15pt{
\rotatebox{-2}{$h$} & & & & & & \rotatebox{-2}{$i$} \\
\\
& & & \rotatebox{-2}{$g$}\ar@[blue][uulll]\ar@[red][uurrr] \\
& \rotatebox{-2}{$e$} & & & & \rotatebox{-2}{$f$} \\
& & \rotatebox{-2}{$c$}\ar@[red][ul]\ar@[blue][uur] & & \rotatebox{-2}{$d$}\ar@[red][uul]\ar@[blue][ur] \\
\rotatebox{-2}{$a$}\ar@{-}[rrrrrr]\ar@[red][uuuuu]\ar@[blue][uur] & & & & & & \rotatebox{-2}{$b$}\ar@[red][uul]\ar@[blue][uuuuu]
}}
\]
where edges with coefficient $+1$ are marked as red, and with $-1$ as blue. The arrows on the chosen edges remind us that the pairs in $\cal O$ are ordered, but this is mere decoration: the definition of $\cal O$ means that all arrows must point upwards.

Note that $v$ is balanced. Graphically, this means that every atom has as many outgoing red edges as outgoing blue edges, and as many incoming red edges as incoming blue edges.

It is easy to draw $\cal O$-cogs in this way. Assuming some additional atom $z>h$ which is adjacent to $a$ and $g$ but not to $h$, the $\cal O$-cog $ah\between gz$ can be drawn as:
\[\rotatebox{5}{
\xymatrix{
\rotatebox{-5}{$h$} & \rotatebox{-5}{$z$} \\
\rotatebox{-5}{$a$}\ar@[red][u]\ar@[blue][ur] & \rotatebox{-5}{$g$}\ar@[blue][ul]\ar@[red][u]  
}}
\]

We would like to present $v$ as a sum of such ${\cal O}$-cogs. Some additional atoms must be used for that, as no four atoms among the original nine form an $\cal O$-duo.
(This is not the case if $\A$ is the ordered atoms: see the proof of \cite[Claim~4.7]{BFKM24}.)
It would be very convenient to introduce a single new atom $z$ to form all the $\cal O$-duos that we will use. 
We can naïvely require $z$ to be:
\begin{itemize}
    \item larger than every atom in $v$,
    \item adjacent to every atom that is a source of a directed edge in $v$ (equivalently: that occurs as the first component of a pair in $v$), and
    \item not adjacent to any atom that is a target of a directed edge in $v$ (equivalently: that occurs as the second component of a pair in $v$).
\end{itemize}
However, such a $z$ does not exist in the triangle-free graph $\A$. There are two problems:
\begin{itemize}
\item The atom $g$ occurs both as the first and as the second component in pairs present in $v$. Our specification of whether $z$ is adjacent to $g$ is therefore inconsistent.
\item Atoms $a$ and $b$ both occur as first components in $v$, and they are adjacent in $\A$. As a result, an atom $z$ as prescribed would create a triangle $abz$ in $\A$, which is forbidden.
\end{itemize}
We resolve such {\em conflicts} by considering auxiliary atoms $g'>g$ and $b'>b$, with just enough edges to make $gh\parallel g'i$ and $bf\parallel b'i$ valid ${\cal O}$-duos. Specifically, we postulate edges $E(g', h)$, $E(g', i)$, $E(b', f)$, $E(b', i)$ and no more. Such atoms exist by Lemma~\ref{lem:free-fresh}. We then define:
\[
v' = v + (gh \between g'i) - (bf \between b'i)
\]
which can be drawn as:
\[
\rotatebox{2}{
\xymatrix@R=6pt@C=15pt{
\rotatebox{-2}{$h$} & & & & & & \rotatebox{-2}{$i$} \\
\\
& & & \rotatebox{-2}{$g$}\ar@{-}[uulll]\ar@{-}[uurrr] & \rotatebox{-2}{$g'$}\ar@[blue]@/_2ex/[uullll]\ar@[red]@/_1ex/[uurr] \\
& \rotatebox{-2}{$e$} & & & & \rotatebox{-2}{$f$} \\
& & \rotatebox{-2}{$c$}\ar@[red][ul]\ar@[blue][uur] & & \rotatebox{-2}{$d$}\ar@[red][uul]\ar@[blue][ur] \\
\rotatebox{-2}{$a$}\ar@{-}[rrrrrr]\ar@[red][uuuuu]\ar@[blue][uur] & & & & & & \rotatebox{-2}{$b$}\ar@{-}[uul]\ar@{-}[uuuuu] & \rotatebox{-2}{$b'$}\ar@[red]@/_1ex/[uull]\ar@[blue]@/_1ex/[uuuuul]
}}
\]
Now an atom $z$ as postulated above does not create any triangles:
\[
\rotatebox{2}{
\xymatrix@R=6pt@C=15pt{
& & & \rotatebox{-2}{$z$}\ar@{-}@/_4ex/[llldddddd]\ar@{-}[lddddd]\ar@{-}[rddddd]\ar@{-}@/_-0.5ex/[rddd]\ar@{-}@/_-3ex/[rrrrdddddd] \\
\rotatebox{-2}{$h$} & & & & & & \rotatebox{-2}{$i$} \\
\\
& & & \rotatebox{-2}{$g$}\ar@{-}[uulll]\ar@{-}[uurrr] & \rotatebox{-2}{$g'$}\ar@[blue]@/_2ex/[uullll]\ar@[red]@/_1ex/[uurr] \\
& \rotatebox{-2}{$e$} & & & & \rotatebox{-2}{$f$} \\
& & \rotatebox{-2}{$c$}\ar@[red][ul]\ar@[blue][uur] & & \rotatebox{-2}{$d$}\ar@[red][uul]\ar@[blue][ur] \\
\rotatebox{-2}{$a$}\ar@{-}[rrrrrr]\ar@[red][uuuuu]\ar@[blue][uur] & & & & & & \rotatebox{-2}{$b$}\ar@{-}[uul]\ar@{-}[uuuuu] & \rotatebox{-2}{$b'$}\ar@[red]@/_1ex/[uull]\ar@[blue]@/_1ex/[uuuuul]
}}
\]
and it is easy to calculate:
\[
v' = (ah \between g'z)  - (ae \between cz) - (cg \between dz) + (b'f \between dz) -(b'i \between g'z)
\]
which presents $v = v' - (gh \between g'i) + (bf \between b'i)$ as a linear combination of $\cal O$-cogs.


\textcolor{red}{
It is worth noting that, unlike with the atom $z$, we do not invent new coefficients here: for instance,
\begin{itemize}
    \item the $-1$ coefficient of $gh \between g'i$ comes from $v(gh) = -1$;
    \item the $-1$ coefficient of $b'i \between g'z$ comes from $v'(b' i) = -1$, which  in turn comes from $v(b i) = -1$.
    \hfill $\blacktriangleleft$
\end{itemize}}
\end{example}

Inspired by the example, we shall now assert:

\begin{theorem}\label{thm:cog-span-generally}
    Any $v \in \Ker_\FF \cal O$ can be written as
    \[
        v = \sum_{a \parallel b} \lambda_{a \parallel b} \cdot a \between b
    \]
    with $\lambda_{a \parallel b} \in \FF$.
\end{theorem}

Leaving the proof to Section~\ref{sec:appendix-cogs} of the appendix, 
we have --- as discussed at start of this section --- established Theorem~\ref{thm:ordered-free-amalg-has-finite-length}.


\subsection{All those equivariant subspaces}

\begin{theorem}
    Fix $d \in \{1, 2, \dots\}$. There exists a finite family of linear maps $\set{\restriction_i}_{i \in I}$, with each one of the form 
    \begin{align*}
     \restriction_i : \Lin_\FF \A^d \to \Lin_{\FF^{n_i}} \cal O_i,
    \end{align*}
    where $\cal O_i$ is an orbit of $\A^e$ with $e < d$, and $n_i \in \set{1,2,\ldots}$, with the following property.  Every equivariant subspace of $\Lin_\FF \A^d$ is equal to
    \begin{equation}\label{eq:equiv-subsp}
        \{v \in \Lin_\FF \A^d \mid \forall i, \forall a \in \cal{O}_i : v{\restriction_i}(a) \in \EE_i \}
    \end{equation}
    where $\EE_i \subseteq F^{n_i}$ are finite-dimensional subspaces for $i = 0, 1, \dots, t$.
    More precisely: these spaces are defined, given an equivariant subspace $W \subseteq \Lin_\FF \A^d$, 
    by \[
        \EE_i = \{w {\restriction_i}(b) \mid w \in W, b \in \cal O_i \}.
    \]
\end{theorem}

\textcolor{red}{$\EE$ vs $\FF$. Thm: These ideas are in some capacity present in \cite[Cor.~3.17]{Gray97}, \cite{HofmanRozycki_2022}, \cite{ghosh2023orbit}, and in Arka's notes. Cor: this tightens the bound in \cite{BFKM24}.}


\begin{example}
    $\A$ is the ordered atoms, 
    $d = 1$.

    \begin{align*}
        v {\restriction_1} (a) &= v(a)
        v {\restriction_2} () &= \sum_{-} v(-) \\
    \end{align*}    
\end{example}

\begin{example}
    $\A$ is the ordered atoms, $d = 2$.

\begin{align*}
    v {\restriction_1} (a < b) &= ( v(a, b),\ v(b, a) ); \\
    v {\restriction_2} (c) &= (\begin{aligned}
        &\sum_{{-} < c} v({-}, c), \sum_{c < {\sim}} v(c, {\sim}), v(c, c), \\
        &\sum_{{-} < c} v({-}, c), \sum_{c < {\sim}} v({\sim}, c)
    \end{aligned}); \\
    v {\restriction_3} () &= ( \sum_{{-} < {\sim}} v({-}, {\sim}), \sum_{{-} < {\sim}} v({\sim}, {-}), \sum_{=} v(=, =)).
\end{align*}
\end{example}

\begin{corollary}
    The length of $Lin_\FF \cal O$, where $\cal O \subseteq \A^d$ is an ordered orbit, is precisely $2^d$.
\end{corollary}

\begin{example}
    continuing from above,
    \begin{align*}
        V_0 &= \{0\} \\
        V_1 &= V_0 + \langle (5, 9) - (5.01, 9) - (5, 9.1) + (5.01, 9.1) \rangle \\
        V_2 &= V_1 + \langle (5, 9) - (5.01, 9) \rangle \\
        V_3 &= V_2 + \langle (5, 9) - (5, 9.1) \rangle \\
        V_4 &= V_3 + \langle (5, 9) \rangle        
    \end{align*}    
\end{example}
