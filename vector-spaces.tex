\section{Orbit-finitely spanned vector spaces}\label{sec:spaces}
We now introduce the main topic of this paper, which is orbit-finitely spanned  vector spaces. We begin with the special case of spaces that have an orbit-finite basis; this special case has an elementary definition, and yet it will be the  relevant case for almost all results of this paper. 

\begin{definition}[Orbit-finite basis]\label{def:orbit-finite-basis}
    A vector space with an orbit-finite basis is any space of the form $\Lin_\field X$ (where $X$ is an orbit-finite set and $\field$ a field), i.e.~the space of finite formal linear combinations of elements in $X$.
\end{definition}

This definition has two important parameters: the field $\field$ and the oligomorphic structure $\A$ over which $X$ is an orbit-finite set.
The spaces defined here have two kinds of structure: that of a vector space, and the action of the automorphism group of $\A$. We will be interested in subsets that preserve both kinds of structure, i.e.~they are closed under taking linear combinations, and applying automorphisms of $\A$. Such subsets are called \emph{equivariant subspaces}. 


\begin{myexample}\label{ex:length-one-dim-one}
    Let $\A$ be the equality atoms and let $\field$ be any field. As explained in~\cite[Ex.~4.2]{BFKM24}, this corresponding vector space with atoms $\Lin_\field \A$
    has only three equivariant subspaces: the zero subspace, the full space, and  subspace which consists of those vectors where the coefficients add up to zero. 
\end{myexample}

 Unfortunately, there is a price to pay for the elementary character of Definition~\ref{def:orbit-finite-basis}, which is the  failure of certain closure  properties. In particular, the spaces are not closed under taking equivariant subspaces, or quotients under such spaces. The problem with subspaces  is apparent already in Example~\ref{ex:length-one-dim-one},  since the unique nontrivial subspace of $\Lin_\field \A$ does not have any equivariant basis, regardless of the underlying field~\cite[Ex.~6.1]{BFKM24}. For this reason, we use a more general notion of vector space, which is presented below, in a style that emphasises the group action, as was done in Definition~\ref{def:orbit-finite-set-alternative}.

\begin{definition}\label{def:orbit-finitely-spanned-abstract}
    [Orbit-finitely spanned vector space, abstractly] An \emph{orbit-finitely spanned vector space} over an oligomorphic structure  $\A$ is a vector space  equipped with an action of  automorphisms of $\A$ such that: 
    \begin{enumerate}
        \item \label{item:equivariance-of-vector-space-structure} vector addition and scalar multiplication are equivariant;\footnote{Equivalently, the group action $v \mapsto \pi(v)$ is linear.}
        \item every vector is supported by a finite set of atoms; 
        \item \label{item:orbit-finite-spanning-subset} the space is spanned by some subset that  is orbit-finite.
    \end{enumerate}
\end{definition}

In~\eqref{item:equivariance-of-vector-space-structure}, by equivariance of scalar multiplication we mean that for every field element $\lambda$, the operation $v \mapsto \lambda v$ is equivariant. The above definition is easily seen to be closed under taking quotients by equivariant subspaces. Closure under taking equivariant subspaces is less obvious: one could imagine that condition~\eqref{item:orbit-finite-spanning-subset} above is violated by moving to an equivariant subspace. It turns out that closure under equivariant subspaces is intimately related to the ascending chain property discussed in the introduction:

\begin{theorem}\label{thm:ACP}
    For any field $\field$ and oligomorphic structure $\A$, the following conditions are equivalent: 
    \begin{enumerate}
        \item orbit-finitely spanned vector spaces are closed under taking equivariant subspaces; 
        \item for every $d \in \set{0,1,\ldots}$,  the vector space 
        $
        \Lin_\field \A^d
        $ does not have any infinite ascending chain of equivariant subspaces;
        \item for every orbit-finitely spanned vector space $V$, there are no infinite ascending chains of equivariant subspaces in $V$.
    \end{enumerate}
    Furthermore, if these conditions hold, then  every orbit-finitely spanned vector space is isomorphic to one of the form $V/U$, where $V \subseteq U$ are equivariant subspaces of $\Lin_\field \A^d$ for some $d \in \set{0,1,\ldots}$.
\end{theorem}


We say that an atom structure $\A$ has the \emph{ascending chain property} over a field $\field$ if any of the equivalent conditions in Theorem~\ref{thm:ACP} are satisfied. One interpretation of the theorem is that   the ascending chain property is necessary for the theory of vector spaces to be well-behaved.  In particular, thanks to the ``furthermore'' part, we get a similar result to the equivalence of Definitions~\ref{def:orbit-finite-set} and~\ref{def:orbit-finite-set-alternative} in the previous section, i.e.~a concrete characterisation of the orbit-finitely spanned vector spaces that can be used in algorithms (provided we know how to represent equivariant subspaces --- see Section~\ref{sec:equivariant-subspaces}). 


We are therefore interested in atom structures that have the ascending chain property. As it turns out, the techniques used in this paper will yield a stronger property, namely a finite bound on the length of chains: 




\begin{definition}
    [Finite length property]
    \label{def:finite-length-property}
    An oligomorphic structure $\A$ has the \emph{finite length property over a field $\field$} if for every orbit-finite set $X$ over $\A$, there is a finite upper bound on the length of chains of equivariant subspaces 
    of $\Lin_\field X$.%
    \footnote{The supremum of the chain lengths is called the \emph{length} of $\Lin_\FF X$, which is a standard notion in algebra. In Appendix~\ref{sec:appendix-modules} we describe a few properties of length that we will use, and we also supply a proof of Theorem~\ref{thm:ACP}.}
\end{definition}

In light of Theorem~\ref{thm:ACP}, we could have used $\A^d$ instead of $X$ in the above theorem.
As mentioned in the introduction, the finite length property can be strictly stronger than the ascending chain property, as witnessed by the vector atoms from Example~\ref{ex:vector-space-atoms}.  (Note that when we talk about the vector atoms, there  two fields involved, namely the finite field used to define $\A$, and the field $\field$ used to define the vector space with atoms. In the counterexample, both fields are the two-element field.)
The finite length property was studied in~\cite{BFKM24}, where it was shown that the equality atoms (Example~\ref{ex:equality-atoms}) and the ordered atoms (Example~\ref{ex:order-atoms}) have this property over any field. 

The main contribution of this paper is to establish the finite length property for more structures. We will use two different techniques for this purpose.






% \begin{myexample}
%     Assume that the atoms are the Rado graph.
%     \begin{center}
%         (todo: give an interesting example of a weighted automaton over the Rado graph)
%     \end{center}
% %      Consider the function
% %     \begin{align*}
% %     \A^* \to \Lin_{\Q} \A.
% %     \end{align*}
% %     A word $w \in \A^*$ can be seen as a vertex weighted graph, where the vertices are the letters of the word, weighted by the number of occurrences. 
% %     \begin{align*}
% %     a_1 \cdots a_n \quad \mapsto \quad \sum_{i \in \set{1,\ldots,n}} \lambda_i a_i,
% %     \end{align*}
% %     where the coefficient $\lambda_i$ is the sum 
% % \begin{align*}
% % |\setbuild{i \in \set{1,\ldots,n}}{$a_j \to  a_i$ is an edge}
% % \end{align*}
% \end{myexample}

