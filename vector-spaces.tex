\section{Orbit-finitely spanned vector spaces}
In this section, we introduce the main topic of this paper, which is orbit-finitely spanned  vector spaces. We begin with the special case of spaces that have an orbit-finite basis; this special case will be the relevant one for almost all results of this paper. 

\begin{definition}[Orbit-finite basis]
    A vector space with an orbit-finite basis is any space of the form $\Lin_\field X$, where $X$ is an orbit-finite set, $\field$ is a field, and $\Lin_\field X$ describes the vector space of finite formal linear combinations of elements in $X$.
\end{definition}

Note that there is an implicit parameter in the above definition, which is the oligomorphic structure over which $X$ is an orbit-finite set.
The spaces described in the above definition have  two kinds of structure: the structure of a vector space, and the action of the automorphism group of $\A$. We will be interested in subsets that preserve both kinds of structure, i.e.~they are closed under taking linear combinations, and applying automorphisms of $\A$. Such subsets are called \emph{equivariant subspaces}. 


\begin{example}
    Let $\A$ be the structure with equality only, let $\field$ be any field, and let the atom dimension be $d=1$. As explained in~\cite[Example 4.2]{BFKM24}, this corresponding vector space with atoms
    \begin{align*}
    \Lin_\field \A
    \end{align*}
    has only three equivariant subspaces: the zero subspace, the full space, and  subspace which consists of vectors where all coefficients sum to zero. 
\end{example}

Spaces  with an orbit-finite basis have a nice elementary definition. Unfortunately, the lack desired closure properties, in particular they are not closed under taking equivariant subspaces. This is apparent already in the above example, since the unique nontrivial subspace of $\Lin_\field A$ does not have a basis that is equivariant. For this reason, one uses a more general notion of vector space, which is the orbit-finitely spanned vector spaces defined below.

\begin{definition}
    [Orbit-finitely spanned vector space] An \emph{orbit-finitely spanned vector space} over an atom structure  $\A$ is a vector space  equipped with an action of atom automorphisms such that: 
    \begin{enumerate}
        \item The vector space structure is equivariant, which means that vector addition $(v,w) \mapsto v+w$ is equivariant, and likewise for  scalar multiplication $v \mapsto \lambda v$ for every field element $\lambda$.
        \item Every vector has finite support, which means that for every $v \in V$ there is some finite set $S \subseteq \A$ such that if an atom automorphism fixes all atoms in $S$, then it also fixes $v$. 
        \item The space is spanned by some subset that  is orbit-finite.
    \end{enumerate}
\end{definition}




\begin{definition}[Orbit-finitely spanned]
    An orbit-finitely spanned vector space is any vector space of the form $V/U$, where $V \subseteq U \subseteq \Lin_\field X$ are equivariant subspaces for some orbit-finite $X$.
\end{definition}


The spaces in the above definition are called orbit-finitely spanned because they have an orbit-finite spanning subset: initially the subset is a basis, but 

$\Lin_\field X$




\begin{lemma}
    Let $\A$ be an oligomorphic structure such that 
    \begin{align*}
    \Lin_k \A^d
    \end{align*}
     has the ascending chain property for every $d \in \set{0,1,\ldots}$. Then orbit-finitely spanned vector spaces are closed under taking equivariant subspaces, and quotients under such subspaces. Furthermore, every orbit-finitely spanned vector space is of the form $V/U$ where $U \subseteq V$ are equivariant subspaces of $\Lin_k \A^d$ for some 
\end{lemma}

The main topic of this paper is the study of strictly growing chains of equivariant spaces 
\begin{align*}
 V_0 \subset V_1  \subset \cdots \subset V_n.
\end{align*}
The number $n$ of strict inclusions is called the \emph{length} of such a chain. 






\begin{definition}
    [Finite length property]
    \label{def:finite-length-property}
    Let $\A$ be an oligomorphic structure. 
    We say that  $\A$ has the \emph{finite length property over a field $\field$} if for \textcolor{red}{every orbit-finite set $X$ over $\A$}, there is some finite upper bound on the length of chains of equivariant subspaces 
    in  $\Lin_\field X$.
\end{definition}

The finite length property was studied in~\cite{BFKM24}, where it was shown that the equality atoms (Example~\ref{ex:equality-atoms}) and the order atoms (Example~\ref{ex:order-atoms}) have this property over any field. The main contribution of this paper is to  establish the finite length property for more structures, including the Rado graph (Example~\ref{ex:rado-graph}) and the vector space atoms (Example~\ref{ex:vector-space-atoms}). For the Rado graph, we will not make any assumptions on the field, but for the case of the vector space structure from Example~\ref{ex:vector-space-atoms}, we will need to assume that the field $\field$ in $\Lin_\field \A^d$ has characteristic zero. (There are two fields involved here, namely the finite field used to define $\A$, and the field $\field$ used to define the vector space with atoms. The assumption is on the latter field.)


\paragraph*{Weighted automata.} One of the main motivations behind the introduction of vector spaces with atoms in~\cite{BFKM24} was to study orbit-finite weighted automata. An orbit-finite weighted automaton is defined like a deterministic orbit-finite automaton, except that the state space $Q$ is a vector space of orbit-finite dimension, the initial state is some equivariant vector in this space, and the transition function 
\begin{align*}
\delta : Q \times \Sigma \to Q
\end{align*}
is an equivariant function such that for every input letter $a \in \Sigma$, the function $\delta(-,a) : Q \to Q$ is a  linear map, and the final function is an equivariant linear map from $Q$ to some output vector space (typically, the field itself).



\begin{example}
    Assume that the atoms are the Rado graph.
    \begin{center}
        (todo: give an interesting example of a weighted automaton over the Rado graph)
    \end{center}
%      Consider the function
%     \begin{align*}
%     \A^* \to \Lin_{\Q} \A.
%     \end{align*}
%     A word $w \in \A^*$ can be seen as a vertex weighted graph, where the vertices are the letters of the word, weighted by the number of occurrences. 
%     \begin{align*}
%     a_1 \cdots a_n \quad \mapsto \quad \sum_{i \in \set{1,\ldots,n}} \lambda_i a_i,
%     \end{align*}
%     where the coefficient $\lambda_i$ is the sum 
% \begin{align*}
% |\setbuild{i \in \set{1,\ldots,n}}{$a_j \to  a_i$ is an edge}
% \end{align*}
\end{example}

