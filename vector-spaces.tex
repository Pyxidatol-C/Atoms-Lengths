\section{Orbit-finitely spanned vector spaces}
In this section, we introduce the main topic of this paper, which is orbit-finitely spanned  vector spaces. We begin with the special case of spaces that have an orbit-finite basis; this special case has an elementary definition, and yet it will be the  relevant case for almost all results of this paper. 

\begin{definition}[Orbit-finite basis]\label{def:orbit-finite-basis}
    A vector space with an orbit-finite basis is any space of the form $\Lin_\field X$, where $X$ is an orbit-finite set, $\field$ is a field, and $\Lin_\field X$ describes the vector space of finite formal linear combinations of elements in $X$.
\end{definition}

There are two  important parameters in the above definition, namely the field $\field$ and  the oligomorphic structure over which $X$ is an orbit-finite set.
The spaces described in the above definition have  two kinds of structure: the structure of a vector space, and the action of the automorphism group of $\A$. We will be interested in subsets that preserve both kinds of structure, i.e.~they are closed under taking linear combinations, and applying automorphisms of $\A$. Such subsets are called \emph{equivariant subspaces}. 


\begin{example}\label{ex:length-one-dim-one}
    Let $\A$ be the structure with equality only, let $\field$ be any field, and let the atom dimension be $d=1$. As explained in~\cite[Example 4.2]{BFKM24}, this corresponding vector space with atoms
    \begin{align*}
    \Lin_\field \A
    \end{align*}
    has only three equivariant subspaces: the zero subspace, the full space, and  subspace which consists of vectors where all coefficients sum to zero. 
\end{example}

 Unfortunately, there is a price to pay for the elementary character of Definition~\ref{def:orbit-finite-basis}, which is the  failure of certain closure  properties. In particular, the spaces are not closed under taking equivariant subspaces, or quotients under such spaces. The problem with subspaces  is apparent already in Example~\ref{ex:length-one-dim-one},  since the unique nontrivial subspace of $\Lin_\field A$ does not have a basis that is equivariant, regardless of the underlying field~\cite[Example 6.1]{BFKM24}. For this reason, we use a more general notion of vector space, which is presented below, in a style that emphasis the group action, similarly to Definition~\ref{def:orbit-finite-set-alternative}.

\begin{definition}
    [Orbit-finitely spanned vector space, abstractly] An \emph{orbit-finitely spanned vector space} over an oligomorphic structure  $\A$ is a vector space  equipped with an action of  automorphisms of $\A$ such that: 
    \begin{enumerate}
        \item \label{item:equivariance-of-vector-space-structure} vector addition and scalar multiplication are equivariant;
        \item vector is supported by a finite set of atoms; 
        \item \label{item:orbit-finite-spanning-subset} the space is spanned by some subset that  is orbit-finite.
    \end{enumerate}
\end{definition}

In item~\ref{item:equivariance-of-vector-space-structure} above, by equivariance of scalar multiplication we mean that for every field element $\lambda$, the operation $v \mapsto \lambda v$ is equivariant. The above definition is easily seen to be closed under taking quotients (when talking about quotients, we mean quotienting under an equivariant subspace). However, closure under equivariant subspaces is not obvious;  one could imagine that condition~\ref{item:orbit-finite-spanning-subset} in the above definition, which talks about orbit-finite spanning subsets, is violated by restricting to an equivariant subspace. It turns out that closure under equivariant subspaces is intimately related to the ascending chain property that was discussed in the introduction, and which is one of the main topics of this paper. 


\begin{theorem}
    Fix some field $\field$ and an oligomorphic structure $\A$. Then the following conditions are equivalent: 
    \begin{enumerate}
        \item orbit-finitely spanned vector spaces are closed under taking equivariant subspaces; 
        \item for every $d \in \set{0,1,\ldots}$,  the vector space 
        $
        \Lin_\field \A^d
        $ does not have any infinite ascending chain of equivariant subspaces;
        \item for every orbit-finitely spanned vector space $V$, there are no infinite ascending chains of equivariant subspaces in $V$.
    \end{enumerate}
\end{theorem}
% \begin{proof}
%     We first show that item (2) is equivalent to 
%     \begin{itemize}
%         \item[(3)] no orbit-finitely spanned vector space has an infinite ascending chain of equivariant subspaces. 
%     \end{itemize}
%     Condition (3) in turn implies (1), since an orbit-finite spanning set can be constructed by a greedy procedure, which keeps on adding orbits of vectors until a spanning set is reached.  On the other hand, (2) implies (1). 

%     By definition, every orbit-finitely spanned vector space can be obtained as a quotient 
%     \begin{align*}
%     (\Lin_\field \A^d)/U
%     \end{align*}
%     for some 
%     Suppose that condition (2) is violated. Then 
% \end{proof}

We say that an atom structure $\A$ has the \emph{ascending chain property} over a field $\field$ if any of the equivalent conditions in the above theorem are satisfied. One interpretation of the above theorem is that it shows that  the ascending chain property is necessary for the theory of vector spaces to be well-behaved. 
Another advantage of the ascending chain condition, which further reinforces its relevance,  is that it gives us an alternative and more concrete description of the vector spaces, as stated in the theorem. When talking about an isomorphism in the theorem, we mean a bijective equivariant linear map 

\begin{theorem}\label{thm:two-definitions-of-orbit-finitely-spanned-equivalent}
    Assume that $\A$ has the ascending chain property over some field. Then every orbit-finitely spanned vector space is isomorphic to one of the form $V/U$, where $V \subseteq U$ are equivariant subspaces of $\Lin_\field \A^d$ for some $d \in \set{0,1,\ldots}$.
\end{theorem}

The theorem can be seen as vector space version of the result from the previous section, which said that  the two definitions of orbit-finitess coincide. The theorem further reinforces the relevance of the ascending chain condition.

\begin{definition}[Orbit-finitely spanned]
    An orbit-finitely spanned vector space is any vector space of the form 
\end{definition}


The spaces in the above definition are called orbit-finitely spanned because they have an orbit-finite spanning subset: initially the subset is a basis, but 

$\Lin_\field X$




\begin{lemma}
    Let $\A$ be an oligomorphic structure such that 
    \begin{align*}
    \Lin_k \A^d
    \end{align*}
     has the ascending chain property for every $d \in \set{0,1,\ldots}$. Then orbit-finitely spanned vector spaces are closed under taking equivariant subspaces, and quotients under such subspaces. Furthermore, every orbit-finitely spanned vector space is of the form $V/U$ where $U \subseteq V$ are equivariant subspaces of $\Lin_k \A^d$ for some 
\end{lemma}

The main topic of this paper is the study of strictly growing chains of equivariant spaces 
\begin{align*}
 V_0 \subset V_1  \subset \cdots \subset V_n.
\end{align*}
The number $n$ of strict inclusions is called the \emph{length} of such a chain. 






\begin{definition}
    [Finite length property]
    \label{def:finite-length-property}
    Let $\A$ be an oligomorphic structure. 
    We say that  $\A$ has the \emph{finite length property over a field $\field$} if for \textcolor{red}{every orbit-finite set $X$ over $\A$}, there is some finite upper bound on the length of chains of equivariant subspaces 
    in  $\Lin_\field X$.
\end{definition}

The finite length property was studied in~\cite{BFKM24}, where it was shown that the equality atoms (Example~\ref{ex:equality-atoms}) and the order atoms (Example~\ref{ex:order-atoms}) have this property over any field. The main contribution of this paper is to  establish the finite length property for more structures, including the Rado graph (Example~\ref{ex:rado-graph}) and the vector space atoms (Example~\ref{ex:vector-space-atoms}). For the Rado graph, we will not make any assumptions on the field, but for the case of the vector space structure from Example~\ref{ex:vector-space-atoms}, we will need to assume that the field $\field$ in $\Lin_\field \A^d$ has characteristic zero. (There are two fields involved here, namely the finite field used to define $\A$, and the field $\field$ used to define the vector space with atoms. The assumption is on the latter field.)


\paragraph*{Weighted automata.} One of the main motivations behind the introduction of vector spaces with atoms in~\cite{BFKM24} was to study orbit-finite weighted automata. An orbit-finite weighted automaton is defined like a deterministic orbit-finite automaton, except that the state space $Q$ is a vector space of orbit-finite dimension, the initial state is some equivariant vector in this space, and the transition function 
\begin{align*}
\delta : Q \times \Sigma \to Q
\end{align*}
is an equivariant function such that for every input letter $a \in \Sigma$, the function $\delta(-,a) : Q \to Q$ is a  linear map, and the final function is an equivariant linear map from $Q$ to some output vector space (typically, the field itself).



\begin{example}
    Assume that the atoms are the Rado graph.
    \begin{center}
        (todo: give an interesting example of a weighted automaton over the Rado graph)
    \end{center}
%      Consider the function
%     \begin{align*}
%     \A^* \to \Lin_{\Q} \A.
%     \end{align*}
%     A word $w \in \A^*$ can be seen as a vertex weighted graph, where the vertices are the letters of the word, weighted by the number of occurrences. 
%     \begin{align*}
%     a_1 \cdots a_n \quad \mapsto \quad \sum_{i \in \set{1,\ldots,n}} \lambda_i a_i,
%     \end{align*}
%     where the coefficient $\lambda_i$ is the sum 
% \begin{align*}
% |\setbuild{i \in \set{1,\ldots,n}}{$a_j \to  a_i$ is an edge}
% \end{align*}
\end{example}

