\section{Vector spaces of orbit-finite dimension}
In this section, we introduce the main topic of this paper, which is vector spaces that are built on an orbit-finite basis.
For a field $\field$ and a set $X$, we write 
\begin{align*}
\Lin_\field X
\end{align*}
for the free vector space over $\field$ with basis $X$. This space consists of finite formal linear combinations of elements in $X$, with coefficients from the field $\field$.  We will be interested in the case when the basis $X$ is an orbit-finite set over some oligomorphic structure $\A$, typically $X = \A^d$. In such a case, the vector space is said to have \emph{orbit-finite dimension}. This space has two kinds of structure: the structure of a vector space, and the action of the automorphism group of $\A$. We will be interested in subsets that preserve both kinds of structure, i.e.~they are closed under taking linear combinations, and applying automorphisms of $\A$. Such subsets are called \emph{equivariant subspaces}. The main topic of this paper is the study of strictly growing chains of equivariant spaces 
\begin{align*}
 V_0 \subset V_1  \subset \cdots \subset V_n.
\end{align*}
The number $n$ of strict inclusions is called the \emph{length} of such a chain. 


\begin{example}
    Let $\A$ be the structure with equality only, let $\field$ be any field, and let the atom dimension be $d=1$. As explained in~\cite[Example 4.2]{BFKM24}, this corresponding vector space with atoms
    \begin{align*}
    \Lin_\field \A
    \end{align*}
    has only three equivariant subspaces: the zero subspace, the whole space, and the subspace which consists of vectors where all coefficients sum to zero. These three spaces form a chain of length two.
\end{example}



\begin{definition}
    [Finite length property]
    \label{def:finite-length-property}
    Let $\A$ be an oligomorphic structure. 
    We say that  $\A$ has the \emph{finite length property over a field $\field$} if for every orbit-finite set $X$ over $\A$, there is some finite upper bound on the length of chains of equivariant subspaces 
    in  $\Lin_\field X$.
\end{definition}

The finite length property was studied in~\cite{BFKM24}, where it was shown that the equality atoms (Example~\ref{ex:equality-atoms}) and the order atoms (Example~\ref{ex:order-atoms}) have this property over any field. The main contribution of this paper is to  establish the finite length property for more structures, including the Rado graph (Example~\ref{ex:rado-graph}) and the vector space atoms (Example~\ref{ex:vector-space-atoms}). For the Rado graph, we will not make any assumptions on the field, but for the case of the vector space structure from Example~\ref{ex:vector-space-atoms}, we will need to assume that the field $\field$ in $\Lin_\field \A^d$ has characteristic zero. (There are two fields involved here, namely the finite field used to define $\A$, and the field $\field$ used to define the vector space with atoms. The assumption is on the latter field.)


\paragraph*{Weighted automata.} One of the main motivations behind the introduction of vector spaces with atoms in~\cite{BFKM24} was to study orbit-finite weighted automata. An orbit-finite weighted automaton is defined like a deterministic orbit-finite automaton, except that the state space $Q$ is a vector space of orbit-finite dimension, the initial state is some equivariant vector in this space, and the transition function 
\begin{align*}
\delta : Q \times \Sigma \to Q
\end{align*}
is an equivariant function such that for every input letter $a \in \Sigma$, the function $\delta(-,a) : Q \to Q$ is a  linear map, and the final function is an equivariant linear map from $Q$ to some output vector space (typically, the field itself).



\begin{example}
    Assume that the atoms are the Rado graph.
    \begin{center}
        (todo: give an interesting example of a weighted automaton over the Rado graph)
    \end{center}
%      Consider the function
%     \begin{align*}
%     \A^* \to \Lin_{\Q} \A.
%     \end{align*}
%     A word $w \in \A^*$ can be seen as a vertex weighted graph, where the vertices are the letters of the word, weighted by the number of occurrences. 
%     \begin{align*}
%     a_1 \cdots a_n \quad \mapsto \quad \sum_{i \in \set{1,\ldots,n}} \lambda_i a_i,
%     \end{align*}
%     where the coefficient $\lambda_i$ is the sum 
% \begin{align*}
% |\setbuild{i \in \set{1,\ldots,n}}{$a_j \to  a_i$ is an edge}
% \end{align*}
\end{example}

