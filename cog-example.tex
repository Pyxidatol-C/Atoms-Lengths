\documentclass[12pt]{article}

\usepackage[all]{xy}
\xyoption{rotate,color}
\usepackage{amsmath}
\usepackage{amssymb}
\usepackage{amsthm}
\usepackage{graphicx,xcolor}
\usepackage{xspace}

\newcommand{\bl}[1]{{\color{blue}{#1}}}
\newcommand{\rd}[1]{{\color{red}{#1}}}

\newcommand{\N}{\mathbb N}
\renewcommand{\P}{\mathcal P}
\newcommand{\Pfin}{{\mathcal P}_{\mathrm{fin}}}
\newcommand{\Pf}{\P_{\textrm{fin}}}
\newcommand{\true}{\top}
\newcommand{\false}{\bot}
\newcommand{\Set}{\textbf{Set}}
\newcommand{\FinSet}{\textbf{FinSet}}
\newcommand{\id}{\mathrm{id}}

\newcommand{\Q}{\mathbb{Q}}
\newcommand{\atoms}{\mathbb{A}}
\newcommand{\Aut}{\mathrm{Aut}}
\newcommand{\End}{\mathrm{End}}
\newcommand{\Bij}{\mathrm{Bij}}
\newcommand{\Sym}{\mathrm{Sym}}
\newcommand{\Alt}{\mathrm{Alt}}
\newcommand{\supp}{\mathrm{supp}}
\newcommand{\tuple}[1]{\langle {#1} \rangle}
\newcommand{\oleq}{\preccurlyeq}

\newcommand{\M}{\mathbf{M}}

\newcommand{\Lin}{\operatorname{Lin}}

\begin{document}

In this example $\atoms_0$ is the universal triangle-free (undirected) graph, and $\atoms$ its totally ordered version. Consider nine atoms $\{a,\ldots,i\}$, ordered alphabetically by $<$, with the edge relation as shown here:
\[
\rotatebox{2}{
\xymatrix@R=10pt{
\rotatebox{-2}{$h$} & & & & & & \rotatebox{-2}{$i$} \\
\\
& & & \rotatebox{-2}{$g$}\ar@{-}[uulll]\ar@{-}[uurrr] \\
& \rotatebox{-2}{$e$} & & & & \rotatebox{-2}{$f$} \\
& & \rotatebox{-2}{$c$}\ar@{-}[ul]\ar@{-}[uur] & & \rotatebox{-2}{$d$}\ar@{-}[uul]\ar@{-}[ur] \\
\rotatebox{-2}{$a$}\ar@{-}[rrrrrr]\ar@{-}[uuuuu]\ar@{-}[uur] & & & & & & \rotatebox{-2}{$b$}\ar@{-}[uul]\ar@{-}[uuuuu]
}}\]
This graph is drawn so that the total order of the atoms corresponds to the vertical order.

Putting $S=\emptyset$ and $d=2$, let $\cal O$ be the $S$-ordered orbit of pairs of atoms which are connected by an edge. Working over an arbitrary field $\sf K$, consider the following vector:
\[
v_0 = ah - ae + ce - cg + dg - df + bf - bi + gi - gh \in \Lin_{\sf K}{\cal O}.
\]
This can be graphically presented as the following graph:
\[
v_0 = 
\vcenter{\rotatebox{2}{
\xymatrix@R=10pt{
\rotatebox{-2}{$h$} & & & & & & \rotatebox{-2}{$i$} \\
\\
& & & \rotatebox{-2}{$g$}\ar@[red][uulll]\ar@[blue][uurrr] \\
& \rotatebox{-2}{$e$} & & & & \rotatebox{-2}{$f$} \\
& & \rotatebox{-2}{$c$}\ar@[blue][ul]\ar@[red][uur] & & \rotatebox{-2}{$d$}\ar@[blue][uul]\ar@[red][ur] \\
\rotatebox{-2}{$a$}\ar@{-}[rrrrrr]\ar@[blue][uuuuu]\ar@[red][uur] & & & & & & \rotatebox{-2}{$b$}\ar@[blue][uul]\ar@[red][uuuuu]
}}}
\]
where edges with coefficient $+1$ are marked as blue, and with $-1$ as red. The arrows on the chosen edges remind us that the pairs in $\cal O$ are ordered, but this is mere decoration: the definition of $\cal O$ means that all arrows must point upwards.

Note that $v_0$ is balanced. Graphically, this means that every atom has as many outgoing red edges as outgoing blue edges, and as many incoming red edges as incoming blue edges.

It is easy to draw $\cal O$-cogs in this way. Assuming some additional atom $z>h$ which is connected by edges to $a$ and $g$, the $\cal O$-cog $ah\between gz$ can be drawn as:
\[
\xymatrix@R=10pt{
& & z \\
h \\
& & \rotatebox{-2}{$g$}\ar@[red][ull]\ar@[blue][uu]  \\
a\ar@[blue][uu]\ar@[red][uuurr]
}
\]

We would like to present $v_0$ as a sum of such ${\cal O}$-cogs. To this end, it would be very convenient to have an atom $z$, larger than every atom in $v_0$, and connected by edges to every atom which is a source of a directed edge in $v_0$ (equivalently: which occurs as the first component of a pair in $v_0$). However, such a $z$ does not exist in the triangle-free graph $\atoms$. There are two problems:
\begin{itemize}
\item The atom $g$ occurs both as the first and as the second component in pairs present in $v_0$. In particular, an atom $z$ as prescribed would create a triangle $dgz$ in $\atoms$, which is forbidden.
\item Atoms $a$ and $b$ both occur as first components in $v$, and they are connected by an edge in $\atoms$. As a result, an atom $z$ as prescribed would create a triangle $abz$ in $\atoms$.
\end{itemize}
We get rid of such {\em obstructions} by considering auxiliary atoms $g'>g$ and $b'>b$, with just enough edges to make $gh\parallel g'i$ and $bf\parallel b'i$ valid ${\cal O}$-duos. Specifically, we postulate edges $g'-h$, $g'-i$, $b'-f$ and $b'-i$ and no more. Such atoms exist by the homogeneity of $\atoms$. We then define:
\[
v_1 = v_0 - (gh \between g'i) - (bf \between b'i)
\]
which can be drawn as:
\[
v_1 = 
\vcenter{
\rotatebox{2}{
\xymatrix@R=10pt{
\rotatebox{-2}{$h$} & & & & & & \rotatebox{-2}{$i$} \\
\\
& & & \rotatebox{-2}{$g$}\ar@{-}[uulll]\ar@{-}[uurrr] & \rotatebox{-2}{$g'$}\ar@[red][uullll]\ar@[blue][uurr] \\
& \rotatebox{-2}{$e$} & & & & \rotatebox{-2}{$f$} \\
& & \rotatebox{-2}{$c$}\ar@[blue][ul]\ar@[red][uur] & & \rotatebox{-2}{$d$}\ar@[blue][uul]\ar@[red][ur] \\
\rotatebox{-2}{$a$}\ar@{-}[rrrrrr]\ar@[blue][uuuuu]\ar@[red][uur] & & & & & & \rotatebox{-2}{$b$}\ar@{-}[uul]\ar@{-}[uuuuu] & \rotatebox{-2}{$b'$}\ar@[blue][uull]\ar@[red][uuuuul]
}}}
\]
Now an atom $z$ as postulated above does not create any forbidden triangles:
\[
v_1 = 
\vcenter{
\rotatebox{2}{
\xymatrix@R=10pt{
& & & \rotatebox{-2}{$z$}\ar@{-}@/_4ex/[llldddddd]\ar@{-}[lddddd]\ar@{-}[rddddd]\ar@{-}[rddd]\ar@{-}[rrrrdddddd] \\
\rotatebox{-2}{$h$} & & & & & & \rotatebox{-2}{$i$} \\
\\
& & & \rotatebox{-2}{$g$}\ar@{-}[uulll]\ar@{-}[uurrr] & \rotatebox{-2}{$g'$}\ar@[red][uullll]\ar@[blue][uurr] \\
& \rotatebox{-2}{$e$} & & & & \rotatebox{-2}{$f$} \\
& & \rotatebox{-2}{$c$}\ar@[blue][ul]\ar@[red][uur] & & \rotatebox{-2}{$d$}\ar@[blue][uul]\ar@[red][ur] \\
\rotatebox{-2}{$a$}\ar@{-}[rrrrrr]\ar@[blue][uuuuu]\ar@[red][uur] & & & & & & \rotatebox{-2}{$b$}\ar@{-}[uul]\ar@{-}[uuuuu] & \rotatebox{-2}{$b'$}\ar@[blue][uull]\ar@[red][uuuuul]
}}}
\]
and we can calculate:

\[
v_2 = v_1 - (ah \between g'z) =
\vcenter{
\rotatebox{2}{
\xymatrix@R=10pt{
& & & \rotatebox{-2}{$z$}\ar@{<-}@[blue]@/_4ex/[llldddddd]\ar@{-}[lddddd]\ar@{-}[rddddd]\ar@{<-}@[red][rddd]\ar@{-}[rrrrdddddd] \\
\rotatebox{-2}{$h$} & & & & & & \rotatebox{-2}{$i$} \\
\\
& & & \rotatebox{-2}{$g$}\ar@{-}[uulll]\ar@{-}[uurrr] & \rotatebox{-2}{$g'$}\ar@{-}[uullll]\ar@[blue][uurr] \\
& \rotatebox{-2}{$e$} & & & & \rotatebox{-2}{$f$} \\
& & \rotatebox{-2}{$c$}\ar@[blue][ul]\ar@[red][uur] & & \rotatebox{-2}{$d$}\ar@[blue][uul]\ar@[red][ur] \\
\rotatebox{-2}{$a$}\ar@{-}[rrrrrr]\ar@{-}[uuuuu]\ar@{-}[uuuuu]\ar@[red][uur] & & & & & & \rotatebox{-2}{$b$}\ar@{-}[uul]\ar@{-}[uuuuu] & \rotatebox{-2}{$b'$}\ar@[blue][uull]\ar@[red][uuuuul]
}}}
\]

\[
v_3 = v_2 - (ae \between cz) =
\vcenter{
\rotatebox{2}{
\xymatrix@R=10pt{
& & & \rotatebox{-2}{$z$}\ar@{-}@/_4ex/[llldddddd]\ar@{<-}@[blue][lddddd]\ar@{-}[rddddd]\ar@{<-}@[red][rddd]\ar@{-}[rrrrdddddd] \\
\rotatebox{-2}{$h$} & & & & & & \rotatebox{-2}{$i$} \\
\\
& & & \rotatebox{-2}{$g$}\ar@{-}[uulll]\ar@{-}[uurrr] & \rotatebox{-2}{$g'$}\ar@{-}[uullll]\ar@[blue][uurr] \\
& \rotatebox{-2}{$e$} & & & & \rotatebox{-2}{$f$} \\
& & \rotatebox{-2}{$c$}\ar@{-}[ul]\ar@[red][uur] & & \rotatebox{-2}{$d$}\ar@[blue][uul]\ar@[red][ur] \\
\rotatebox{-2}{$a$}\ar@{-}[rrrrrr]\ar@{-}[uuuuu]\ar@{-}[uur] & & & & & & \rotatebox{-2}{$b$}\ar@{-}[uul]\ar@{-}[uuuuu] & \rotatebox{-2}{$b'$}\ar@[blue][uull]\ar@[red][uuuuul]
}}}
\]

\[
v_4 = v_3 + (cg \between dz) =
\vcenter{
\rotatebox{2}{
\xymatrix@R=10pt{
& & & \rotatebox{-2}{$z$}\ar@{-}@/_4ex/[llldddddd]\ar@{<-}@{-}[lddddd]\ar@{<-}@[blue][rddddd]\ar@{<-}@[red][rddd]\ar@{-}[rrrrdddddd] \\
\rotatebox{-2}{$h$} & & & & & & \rotatebox{-2}{$i$} \\
\\
& & & \rotatebox{-2}{$g$}\ar@{-}[uulll]\ar@{-}[uurrr] & \rotatebox{-2}{$g'$}\ar@{-}[uullll]\ar@[blue][uurr] \\
& \rotatebox{-2}{$e$} & & & & \rotatebox{-2}{$f$} \\
& & \rotatebox{-2}{$c$}\ar@{-}[ul]\ar@{-}[uur] & & \rotatebox{-2}{$d$}\ar@{-}[uul]\ar@[red][ur] \\
\rotatebox{-2}{$a$}\ar@{-}[rrrrrr]\ar@{-}[uuuuu]\ar@{-}[uur] & & & & & & \rotatebox{-2}{$b$}\ar@{-}[uul]\ar@{-}[uuuuu] & \rotatebox{-2}{$b'$}\ar@[blue][uull]\ar@[red][uuuuul]
}}}
\]

\[
v_5 = v_4 + (b'f \between dz) =
\vcenter{
\rotatebox{2}{ 
\xymatrix@R=10pt{
& & & \rotatebox{-2}{$z$}\ar@{-}@/_4ex/[llldddddd]\ar@{<-}@{-}[lddddd]\ar@{-}[rddddd]\ar@{<-}@[red][rddd]\ar@{<-}@[blue][rrrrdddddd] \\
\rotatebox{-2}{$h$} & & & & & & \rotatebox{-2}{$i$} \\
\\
& & & \rotatebox{-2}{$g$}\ar@{-}[uulll]\ar@{-}[uurrr] & \rotatebox{-2}{$g'$}\ar@{-}[uullll]\ar@[blue][uurr] \\
& \rotatebox{-2}{$e$} & & & & \rotatebox{-2}{$f$} \\
& & \rotatebox{-2}{$c$}\ar@{-}[ul]\ar@{-}[uur] & & \rotatebox{-2}{$d$}\ar@{-}[uul]\ar@{-}[ur] \\
\rotatebox{-2}{$a$}\ar@{-}[rrrrrr]\ar@{-}[uuuuu]\ar@{-}[uur] & & & & & & \rotatebox{-2}{$b$}\ar@{-}[uul]\ar@{-}[uuuuu] & \rotatebox{-2}{$b'$}\ar@{-}[uull]\ar@[red][uuuuul]
}}} 
\]
and finally $v_5 = -(b'i \between g'z)$. This presents $v_0$ as a sum of ${\cal O}$-cogs.


\end{document}







