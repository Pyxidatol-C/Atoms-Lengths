\section{Spanning by cogs: a proof}\label{sec:appendix-cogs}
Recall that setting of Section~\ref{sec:free-amalg} and the notations of Section~\ref{sec:cogs-turn}.
Consider an $S$-ordered orbit $\cal O \subseteq \A^I$.
The goal here is to prove Theorem~\ref{thm:cog-span-generally}.

\subsection{Subvectors, locations, conflicts}
We begin by introducing some additional terminology and notation. First, let us make explicit a view we have tacitly taken:
with $\cal O$ as a standard basis, a vector $v \in \Lin_\FF \cal O$ is just a finite set of pairs in $\FF \times \cal O$.
A \emph{subvector} of $v$ is a subset of these pairs. We write $\vsup{v}\subseteq {\cal O}$ for the set of tuples which are present in $v$. For a finite subset $T\subseteq\cal O$, we write $\sqrt T\subseteq\A$ for the set of atoms present anywhere in $T$.

For any $i \in I$ and $a \in \A$ (which is equal to $b_i$ for some $b \in \cal O$), we write 
\[
    \cal O^{i:a} = \{c \in \cal O \mid c_i = a\};
\]
this is an $\Aut(\A)_{(S a)}$-orbit (containing $b$), and its projection $\cal O^{i:a} |^{-i}$ is $Sa$-ordered.
For a vector $v \in \Lin_\FF \cal O$, by
\[
    v^{i:a} \in \Lin_\FF \cal O^{i:a}
\]
we mean the subvector of $v$ consisting of all pairs in $\FF \times \cal O^{i:a}$.

\begin{lemma}\label{lem:balanced-projected-subvector}
    Let $v \in \Lin_\FF \cal O$ be balanced. 
    Then any projected subvector $v^{i:a}|^{-i} \in \Lin_\FF \cal O^{i:a}|^{-i}$ is also balanced.
\end{lemma}
\begin{proof}
    Let $j \in I \setminus \{i\}$. 
    By assumption we have \[
        0 = v|^{-j} = \sum_a v^{i:a}|^{-j}.
    \]
    This sum is finite: it runs over those atoms $a$ that occur as the $i$-th entries in $\vsup{v}$.
    By looking at $i$-th entries, we see that each $v^{i:a}|^{-j}$ must be the zero vector.
    Hence so is $v^{i:a}|^{-j}|^{-i} = v^{i:a}|^{-i}|^{-j}$,
    which shows that $v^{i:a}|^{-i}$ is balanced.
\end{proof}

For a finite subset $T \subseteq \cal O$, a {\em location} in $T$ is a pair $(i,a)\in I\times\A$ such that $a=c_i$ for some $c\in T$. 
Note that for any fixed $i,j\in I$, for all $c\in\cal O$ the atoms $c_i$ and $c_j$ are related in the same way in $\A_0$ (i.e., with respect to equality and binary relations in $\sigma_0$). We say that two locations $(i,a)$ and $(j,b)$ in $T$ are in:
\begin{itemize}
\item an {\em equational conflict}, if $i\neq j$ but $a=b$, and
\item a {\em relational conflict}, if $a$ and $b$ are related in $\A_0$ but not in the same way as $c_i$ and $c_j$ for $c\in\cal O$.
\end{itemize}
(A situation where $a \neq b$ are not related by any relation in $\sigma_0$ at all does not constitute a conflict, even if $c_i$ and $c_j$ are related.) 
Recalling that $a$ and $b$ are related if they are equal, an equational conflict is a special case of a relational one.
A location in a vector $v$ means a location in the set $[v]$.

The prototypical examples of vectors which are free from any conflicts are cogs (or any subvectors of cogs). Note that the locations in a cog $a^+ \between a^-$ are exactly those in $\{a^+ ,a^-\}$, and these have no conflicts if $a^+ \parallel a^-$ is a duo.

In the following proof we will often manipulate many duos and cogs at once, so we will benefit from a concise notation for them. 
An $\cal O$-duo will be denoted by a single letter as $a^\pm$; its constituent parts will then be denoted by $a^+$ and $a^-$, so that $a^\pm=a^+\parallel a^-$. Sets of duos will be denoted with capital letters such as $A^\pm$, and sometimes we will slightly abuse this notation and write $A^\pm$ to mean $\bigcup_{a^\pm\in A^\pm}\{a^+,a^-\}$,
$A^+$ to mean $\bigcup_{a^\pm\in A^\pm}\{a^+\}$, 
and $A^+ a$ to mean $\bigcup_{a^\pm\in A^\pm}\{a^+ a\}$.

\subsection{Conflict resolution lemmas}
The following sister lemmas, relying on free amalgamation as distilled in Lemma~\ref{lem:free-fresh}, show how to merge conflict-free subsets of $\cal O$ in a way that avoids introducing new conflicts. This will be useful in Sections~\ref{subsec:unobstructed} and~\ref{subsec:unambiguous}.

\begin{lemma}\label{lem:?}
    Let $K, V_0 \subseteq V$ be finite subsets of $\cal O$ such that both $V_0\cup K$ and $V$ are free from equational conflicts. Then there exists a $\pi\in\Aut(\A)$ that, while fixeing all atoms in $S$ and in $\sqrt{V_0}$, makes $V \cup \pi(K)$ free from equational conflicts.
\end{lemma}
\begin{proof}
    Fix $V_0, V$ and induct on the number of equationally conflicting locations in $V \cup K$. 
    Take any such locations $(i,a)$ and $(j,a)$, where $i \neq j$; without loss of generality $(i,a)$ is a location in $K$ and $(j,a)$ a location in $V$. 

    Suppose that $(i', a)$ were a location in $V_0$.
    Since there are no equational conflicts in $V_0\cup K$ or in $V$, we see that $i = i'$ and $i' = j$, which is impossible.
    So $a \not\in \sqrt{V_0}$.
    Also $a\not\in S$, as $\cal O$ is $S$-ordered.
    Put: 
    \[
        X = S \cup \sqrt{K \cup V} \setminus \{a\},
    \]
    and note that $X$ contains $S \cup \sqrt{V_0}$.
    Use Lemma~\ref{lem:free-fresh} (putting $z=a$ and $Y=\emptyset$) to obtain an automorphism $\pi \in \Aut(\A/X)$ such that $\pi(a) \not\in X \cup \{a\}$. 
    
    In the set $V\cup\pi(K)$, the conflicting location $(i,a)$ disappears and no new equational conflicts are created, so the number of equationally conflicting locations drops compared to $V\cup K$.
    Because $V_0 \cup \pi(K)$ is still conflict-free,
    the inductive hypothesis gives us some $\pi' \in \Aut(\A)_{(S \cup \sqrt{V_0})}$ such that $V \cup \pi' \pi(K)$ is free from equational conflicts.
\end{proof}

\begin{lemma}\label{lem:!}
    Let $K, V_0 \subseteq V$ be finite subsets of $\cal O$ such that both $V_0\cup K$ and $V$ are free from relational conflicts. Then there exists a $\pi\in\Aut(\A)$ that, while fixing all atoms in $S$ and in $\sqrt{V_0}$, makes $V \cup \pi(K)$ free from relational conflicts.
\end{lemma}
\begin{proof}
    By \autoref{lem:?} we may assume that $V \cup K$ is free from {\em equational} conflicts.  As before, fix $V_0, V$ and proceed by induction on the number of relationally conflicting locations in $V \cup K$.
    
    Let $(i, a)$ and $(j, b)$ be in a relational conflict; without loss of generality $(i, a)$ is a location in $K$ and $(j, b)$ in $V$. This is not an equational conflict, so $a\neq b$ (but possibly $i=j$). 
    
    Since there are no conflicts in $V_0\cup K$ or in $V$, we see that $(i, a)$ is not a location in $V$ and $(j, b)$ is not a location in $V_0 \cup K$
    --- i.e., since there are no equational conflicts, that $a\not\in \sqrt{V}$ and $b\not\in\sqrt{V_0\cup K}$.
    Also, $a,b\not\in S$.
    Let $Y$ consist of all the atoms $b$ that are in a relational conflict with $(i,a)$ in $V\cup K$; we have just shown that $Y$ does not contain $a$ and is disjoint with $S\cup\sqrt{V_0\cup K}$. Put:
    \[
        X = S \cup \sqrt{K \cup V} \setminus (Y \cup \{a\}).
    \]
    Then $X, Y, \{a\}$ are pairwise disjoint, and $X$ contains $S \cup \sqrt{V_0}$. It also contains all atoms in $\sqrt{K}$ except $a$.
   Using Lemma~\ref{lem:free-fresh}, find some $\pi\in \Aut(\A/X)$ such that $\pi(a) \not\in X \cup Y \cup \{a\}$ and $\pi(a)$ is not related to any atom in $Y$. In $V\cup\pi(K)$ the conflicting location $(i,a)$ disappears and no new conflicts are created, so the conclusion follows from the inductive hypothesis.
\end{proof}

\subsection{Conflict-free vectors}\label{subsec:unobstructed}
\begin{claim}\label{claim:!-free-decomposition}
    If $v \in \Ker_\FF \cal O$ is free from conflicts, then it can be written as a linear combination of $\cal O$-cogs:
    \[
        v = \sum_{a^\pm \in A^\pm} \lambda_{a^\pm} \cdot a^+ \between a^-
    \]
    with $\lambda_{a^\pm} \in \FF$, where moreover $\vsup{v} \cup A^\pm$ is free from conflicts.
\end{claim}
\begin{proof}
We proceed by induction on the dimension $|I|$, 
noting that when $I = \emptyset$ we just have $v = \lambda \cdot () = \lambda \cdot ( \between )$.

So suppose $I$ is non-empty; let $j \in I$ be the greatest element. 
Group the terms in $v$ by their greatest atom so that $v = v^1 + v^2 + \cdots + v^k$.
We now induct on $k$.
If $k \leq 1$, we are done: as $v|^{-j} = 0$ we must have $v = 0$ (and $k = 0$), so the empty sum will do.
Otherwise \[
    v = v^{j:a} + v^{j:b} + v'
\]
for some $a\neq b\in \A$. By Lemma~\ref{lem:balanced-projected-subvector}, $v^{j:a}|^{-j}$ is balanced, and it is conflict-free, as every location in it is also a location in $v$.
By the outer inductive hypothesis, we get \[
    v^{j:a} = (v^{j:a}|^{-j}) a = \sum_{a^\pm\in A^\pm} (\lambda_{a^\pm} \cdot a^+ \between a^-)a
\]
where $[v^{j:a}|^{-j}]\cup A^\pm$ is free from conflicts, which immediately implies that $[v^{j:a}]\cup A^\pm a$ is free from conflicts as well. 
Note that if a $\pi \in \Aut(\A/S)$ fixes every atom in $v^{j:a}$ --- in particular, $a$ ---  then
\[
    v^{j:a} = \pi(v^{j:a}) = \sum_{a^\pm \in A^\pm} \lambda_{a^\pm} \cdot \pi a^+ \between \pi a^-,
\]
so by Lemma~\ref{lem:!} (putting $K=A^\pm a$, $V_0=[v^{j:a}]$, and $V=[v]$) we may assume without loss of generality that
$[v]\cup A^\pm a$ is free from conflicts.

Similarly, we can write \[
    v^{j:b} = \sum_{b^\pm\in B^\pm} (\mu_{b^\pm} \cdot b^+ \between b^-)b
\]
and apply Lemma~\ref{lem:!} again (putting $K=B^\pm b$, $V_0=[v^{j:b}]$, and $V=[v]\cup A^\pm a$) to conclude that
\[
    \vsup{v} \cup A^\pm a \cup B^\pm b
\]
is free from conflicts.

We now invent a new element $z$, on which we impose the following relations with $S \cup \sqrt{A^\pm a \cup B^\pm b} \subseteq \A$: 
\begin{enumerate}
    \item $a, b < z$, and $z < s$ iff $a, b < s$ for any $s \in S$;
    
    \item for any unary relation $P \in \sigma_0$:
    \[
        P(z) \;:\Longleftrightarrow\; P(a) \iff P(b);
    \]
    \item for any binary relation $R \in \sigma_0$ and $s \in S$, $a^\pm \in A^\pm$, $b^\pm \in B^\pm$, $i \in I \setminus \{d\}$:
    \begin{itemize}
        \item $R(z, s) \;:\Longleftrightarrow\; R(a, s) \iff R(b, s)$,
        \item $R(z, a^+_i) \;:\Longleftrightarrow\; R(a, a^+_i)$,
        \item $R(z, b^+_i) \;:\Longleftrightarrow\; R(b, b^+_i)$,
        \item $R(z, a^-_i) \;:\Longleftrightarrow\; R(a, a^-_i)$,
        \item $R(z, b^-_i) \;:\Longleftrightarrow\; R(b, b^-_i)$,
        \item $R(z, a)$ and $R(z,b)$ are both false,
        \item and symmetrically for $R(-, z)$.
    \end{itemize}
    These are consistent as there are no equational conflicts.
    (For instance, if $a^+_i = b^-_{i'}$ then $i = i'$, and $R(a, a^+_i) \iff R(b, b^-_i)$ holds since $a^+ a$ and $b^- b$ are both in $\cal O$.)
\end{enumerate}
To see that the $\sigma$-structure $S \cup \sqrt{A^\pm a \cup B^\pm b} \cup \{z\}$ still embeds into $\A$, 
suppose towards a contradiction that it contains a forbidden $\sigma_0$-substructure $F$.
Then $F$ must contain $z$.
Since any two elements in $F$ are necessarily related, we must have $a, b \not\in F$.
Similarly, whenever $F$ contains an atom $x_i$ for any $x\in A^\pm\cup B^\pm$, it does not contain $y_i$ for any other $y\in A^\pm\cup B^\pm$.
It follows that, fixing any $a^\pm\in A^\pm$,
\[
    s \mapsto s,\quad x_i \mapsto a^+_i,\quad z \mapsto a
\]
defines an injective function $\phi : F \to \A_0$,
which is furthermore an embedding (we only need to check this for pairs!) because $A^\pm \cup B^\pm$ is conflict-free and any $x_i, y_{i'}$ for $i \neq i'$ are related.
This is a contradiction. We may therefore assume that $z \in \A$.

It is now routine to check that $a^+ a \parallel a^- z$ and $b^+ b \parallel b^- z$ are $\cal O$-duos for all $a^\pm \in A^\pm, b^\pm \in B^\pm$,
and that 
\[
A^+ a \cup A^- z \cup B^+ b \cup B^- z
\]
is free from conflicts.
From Lemma~\ref{lem:!} we may assume that 
\[
\vsup{v} \cup A^+ a \cup A^- z \cup B^+ b \cup B^- z
\] 
is also free from conflicts.
(Alternatively, we could have explicitly ensured this when defining $z$.)
Then the vector:
\begin{align*}
    v'' &= v
    - \sum_{a^\pm\in A^\pm} \lambda_{a^\pm} \cdot a^+ a \between a^- z 
    - \sum_{b^\pm\in B^\pm} \mu_{b^\pm} \cdot b^+ b \between b^- z  \\
    &= v^{j:a}|^{-j} z + v^{j:b}|^{-j} z + v',
\end{align*}
when grouped into subvectors by the largest atom in each term, has at least one fewer component than $v$.
By the inner inductive hypothesis, we may write
\[
    v'' = \sum_{c^\pm\in C^\pm} \kappa_{c^\pm} \cdot c^+ \between c^-
\]
with $\vsup{v''} \cup C^\pm$ conflict-free, and one
last application of Lemma~\ref{lem:!} allows us to conclude that
\[
    \vsup{v} \cup A^+ a \cup A^- z \cup B^+ b \cup B^- z \cup C^\pm
\]
is conflict-free as well.
We conclude that
\begin{align*}
    v = 
      \sum_{a^\pm\in A^\pm} \lambda_{a^\pm} \cdot a^+ a \between a^- z 
    + \sum_{b^\pm\in B^\pm} \mu_{b^\pm} \cdot b^+ b \between b^- z \\
    + \sum_{c^\pm\in C^\pm} \kappa_{c^\pm} \cdot c^+ \between c^-
\end{align*}
is a decomposition of $v$ into a linear combination of $\cal O$-cogs as required.
\end{proof}

\subsection{Vectors without equational conflicts}\label{subsec:unambiguous}
\begin{claim}\label{claim:?-free-decomposition}
     If $v \in \Ker_\FF \cal O$ has no equational conflicts, then it can be written as a linear combination of $\cal O$-cogs:
    \[
        v = \sum_{a^\pm \in A^\pm} \lambda_{a^\pm} \cdot a^+ \between a^-
    \]
    with $\lambda_{a^\pm} \in \FF$, where moreover $\vsup{v} \cup A^\pm$ has no equational conflicts.
\end{claim}
\begin{proof}
We proceed again by induction, first on $|I|$ then on the number of relational conflicts in $v$.
The outer base case $I = \emptyset$ is trivial 
--- we have $v = \lambda \cdot ( \between )$, and no conflicts arise
--- and the inner base case is just Claim~\ref{claim:!-free-decomposition}.

Suppose that a location $(i, a)$ is part of a relational conflict in $v$.
Since every location in $v^{i:a}|^{-i}$ is also a location in $v$, we know that $v^{i:a}|^{-i}$ has no equational conflicts, and by Lemma~\ref{lem:balanced-projected-subvector} it is balanced.
By the outer inductive hypothesis, we get:
\[
    v^{i:a} = (v^{i:a}|^{-i}) a = \sum_{a^\pm\in A^\pm} (\lambda_{a^\pm} \cdot a^+ \between a^-) a
\]
where $[v^{i:a}|^{-i}]\cup A^\pm$ has no equational conflicts, which immediately implies that $[v^{i:a}]\cup A^\pm a$ is free from equational conflicts as well. 
By Lemma~\ref{lem:?} (putting $K=A^\pm a$, $V_0=[v^{i:a}]$, and $V=[v]$) we may assume that
$[v]\cup A^\pm a$ has no equational conflicts.

Take any location $(j,b)$ which is a in a relational conflict with $(i,a)$ in $v$. Then $b\not\in\sqrt{A^\pm a}$. To see this, note that $b=a$ would imply $j=i$ (since $v$ has no equational conflicts), but that would mean no conflict. On the other hand, if $b = a^+_k$ for some $a^+ \in A^\pm$ and $k \in I \setminus \{i\}$ (the case of $a^-$ is identical) then $j=k$, but this is not a conflict either, since $a^+a\in\cal O$ and $a=(a^+a)_i$.
Also, $b \not\in S$.

   Let $Y$ consist of all the atoms $b$ that are in a relational conflict with $(i,a)$ in $V$. 
   As $a \not\in S$ and $a \not\in \sqrt{A^\pm}$, we have shown that
   \[
X = S \cup \sqrt{\vsup{v} \cup A^\pm} \setminus (Y \cup \{a\})   
\]
contains $S \cup \sqrt{A^\pm}$ and that $X, Y, \{a\}$ are pairwise disjoint.
Use Lemma~\ref{lem:free-fresh} (putting $z=a$) to find $\tau \in \Aut(\A/X)$ such that $\tau(a) \not\in X \cup Y \cup \{a\}$, is greater than $a$, and is not related to any of $Y \cup \{a\}$. Denote $a'=\tau(a)$.
Then, for any $a^\pm \in A^\pm$, it follows by Lemma~\ref{lem:cog-fresh-single} that $a^+ a \parallel a^- a'$ is an $\cal O$-duo. Moreover, 
$\vsup{v} \cup A^+ a \cup A^- a'$
has no equational conflicts, and the vector
\begin{align*}
    v' = v - \sum_{a^\pm\in A^\pm} \lambda_{a^\pm} \cdot a^+ a \between a^- a'
    &= v -  v^{i:a} + v^{i:a'}%\\
  %  &= v^{i : a_i \mapsto \tau(a_i)}    
\end{align*}
has strictly fewer relationally conflicting locations than $v$, as the location $(i,a)$ disappears from it.
The inner inductive hypothesis tells us 
that we may write
\[
v' = \sum_{b^\pm\in B^\pm} \mu_{B^\pm} \cdot b^+ \between b^-
\]
with $\vsup{v'} \cup B^\pm$ free from equational conflicts.

Since $\vsup{v'} \subseteq \vsup{v} \cup A^+ a \cup A^- a'$, Lemma~\ref{lem:?} allows us to assume that 
\[
    \vsup{v} \cup A^+ a \cup A^-a' \cup B^\pm
\]
is also free from equational conflicts. We conclude that 
\[
    v =\sum_{a^\pm\in A^\pm} \lambda_{a^\pm} \cdot a^+ a \between a^-a'+ \sum_{b^\pm\in B^\pm} \mu_{b^\pm} \cdot b^+ \between b^-
\]
as required.
\end{proof}

\subsection{Arbitrary vectors}
We restate and prove Theorem~\ref{thm:cog-span-generally}:
\begin{theorem}
    Any $v \in \Ker_\FF \cal O$ can be written as
    \[
        v = \sum_{a^\pm \in A^\pm} \lambda_{a^\pm} \cdot a^+ \between a^-
    \]
    with $\lambda_{a^\pm} \in \FF$.
\end{theorem}
\begin{proof}
This is similar to the proof of Claim~\ref{claim:?-free-decomposition}, but simpler.
We proceed again by induction, first on $|I|$ then on the number of equational conflicts in $v$.
The outer base case $I = \emptyset$ is trivial as before, and 
and the inner base case is Claim~\ref{claim:?-free-decomposition}.

Suppose that a location $(i, a)$ is part of an equational conflict in $v$. By the outer inductive hypothesis, we get:
\[
    v^{i:a}
    = (v^{i:a}|^{-i}) a 
    = \sum_{a^\pm\in A^\pm} (\lambda_{a^\pm} \cdot a^+ \between a^-) a.
\]
Then neither $S$ nor $\sqrt{A^\pm}$ contains $a$,
so 
\[
X = S \cup \sqrt{\vsup{v} \cup A^\pm} \setminus \{a\}
\]
contains $S \cup \sqrt{A^\pm}$.
Using Lemma~\ref{lem:free-fresh} (putting $z=a$ and $Y=\emptyset$), find $\pi \in \Aut(\A/X)$ such that $\pi(a)$ is not in $X$, is greater than $a$, and is otherwise unrelated to $a$. Denote $a'=\tau(a)$. 
Then, for any $a^\pm \in A^\pm$, it follos by Lemma~\ref{lem:cog-fresh-single} that $a^+ a \parallel a^- a'$ is an $\cal O$-duo. Moreover, 
the vector
\begin{align*}
    v' = v - \sum_{a^\pm\in A^\pm} \lambda_{a^\pm} \cdot a^+ a \between a^- a'
    &= v -  v^{i:a} + v^{i:a'} 
\end{align*}
has strictly fewer equationally conflicting locations than $v$, as the location $(i,a)$ disappears from it.
It follows from the inner inductive hypothesis that we can write
\[
    v' = \sum_{b^\pm \in B^\pm} \mu_{b^\pm} \cdot b^+ \between b^-,
\]
which gives a decomposition of $v = \sum_{a^\pm \in A^\pm} \lambda_{a^\pm} \cdot a^+ a \between a^- a'$ as required.
\end{proof}