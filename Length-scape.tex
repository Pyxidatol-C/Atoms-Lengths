\section{The length-scape}
With Theorems~\ref{thm:weak-smooth-approximation-finite-length} and \ref{thm:ordered-free-amalg-has-finite-length} and their corollaries, we have extended the finite length property far beyond equality and ordered atoms, the two examples considered in~\cite{BFKM24}.
One quick way to produce even more examples is by considering an \emph{interpretation} in $\A$, 
which consists of an orbit-finite set $\B$ over $\A$ equipped with $\Aut(\A)$-equivariant relations.
($\B$ is called a \emph{first-order reduct} of $\A$ if its underlying set is just $\A$.)

\begin{lemma}\label{lem:interpretations-preserve-fin-len}
    If $\A$ has the finite length property over a field $\FF$, then so does any interpretation $\B$ in $\A$.
\end{lemma}
\begin{proof}
It is a standard result in model theory (see e.g.~\cite[Chap.~5,7]{hodges1993model}) that $\B$ is oligomorphic if $\A$ is.
%
    Now, consider a chain $V_0 \subset V_1 \subset \cdots \subset V_l \subseteq \Lin_\FF \B^d$ of $\Aut(\B)$-equivariant subspaces. 
    Then each $V_i$ is also $\Aut(\A)$-equivariant,
    and the finite length property of $\A$ yields an upper bound on $l$.
\end{proof}

As an example application:

\begin{theorem}
    Every countable homogeneous undirected graph has the finite length property over any field.
\end{theorem}

\begin{proof}
    We rely on the intricate classification result of Lachlan and Woodrow~\cite{LachlanWoodrow_80} which says that any countable homogeneous graph, or its complement,\footnote{Since a graph $\A$ and its complement share the same automorphism group, 
    when studying the length of $\Lin_\FF \A^d$ we only need to examine the former graph.} is isomorphic to: 
    \begin{enumerate}
        \item the Rado graph;
        \item the $K_n$-free Henson graph, where $n \geq 3$; or
        \item $m$ disjoint copies of the complete graph $K_n$, where at least one of $m$ and $n$ is infinite.
    \end{enumerate}
    
    For (1) and (2), the finite length property follows directly from Corollary~\ref{cor:free-finite-length}.
    The case (3) follows from Lemma~\ref{lem:interpretations-preserve-fin-len}, as those graphs admit interpretations in the equality atoms.
    \end{proof}

% Another recipe for producing more structures from $\A$ is by considering $(\A, {=}c_1, \dots, {=}c_k)$, where $c_1, \dots, c_k \in \A$ are seen as constants.
% Because $\A^{d+k}$ is orbit-finite for every $d$,
% this expansion is again oligomorphic if $\A$ is so. But we do not know whether the finite length property of the expansion can be deduced from the property for $\A$ in general. We only know a few positive examples:
% \begin{itemize}
%     \item 
%     Expansions of $(\N, =)$ or $(\Q, \leq)$ by finitely many constants are interpretable in the original structure~\cite[Lem~2.22]{BodirskyBodorMarimon_25},
%     so they have the finite length property by Theorem~\ref{thm:ordered-free-amalg-has-finite-length} in light of 
%     Lemma~\ref{lem:interpretations-preserve-fin-len}. (See also~\cite[Thm.~4.10]{BFKM24}.)
%     \item 
%     On the other hand, for $\A$ the Rado graph, it was shown in \cite{no-constant-interpreted-in-Rado} that $(\A, {=}c)$ does not interpret in $\A$ for any $c \in \A$. 
%     Instead, we explicitly took the constant into account when we established the finite length property in Corollary~\ref{cor:free-finite-length}.
% \end{itemize}


Let us summarise the scope of our results. In the cog-based approach of Sections~\ref{sec:free-amalg}--\ref{sec:equivariant-subspaces}, we cover the generically ordered Fraïssé limit of any free amalgamation class over a finite relational vocabulary of relational symbols of arity at most $2$. We do not know how to drop the arity restriction from our proofs, so it is interesting that recently, using a different approach, Evans proved~\cite[Prop.~3.10]{Evans_pm1} that $\Lin \A^2$ has finite length for a vocabulary of any arity. We do not know how to combine these results. 

Our main cog-based result is Theorem~\ref{thm:cog-span-generally}, 
which allows us in Section~\ref{sec:equivariant-subspaces} to describe all equivariant subspaces in any orbit-finite-dimensional vector space, and to deduce the finite length property with a tight bound.
However, as we saw in Section~\ref{sec:duals}, these structures can give rise to ill-behaved models of computation:
over the Rado and Henson graphs, orbit-finitely spanned sets do not admit orbit-finitely spanned function spaces, and weighted register automata are not closed under reversal.

The cog-less approach to oligomorphically approximated structures in Section~\ref{sec:characteristic-zero} is almost complementary.
The finite length property is proved there only over fields of characteristic $0$; indeed some structures covered in that setting, notably vector atoms (Ex.~\ref{ex:vector-space-atoms}), do not have the finite length property over finite fields~\cite[Sec.~4.4]{BFKM24}.
But the proof here is quick and elegant, and covers two important subclasses of oligomorphic structures. 
The first is $\omega$-stable structures, which admit orbit-finitely spanned dual spaces according to~\cite[Thm.~3.7]{Przybyłek_2024}.
The second is weakly Lie coordinatisable structures, which are thoroughly studied in \cite{CherlinHrushovski_03} and provide a rich variety of examples.

A few major questions remain open:
\begin{itemize}
\item \cite[Q.~2]{CaminaEvans_91} Does every oligomorphic structure have the ascending chain property? 
\item ~\cite[Q.~1.4]{Evans_pm1} Does every structure homogeneous over a finite vocabulary have the finite length property?
\item Does every oligomorphic structure have the finite length property over fields of characteristic $0$?
\end{itemize}
We do not even know whether the ascending chain property or the finite length property hold for some concrete and well-studied Fraïssé limits, notably the universal partial order or the countable atomless Boolean algebra.

