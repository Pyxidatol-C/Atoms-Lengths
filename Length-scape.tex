\section{The length-scape}
With Corollaries~\ref{cor:weak-finite-length} and \ref{cor:free-finite-length} in hand, 
the finite length property is far beyond a theory of two examples.
One quick way to produce even more examples is by considering an \emph{interpretation} in $\A$, 
which consists of an orbit-finite set $\B$ over $\A$ equipped with $\Aut(\A)$-equivariant relations.
(In particular, $\B$ is a \emph{first-order reduct} of $\A$ if the underlying orbit-finite set is just $\A$.)

\begin{lemma}\label{lem:interpretations-preserve-fin-len}
    If $\A$ is oligomorphic (resp., has the finite length property over the field $\FF$), then so does any interpretation $\B$ in $\A$.
\end{lemma}
\begin{proof}
    Recall that $\B$ is acted upon by $\Aut(\A)$.
    In fact, any $\pi \in \Aut(\A)$ defines a bijection $\hat \pi : \B \to \B ; a \mapsto \pi(a)$ which by definition belongs to $\Aut(\B)$. 
    As $\A$ is oligomorphic, $\B^d$ is an orbit-finite set over $\A$ for every $d$.
    But \[
        \Aut(\B) \cdot b \mapsto \Aut(\A) \cdot b, \quad b \in \B^d
    \] defines a surjection of the  $\Aut(\B)$-orbits onto the $\Aut(\A)$-orbits,
    so $\B^d$ is orbit-finite over $\B$ as well
    --- this shows $\B$ is oligomorphic.

    Now, consider a chain $V_0 \subset V_1 \subset \cdots \subset V_l \subseteq \Lin_\FF \B^d$ of $\Aut(\B)$-equivariant subspaces. 
    As $\pi(v) = \hat \pi(v)$, each $V_i$ is also $\Aut(\A)$-equivariant.
    The finite length property of $\A$ then yields an upper bound on $l$.
\end{proof}

\begin{theorem}
    Every countable homogeneous undirected graph has the finite length property over any field.
\end{theorem}

\begin{proof}
    We rely on the intricate classification result of Lachlan and Woodrow~\cite{LachlanWoodrow_80} which says that any countable homogeneous graph, or its complement, is isomorphic to: 
    \begin{enumerate}
        \item the Rado graph;
        \item the $K_n$-free Henson graph, where $n \geq 3$; or
        \item $m$ disjoint copies of the complete graph $K_n$, where at least one of $m$ and $n$ is infinite.
    \end{enumerate}
    Since a graph $(\A, \sim)$ and its complement $(\A, \nsim)$ share the same automorphism group, 
    when studying the length of $\Lin_\FF \A^d$ we only need to examine the former graph.
    
    In cases (1) and (2), the finite length property follow directly from Corollary~\ref{cor:free-finite-length}.
    For case (3), observe that $\{1, \dots, m\} \times \N$, $\N \times \{1, \dots, n\}$, and $\N \times \N$ are all orbit-finite sets over the equality atoms $(\N, =)$, where $m$ and $n$ are finite;
    on each of these, the binary relation $\sim$ checking equality on the first coordinate is certainly equivariant.
    We thus conclude using Lemma~\ref{lem:interpretations-preserve-fin-len}:
    again by Corollary~\ref{cor:free-finite-length}, the equality atoms have the finite length property, as do any interpretations.
\end{proof}

Another recipe for producing more structures from $\A$ is by considering $(\A, {=}c_1, \dots, {=}c_k)$, where $c_1, \dots, c_k \in \A$ are now constants.
Because $\A^{d+k}$ is orbit-finite for every $d$,
this expansion is again oligomorphic if $\A$ is so.
Whether the finite length property is preserved is more subtle;
we list some positive examples:
\begin{itemize}
    \item 
    Expansions of $(\N, =)$ or $(\Q, \leq)$ by finitely many constants are interpretable in the original structure~\cite[Lemma~2.22]{BodirskyBodorMarimon_25},
    so they have the finite length property by Theorem~\ref{thm:ordered-free-amalg-has-finite-length} in light of 
    Lemma~\ref{lem:interpretations-preserve-fin-len}. 
    (We just paraphrased the proof of \cite[Theorem~4.10]{BFKM24} model-theoretically.)

    \item 
    On the other hand, when $\A$ is the Rado graph, it was shown in \cite{no-constant-interpreted-in-Rado} that for no $c \in \A$ does $(\A, {=}c)$ interpret in $\A$. 
    Instead, we explicitly took the constant into account when we established the finite length property in Corollary~\ref{cor:free-finite-length}.

    \item
    \textcolor{red}{Homogeneous, smoothly approximable structures are also closed under expansions by constants? Do we mention this?}
\end{itemize}


Let us summarise where the cog-based approach in Sections~\ref{sec:free-amalg}--\ref{sec:equivariant-subspaces} brings us. 
We treat the generically ordered Fraïssé limit of any free amalgamation class, over a finite relational vocabulary that has to be binary for technical reasons. 
(This restriction is necessary for tackling forbidden substructures. Whilst we believe the proofs can be adapted to handle the Fraïssé limit of ordered $k$-uniform hypergraphs,
Evans's result \cite[Proposition~3.10]{Evans_pm1} that $\Lin \A^2$ has finite length without this arity assumption seems out of reach.) 
This approach culminates in Theorem~\ref{thm:cog-span-generally}, 
which allows us in Section~\ref{sec:equivariant-subspaces} to describe all equivariant subspaces in any orbit-finite-dimensional vector space --- in the presence of constants, in any field --- and deduce the finite length property with a tight bound.
However, as we saw in Section~\ref{sec:unambiguous}, these structures can give rise to ill-behaved models of computation:
over the Rado and Henson graphs, orbit-finitely spanned sets do not admit orbit-finitely spanned function spaces, and weighted register automata fail to be closed under reversal.

The cogless approach for oligomorphically approximated structures in Section~\ref{sec:characteristic-zero} is almost the opposite.
The finite length property only holds over characteristic zero, 
orthogonal to the model-theoretic motivation to refute Thomas's conjecture.
(And indeed some structures do not have the finite length property over finite fields.)
The proof here is quick and elegant, and cover two important subclasses of oligomorphic structures. 
The first is those that are $\omega$-stable: 
these admit orbit-finitely spanned dual spaces according to \cite[Theorem~3.7]{Przybyłek_2024}.
The second is those that are weakly Lie coordinitisable, which are thoroughly studied in \cite{CherlinHrushovski_03} and provide a rich arsenal of examples for us to understand.

\textcolor{red}{
We leave two concrete open questions. Which has the finite length property:
\begin{enumerate}
    \item the Fraïssé limit of finite partially ordered sets?
    \item the Fraïssé limit of finite Boolean algebras, over characteristic zero?
\end{enumerate}
}