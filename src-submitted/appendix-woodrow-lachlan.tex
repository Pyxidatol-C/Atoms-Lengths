\section{Proof of Theorem~\ref{thm:lachlan-woodrow}}
    We rely on the intricate classification result of Lachlan and Woodrow~\cite{LachlanWoodrow_80} which says that any countable homogeneous graph, or its complement, is isomorphic to one of: (a) the Rado graph;  (b) the $K_n$-free Henson graph, where $n \geq 3$; or (c) $m$ disjoint copies of the complete graph $K_n$, where at least one of $m$ and $n$ is infinite.
    Since a graph $\A$ and its complement share the same automorphism group, 
    when studying the length of $\Lin_\FF \A^d$ we only need to examine the structures of kinds (a), (b) and (c). For the first two kinds, the finite length property follows directly from Corollary~\ref{cor:free-finite-length}. The structures of kind (c) admit a first-order interpretation in the equality atoms, and hence they inherit the finitel length property from them. This is because by doing a first-order interpretation, we can only have fewer orbit-finite sets and orbit-finitely vector spaces. 
