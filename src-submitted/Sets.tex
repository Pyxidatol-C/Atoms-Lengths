\section{Orbit-finite sets}
\label{sec:orbit-finite-sets}
We shall now briefly explain the concept of orbit-finiteness.
We start with a countable oligomorphic structure $\A$, as described in Section~\ref{sec:structures}, whose elements we call \emph{atoms}. These can be used to construct sets that are finite up to the symmetries of $\A$, such as
\begin{align*}
\myunderbrace{\A^2}{pairs}
\qquad 
\myunderbrace{\setbuild{ (a,b) \in \A^2}{ $a \neq b$ }}{non-repeating pairs}
\qquad 
\myunderbrace{\binom \A 2}{unordered pairs}.
\end{align*}
% Since we want to be able to construct tuples of arbitrary finite length, and we want them to be finite up to symmetries, we will need to assume $\A$ is oligomorphic.

There are several equivalent definitions of orbit-finiteness. Of these, we use one that is based on first-order interpretations~\cite[Sec.~4.3]{hodges1993model}. To construct an orbit-finite set, we  proceed in three steps: take a finite power of the atoms (as in the first example above), restrict this power to an equivariant subset (as in the second example above), and then take a quotient under an equivariant equivalence relation (as in the third example above, where the equivalence relation identifies two pairs if they differ only in their order). The formal definition is given below.

 \begin{definition}[Orbit-finite set]\label{def:orbit-finite-set}
    An \emph{orbit-finite set} over an oligomorphic structure $\A$ is any set that is obtained as follows: 
    \begin{enumerate}
        \item Start with a finite power $\A^d$ for some $d \in \set{0,1,\ldots}$.
        \item Restrict it to an equivariant subset $X \subseteq \A^d$.
        \item Quotient  $X$ under an equivariant equivalence relation. 
    \end{enumerate}
 \end{definition}

Let us justify the name ``orbit-finite'' in this definition. 
An orbit-finite set is  equipped with an action of the automorphism group of the original structure $\A$, namely the action inherited  from $\A^d$, suitably extended to the quotient. Under this action, the set has finitely many orbits: $\A^d$ has finitely many orbits by assumption on oligomorphism, and the number of orbits can only go down when  restricting to an equivariant subset and quotienting under an equivariant equivalence relation. 
  
 There are other equivalent definitions of orbit-finite sets. One, see Definition~\ref{def:orbit-finite-set-alternative} below, emphasises the role of the  group action. In this definition, we use the following notion of support: if $X$ is a set equipped with an action of the automorphism group of $\A$, then a \emph{support} for an element $x \in X$ is any set $S \subseteq \A$ such that every automorphism $\pi$ of $\A$ satisfies
  \begin{align*}
  \myunderbrace{\forall a \in S\ \pi(a)=a}{action of $\pi$ on $\A$}
  \quad \implies \quad 
  \myunderbrace{\pi(x)=x}{action of $\pi$ on $X$}.
  \end{align*}
\begin{definition}
   [Orbit-finite sets, abstractly] \label{def:orbit-finite-set-alternative} An \emph{orbit-finite set} over an oligomorphic structure $\A$ is a set $X$ equipped with an action of the automorphism group of $\A$ such that: 
   \begin{enumerate}
      \item there are finitely many orbits under the action;
      \item every element has some finite support.
   \end{enumerate}
\end{definition}

Thanks to oligomorphism of the underlying atom structure, the  two definitions above are equivalent in the sense that they describe the same sets, up to equivariant bijections --- see~\cite[Thm.~5.13]{bojanczyk_slightly}. Both definitions have their uses. Definition~\ref{def:orbit-finite-set} is more concrete, and it comes with a finite representation, which can be used for algorithms that process orbit-finite sets. On the other hand, some constructions --- e.g.~disjoint unions, products --- will result in sets that are consistent with the second, more relaxed, definition. %An example of this arises in automata theory, where the  minimal automaton, whose states are equivalence classes of input word with respect to some equivalence relation~\cite[Thm.~5.2]{bojanczykAutomataTheoryNominal2014}. 
%  \paragraph*{Orbit-finite automata.} As mentioned in the introduction,  the study of orbit-finite sets was originally motivated by automata and regular languages over infinite alphabets~\cite{bojanczykNominalMonoids2013,bojanczykAutomataTheoryNominal2014}. The idea is to use standard models of computation, but to replace finite sets with orbit-finite ones, while keeping all structure equivariant. The standard example is orbit-finite automata, defined as follows.
%     An orbit-finite nondeterministic automaton over atoms $\A$ is defined like a nondeterministic finite automaton, except that the states and alphabet are orbit-finite sets over $\A$, instead of finite ones, and all structure (the sets of initial and final states and the transition relation) are equivariant. A deterministic orbit-finite automaton is the special case which has only one initial state, and where the transition relation is a function. 
%  \begin{myexample}[Cycles in the Rado graph]
%     Assume that the atoms are the Rado graph. The cycles in the Rado graph can be viewed as a language $L \subseteq \A^*$, which consists of words where every two consecutive letters are related by an edge in the atoms, and furthermore there is an edge between the last and the first letter. This language can be recognised by a deterministic  orbit-finite automaton, which uses its states to remember the first letter of the input word as well as the most recently read letter. The state space of this automaton is the disjoint union
%     \begin{align*}
%       \myunderbrace{\set{\varepsilon}}{initial \\ state}
%       \quad + \quad 
%       \myunderbrace{\A^2}{first letter and \\ most recent letter}
%         \quad + \quad
%         \myunderbrace{\set{\bot}}{rejecting \\ sink state}.
%     \end{align*}
%     This is an orbit-finite set. This is because the initial and sink states can be seen as two copies of $\A^0$, and  orbit-finite sets are closed under disjoint unions, assuming that the atom structure has at least two elements.
%  \end{myexample}