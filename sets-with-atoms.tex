
\section{Who has finite length?}
\begin{definition}
    An \emph{interpretation in $\A$} is a structure $\A' = (D/E ; R_1, R_2, \dots)$, where
    \begin{itemize}
        \item 
        $D$ is an equivariant subset of $\A^n$ for some $n \geq 1$;

        \item 
        $E$ is an equivariant equivalence relation on $D$;

        \item 
        $D/E$ consists of equivalence classes $d/E \subseteq D$ for $d \in D$,
        which $\pi \in \Aut(\A)$ acts on via $\pi \cdot d/E = (\pi \cdot d) / E$;

        \item 
        every relation $R_i$ of arity $r_i$ is an equivariant subset of $(D/E)^{r_i}$.
    \end{itemize}
    We call $\A'$ a \emph{reduct of $\A$} if $D = \A$ and $E$ is just equality.
\end{definition}



\begin{proposition}
    If $\A$ has the finite length property over $\FF$,
    then so does any interpretation $\A'$.
\end{proposition}
\begin{proof}
    Let $V_0 \subsetneq V_1 \subsetneq \cdots \subsetneq V_l$ be a chain of $\Aut(\A')$-equivariant subspaces in $\Lin_\FF (\A')^k = \Lin_\FF (D/E)^k$.
    Each $V_j$ is then $\Aut(\A)$-equivariant: 
    given $\pi \in \Aut(\A)$, notice that $d/E \mapsto (\pi \cdot d)/E$ is an automorphism of $\A'$.
    Now $l$ is bounded above by the $\Aut(\A)$-length of $\Lin_\FF (D/E)^k$, which is finite
    --- $\Lin_\FF (D/E)^k$ is isomorphic to the quotient of $\Lin_\FF D^k$ by the span of 
    \[
        \left\{ 
            \begin{aligned}
                &(d_1, \dots, d_k) \\
                -{} &(d'_1, \dots, d'_k) 
            \end{aligned}
            \middle| d_1/E = d'_1/E, \dots, d_k/E = d'_k/E
        \right\}.
    \] \qedhere
\end{proof}